\chapter*{Indledning}
\addcontentsline{toc}{chapter}
{\protect\numberline{}{Indledning}}
\evenpageheader{\bsf Indledning}{}{}
\oddpageheader{}{}{\bsf Indledning}
\label{cha:Indledning}
``Der var engang, hvor kemiske reaktioner egentlig ikke var
s{\ae}rligt fasci\-ner\-ende''. Enhver videnskabsmand med sin
fornuft i behold vil selvf{\o}lgelig kr{\ae}ve et
pr{\ae}cist argument for denne noget l{\o}se p{\aa}stand.
S{\aa} i stedet b{\o}r man m{\aa}ske for at undg{\aa}
kontroverser indskr{\ae}nke sig til det lidt svagere
udsagn, ``der var engang, hvor kemiske reaktioner ikke var
halvt s{\aa} fascinerende, som de er idag''. P{\aa}
nuv{\ae}rende tidspunkt vil de fleste afd{\o}de kemikere
nok vende sig mindst et par gange i deres respektive grave,
mens de flerstemmigt udbryder, ``med hvilken uds{\o}gt
fr{\ae}khed og i tarveligt formulerede vendinger t{\o}r
nogen stemple vores arbejde som v{\ae}rende
usp{\ae}ndende''. M{\aa}ske b{\o}r man for en stund lade
v{\ae}re med at dv{\ae}le ved fortiden og kort og godt sl{\aa}
fast, ``at kemiske reaktioner idag er utroligt
fascinerende''.

\vspace{4.0mm}
Anvendelse af {\em ikke-line{\ae}r dynamik\/} har i lang
tid bidraget med mange frugtbare resultater i dele af den
naturvidenskabelig forskning, der bl.a.\ har resulteret i,
at {\em kemisk reaktionskinetik\/} indenfor de sidste 25
{\aa}r har gennemg{\aa}et en gevaldig
``ansigtsl{\o}ftning''. Videnskaben har opdaget, at
forl{\o}bet af kemiske reaktioner sagtens kan spr{\ae}nge
de gr{\ae}nser, som man i lang tid troede termodynamikkens
2.\ hoveds{\ae}tning forb{\o}d dem at oveskride. Kemiske
reaktioner kan oscillerer, oscillationerne kan
periodefordoble og kan tilmed v{\ae}re kaotiske. Man kan
tage reagenser, der er nu om dage st{\aa}r p{\aa} hylden i
ethvert laboratorium, og h{\ae}lde et tyndt lag ud i en
petrisk{\aa}l, hvorefter spiralm{\o}nstrer og
turbulensf{\ae}nomener vil kunne studeres med det blotte
{\o}je. Ofte kalder man s{\aa}danne reaktioner for {\em
komplekse kemiske reaktionssystemer\/}.

\vspace{4.0mm}
Men en videnskabsmand tager sig ikke til takke med blot at
betragte disse f{\ae}nomener, der udfolder sig foran ham.
Han vil forst{\aa} deres natur p{\aa} en pr{\ae}cis
m{\aa}de. Hvordan kan disse reaktioners egenskaber
beskrives p{\aa} en kvantitativ m{\aa}de? Hvorledes
unders{\o}ges reaktionerne eksperimentelt s{\aa}ledes, at
de kan kontrolleres. Hvordan sammenlignes de
eksperimentelle m{\aa}linger med den teoretiske
beskrivelse?

\newpage
Nogen af disse sp{\o}rgsm{\aa}l er idag delvist besvaret.
Man form{\aa}r idag til et vist punkt at beskrive og
forklare nogle af de mange f{\ae}nomener, der udvises af
komplekse reaktioner. Men der er ogs{\aa} mange l{\o}se
ender. Der er f{\ae}nomener, der volder problemer, n{\aa}r
de skal beskrives teoretisk; og der er mindst liges{\aa}
mange problemer involveret, n{\aa}r en bestemt type
opf{\o}rsel skal kontrolleres og fastholdes eksperimentelt.

\vspace{2.0mm}
Det er svarene p{\aa} nogle af de ovenst{\aa}ende
sp{\o}rgsm{\aa}l vi vil fors{\o}ge at beskrive i denne
prisopgave. Det ville have v{\ae}ret en umenneskelig og
h{\aa}bl{\o}s opgave p{\aa} tilfredsstillende vis at
beskrive alt, hvad der idag vides om komplekse reaktioner.
Vi har derfor udvalgt en r{\ae}kke emner indenfor
ikke-line{\ae}r reaktionskinetik, som vi selv synes er
interssante. Den teoretiske behandling af stoffet er
v{\ae}gtet h{\o}jest, da den nu engang passer bedst til
vores temperament. Men den eksperimentelle del er da
bestemt ogs{\aa} tilgodeset. Kort kan indholdet af de
enkelte afsnit opsummeres som f{\o}lger (de enkelte forfatteres
navne er anf{\o}rt i parentes)

%%%%%%%%%%%%%%%%%%%%%%%%%%%%%%%%%%%%%%%%%%%%%%%%%%%%%%%%%%%%%%%%%%%%%%%%
%% Kapitel oversigt
%%%%%%%%%%%%%%%%%%%%%%%%%%%%%%%%%%%%%%%%%%%%%%%%%%%%%%%%%%%%%%%%%%%%%%%%
\begin{quote}
  \vspace{2.5mm}
  {\bsf Kapitel 1}. En historisk gennemgang af de
  opfattelser og {\ae}ndringer kemisk reaktionsskinetik har
  underg{\aa}et i l{\o}bet af det 20.\ {\aa}rhund\-rede. (Mads Ipsen)

  \vspace{2.2mm}
  {\bsf Kapitel 2}. Pr{\ae}sentation af teorien for {\em
  quenching af kemiske oscillationer\/}. Udover de
  teoretiske afsnit gennemg{\aa}r kapitlet ogs{\aa} en
  r{\ae}kke eksperimentelle resultater. (Mads Ipsen)

  \vspace{2.2mm}
  {\bsf Kapitel 3}. Teori og anvendelse af {\em kemiske
  netv{\ae}rk\/}, som benytter eksperimentelle data fra
  quenchingerne. (Kenneth Geisshirt)

  \vspace{2.2mm}
  {\bsf Kapitel 4}. I dette kapitel pr{\ae}senteres en
  forholdvis ny teori om {\em pertubation af station{\ae}re
  tilstande\/}, der muligg{\o}r en udledning af
  reaktionsmekanismer for visse kemiske reaktioner. (Kenneth Geiss\-hirt)

  \vspace{2.2mm}
  {\bsf Kapitel 5}. Beskrivelse af oscillationer i kemiske
  reaktioner. (Mads Ipsen)

  \vspace{2.2mm}
  {\bsf Kapitel 6}. {\em Rekonstruktion af tiltr{\ae}kkere\/}
  udfra en analyse af tids\-r{\ae}kker m{\aa}lt p{\aa}
  kemiske reaktioner. (Kenneth Geisshirt)

  \vspace{2.2mm}
  {\bsf Kapitel 7}. Teoretisk gennemgang af
  {\em Poincar\'{e}afbildningen\/} og dennes anvendelse til at
  beskrive egenskaber for komplekse kemiske reaktioner. (Mads Ipsen)
\end{quote}

\newpage
Mange af kapitlerne kr{\ae}ver et kendskab til den
matematiske beskrivelse af ikke-line{\ae}r dynamik. Nogen
ville m{\aa}ske indvende, at det derfor ville have
v{\ae}ret p{\aa} sin plads, hvis et enkelt kapitel var
blevet reserveret til at redeg{\o}re for de forskellige
matematiske begreber. Alligevel har vi valgt at undlade
dette. Der findes idag s{\aa} mange v{\ae}rker, der giver
udm{\ae}rkede introduktioner til disse begreber, at vi har
fundet det omsonst at gentage, hvad der er sagt bedre og
mere pr{\ae}cist andensteds. I stedet henviser der
l{\o}bende i teksten til litteratur, hvor et
p{\aa}g{\ae}ldende begreb er beskrevet.

\vspace{4.0mm}
Vi {\o}nsker at takke {\em Christa Trandum\/} for hendes
danske overs{\ae}ttelse af citatet p{\aa}
side~\pageref{`cite:Nernst'}.

\vspace{2.5mm}
Derudover rettes en s{\ae}rlig tak til {\em Keld
Nielsen\/}, {\em Dorte Ipsen\/} og {\em Mikael Larsen\/}
for deres omhyggelige hj{\ae}lp med korrekturl{\ae}sningen.

\vspace{2.5mm}
Sidst men ikke mindst takker vi {\em Preben Graae
S{\o}rensen\/} og {\em Finn Hynne\/} for deres velvilje til
at diskutere og forklare en masse af de uklarheder, der
opstod undervejs under udarbejdelsen af dette projekt.

\vspace{1.0cm}
\begin{flushright}
  Kenneth Geisshirt, Mads Ipsen\\
  K{\o}benhavn, d.\ 17 januar, 1994
\end{flushright}









