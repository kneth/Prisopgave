\pageheaderlinetrue
\evenpageheader{\bsf\leftmark}{}{}
\oddpageheader{}{}{\bsf\rightmark}
\evenpagefooter{\thepage}{}{}
\oddpagefooter{}{}{\thepage}

\chapter{Historisk indledning}
\label{cha:Historie}
\pagenumbering{arabic}
\section{Kemisk kinetik}
{\em Kemisk kinetik\/} omhandler tidsforl{\o}bet af kemiske
reaktioner. N{\aa}r en reaktion finder sted, {\ae}ndrer
koncentrationerne af de implicerede stoffer sig. Visse
reaktioner foreg{\aa}r langsomt, hvorimod andre finder sted
n{\ae}sten momentant. Til en given reaktion kan vi knytte
en reaktions\-hastighed, der beskriver hvor meget de
enkelte stoffers koncentration {\ae}ndres pr.\ tidsenhed.
Den hastighed, hvormed stoffernes koncentration {\ae}ndres
som funktion af tiden, defineres ved en
differentialkvotient $\frac{dc_i}{dt}$, hvor $c_i$ angiver
koncentrationen af det $i$'te stof til tiden $t$. Betragter
vi for eksempel reaktionen

\begin{equation}
  {\rm A} + {\rm B}  \rightarrow {\rm C} + {\rm D}
  \label{eq:VerySimpleReaction}
\end{equation}

f{\aa}s f{\o}lgende sammenh{\ae}ng mellem
hastigheds\-udtrykkene for de fire stoffer A, B, C og D

\begin{equation}
 - \frac{d{\rm [A]}}{dt} =
 - \frac{d{\rm [B]}}{dt} =
   \frac{d{\rm [C]}}{dt} =
   \frac{d{\rm [D]}}{dt}
\end{equation}

I kemisk kinetik er det oftest fordelagtigt at unders{\o}ge
s{\aa}kaldte {\em elementarreaktioner\/}, da
hastigheds\-udtrykkene for s{\aa}danne er specielt simple.
Lidt l{\o}st kan vi definere en elementarreaktion som en
reaktion, {\em hvis reaktions\-lig\-ning repr{\ae}senterer
det faktiske fysiske forl{\o}b, efter hvilken reaktionen
finder sted\/}. Ofte vil elementarreaktioner enten v{\ae}re
{\em uni\/}- eller {\em bimolekyl{\ae}re\/}, svarende til
at disse enten beskriver en spontan omdannelse af et enkelt
atom/molekyle eller et sammenst{\o}d mellem to
atomer/molekyler. Antager vi eksempelvis, at
reaktion~\ref{eq:VerySimpleReaction} er en
elementarreaktion, da vil reaktionen i det
p{\aa}g{\ae}ldende reaktions\-medium realiseres ved et
sammenst{\o}d mellem et A og et B molekyle under dannelse
af et C og et D molekyle. Derimod vil en reaktion som

{
\newcommand{\ho}        {\mbox{H$_2$O}}
\newcommand{\permn}     {\mbox{MnO$^-_4$}}
\newcommand{\mn}        {\mbox{Mn$^{2+}$}}
\newcommand{\bromid}    {\mbox{Br$^-$}}
\newcommand{\brom}      {\mbox{Br$_2$}}
\newcommand{\prot}      {\mbox{H$^+$}}
\begin{equation}
 2 \permn + 10 \bromid + 16 \prot \rightleftharpoons 
 2 \mn    + 5  \brom + 8 \ho
\end{equation}

ikke v{\ae}re nogen elementarreaktion, da den tilknyttede
omdannelse rimeligvis ikke finder sted via et samtidigt
sammenst{\o}d mellem 2 \permn-ioner, 10 \bromid-ioner og 16
protoner.} I dette tilf{\ae}lde fors{\o}ger man i stedet at
dekomponere reaktionen i et s{\ae}t elementarreaktioner,
efter hvilket reaktionen menes at kunne forl{\o}be. Et
s{\aa}dant s{\ae}t elementarreaktioner kaldes {\em en
reaktions\-meka\-nis\-me\/}. Bestemmelsen af en
reaktions\-mekanisme er ofte et meget kompliceret
puslespil, der involverer et omfattende eksperimentelt
arbejde. Vi vil senere i kapitel~\ref{cha:PertStat} se
eksempler p{\aa}, hvorledes man udfra en r{\ae}kke
teoretiske overvejelser kan bestemme reaktions\-mekanismen
for visse typer af kemiske reaktioner.

\vspace{4.0mm}
{
\newcommand{\brint}     {\mbox{H$_2$}}
\newcommand{\iod}       {\mbox{I$_2$}}
\newcommand{\hi}        {\mbox{HI}}
\newcommand{\reactionarrow}[2]
{\begin{array}{c}
  \mbox{\scriptsize $#1$} \\[-1.5mm]
  \rightleftharpoons      \\[-2.5mm]
  \mbox{\scriptsize $#2$} 
\end{array}}

Som n{\ae}vnt tidligere er de hastigheds\-udtryk, der
knyttes til en elementarreaktion forholdsvis ukomplicerede,
idet disse opfylder en s{\aa}kaldt {\em
potenslov\-kinetik\/}. Vi kan nemlig med rimelighed antage,
at den hastighed, hvormed et stofs koncentration
{\ae}ndres, er proportional med koncentrationen af de
reagerende stoffers koncentration opl{\o}ftet til deres
respektive st{\o}kiometriske koeffi\-cienter.
Proportionalitetskonstanten, der knyttes til et s{\aa}dant
hastig\-heds\-udtryk, kaldes en {\em
hastig\-heds\-konstant\/}. For at afmystificere dette en
smule betragter vi et klassisk eksempel, nemlig reaktionen
mellem hydrogen og iod i gasfase~\cite{KAJensen}

\begin{equation}
  \brint + \iod \reactionarrow{k_1}{k_{-1}} \hi
\end{equation}

Her angiver $k_1$ og $k_{-1}$ hastigheds\-konstanterne for
den fremadg{\aa}ende henholds\-vis tilbageg{\aa}ende
reaktion. F{\o}lger vi de n{\ae}vnte foreskrifter for
potenslovs\-kinetik, da bliver hastigheds\-udtrykket for
\brint

\begin{equation}
  \frac{d[\brint]}{dt} = -k_1 [\brint][\iod] + k_{-1}[\hi]^2
\end{equation}

Hastighedsudtrykkene for \iod~og \hi~findes ved en
lignende fremgangsm{\aa}de. 
}

\vspace{4.0mm}
{\O}nsker man at beskrive tidsudviklingen for en
vilk{\aa}rlig kemisk reaktion, er det n{\o}dvendigt, at der
for den p{\aa}g{\ae}ldende reaktion er opstillet en
reaktions\-mekanisme. For denne mekanisme udregnes samtlige
involverede stoffers ha\-stig\-heds\-udtryk ved hj{\ae}lp
af de indg{\aa}ende elementarreaktioner. Dette giver
normalt anledning til et system af koblede ikke-line{\ae}re
1.~ordens differentiallig\-ninger. P{\aa} kompakt form har
vi, at en kemisk reaktions kinetik er beskrevet ved

\begin{equation}
  \dot{\bf c} = {\bf f}({\bf c}),
  \ms{hvor} {\bf c} \in \R^n 
  \ms{og} {\bf f}: \R^n \mapsto \R^n
  \label{eq:GeneralRateEq}
\end{equation}

hvor vektoren ${\bf c}$ angiver koncentrationerne,
$c_1,\ldots,c_n$, af de stoffer, der deltager i reaktionen.
Vektorfunktionen ${\bf f}$ vil oftest v{\ae}re et
vektorpolynomie i koncentrationerne $c_1,\ldots, c_n$, men
kan dog ogs{\aa} indeholde andre funktionstyper. 

\vspace{4.0mm}
Det er differentiallig\-ninger af denne type, der idag
spiller den st{\o}rste rolle ved kvalitative s{\aa}vel som
kvantitative sammenlig\-ninger mellem empiriske data fra
kemiske reaktioner og modeller for disse. S{\aa}danne
ikke-line{\ae}re 1.\ ordens differentiallig\-ninger kan kun
yderst sj{\ae}ldent l{\o}ses analytisk, hvorfor
computer\-simuleringer ofte benyttes, n{\aa}r lig\-ningernes
egen\-skaber skal analyseres.

\vspace{4.0mm}
Egenskaberne ved s{\aa}danne ikke-line{\ae}re
differentiallig\-ninger kan v{\ae}re utrolig komplicerede,
hvilket vi i de efterf{\o}lgende kapitler skal se mange
eksempler p{\aa}. Det er dog f{\o}rst inden for de sidste
tyve {\aa}r, at videnskaben rigtigt er blevet klar over den
mangfoldighed af kompleks opf{\o}rsel, der kan gemme sig
bag differentiallig\-ninger af ovenst{\aa}ende type. I
starten af dette {\aa}rhundrede var der n{\ae}ppe nogen,
der havde forestillet sig, at en kemisk reaktion ville
kunne besidde nogen af disse egen\-skaber. Lad os derfor
kaste et blik p{\aa} den m{\aa}de hvorp{\aa} kemiske
reaktioner opfattedes p{\aa} den tid.

\vspace{4.0mm}
Allerede i slutningen af forrige {\aa}rhundrede foretager
den franske matematiker Henri Poincar\'{e} en r{\ae}kke
epokeg{\o}rende studier af dynamiske systemers natur
\cite{PoinOrig} med henblik p{\aa} at forst{\aa} mekaniske
systemer i fysikken. Det er derfor rimeligt at antage, at
man p{\aa} dette tidspunkt er klar over, at
differentiallig\-ninger med oscillerende l{\o}sninger har
en betydning i fysikkens beskrivelse af verden. En lignende
analogi genfindes dog ikke i det samtidige kemiske
verdensbillede.

\vspace{4.0mm}
Den kemiske reaktion opfattes som et forl{\o}b af ``en
r{\ae}kke stramt knyttede b{\aa}nd'', der ul{\o}seligt er
knyttet til reaktionens vej mod ligev{\ae}gt. En kemisk
reaktion betragtes som v{\ae}rende uden frihedsgrader og
tillades derfor \'{e}n og kun \'{e}n passage mod
ligev{\ae}gt. I 1921 skriver Walther Nernst i tiende oplag
af ``Theoretische Chemie'' \cite{Nernst} f{\o}lgende

\begin{quote}
  ``\ldots\ Integration \label{`cite:Nernst'} af
  differentiallig\-ningen for den kemiske omdannelse giver
  i alle tilf{\ae}lde det resultat, at ligev{\ae}gt
  f{\o}rst bliver oprettet efter uendelig lang tid;
  f{\o}rst for $t=\infty$ bliver $\frac{dx}{dt}=0$; et
  kemisk system tilstr{\ae}ber, ligesom et st{\ae}rkt
  d{\ae}mpet pendul, aperiodisk mod
  ligev{\ae}gts\-tilstanden. En overskridelse af dette
  m{\aa}l er faktisk uforenelig med vores samlede
  betragtninger af kemiske processer; dette ville betyde,
  at under s{\aa}danne omst{\ae}ndigheder ville en
  reaktions betydning afh{\ae}nge af systemets forhistorie,
  s{\aa}ledes f.eks.\ for to absolut identiske l{\o}sninger
  ville reaktionen kunne forl{\o}be i modsat retning, idet
  den ene n{\ae}rmede sig ligev{\ae}gt og den anden sk{\o}d
  sig ud over. Faktum er, at en s{\aa}dan reaktion endnu
  aldrig er blevet iagttaget\footnote{Citatet er oversat
  fra tysk af Christa Trandum.}.''
\end{quote}

Nernst argumenterer, at hvis en kemisk reaktion skal have
et forl{\o}b, s{\aa}ledes at denne ``rammer'' ved siden af
ligev{\ae}gts\-punktet, da ville reaktionen til forskellige
tids\-rum passere den samme tilstand, men i forskellige
retninger. Umiddelbart giver dette da ogs{\aa} en modstrid.
Opfattes en kemisk reaktions natur som v{\ae}rende
deterministisk, er reaktionens opf{\o}rsel bagud og frem i
tiden til enhver tid entydigt bestemt udfra reaktionens
nuv{\ae}rende tilstand. Hvis reaktionen passerer
ligev{\ae}gts\-punktet, m{\aa} denne n{\o}dvendigvis p{\aa}
et tids\-punkt vende tilbage mod dette, da reaktionen under
alle omst{\ae}ndigheder skal konvergere mod
ligev{\ae}gts\-punktet for $t\rightarrow\infty$. Dette
betyder, at koncentrationen af de stoffer, der deltager i
reaktionen, til to forskellige ``reaktions\-retninger'' vil
antage samme v{\ae}rdier.

\vspace{4.0mm}
Nernst p{\aa}peger udfra dette, at en kemisk reaktion, der
``rammer ved siden af'' ligev{\ae}gts\-tilstanden, vil
passere to identiske tilstande med forskellig fortid og
fremtid, hvilket forekommer absurd. Ideen bag dette
r{\ae}sonnement er illustreret skematisk i
figur~\ref{fig:Nernst}a.

%%%%%%%%%%%%%%%%%%%%%%%%%%%%%%%%%%%%%%%%%%%%%%%%%%%%%%%%%%%%%%%%%%%%%%%%
%% figur
%%
%% beskrivelse : Illustration af Nernsts ide + fysisk pendul
%% type        : PSTricks
%%%%%%%%%%%%%%%%%%%%%%%%%%%%%%%%%%%%%%%%%%%%%%%%%%%%%%%%%%%%%%%%%%%%%%%%
\boxfigure{t}{\textwidth}
{
\vspace{0.5cm}
\begin{center}
  \begin{pspicture}(0,0)(12,5)
    %%%%%%%%%%%%%%%%%%%%%%%%%%%%%%%%%%%%%%%%%%%%%%%%%%%%%%%%%%%%%%%%%%%%%%%%
    %% Nernsts ligev{\ae}gt
    %%%%%%%%%%%%%%%%%%%%%%%%%%%%%%%%%%%%%%%%%%%%%%%%%%%%%%%%%%%%%%%%%%%%%%%%
    \rput[br]{*0}(1.0,4.5){\footnotesize a)}
    \rput[br]{*0}(3.9,2.1){\footnotesize eq.}
    \rput[c ]{*0}(2.95,-0.4){\footnotesize koncentration}
    \psbezier[linewidth=1.8pt,arrowinset=0]{->}(0.5,4)(7.5,2)(6.5,1.5)(0.5,1)
%   \psgrid[subgriddiv=1,griddots=10,gridlabels=7pt](0,0)(0,0)(12,5)
    \psline[linearc=0.1,linewidth=1.8pt,arrowinset=0]{->}(1.0,0.2)(5.4,0.2)(5.4,0)(0.5,0)
    \pscircle*[](4,2){0.075}
    %%%%%%%%%%%%%%%%%%%%%%%%%%%%%%%%%%%%%%%%%%%%%%%%%%%%%%%%%%%%%%%%%%%%%%%%
    %% Fysisk pendul
    %%%%%%%%%%%%%%%%%%%%%%%%%%%%%%%%%%%%%%%%%%%%%%%%%%%%%%%%%%%%%%%%%%%%%%%%
    \psline[linewidth=1.8pt]{-}(9.5,0.5)(9.5,4)(11.5,1.5)
    \pscircle*[](11.5,1.5){0.1}
    \pswedge[linestyle=dashed,dash=3pt 2pt](9.5,4){1.25}{270}{308.66}
    \rput[tc]{*0}(9.9,2.7){\footnotesize $\theta$}
    \rput[br]{*0}(8.5,4.5){\footnotesize b)}
  \end{pspicture}
\end{center}
\vspace{0.5cm}
}
{
\caption{\protect\capsize
a) Skematisk gengivelse af Nernsts argument for at
oscillationer ikke kan forekomme i kemiske reaktioner. Den
{\o}verste kurve viser en reaktionskoordinat, der skyder
forbi ligev{\ae}gtspunktet. Dette betyder at
koncentrationen af de stoffer, der deltager i reaktionen,
til to forskellige punkter p{\aa} reaktionskoordinaten
m{\aa} antage samme v{\ae}rdier. Dette er illustreret ved
den nederste kurve. Til to ens s{\ae}t koncentrationer
findes s{\aa}ledes to forskellige l{\o}sninger til det
deterministiske kemiske system. b) Det hamiltonske pendul.}
\label{fig:Nernst}
}

\vspace{4.0mm}
Lad os fors{\o}ge at sammenligne denne situation med det
hamiltonske pendul fra fysikken. Betragter vi pendulet i
positionen $\theta$, kan vi godt tillade pendu\-let at
passere dette punkt i to forskellige retninger svarende
til, at pendulets tilstand f{\o}rst er fuldst{\ae}ndigt
defineret, n{\aa}r b{\aa}de dets position $\theta$ og
hastighed $\dot{\theta}$ er opgivet
(figur~\ref{fig:Nernst}b).

\vspace{4.0mm}
P{\aa} Nernsts tid var det endnu ikke kendt, at en given
reaktion sj{\ae}ldent forl{\o}ber via selve
reaktions\-lig\-ningen, men derimod finder sted via et
kompliceret samspil mellem en r{\ae}kke intermedi{\ae}re
reaktioner svarende til den tidligere omtalte
reaktions\-mekanisme. Det er tilstedev{\ae}relsen af disse
intermedi{\ae}re reaktioner og stoffer mellem reaktionens
start- og sluttilstand, der for{\aa}rsager Nernsts forkerte
r{\ae}sonnement. Disse intermedi{\ae}re tilstande og dermed
et ekstra s{\ae}t frihedsgrader tillader reaktionen ``at
skyde over ligev{\ae}gts\-punktet''. Vi skal senere se,
hvorledes en reaktion under de rette betingelser faktisk
kan blive ved med ``at skyde over ligev{\ae}gts\-punktet''.

\vspace{4.0mm}
Lad os ved hj{\ae}lp af et simpelt og klassisk
l{\ae}rebogseksempel i kemisk kinetik fors{\o}ge at
illustrere den natur, som kemiske reaktioner i starten af
dette {\aa}rhundrede tillagdes \cite[s.\ 690]{Alberty}. Vi
forestiller os en reaktant A, der via et intermediat B
omdannes til stoffet C. Opskriver vi
reaktions\-lig\-ningerne for denne simple unimolekyl{\ae}re
omdannelse f{\aa}s

\begin{equation}
  {\rm A} \stackrel{k_1}{\longrightarrow}
  {\rm B} \stackrel{k_2}{\longrightarrow} 
  {\rm C}
  \label{eq:Unimolecular}
\end{equation}

hvor $k_1$ og $k_2$ er hastigheds\-konstanterne for de to
reaktioner. 

\newpage
Ved hj{\ae}lp af potenslov\-kinetik f{\aa}s
f{\o}lgende tre hastigheds\-udtryk for de tre stoffer

\begin{eqnarray}
  \frac{dA}{dt} & = & - k_1 A       \nonumber\\
                &   &               \nonumber\\
  \frac{dB}{dt} & = & k_1 A - k_2 B          \\
                &   &               \nonumber\\
  \frac{dC}{dt} & = & k_2 B         \nonumber
  \label{eq:DerivativeABC}
\end{eqnarray}

L{\o}ser vi nu disse tre koblede differentiallig\-ninger, med
begyndelsesbeting\-el\-ser\-ne $A(0) = A_0$ og $B(0) = C(0) = 0$
til tiden $t=0$, finder vi, at tidsudviklingen for de tre
stoffer bliver

\begin{eqnarray}
  A(t) & = & A_0 e^{-k_1 t}                   \nonumber\\
  B(t) & = & A_0 \frac{k_1}{k_2-k_1} 
  \left[ e^{-k_1 t} - e^{-k_2 t} \right]               \\
  C(t) & = & A_0 \left[ 1 + \frac{1}{k_1-k_2} 
  (k_2 e^{-k_1 t} - k_1 e^{-k_2 t}) \right]   \nonumber
\end{eqnarray}

%%%%%%%%%%%%%%%%%%%%%%%%%%%%%%%%%%%%%%%%%%%%%%%%%%%%%%%%%%%%%%%%%%%%%%%%
%% figur
%%
%% beskrivelse : tidsudvikling for unimolekyl{\ae}r reaktion
%% plt         : fig40.plt
%% tex         : fig40.tex
%% type        : TeXDraw
%%%%%%%%%%%%%%%%%%%%%%%%%%%%%%%%%%%%%%%%%%%%%%%%%%%%%%%%%%%%%%%%%%%%%%%%
\boxfigure{t}{\textwidth}
{
\begin{center}
 \vspace{1cm}
  % GNUPLOT: LaTeX using TEXDRAW macros
\begin{texdraw}
\normalsize
\ifx\pathDEFINED\relax\else\let\pathDEFINED\relax
 \def\QtGfr{\ifx (\TGre \let\YhetT\cpath\else\let\YhetT\relax\fi\YhetT}
 \def\path (#1 #2){\move (#1 #2)\futurelet\TGre\QtGfr}
 \def\cpath (#1 #2){\lvec (#1 #2)\futurelet\TGre\QtGfr}
\fi
\drawdim pt
\setunitscale 0.24
\linewd 3
\textref h:L v:C
\path (220 113)(1436 113)
\path (220 113)(220 877)
\path (220 266)(1436 266)
\linewd 4
\path (220 266)(240 266)
\path (1436 266)(1416 266)
\move (198 266)\textref h:R v:C \htext{{\footnotesize$0.2$}}
\linewd 3
\path (220 419)(1436 419)
\linewd 4
\path (220 419)(240 419)
\path (1436 419)(1416 419)
\move (198 419)\htext{{\footnotesize$0.4$}}
\linewd 3
\path (220 571)(1436 571)
\linewd 4
\path (220 571)(240 571)
\path (1436 571)(1416 571)
\move (198 571)\htext{{\footnotesize$0.6$}}
\linewd 3
\path (220 724)(1436 724)
\linewd 4
\path (220 724)(240 724)
\path (1436 724)(1416 724)
\move (198 724)\htext{{\footnotesize$0.8$}}
\linewd 3
\path (463 113)(463 877)
\linewd 4
\path (463 113)(463 133)
\path (463 877)(463 857)
\move (463 68)\textref h:C v:C \htext{{\footnotesize$20$}}
\linewd 3
\path (706 113)(706 877)
\linewd 4
\path (706 113)(706 133)
\path (706 877)(706 857)
\move (706 68)\htext{{\footnotesize$40$}}
\linewd 3
\path (950 113)(950 877)
\linewd 4
\path (950 113)(950 133)
\path (950 877)(950 857)
\move (950 68)\htext{{\footnotesize$60$}}
\linewd 3
\path (1193 113)(1193 877)
\linewd 4
\path (1193 113)(1193 133)
\path (1193 877)(1193 857)
\move (1193 68)\htext{{\footnotesize$80$}}
\path (220 113)(1436 113)(1436 877)(220 877)(220 113)
\move (45 495)\vtext{konc.}
\move (828 23)\htext{$t$/s }
\move (1339 690)\textref h:R v:C \htext{\scriptsize $A(t)$}
\linewd 3
\path (1361 690)(1427 690)
\path (220 877)(220 877)(232 804)(245 737)(257 677)(269 623)
\cpath (281 574)(294 530)(306 490)(318 454)(331 421)
\cpath (343 391)(355 365)(367 340)(380 318)(392 299)
\cpath (404 281)(417 265)(429 250)(441 237)(453 225)
\cpath (466 214)(478 205)(490 196)(503 188)(515 181)
\cpath (527 174)(539 168)(552 163)(564 158)(576 154)
\cpath (588 150)(601 146)(613 143)(625 140)(638 138)
\cpath (650 135)(662 133)(674 131)(687 129)(699 128)
\cpath (711 126)(724 125)(736 124)(748 123)(760 122)
\cpath (773 121)(785 120)(797 120)(810 119)(822 118)
\cpath (834 118)(846 117)(859 117)(871 117)(883 116)
\cpath (896 116)(908 116)(920 115)(932 115)(945 115)
\cpath (957 115)(969 115)(982 114)(994 114)(1006 114)
\cpath (1018 114)(1031 114)(1043 114)(1055 114)(1068 114)
\cpath (1080 114)(1092 114)(1104 114)(1117 113)(1129 113)
\cpath (1141 113)(1153 113)(1166 113)(1178 113)(1190 113)
\cpath (1203 113)(1215 113)(1227 113)(1239 113)(1252 113)
\cpath (1264 113)(1276 113)(1289 113)(1301 113)(1313 113)
\cpath (1325 113)(1338 113)(1350 113)(1362 113)(1375 113)
\cpath (1387 113)(1399 113)(1411 113)(1424 113)(1436 113)
\move (1339 645)\htext{\scriptsize $B(t)$}
\linewd 4
\path (1361 645)(1427 645)
\path (220 113)(220 113)(232 185)(245 246)(257 298)(269 341)
\cpath (281 378)(294 408)(306 433)(318 452)(331 467)
\cpath (343 479)(355 487)(367 492)(380 494)(392 495)
\cpath (404 494)(417 490)(429 486)(441 481)(453 474)
\cpath (466 467)(478 459)(490 450)(503 442)(515 432)
\cpath (527 423)(539 413)(552 404)(564 394)(576 385)
\cpath (588 375)(601 366)(613 356)(625 347)(638 338)
\cpath (650 329)(662 321)(674 312)(687 304)(699 296)
\cpath (711 289)(724 281)(736 274)(748 267)(760 261)
\cpath (773 254)(785 248)(797 242)(810 236)(822 231)
\cpath (834 226)(846 220)(859 216)(871 211)(883 206)
\cpath (896 202)(908 198)(920 194)(932 190)(945 187)
\cpath (957 183)(969 180)(982 177)(994 174)(1006 171)
\cpath (1018 168)(1031 166)(1043 163)(1055 161)(1068 158)
\cpath (1080 156)(1092 154)(1104 152)(1117 150)(1129 149)
\cpath (1141 147)(1153 145)(1166 144)(1178 142)(1190 141)
\cpath (1203 139)(1215 138)(1227 137)(1239 136)(1252 135)
\cpath (1264 134)(1276 133)(1289 132)(1301 131)(1313 130)
\cpath (1325 129)(1338 128)(1350 128)(1362 127)(1375 126)
\cpath (1387 125)(1399 125)(1411 124)(1424 124)(1436 123)
\move (1339 600)\htext{\scriptsize $C(t)$}
\linewd 6
\path (1361 600)(1427 600)
\path (220 113)(220 113)(232 115)(245 120)(257 128)(269 139)
\cpath (281 151)(294 165)(306 181)(318 197)(331 215)
\cpath (343 233)(355 252)(367 271)(380 290)(392 309)
\cpath (404 329)(417 348)(429 367)(441 385)(453 404)
\cpath (466 422)(478 440)(490 457)(503 474)(515 490)
\cpath (527 506)(539 521)(552 536)(564 551)(576 565)
\cpath (588 578)(601 591)(613 604)(625 616)(638 627)
\cpath (650 638)(662 649)(674 659)(687 669)(699 679)
\cpath (711 688)(724 696)(736 705)(748 713)(760 720)
\cpath (773 728)(785 735)(797 741)(810 748)(822 754)
\cpath (834 760)(846 765)(859 770)(871 776)(883 780)
\cpath (896 785)(908 789)(920 794)(932 798)(945 801)
\cpath (957 805)(969 808)(982 812)(994 815)(1006 818)
\cpath (1018 821)(1031 823)(1043 826)(1055 829)(1068 831)
\cpath (1080 833)(1092 835)(1104 837)(1117 839)(1129 841)
\cpath (1141 843)(1153 844)(1166 846)(1178 848)(1190 849)
\cpath (1203 850)(1215 852)(1227 853)(1239 854)(1252 855)
\cpath (1264 856)(1276 857)(1289 858)(1301 859)(1313 860)
\cpath (1325 861)(1338 862)(1350 862)(1362 863)(1375 864)
\cpath (1387 864)(1399 865)(1411 866)(1424 866)(1436 867)
\end{texdraw}

 \vspace{1cm}
\end{center}
}
{
\caption{\protect\capsize Afbildning af hvorledes
koncentrationen af de tre stoffer A, B og C {\ae}ndres som
funktion af tiden under et reaktions\-forl{\o}b beskrevet
ved reaktions\-lig\-ningerne~\protect\ref{eq:Unimolecular}.
I dette eksempel er v{\ae}rdierne $k_1 = {\rm 0.10~s}^{-1}$
og $k_2 = {\rm 0.05~s}^{-1}$ valgt for de to
hastigheds\-konstanter.}
\label{fig:UnimolecularPlot}
}

For et s{\ae}t v{\ae}rdier for $k_1$ og $k_2$ er disse tre
koncentrationer afbildet som funktion af tiden i
figur~\ref{fig:UnimolecularPlot}. Af figuren fremg{\aa}r,
at samtlige stoffers koncentration konvergerer mod et
ligev{\ae}gtspunkt svarende til, at disse ikke {\ae}ndrer
sig i tiden.

\vspace{4.0mm}
Havde vi yderligere inkluderet de modsatrettede reaktioner
i lig\-ning~\ref{eq:Unimolecular} og dermed gjort modellen
mere realistisk ville denne have v{\ae}ret beskrevet ved
f{\o}lgende reaktioner

{
\newcommand{\reactionarrow}[2]
{
\begin{array}{c}
  \mbox{\scriptsize $#1$} \\[-1.5mm]
  \rightleftharpoons      \\[-2.5mm]
  \mbox{\scriptsize $#2$} 
\end{array}
}

\begin{equation}
  {\rm A} \reactionarrow {k_1}{k_2} 
  {\rm B} \reactionarrow {k_3}{k_4} {\rm C} 
  \label{eq:UnimolecularReversible}
\end{equation}
}

Her ville vi atter have fundet, at systemets
koncentrationer i gr{\ae}nsen \mbox{$t \rightarrow \infty$}
ville konvergere mod et s{\ae}t af gr{\ae}nsev{\ae}rdier,
der opfylder udtrykkene for de to ligev{\ae}gts\-konstanter

\begin{equation}
  \frac{B_{\rm eq}}{A_{\rm eq}} = \frac{k_1}{k_2} \ms{og}
  \frac{C_{\rm eq}}{B_{\rm eq}} = \frac{k_3}{k_4}
  \label{eq:EquiConst}
\end{equation}

P{\aa} trods af en st{\o}rre kompleksitet i det s{\ae}t
differentiallig\-ninger, der be\-skri\-ver tidsudviklingen
for reaktion~\ref{eq:UnimolecularReversible}, ville vi
stadig have fundet, at systemets langtidsopf{\o}rsel ville
v{\ae}re beskrevet ved et ligev{\ae}gts\-punkt. Videre
ville vi analogt med observationerne fra
figur~\ref{fig:UnimolecularPlot} ogs{\aa} have fundet, at
de kurver, der beskriver selve tidsudviklingen, ville
v{\ae}re af en forholdsvis simpel karakter.

\section{Lotka-Volterra modellen}
\label{sec:LotkaSection}
Denne opfattelse af kemisk kinetik og tidsudviklingen i et
reaktions\-forl{\o}b var ener{\aa}dende indenfor kemisk
reaktionskinetik indtil midten af 50'erne. Man opfatter
forl{\o}bet af en kemisk reaktion som v{\ae}rende bygget op
af en simpel transientperiode, langs hvilken
koncentrationerne af de kemiske stoffer konvergerer mod
deres respektive ligev{\ae}gts\-v{\ae}rdier. Til trods for
dette har der alligevel tidligere v{\ae}ret publiceret
artikler, der beskriver kemiske modelsystemer, hvis
opf{\o}rsel er i strid med ovenst{\aa}ende opfattelse.

\vspace{4.0mm}
Her har den model, der idag kendes under navnet
``Volterra-Lotka modellen'', v{\ae}ret af st{\o}rre
betydning. Modellen forkastes dog af samtidens
eta\-ble\-rede kemi\-kere som v{\ae}rende spekulativ
matematik uden ``virkeligt'' kemisk indhold. Ironisk nok
bet{\o}d dette, at modellens kemiske betydning ogs{\aa}
blev forkastet af dennes ophavsm{\ae}nd. Retrospektivt kan
dette faktisk kun beklages, da det viser sig, at
``Volterra-Lotka modellen'' besidder mange af de typiske
egen\-skaber, der karakteriserer de komplekse kemiske
reaktioner, der kendes idag. P{\aa} grund af dennes
historiske betydning vil ``Volterra-Lotka modellen'' blive
gennemg{\aa}et her.

\vspace{4.0mm}
I 1910 publicerer kemikeren Alfred J.\ Lotka en r{\ae}kke
resultater \cite{Lotka1}, hvor han studerer f{\o}lgende
reaktions\-mekanisme

\begin{subequations}
 \begin{eqalignno}
  {\rm a} & \rightarrow {\rm A}\\
  {\rm A} & \stackrel{k_1}{\rightarrow} {\rm B}\\
  {\rm B} & \stackrel{k_2}{\rightarrow} {\rm C}
 \end{eqalignno}
 \label{eq:LotkaSimple}
\end{subequations}


hvor $k_1$ og $k_2$ er hastigheds\-konstanter.
Koncentrationen $H$ af stoffet a antages at v{\ae}re s{\aa}
stor, at denne kan betragtes som v{\ae}rende konstant under
reaktions\-forl{\o}bet. Endvidere antages, at stoffet B har
en autokatalytisk effekt p{\aa} sin egen dannelse, hvorfor
produktionen af stoffet B er proportional med
koncentrationen af B. For ovenst{\aa}ende
reaktions\-mekanisme f{\aa}s f{\o}lgende s{\ae}t af
hastigheds\-udtryk

\begin{subequations}
 \begin{eqalignno}
  \frac{dA}{dt} & = H - k_1 A B \\
  \frac{dB}{dt} & = k_1 A B - k_2 B
 \end{eqalignno}
 \label{eq:LotkaSimpleRate}
\end{subequations}

hvor $A$ og $B$ er koncentrationerne af stofferne A og B,
og $H$ er en konstant. Lotka v{\ae}lger nu at transformere
differentiallig\-ningssystemet~\ref{eq:LotkaSimpleRate} via
f{\o}lgende variabelskift

\begin{subequations}
 \begin{eqalignno}
  T & = k_1 t           \\
  h & = \frac{H}{k_1}   \\
  K & = \frac{k_2}{k_1}
 \end{eqalignno}
\label{eq:LotkaTransform}
\end{subequations}

der f{\o}rer til systemet 

\begin{subequations}
 \begin{eqalignno}
  \frac{dA}{dt} & = h - A B   \label{eq:Trans-a} \\
  \frac{dB}{dt} & = A B - K B \label{eq:Trans-b}
 \end{eqalignno}
\end{subequations}

S{\ae}ttes ligningerne~\ref{eq:Trans-a}~og~\ref{eq:Trans-b}
lig nul, ses dette system for $t \rightarrow \infty$ at
konvergere mod ligev{\ae}gts\-koncentrationerne

\begin{equation}
  A_{\rm eq} = K \ms{og}  B_{\rm eq} = \frac{h}{K}
  \label{eq:LotkaSimpleEqui}
\end{equation}

Lotka unders{\o}ger nu systemets opf{\o}rsel infinitesimalt
t{\ae}t p{\aa} ligev{\ae}gts\-\-punktet $(K,\frac{h}{K})$
ved hj{\ae}lp af variabelskiftet

\begin{subequations}
  \begin{eqalignno}
   x = A - A_{\rm eq} \\
   y = B - B_{\rm eq}
  \end{eqalignno}
\end{subequations}

Indf{\o}res denne transformation f{\aa}s

\begin{subequations}
  \begin{eqalignno}
    \frac{dx}{dt} & = -xy - Ky - \frac{h}{K}x \\
    \frac{dy}{dt} & =  xy +      \frac{h}{K}x 
  \end{eqalignno}
\end{subequations}

Da $x$ og $y$ betragtes t{\ae}t p{\aa}
ligev{\ae}gts\-punktet~\ref{eq:LotkaSimpleEqui}, ses der
bort fra h{\o}jere\-ordens\-led af typen $xy$, hvorfor
systemets opf{\o}rsel omkring ligev{\ae}gts\-punktet kan
udtrykkes ved f{\o}lgende line{\ae}re differentiallig\-ning

\begin{equation}
  \left[ \begin{array}{c} \dot{x} \\ \dot{y} \end{array} \right] =
  \left[ \begin{array}{rc}
          -\frac{h}{K} & -K \\
           \frac{h}{K} &  0
  \end{array} \right]
  \left[ \begin{array}{c} x \\ y \end{array} \right]
  \label{eq:LinearLotka}
\end{equation}

Lotka v{\ae}lger at omskrive dette til to uafh{\ae}ngige
2.\ ordens differentiallig\-ninger, som derefter l{\o}ses
analytisk. Denne fremgangsm{\aa}de viser sig, at v{\ae}re
en smule uhensigts\-m{\ae}ssig, da vores m{\aa}l med at
gennemg{\aa} Lotkas model udelukkende er at drage nogle
kvalitative konklusioner vedr{\o}rende systemets
opf{\o}rsel t{\ae}t p{\aa} ligev{\ae}gt. I stedet kan vi
n{\o}jes med at betragte $2 \times 2$ matricen i
lig\-ning~\ref{eq:LinearLotka}. Denne har egenv{\ae}rdierne

\begin{equation}
  \lambda_{1,2} = 
  \frac{ -\frac{h}{K} \pm \sqrt{\left(\frac{h}{K}\right)^2-4h}}{2}
\end{equation}

%%%%%%%%%%%%%%%%%%%%%%%%%%%%%%%%%%%%%%%%%%%%%%%%%%%%%%%%%%%%%%%%%%%%%%%%
%% figur
%%
%% beskrivelse : Lotka f{\o}rste model, faseprotr{\ae}t
%% plt         : fig41.plt
%% dat         : fig41.dat
%% tex         : fig41.tex
%% type        : TeXDraw
%%%%%%%%%%%%%%%%%%%%%%%%%%%%%%%%%%%%%%%%%%%%%%%%%%%%%%%%%%%%%%%%%%%%%%%%
\boxfigure{t}{\textwidth}
{
\begin{center}
 \vspace{1cm}
 % GNUPLOT: LaTeX using TEXDRAW macros
\begin{texdraw}
\normalsize
\ifx\pathDEFINED\relax\else\let\pathDEFINED\relax
 \def\QtGfr{\ifx (\TGre \let\YhetT\cpath\else\let\YhetT\relax\fi\YhetT}
 \def\path (#1 #2){\move (#1 #2)\futurelet\TGre\QtGfr}
 \def\cpath (#1 #2){\lvec (#1 #2)\futurelet\TGre\QtGfr}
\fi
\drawdim pt
\setunitscale 0.24
\linewd 3
\textref h:L v:C
\linewd 4
\path (176 169)(196 169)
\path (1436 169)(1416 169)
\move (154 169)\textref h:R v:C \htext{{\footnotesize$-1.5$}}
\path (176 270)(196 270)
\path (1436 270)(1416 270)
\move (154 270)\htext{{\footnotesize$-1.0$}}
\path (176 371)(196 371)
\path (1436 371)(1416 371)
\move (154 371)\htext{{\footnotesize$-0.5$}}
\path (176 473)(196 473)
\path (1436 473)(1416 473)
\move (154 473)\htext{{\footnotesize$0.0$}}
\path (176 574)(196 574)
\path (1436 574)(1416 574)
\move (154 574)\htext{{\footnotesize$0.5$}}
\path (176 675)(196 675)
\path (1436 675)(1416 675)
\move (154 675)\htext{{\footnotesize$1.0$}}
\path (176 776)(196 776)
\path (1436 776)(1416 776)
\move (154 776)\htext{{\footnotesize$1.5$}}
\path (334 68)(334 88)
\path (334 877)(334 857)
\move (334 23)\textref h:C v:C \htext{{\footnotesize$-15$}}
\path (491 68)(491 88)
\path (491 877)(491 857)
\move (491 23)\htext{{\footnotesize$-10$}}
\path (649 68)(649 88)
\path (649 877)(649 857)
\move (649 23)\htext{{\footnotesize$-5$}}
\path (806 68)(806 88)
\path (806 877)(806 857)
\move (806 23)\htext{{\footnotesize$0$}}
\path (964 68)(964 88)
\path (964 877)(964 857)
\move (964 23)\htext{{\footnotesize$5$}}
\path (1121 68)(1121 88)
\path (1121 877)(1121 857)
\move (1121 23)\htext{{\footnotesize$10$}}
\path (1279 68)(1279 88)
\path (1279 877)(1279 857)
\move (1279 23)\htext{{\footnotesize$15$}}
\path (176 68)(1436 68)(1436 877)(176 877)(176 68)
\linewd 3
\path (1178 211)(1178 211)(1179 200)(1178 211)(1165 206)(1178 211)
\cpath (1178 212)(1180 214)(1185 222)(1196 241)(1207 261)
\cpath (1216 281)(1223 301)(1223 301)(1228 291)(1223 301)
\cpath (1213 293)(1223 301)(1230 321)(1235 341)(1238 360)
\cpath (1241 380)(1241 380)(1248 370)(1241 380)(1232 371)
\cpath (1241 380)(1242 399)(1243 417)(1242 435)(1241 452)
\cpath (1239 467)(1239 467)(1248 459)(1239 467)(1233 458)
\cpath (1239 467)(1237 481)(1234 494)(1231 506)(1227 521)
\cpath (1223 536)(1217 552)(1217 552)(1228 545)(1217 552)
\cpath (1213 541)(1217 552)(1210 569)(1202 586)(1193 603)
\cpath (1182 620)(1170 638)(1170 638)(1183 633)(1170 638)
\cpath (1169 627)(1170 638)(1157 655)(1143 673)(1127 690)
\cpath (1110 707)(1110 707)(1124 704)(1110 707)(1112 696)
\cpath (1110 707)(1092 722)(1073 737)(1054 750)(1034 762)
\cpath (1034 762)(1049 762)(1034 762)(1039 752)(1034 762)
\cpath (1015 773)(995 783)(975 792)(955 799)(936 806)
\cpath (936 806)(950 808)(936 806)(944 797)(936 806)
\cpath (916 811)(897 816)(878 819)(860 822)(843 824)
\cpath (826 825)(826 825)(838 831)(826 825)(838 819)
\cpath (826 825)(810 825)(797 825)(785 825)(772 824)
\cpath (758 823)(742 821)(727 818)(727 818)(736 826)
\cpath (727 818)(740 815)(727 818)(710 814)(693 810)
\cpath (676 804)(659 798)(642 791)(624 782)(624 782)
\cpath (629 793)(624 782)(639 783)(624 782)(607 773)
\cpath (590 762)(574 750)(558 737)(543 722)(543 722)
\cpath (543 734)(543 722)(556 727)(543 722)(529 708)
\cpath (517 693)(506 678)(496 662)(487 646)(487 646)
\cpath (484 657)(487 646)(499 653)(487 646)(480 630)
\cpath (473 614)(468 598)(464 582)(461 566)(461 566)
\cpath (454 576)(461 566)(470 575)(461 566)(458 551)
\cpath (457 536)(457 521)(457 506)(458 493)(459 480)
\cpath (459 480)(450 488)(459 480)(466 489)(459 480)
\cpath (461 468)(463 457)(465 447)(468 435)(472 423)
\cpath (477 410)(482 397)(482 397)(471 404)(482 397)
\cpath (486 408)(482 397)(488 384)(496 370)(504 356)
\cpath (513 342)(523 328)(535 314)(535 314)(521 318)
\cpath (535 314)(535 325)(535 314)(547 300)(561 287)
\cpath (576 274)(590 262)(606 251)(606 251)(591 253)
\cpath (606 251)(602 262)(606 251)(621 242)(637 233)
\cpath (653 225)(669 218)(685 212)(700 206)(700 206)
\cpath (686 204)(700 206)(692 215)(700 206)(716 202)
\cpath (731 198)(746 195)(761 193)(775 191)(788 191)
\cpath (801 190)(801 190)(789 184)(801 190)(789 196)
\cpath (801 190)(812 190)(822 190)(832 191)(844 192)
\cpath (856 194)(868 196)(881 199)(895 202)(908 206)
\cpath (908 206)(901 197)(908 206)(894 208)(908 206)
\cpath (922 211)(936 217)(950 224)(964 231)(977 240)
\cpath (991 249)(1004 260)(1004 260)(1002 249)(1004 260)
\cpath (990 257)(1004 260)(1016 271)(1027 283)(1036 295)
\cpath (1045 307)(1053 319)(1060 332)(1060 332)(1064 321)
\cpath (1060 332)(1049 326)(1060 332)(1067 345)(1072 358)
\cpath (1076 371)(1079 384)(1082 396)(1084 409)(1084 409)
\cpath (1091 399)(1084 409)(1075 400)(1084 409)(1085 421)
\cpath (1086 433)(1086 444)(1085 455)(1084 466)(1082 475)
\cpath (1081 484)(1079 492)(1079 492)(1089 484)(1079 492)
\cpath (1073 482)(1079 492)(1077 501)(1074 511)(1070 521)
\cpath (1066 532)(1061 542)(1055 553)(1049 565)(1041 576)
\cpath (1041 576)(1054 571)(1041 576)(1040 565)(1041 576)
\cpath (1033 587)(1024 598)(1014 609)(1003 620)(992 630)
\cpath (980 640)(980 640)(994 638)(980 640)(983 629)
\cpath (980 640)(968 648)(955 656)(943 664)(930 670)
\cpath (917 676)(905 681)(892 685)(892 685)(906 687)
\cpath (892 685)(900 676)(892 685)(879 689)(867 692)
\cpath (855 694)(843 696)(832 697)(821 698)(811 699)
\cpath (802 699)(794 698)(794 698)(805 705)(794 698)
\cpath (807 693)(794 698)(786 698)(777 697)(767 696)
\cpath (757 694)(747 692)(736 689)(725 686)(714 682)
\cpath (703 677)(692 672)(692 672)(697 682)(692 672)
\cpath (706 673)(692 672)(681 666)(670 659)(659 652)
\cpath (649 643)(639 635)(630 625)(622 616)(615 606)
\cpath (615 606)(613 617)(615 606)(628 611)(615 606)
\cpath (609 596)(603 586)(598 576)(594 565)(590 555)
\cpath (587 545)(585 535)(584 525)(584 525)(577 534)
\cpath (584 525)(593 533)(584 525)(583 515)(582 505)
\cpath (582 496)(583 487)(584 479)(585 471)(586 464)
\cpath (587 458)(589 450)(592 443)(592 443)(581 450)
\cpath (592 443)(596 453)(592 443)(594 435)(598 426)
\cpath (602 417)(606 409)(611 400)(617 391)(623 382)
\cpath (631 373)(638 364)(638 364)(625 368)(638 364)
\cpath (638 375)(638 364)(647 355)(656 347)(666 339)
\cpath (675 332)(685 326)(695 320)(706 315)(716 310)
\cpath (716 310)(701 309)(716 310)(709 320)(716 310)
\cpath (726 306)(736 303)(746 300)(756 297)(766 295)
\cpath (775 294)(784 293)(793 292)(801 292)(809 291)
\cpath (816 292)(816 292)(804 285)(816 292)(803 297)
\cpath (816 292)(822 292)(829 293)(837 294)(845 295)
\cpath (853 297)(862 299)(871 302)(880 305)(888 308)
\cpath (897 313)(906 318)(915 323)(923 329)(923 329)
\cpath (921 318)(923 329)(909 326)(923 329)(932 336)
\cpath (940 343)(947 350)(953 358)(959 366)(964 374)
\cpath (969 382)(973 390)(976 398)(976 398)(981 388)
\cpath (976 398)(965 391)(976 398)(979 407)(981 415)
\cpath (983 423)(984 431)(985 439)(985 446)(985 454)
\cpath (985 461)(984 467)(983 474)(982 479)(982 479)
\cpath (992 471)(982 479)(976 469)(982 479)(981 485)
\cpath (980 490)(978 496)(975 503)(973 510)(970 517)
\cpath (966 524)(962 531)(957 538)(952 545)(946 552)
\cpath (946 552)(960 548)(946 552)(946 541)(946 552)
\cpath (940 559)(933 566)(926 573)(918 579)(910 585)
\cpath (902 590)(894 594)(886 599)(878 602)(870 606)
\cpath (870 606)(884 607)(870 606)(878 596)(870 606)
\cpath (862 608)(854 611)(846 613)(838 614)(831 615)
\cpath (823 617)(816 617)(810 617)(804 617)(798 617)
\cpath (793 617)(787 616)(781 615)(775 614)(768 613)
\cpath (768 613)(777 622)(768 613)(782 610)(768 613)
\cpath (761 611)(754 609)(747 607)(740 604)(733 600)
\cpath (726 596)(719 592)(712 587)(705 582)(699 576)
\cpath (693 570)(688 564)(684 558)(679 552)(676 545)
\cpath (673 538)(673 538)(669 549)(673 538)(684 545)
\cpath (673 538)(670 532)(668 525)(666 519)(665 512)
\cpath (664 506)(663 500)(663 493)(663 487)(663 482)
\cpath (663 477)(664 471)(665 467)(666 463)(666 463)
\cpath (656 470)(666 463)(671 473)(666 463)(667 458)
\cpath (669 453)(670 448)(673 443)(675 437)(678 431)
\cpath (681 426)(685 420)(689 414)(694 409)(699 403)
\cpath (704 397)(710 392)(710 392)(696 395)(710 392)
\cpath (708 403)(710 392)(716 387)(723 383)(729 378)
\cpath (735 375)(742 371)(748 369)(755 366)(761 364)
\cpath (768 362)(774 360)(780 359)(786 358)(792 357)
\cpath (798 357)(803 357)(808 357)(808 357)(796 350)
\cpath (808 357)(796 362)(808 357)(812 357)(816 357)
\cpath (821 357)(826 358)(831 359)(837 360)(842 361)
\cpath (848 363)(853 365)(859 368)(865 370)(870 373)
\cpath (876 377)(881 381)(887 385)(892 390)(896 394)
\cpath (896 394)(896 383)(896 394)(883 390)(896 394)
\cpath (900 399)(904 404)(907 409)(910 415)(913 420)
\cpath (915 425)(917 430)(918 436)(919 441)(920 446)
\cpath (921 451)(921 456)(921 461)(920 465)(920 469)
\cpath (919 474)(919 477)(918 480)(917 484)(916 488)
\cpath (916 488)(927 481)(916 488)(911 478)(916 488)
\cpath (914 492)(913 496)(911 501)(908 505)(906 510)
\cpath (903 515)(899 519)(896 524)(892 528)(887 533)
\cpath (883 537)(878 541)(873 545)(868 548)(862 551)
\cpath (862 551)(877 551)(862 551)(868 540)(862 551)
\cpath (857 553)(852 556)(847 558)(842 560)(836 561)
\cpath (831 563)(826 564)(822 564)(817 565)(813 565)
\cpath (808 565)(804 565)(801 565)(798 565)(794 565)
\cpath (790 564)(786 563)(781 563)(777 561)(773 560)
\cpath (768 558)(764 556)(764 556)(769 567)(764 556)
\cpath (778 556)(764 556)(759 554)(754 552)(750 549)
\cpath (746 546)(741 542)(737 539)(734 535)(730 531)
\cpath (728 527)(725 523)(723 519)(721 515)(719 510)
\cpath (717 506)(716 502)(715 498)(715 494)(714 490)
\cpath (714 490)(707 499)(714 490)(723 499)(714 490)
\cpath (714 486)(714 482)(714 478)(715 475)(715 472)
\cpath (716 469)(716 466)(717 463)(718 460)(719 457)
\cpath (721 453)(722 450)(724 446)(726 443)(729 439)
\cpath (731 435)(734 431)(737 428)(741 424)(745 421)
\cpath (749 418)(749 418)(734 420)(749 418)(745 428)
\cpath (749 418)(753 415)(757 412)(761 410)(765 408)
\cpath (769 406)(773 404)(778 403)(782 402)(786 400)
\cpath (790 400)(794 399)(797 399)(801 398)(804 398)
\cpath (807 398)(810 398)(812 398)(815 399)(819 399)
\cpath (822 400)(825 400)(829 401)(832 402)(836 404)
\cpath (840 405)(843 407)(847 409)(850 411)(854 413)
\cpath (857 416)(861 419)(864 422)(866 425)(869 428)
\cpath (871 432)(873 435)(874 439)(876 442)(877 445)
\cpath (878 449)(878 452)(879 455)(879 458)(879 462)
\cpath (879 465)(879 467)(879 470)(879 473)(878 475)
\cpath (878 477)(877 479)(877 482)(876 484)(875 487)
\cpath (873 490)(872 493)(870 496)(868 499)(866 502)
\cpath (864 505)(862 507)(859 511)(856 513)(853 516)
\cpath (849 518)(846 520)(843 522)(840 524)(836 525)
\cpath (833 527)(830 528)(826 529)(823 530)(820 531)
\cpath (817 531)(814 532)(811 532)(808 532)(805 532)
\cpath (803 532)(801 532)(799 532)(796 531)(794 531)
\cpath (791 530)(788 530)(785 529)(782 528)(779 527)
\cpath (777 525)(774 524)(771 522)(768 520)(765 518)
\cpath (763 516)(760 513)(758 511)(756 508)(754 505)
\cpath (753 503)(752 500)(750 497)(749 495)(749 492)
\cpath (748 489)(748 487)(747 484)(747 482)(747 479)
\cpath (747 477)(747 475)(748 473)(748 471)(748 469)
\cpath (749 467)(749 465)(750 463)(751 461)(752 459)
\cpath (753 457)(754 454)(756 452)(757 449)(759 447)
\cpath (761 445)(763 443)(766 440)(768 438)(771 436)
\cpath (773 434)(776 433)(779 432)(781 430)(784 429)
\cpath (787 428)(789 427)(792 427)(795 426)(797 426)
\cpath (799 425)(802 425)(804 425)(806 425)(808 425)
\cpath (809 425)(811 425)(813 425)(815 426)(817 426)
\cpath (819 427)(822 427)(824 428)(826 429)(829 430)
\cpath (831 431)(833 432)(836 434)(838 435)(840 437)
\cpath (842 439)(844 441)(845 443)(847 445)(848 447)
\cpath (849 450)(850 452)(851 454)(852 456)(852 458)
\cpath (853 460)(853 462)(853 464)(853 466)(853 468)
\cpath (853 470)(853 472)(852 473)(852 475)(852 476)
\cpath (851 478)(851 479)(850 481)(850 483)(849 485)
\cpath (848 486)(847 488)(845 490)(844 492)(842 494)
\cpath (841 496)(839 498)(837 499)(835 501)(833 502)
\cpath (831 504)(829 505)(826 506)(824 507)(822 508)
\cpath (820 509)(818 509)(816 510)(814 510)(812 510)
\cpath (810 510)(808 511)(806 511)(805 511)(804 511)
\cpath (802 510)(801 510)(799 510)(797 510)(796 509)
\cpath (794 509)(792 508)(790 508)(788 507)(786 506)
\cpath (785 505)(783 504)(781 502)(779 501)(778 499)
\cpath (776 498)(775 496)(774 495)(773 493)(772 491)
\cpath (771 489)(770 488)(770 486)(769 484)(769 483)
\cpath (769 481)(768 479)(768 478)(768 476)(768 475)
\cpath (769 473)(769 472)(769 471)(769 470)(770 468)
\cpath (770 467)(770 466)(771 464)(772 463)(773 461)
\cpath (773 460)(774 458)(776 457)(777 456)(778 454)
\cpath (780 452)(781 451)(783 450)(784 449)(786 448)
\cpath (788 447)(789 446)(791 445)(793 444)(795 444)
\cpath (796 443)(798 443)(800 443)(801 442)(803 442)
\cpath (804 442)(805 442)(807 442)(808 442)(809 442)
\cpath (810 442)(811 442)(813 443)(814 443)(816 443)
\cpath (817 444)(819 444)(820 445)(821 446)(823 446)
\cpath (824 447)(826 448)(827 450)(829 451)(830 452)
\cpath (831 453)(832 455)(833 456)(833 457)(834 459)
\cpath (835 460)(835 462)(835 463)(836 464)(836 466)
\cpath (836 467)(836 468)(836 469)(836 471)(836 472)
\cpath (836 473)(836 474)(835 475)(835 476)(835 477)
\cpath (834 478)(834 479)(833 480)(833 481)(832 482)
\cpath (831 484)(831 485)(830 486)(828 487)(827 488)
\cpath (826 489)(825 491)(823 492)(822 492)(821 493)
\cpath (819 494)(818 494)(817 495)(815 495)(814 496)
\cpath (813 496)(811 496)(810 497)(809 497)(808 497)
\cpath (807 497)(806 497)(805 497)(804 497)(803 497)
\cpath (802 497)(801 496)(800 496)(798 496)(797 496)
\cpath (796 495)(795 495)(794 494)(793 493)(791 493)
\cpath (790 492)(789 491)(788 490)(787 489)(786 488)
\cpath (785 487)(785 486)(784 485)(784 484)(783 483)
\cpath (783 481)(782 480)(782 479)(782 478)(782 477)
\cpath (782 476)(782 475)(782 474)(782 473)(782 472)
\cpath (782 471)(782 471)(783 470)(783 469)(783 468)
\cpath (784 467)(784 467)(784 466)(785 465)(786 464)
\cpath (786 463)(787 462)(788 461)(789 460)(790 459)
\cpath (791 458)(792 457)(793 457)(794 456)(795 456)
\cpath (796 455)(797 454)(798 454)(800 454)(801 454)
\cpath (802 453)(803 453)(804 453)(805 453)(805 453)
\cpath (806 453)(807 453)(808 453)(808 453)(809 453)
\cpath (810 453)(811 453)(812 454)(813 454)(814 454)
\cpath (815 455)(816 455)(817 456)(818 456)(819 457)
\cpath (820 458)(820 459)(821 459)(822 460)(822 461)
\cpath (823 462)(824 463)(824 464)(824 464)(825 465)
\cpath (825 466)(825 467)(825 468)(825 469)(825 470)
\cpath (825 470)(825 471)(825 472)(825 473)(825 473)
\cpath (825 474)(825 474)(825 475)(824 476)(824 477)
\cpath (824 477)(823 478)(823 479)(822 480)(822 480)
\cpath (821 481)(820 482)(820 483)(819 483)(818 484)
\cpath (817 485)(816 485)(816 486)(815 486)(814 486)
\cpath (813 487)(812 487)(811 487)(810 488)(809 488)
\cpath (809 488)(808 488)(807 488)(806 488)(806 488)
\cpath (805 488)(805 488)(804 488)(803 488)(803 488)
\cpath (802 488)(801 487)(800 487)(800 487)(799 487)
\cpath (798 486)(797 486)(797 485)(796 485)(795 484)
\cpath (794 484)(794 483)(793 482)(793 482)(792 481)
\cpath (792 480)(792 480)(791 479)(791 478)(791 478)
\cpath (791 477)(791 476)(791 475)(791 475)(791 474)
\cpath (791 474)(791 473)(791 472)(791 472)(791 471)
\cpath (791 471)(791 470)(791 470)(792 469)(792 469)
\cpath (792 468)(793 467)(793 467)(793 466)(794 466)
\cpath (794 465)(795 464)(796 464)(796 463)(797 463)
\cpath (798 462)(798 462)(799 462)(800 461)(800 461)
\cpath (801 461)(802 461)(803 460)(803 460)(804 460)
\cpath (804 460)(805 460)(806 460)(806 460)(807 460)
\cpath (807 460)(808 460)(808 460)(809 460)(809 460)
\cpath (810 461)(810 461)(811 461)(812 461)(812 461)
\cpath (813 462)(814 462)(814 463)(815 463)(815 464)
\cpath (816 464)(816 465)(817 465)(817 466)(817 466)
\cpath (817 467)(818 467)(818 468)(818 468)(818 469)
\cpath (818 470)(818 470)(818 471)(818 471)(818 472)
\cpath (818 472)(818 473)(818 473)(818 473)(818 474)
\cpath (818 474)(818 475)(818 475)(817 476)(817 476)
\cpath (817 477)(816 477)(816 478)(816 478)(815 479)
\cpath (815 479)(814 479)(814 480)(813 480)(813 481)
\cpath (812 481)(812 481)(811 481)(810 482)(810 482)
\cpath (809 482)(809 482)(808 482)(808 482)(807 482)
\cpath (807 482)(806 482)(806 482)(805 482)(805 482)
\cpath (805 482)(804 482)(804 482)(803 482)(803 482)
\cpath (802 482)(802 482)(801 482)(801 481)(800 481)
\cpath (800 481)(800 480)(799 480)(799 480)(798 479)
\cpath (798 479)(798 478)(797 478)(797 478)(797 477)
\cpath (797 477)(796 476)(796 476)(796 475)(796 475)
\cpath (796 474)(796 474)(796 474)(796 473)(796 473)
\cpath (796 473)(796 472)(796 472)(796 471)(797 471)
\cpath (797 471)(797 470)(797 470)(797 470)(797 469)
\cpath (798 469)(798 468)(798 468)(799 468)(799 467)
\cpath (799 467)(800 467)(800 466)(801 466)(801 466)
\cpath (802 465)(802 465)(802 465)(803 465)(803 465)
\cpath (804 465)(804 465)(805 465)(805 465)(805 464)
\cpath (806 464)(806 464)(806 464)(807 465)(807 465)
\cpath (807 465)(808 465)(808 465)(808 465)(809 465)
\cpath (809 465)(810 465)(810 465)(810 466)(811 466)
\cpath (811 466)(812 466)(812 467)
\path (1178 211)(1178 211)(1176 209)(1171 201)(1157 184)(1141 166)
\cpath (1124 149)(1105 133)(1105 133)(1102 122)(1105 133)
\cpath (1091 131)(1105 133)(1085 117)(1063 103)(1040 89)
\cpath (1016 77)
\path (1332 96)(1332 96)(1333 85)(1332 96)(1319 90)(1332 96)
\cpath (1332 95)(1330 92)(1322 81)
\path (1332 96)(1332 96)(1334 99)(1342 110)(1359 137)(1374 165)
\cpath (1374 165)(1378 155)(1374 165)(1363 159)(1374 165)
\cpath (1387 194)(1398 223)(1408 251)(1408 251)(1413 241)
\cpath (1408 251)(1398 243)(1408 251)(1415 279)(1421 307)
\cpath (1425 335)(1425 335)(1432 325)(1425 335)(1416 326)
\cpath (1425 335)(1427 362)(1428 388)(1427 414)(1427 414)
\cpath (1436 405)(1427 414)(1420 404)(1427 414)(1426 438)
\cpath (1423 461)(1420 480)(1417 499)(1417 499)(1427 491)
\cpath (1417 499)(1411 489)(1417 499)(1413 517)(1407 538)
\cpath (1401 559)(1392 582)(1392 582)(1404 575)(1392 582)
\cpath (1388 571)(1392 582)(1383 605)(1372 629)(1359 654)
\cpath (1359 654)(1371 648)(1359 654)(1357 643)(1359 654)
\cpath (1344 678)(1328 703)(1310 728)(1310 728)(1323 724)
\cpath (1310 728)(1309 717)(1310 728)(1290 753)(1268 777)
\cpath (1243 802)(1243 802)(1257 799)(1243 802)(1245 791)
\cpath (1243 802)(1218 824)(1191 845)(1164 865)
\path (340 798)(340 798)(339 809)(340 798)(353 803)(340 798)
\cpath (340 797)(338 795)(332 785)(317 761)(304 736)
\cpath (293 711)(293 711)(289 721)(293 711)(304 718)
\cpath (293 711)(284 686)(276 661)(270 636)(270 636)
\cpath (264 646)(270 636)(279 644)(270 636)(265 611)
\cpath (262 587)(261 564)(260 541)(260 541)(252 550)
\cpath (260 541)(268 550)(260 541)(261 519)(262 498)
\cpath (265 478)(267 462)(267 462)(258 470)(267 462)
\cpath (273 472)(267 462)(271 446)(274 429)(279 411)
\cpath (285 392)(293 372)(293 372)(281 379)(293 372)
\cpath (297 383)(293 372)(301 351)(312 330)(323 309)
\cpath (336 287)(336 287)(324 292)(336 287)(338 298)
\cpath (336 287)(351 265)(367 243)(385 221)(405 200)
\cpath (405 200)(392 203)(405 200)(404 211)(405 200)
\cpath (427 179)(450 159)(473 141)(473 141)(459 143)
\cpath (473 141)(470 152)(473 141)(497 125)(521 109)
\cpath (546 96)(571 84)(571 84)(556 83)(571 84)
\cpath (564 93)(571 84)(596 73)
\path (340 798)(340 798)(342 801)(349 810)(367 833)(386 855)
\cpath (408 876)
\end{texdraw}

 \vspace{1cm}
\end{center}
}
{
\caption{\protect\capsize Faseprotr{\ae}t for Lotkas
f{\o}rste model for f{\o}lgende parameterv{\ae}rdier: $a$ =
10 og $b$ = 10. L{\o}sningskurverne foretager en
spiralerende bev{\ae}gelse ind mod ligev{\ae}gtspunktet.
}
\label{fig:PlotSimpleLotka}
}

Kr{\ae}ver vi, at betingelsen $2K > \sqrt{h} > 0$ er
opfyldt vil $\lambda_{1,2}$ v{\ae}re et komplekst tal
med formen $-\alpha \pm i \omega$, hvor $\alpha$ og
$\omega$ begge er st{\o}rre end nul. For disse
egenv{\ae}rdier sikrer teorien om line{\ae}re
differentiallig\-ninger os nu, at systemets opf{\o}rsel vil
v{\ae}re en spiralerende bev{\ae}gelse ind mod
ligev{\ae}gts\-punktet sammensat af en oscillation med
vinkelfrekvens $\omega$ og et eksponentielt henfald
$e^{-\alpha t}$. Disse egen\-skaber ved det
p{\aa}g{\ae}ldende system er illustreret i
figur~\ref{fig:PlotSimpleLotka} i form af et
faseportr{\ae}t i $(x,y)$-planet.

\vspace{4.0mm}
Lotkas model beskriver en kemisk reaktion, hvis forl{\o}b
imod kemisk ligev{\ae}gt er karakteriseret ved
oscillationer af koncentrationerne af de involverede
kemi\-ske stoffer. Som tidligere n{\ae}vnt er dette
f{\ae}nomen i 1910 helt ukendt og stik imod enhver
videnskabsmands sunde fornuft. Lotka udviser da ogs{\aa}
megen skepsis overfor dette mystiske f{\ae}nomen, idet han
afslutter artiklen med f{\o}lgende kommentar

\begin{quote}
  ``\ldots\ No reaction is known which follows the above
  law, and as a matter of fact the case here considered was
  suggested by the consideration of matters lying outside
  the field of physical chemistry''. \cite{Lotka1}
\end{quote}

Man kunne umiddelbart fristes til at tro, at en s{\aa}dan
opfattelse ville virke afskr{\ae}kkende med hensyn til at
unders{\o}ge s{\aa}danne kemiske modeller n{\ae}rmere.
Ikke desto mindre publicerer Lotka ti {\aa}r senere endnu
en artikel om emnet \cite{Lotka2}, hvis resultater m{\aa}
anses for at v{\ae}re endnu mere opsigts\-v{\ae}kkende,
end dem vi hidtil har diskuteret.

\vspace{4.0mm}
Lotka diskuterer f{\o}lgende situation: Betragt et uendelig
stort reservoir af et stof C og antag yderligere, at
stoffet A dannes autokatalytisk udfra C. Lad stoffet B
blive dannet autokatalytisk udfra A og lad endvidere B
blive fjernet via en unimolekyl{\ae}r reaktion. Hvilken
afh{\ae}ngighed af tiden vil koncentrationerne af de
indg{\aa}ende kemiske stoffer udvise i dette system? For at
besvare dette sp{\o}rgsm{\aa}l opstiller Lotka analogt med
forrige eksempel systemets kinetiske lig\-ninger

\begin{subequations}
  \begin{eqalignno}
   \frac{dA}{dt} & = k_1 A - k_2 A B \\
   \frac{dB}{dt} & = k_2 A B - k_3 B
  \end{eqalignno}
  \label{eq:LotkaRate}
\end{subequations}

Differentiallig\-ningssystemet~\ref{eq:LotkaRate} har
ligev{\ae}gts\-punktet

\begin{equation}
  A_{\rm eq} = \frac{k_3}{k_2} \ms{og}
  B_{\rm eq} = \frac{k_1}{k_2}
  \label{eq:LotkaEqui}
\end{equation}

%%%%%%%%%%%%%%%%%%%%%%%%%%%%%%%%%%%%%%%%%%%%%%%%%%%%%%%%%%%%%%%%%%%%%%%%
%% figur
%%
%% beskrivelse : Volterra-Lotka model, faseprotr{\ae}t
%% plt         : fig42.plt
%% dat         : fig42.dat
%% tex         : fig42.tex
%% type        : TeXDraw
%%%%%%%%%%%%%%%%%%%%%%%%%%%%%%%%%%%%%%%%%%%%%%%%%%%%%%%%%%%%%%%%%%%%%%%%
\boxfigure{t}{\textwidth}
{
\begin{center}
 \vspace{1cm}
  % GNUPLOT: LaTeX using TEXDRAW macros
\begin{texdraw}
\normalsize
\ifx\pathDEFINED\relax\else\let\pathDEFINED\relax
 \def\QtGfr{\ifx (\TGre \let\YhetT\cpath\else\let\YhetT\relax\fi\YhetT}
 \def\path (#1 #2){\move (#1 #2)\futurelet\TGre\QtGfr}
 \def\cpath (#1 #2){\lvec (#1 #2)\futurelet\TGre\QtGfr}
\fi
\drawdim pt
\setunitscale 0.24
\linewd 3
\textref h:L v:C
\path (176 215)(1436 215)
\path (401 68)(401 877)
\linewd 4
\path (176 142)(196 142)
\path (1436 142)(1416 142)
\move (154 142)\textref h:R v:C \htext{{\footnotesize$-5.0$}}
\path (176 215)(196 215)
\path (1436 215)(1416 215)
\move (154 215)\htext{{\footnotesize$0.0$}}
\path (176 289)(196 289)
\path (1436 289)(1416 289)
\move (154 289)\htext{{\footnotesize$5.0$}}
\path (176 362)(196 362)
\path (1436 362)(1416 362)
\move (154 362)\htext{{\footnotesize$10.0$}}
\path (176 436)(196 436)
\path (1436 436)(1416 436)
\move (154 436)\htext{{\footnotesize$15.0$}}
\path (176 509)(196 509)
\path (1436 509)(1416 509)
\move (154 509)\htext{{\footnotesize$20.0$}}
\path (176 583)(196 583)
\path (1436 583)(1416 583)
\move (154 583)\htext{{\footnotesize$25.0$}}
\path (176 656)(196 656)
\path (1436 656)(1416 656)
\move (154 656)\htext{{\footnotesize$30.0$}}
\path (176 730)(196 730)
\path (1436 730)(1416 730)
\move (154 730)\htext{{\footnotesize$35.0$}}
\path (176 803)(196 803)
\path (1436 803)(1416 803)
\move (154 803)\htext{{\footnotesize$40.0$}}
\path (266 68)(266 88)
\path (266 877)(266 857)
\move (266 23)\textref h:C v:C \htext{{\footnotesize$-3$}}
\path (401 68)(401 88)
\path (401 877)(401 857)
\move (401 23)\htext{{\footnotesize$0$}}
\path (536 68)(536 88)
\path (536 877)(536 857)
\move (536 23)\htext{{\footnotesize$3$}}
\path (671 68)(671 88)
\path (671 877)(671 857)
\move (671 23)\htext{{\footnotesize$6$}}
\path (806 68)(806 88)
\path (806 877)(806 857)
\move (806 23)\htext{{\footnotesize$9$}}
\path (941 68)(941 88)
\path (941 877)(941 857)
\move (941 23)\htext{{\footnotesize$12$}}
\path (1076 68)(1076 88)
\path (1076 877)(1076 857)
\move (1076 23)\htext{{\footnotesize$15$}}
\path (1211 68)(1211 88)
\path (1211 877)(1211 857)
\move (1211 23)\htext{{\footnotesize$18$}}
\path (1346 68)(1346 88)
\path (1346 877)(1346 857)
\move (1346 23)\htext{{\footnotesize$21$}}
\path (176 68)(1436 68)(1436 877)(176 877)(176 68)
\linewd 3
\path (817 318)(817 318)(814 308)(817 318)(802 316)(817 318)
\cpath (818 319)(823 322)(827 326)(832 330)(835 334)
\cpath (838 339)(841 343)(843 347)(844 352)(845 357)
\cpath (846 362)(845 367)(845 372)(843 377)(841 382)
\cpath (841 382)(853 376)(841 382)(837 372)(841 382)
\cpath (839 387)(836 393)(832 399)(828 404)(823 410)
\cpath (818 416)(812 423)(805 429)(798 436)(789 443)
\cpath (789 443)(804 440)(789 443)(791 433)(789 443)
\cpath (780 450)(770 458)(760 465)(749 472)(738 479)
\cpath (727 486)(716 493)(705 500)(705 500)(720 498)
\cpath (705 500)(709 490)(705 500)(693 506)(681 513)
\cpath (668 520)(655 527)(641 533)(629 540)(618 545)
\cpath (618 545)(633 544)(618 545)(623 535)(618 545)
\cpath (607 550)(598 554)(589 558)(580 562)(573 565)
\cpath (565 568)(558 571)(552 573)(546 576)(540 578)
\cpath (534 580)(529 582)(524 584)(524 584)(539 585)
\cpath (524 584)(531 575)(524 584)(519 585)(514 587)
\cpath (510 588)(505 590)(501 591)(497 592)(493 593)
\cpath (490 594)(486 595)(483 596)(480 596)(477 597)
\cpath (474 598)(471 598)(468 599)(466 599)(463 600)
\cpath (461 600)(459 600)(456 600)(454 601)(452 601)
\cpath (450 601)(448 601)(447 601)(445 601)(443 601)
\cpath (442 601)(440 601)(439 601)(437 600)(436 600)
\cpath (435 600)(433 600)(432 600)(431 599)(430 599)
\cpath (429 599)(428 598)(427 598)(426 597)(425 597)
\cpath (424 597)(423 596)(423 596)(427 606)(423 596)
\cpath (438 597)(423 596)(422 596)(421 595)(421 595)
\cpath (420 594)(419 594)(419 593)(418 592)(417 592)
\cpath (417 591)(416 591)(415 590)(415 589)(414 589)
\cpath (414 588)(413 587)(413 587)(413 586)(412 585)
\cpath (412 585)(411 584)(411 583)(410 583)(410 582)
\cpath (410 581)(410 580)(409 579)(409 579)(409 578)
\cpath (408 577)(408 576)(408 576)(408 575)(407 574)
\cpath (407 573)(407 572)(407 571)(406 571)(406 570)
\cpath (406 569)(406 568)(406 567)(405 566)(405 566)
\cpath (405 565)(405 564)(405 563)(405 562)(405 561)
\cpath (404 560)(404 559)(404 558)(404 558)(404 557)
\cpath (404 556)(404 555)(404 554)(404 553)(403 552)
\cpath (403 551)(403 550)(403 549)(403 548)(403 547)
\cpath (403 546)(403 545)(403 544)(403 543)(403 542)
\cpath (403 541)(403 540)(402 539)(402 539)(402 538)
\cpath (402 537)(402 536)(402 535)(402 534)(402 533)
\cpath (402 532)(402 530)(402 529)(402 529)(394 538)
\cpath (402 529)(411 538)(402 529)(402 528)(402 527)
\cpath (402 526)(402 525)(402 524)(402 523)(402 522)
\cpath (402 521)(402 520)(402 519)(402 518)(402 517)
\cpath (402 516)(402 515)(402 514)(402 513)(402 512)
\cpath (401 510)(401 509)(401 508)(401 507)(401 506)
\cpath (401 505)(401 504)(401 503)(401 502)(401 500)
\cpath (401 499)(401 498)(401 497)(401 496)(401 495)
\cpath (401 494)(401 492)(401 491)(401 490)(401 489)
\cpath (401 488)(401 486)(401 485)(401 484)(401 483)
\cpath (401 481)(401 480)(401 479)(401 478)(401 476)
\cpath (401 475)(401 474)(401 473)(401 471)(401 470)
\cpath (401 469)(401 467)(401 466)(401 465)(401 463)
\cpath (401 462)(401 461)(401 459)(401 459)(393 468)
\cpath (401 459)(410 468)(401 459)(401 458)(401 456)
\cpath (401 455)(401 454)(401 452)(401 451)(401 449)
\cpath (401 448)(401 446)(401 444)(401 443)(401 441)
\cpath (401 440)(401 438)(401 436)(401 435)(401 433)
\cpath (401 431)(401 429)(401 428)(401 426)(401 424)
\cpath (401 422)(401 419)(401 417)(401 415)(401 412)
\cpath (401 409)(401 407)(401 404)(401 402)(401 399)
\cpath (401 397)(401 395)(401 393)(401 391)(401 389)
\cpath (401 389)(392 397)(401 389)(410 397)(401 389)
\cpath (401 386)(401 384)(401 382)(401 380)(401 379)
\cpath (401 377)(401 375)(401 373)(401 371)(401 369)
\cpath (401 367)(401 365)(401 363)(401 361)(401 360)
\cpath (401 358)(401 356)(401 354)(401 352)(401 351)
\cpath (401 349)(401 347)(401 346)(401 344)(401 342)
\cpath (401 340)(401 339)(401 337)(401 336)(401 334)
\cpath (401 332)(401 331)(401 329)(401 328)(401 326)
\cpath (401 325)(401 324)(401 322)(401 321)(401 320)
\cpath (401 318)(401 318)(392 327)(401 318)(410 327)
\cpath (401 318)(401 317)(401 316)(401 315)(401 314)
\cpath (401 313)(401 312)(401 311)(401 310)(401 309)
\cpath (401 308)(401 307)(401 306)(401 305)(401 304)
\cpath (401 303)(401 302)(401 302)(401 301)(401 300)
\cpath (401 299)(401 299)(401 298)(401 297)(401 296)
\cpath (401 296)(401 295)(401 294)(401 294)(401 293)
\cpath (401 293)(401 292)(401 291)(401 291)(401 290)
\cpath (401 290)(401 289)(401 288)(401 288)
\path (815 317)(815 317)(811 314)(805 311)(799 308)(792 305)
\cpath (786 302)(779 299)(772 296)(764 294)(757 291)
\cpath (749 289)(741 287)(733 285)(725 283)(725 283)
\cpath (716 275)(725 283)(710 285)(725 283)(716 281)
\cpath (708 279)(699 277)(691 275)(682 274)(673 272)
\cpath (665 271)(657 269)(649 268)(640 267)(632 266)
\cpath (625 265)(617 264)(617 264)(606 257)(617 264)
\cpath (603 268)(617 264)(609 263)(602 262)(594 261)
\cpath (587 260)(579 259)(572 259)(565 258)(558 257)
\cpath (552 257)(545 256)(539 256)(534 255)(529 255)
\cpath (524 254)(519 254)(514 254)(510 253)(510 253)
\cpath (498 247)(510 253)(496 258)(510 253)(505 253)
\cpath (501 253)(498 253)(494 252)(490 252)(487 252)
\cpath (484 252)(480 252)(477 252)(474 251)(472 251)
\cpath (469 251)(466 251)(464 251)(462 251)(459 251)
\cpath (457 251)(455 251)(453 251)(451 251)(449 251)
\cpath (447 251)(446 251)(444 251)(442 251)(441 251)
\cpath (439 251)(438 251)(436 251)(435 251)(434 251)
\cpath (433 251)(431 251)(430 251)(429 251)(428 251)
\cpath (427 251)(426 251)(425 252)(424 252)(423 252)
\cpath (423 252)(422 252)(421 252)(420 252)(420 252)
\cpath (419 252)(418 252)(418 253)(417 253)(416 253)
\cpath (416 253)(415 253)(415 253)(414 253)(414 253)
\cpath (413 254)(413 254)(412 254)(412 254)(411 254)
\cpath (411 254)(411 255)(410 255)(410 255)(410 255)
\cpath (409 255)(409 255)(409 256)(408 256)(408 256)
\cpath (408 256)(408 256)(407 256)(407 257)(407 257)
\cpath (407 257)(406 257)(406 257)(406 258)(406 258)
\cpath (406 258)(406 258)(392 263)(406 258)(407 268)
\cpath (406 258)(406 258)(405 258)(405 259)(405 259)
\cpath (405 259)(405 259)(405 260)(404 260)(404 260)
\cpath (404 260)(404 260)(404 261)(404 261)(404 261)
\cpath (404 261)(404 262)(403 262)(403 262)(403 262)
\cpath (403 263)(403 263)(403 263)(403 263)(403 264)
\cpath (403 264)(403 264)(403 264)(403 265)(403 265)
\cpath (403 265)(402 266)(402 266)(402 266)(402 266)
\cpath (402 267)(402 267)(402 267)(402 268)(402 268)
\cpath (402 268)(402 269)(402 269)(402 269)(402 269)
\cpath (402 270)(402 270)(402 270)(402 271)(402 271)
\cpath (402 271)(402 272)(402 272)(402 272)(402 273)
\cpath (402 273)(402 274)(402 274)(402 274)(401 275)
\cpath (401 275)(401 275)(401 276)(401 276)(401 277)
\cpath (401 277)(401 277)(401 278)(401 278)(401 279)
\cpath (401 279)(401 279)(401 280)(401 280)(401 281)
\cpath (401 281)(401 282)(401 282)(401 283)(401 283)
\cpath (401 284)(401 284)(401 285)(401 285)(401 286)
\cpath (401 286)(401 287)(401 287)(401 288)(401 288)
\cpath (401 289)(401 289)(401 290)(401 290)(401 291)
\cpath (401 292)(401 292)(401 293)(401 293)(401 294)
\cpath (401 295)(401 295)(401 296)(401 297)(401 297)
\cpath (401 298)(401 299)(401 299)(401 300)(401 301)
\cpath (401 302)(401 303)(401 303)(401 304)(401 305)
\cpath (401 306)(401 307)(401 308)(401 309)(401 309)
\cpath (401 310)(401 311)(401 312)(401 314)(401 315)
\cpath (401 316)(401 317)(401 318)(401 319)(401 320)
\cpath (401 322)(401 323)(401 324)(401 326)(401 327)
\cpath (401 329)(401 330)(401 332)(401 333)(401 335)
\cpath (401 336)(401 338)(401 340)(401 341)(401 343)
\cpath (401 345)(401 347)(401 348)(401 350)(401 352)
\cpath (401 354)(401 356)(401 357)(401 359)(401 361)
\cpath (401 363)(401 365)(401 367)(401 369)(401 371)
\cpath (401 372)(401 374)(401 376)(401 378)
\path (699 354)(699 354)(706 345)(699 354)(689 346)(699 354)
\cpath (699 355)(699 358)(700 362)(700 366)(699 370)
\cpath (698 374)(697 379)(695 383)(693 387)(690 392)
\cpath (687 397)(684 401)(680 406)(675 411)(671 416)
\cpath (665 421)(665 421)(680 418)(665 421)(666 411)
\cpath (665 421)(659 427)(653 432)(647 437)(640 442)
\cpath (634 447)(627 452)(620 457)(612 462)(605 466)
\cpath (597 471)(589 475)(589 475)(605 474)(589 475)
\cpath (594 466)(589 475)(581 480)(573 485)(564 489)
\cpath (556 493)(548 497)(541 500)(534 503)(528 506)
\cpath (522 508)(517 511)(512 513)(508 514)(503 516)
\cpath (499 517)(499 517)(514 518)(499 517)(506 509)
\cpath (499 517)(495 519)(491 520)(487 521)(484 522)
\cpath (481 523)(478 524)(474 525)(472 525)(469 526)
\cpath (466 526)(464 527)(461 527)(459 528)(457 528)
\cpath (454 528)(452 528)(450 528)(448 529)(447 529)
\cpath (445 529)(443 529)(442 528)(440 528)(439 528)
\cpath (437 528)(436 528)(434 527)(433 527)(432 527)
\cpath (431 527)(430 526)(429 526)(428 525)(427 525)
\cpath (426 524)(425 524)(424 523)(423 523)(422 522)
\cpath (421 522)(421 521)(420 520)(419 520)(418 519)
\cpath (418 518)(417 518)(417 517)(416 516)(415 516)
\cpath (415 515)(414 514)(414 513)(413 512)(413 512)
\cpath (412 511)(412 510)(412 509)(411 508)(411 508)
\cpath (406 518)(411 508)(423 515)(411 508)(411 507)
\cpath (410 507)(410 506)(410 505)(409 504)(409 503)
\cpath (409 502)(409 501)(408 500)(408 499)(408 498)
\cpath (407 497)(407 496)(407 495)(407 494)(407 493)
\cpath (406 492)(406 491)(406 490)(406 489)(406 488)
\cpath (405 487)(405 485)(405 484)(405 483)(405 482)
\cpath (405 481)(405 480)(404 479)(404 477)(404 476)
\cpath (404 475)(404 474)(404 473)(404 471)(404 470)
\cpath (404 469)(403 468)(403 466)(403 465)(403 464)
\cpath (403 463)(403 461)(403 460)(403 459)(403 457)
\cpath (403 456)(403 454)(403 453)(403 452)(402 450)
\cpath (402 449)(402 447)(402 446)(402 444)(402 443)
\cpath (402 441)(402 440)(402 438)(402 436)(402 435)
\cpath (402 433)(402 431)(402 429)(402 428)(402 426)
\cpath (402 424)(402 422)(402 420)(402 418)(402 415)
\cpath (402 413)(402 410)(402 407)(402 404)(402 402)
\cpath (402 400)(401 397)(401 395)(401 393)(401 391)
\cpath (401 389)(401 387)(401 385)(401 383)(401 381)
\cpath (401 379)(401 377)(401 375)(401 373)(401 371)
\cpath (401 369)(401 367)(401 365)(401 364)(401 362)
\cpath (401 360)(401 358)(401 356)(401 355)(401 353)
\cpath (401 351)(401 349)(401 348)(401 346)(401 344)
\cpath (401 343)(401 341)(401 339)(401 338)(401 336)
\cpath (401 334)(401 333)(401 331)(402 330)(402 328)
\cpath (402 327)(402 325)(402 324)(402 323)(402 321)
\cpath (402 320)(402 319)(402 318)(402 316)(402 315)
\cpath (402 314)(402 313)(402 312)(402 311)(402 310)
\cpath (402 309)(402 308)(402 307)(402 306)(402 306)
\cpath (402 305)(402 304)(402 303)(402 302)(402 301)
\cpath (402 301)(402 300)(402 299)(403 299)(403 298)
\cpath (403 297)(403 296)(403 296)(403 295)(403 295)
\cpath (403 294)(403 293)(403 293)(403 292)(403 292)
\cpath (403 291)(404 290)(404 290)(404 289)(404 289)
\cpath (404 288)(404 288)(404 287)(404 287)(404 286)
\cpath (405 286)(405 285)(405 285)(405 284)(405 284)
\cpath (405 283)(406 283)(406 283)(406 282)(406 282)
\cpath (406 281)(406 281)(407 281)(407 280)(407 280)
\cpath (407 279)(408 279)(408 279)
\path (699 353)(699 353)(698 351)(697 347)(695 343)(693 340)
\cpath (690 336)(687 333)(684 329)(680 326)(676 323)
\cpath (672 320)(667 317)(662 314)(657 311)(652 309)
\cpath (647 306)(642 304)(636 302)(630 300)(630 300)
\cpath (623 291)(630 300)(615 301)(630 300)(624 298)
\cpath (619 296)(613 294)(607 292)(600 291)(594 289)
\cpath (588 287)(582 286)(576 285)(570 283)(564 282)
\cpath (559 281)(553 280)(547 279)(542 278)(536 277)
\cpath (531 276)(526 276)(520 275)(515 274)(510 274)
\cpath (506 273)(501 272)(497 272)(493 272)(490 271)
\cpath (486 271)(483 271)(479 270)(476 270)(473 270)
\cpath (471 270)(468 269)(465 269)(463 269)(460 269)
\cpath (458 269)(456 269)(454 269)(452 269)(450 269)
\cpath (448 269)(446 269)(445 269)(443 269)(441 269)
\cpath (440 269)(438 269)(437 269)(436 269)(434 269)
\cpath (433 269)(432 269)(431 269)(429 269)(428 270)
\cpath (427 270)(426 270)(425 270)(425 270)(424 270)
\cpath (423 270)(422 271)(421 271)(420 271)(420 271)
\cpath (419 271)(418 272)(418 272)(417 272)(416 272)
\cpath (416 273)(415 273)(415 273)(414 273)(414 274)
\cpath (413 274)(413 274)(412 274)(412 275)(412 275)
\cpath (411 275)(411 276)(410 276)(410 276)(410 276)
\cpath (409 277)(409 277)(409 277)(409 278)(408 278)
\cpath (408 278)(408 279)(407 279)(407 280)(407 280)
\cpath (407 280)(407 281)(406 281)(406 282)(406 282)
\cpath (406 282)(406 283)(405 283)(405 284)(405 284)
\cpath (405 285)(405 285)(405 285)(405 286)(404 286)
\cpath (404 287)(404 287)(404 288)(404 288)(404 289)
\cpath (404 289)(404 290)(404 291)(403 291)(403 292)
\cpath (403 292)(403 293)(403 293)(403 294)(403 295)
\cpath (403 295)(403 296)(403 297)(403 297)(403 298)
\cpath (403 299)(402 299)(402 300)(402 301)(402 301)
\cpath (402 302)(402 303)(402 304)(402 305)(402 305)
\cpath (402 306)(402 307)(402 308)(402 309)(402 310)
\cpath (402 311)(402 312)(402 313)(402 314)(402 315)
\cpath (402 316)(402 317)(402 318)(402 320)(402 321)
\cpath (402 322)(402 323)(402 325)(402 326)(402 327)
\cpath (402 329)(402 330)(401 332)(401 334)(401 335)
\cpath (401 337)(401 338)(401 340)(401 342)(401 343)
\cpath (401 345)(401 347)(401 349)(401 350)(401 352)
\cpath (401 354)(401 356)(401 358)(401 359)(401 361)
\cpath (401 363)(401 365)(401 367)(401 369)(401 371)
\cpath (401 373)(401 374)(401 376)(401 378)(401 380)
\cpath (401 382)(401 384)(401 386)(401 388)(401 391)
\cpath (401 393)(401 395)(401 397)(402 399)(402 402)
\cpath (402 404)(402 406)(402 409)(402 412)(402 414)
\cpath (402 417)(402 419)(402 422)(402 424)(402 426)
\cpath (402 428)(402 430)(402 432)(402 433)(402 435)
\cpath (402 437)(402 439)(402 440)(402 442)
\path (590 352)(590 352)(596 343)(590 352)(579 345)(590 352)
\cpath (590 353)(590 355)(591 358)(591 361)(591 364)
\cpath (591 367)(590 370)(589 373)(588 377)(587 380)
\cpath (585 383)(584 387)(581 390)(579 393)(576 397)
\cpath (573 401)(570 404)(567 408)(563 411)(559 415)
\cpath (555 418)(555 418)(570 415)(555 418)(557 408)
\cpath (555 418)(551 422)(547 425)(543 428)(538 431)
\cpath (533 434)(529 437)(524 440)(519 443)(514 446)
\cpath (509 449)(504 451)(499 454)(494 456)(489 458)
\cpath (485 459)(482 461)(478 462)(475 463)(472 464)
\cpath (469 465)(466 465)(463 466)(461 467)(458 467)
\cpath (456 467)(454 468)(452 468)(450 468)(448 468)
\cpath (446 468)(444 468)(443 468)(441 468)(440 468)
\cpath (438 468)(437 467)(435 467)(434 467)(433 466)
\cpath (431 466)(430 465)(429 465)(428 464)(427 464)
\cpath (426 463)(425 463)(424 462)(423 461)(422 460)
\cpath (422 460)(421 459)(420 458)(419 457)(419 456)
\cpath (418 455)(417 455)(417 454)(416 453)(416 452)
\cpath (415 451)(415 450)(414 449)(414 447)(413 446)
\cpath (413 445)(412 444)(412 443)(411 442)(411 440)
\cpath (411 439)(410 438)(410 436)(410 435)(409 434)
\cpath (409 432)(409 431)(408 429)(408 428)(408 426)
\cpath (408 425)(407 423)(407 421)(407 419)(407 417)
\cpath (406 416)(406 413)(406 411)(406 409)(405 406)
\cpath (405 403)(405 401)(405 398)(405 396)(405 394)
\cpath (405 392)(405 390)(404 388)(404 386)(404 384)
\cpath (404 382)(404 380)(404 378)(404 376)(404 374)
\cpath (404 372)(404 371)(404 369)(404 367)(404 365)
\cpath (404 363)(404 362)(404 360)(404 358)(404 356)
\cpath (404 355)(404 353)(404 351)(404 350)(404 348)
\cpath (404 346)(404 345)(404 343)(404 342)(404 340)
\cpath (405 338)(405 337)(405 335)(405 334)(405 333)
\cpath (405 331)(405 330)(405 328)(405 327)(405 326)
\cpath (406 324)(406 323)(406 322)(406 321)(406 320)
\cpath (406 319)(407 317)(407 316)(407 315)(407 314)
\cpath (408 314)(408 313)(408 312)(408 311)(409 310)
\cpath (409 309)(409 308)(410 308)(410 307)(410 306)
\cpath (411 305)(411 305)(411 304)(412 303)(412 303)
\cpath (413 302)(413 302)(413 301)(414 300)(414 300)
\cpath (415 299)(415 299)(416 298)(417 298)(417 297)
\cpath (418 297)(419 297)(419 296)(420 296)(421 295)
\cpath (421 295)(422 295)(423 294)(424 294)(425 294)
\cpath (426 293)(427 293)(428 293)(429 293)(430 292)
\cpath (431 292)(432 292)(433 292)(435 292)(436 291)
\cpath (437 291)(439 291)(440 291)(442 291)(444 291)
\cpath (445 291)(447 291)(449 291)(451 291)(453 291)
\cpath (455 291)(457 291)(460 292)(462 292)
\path (589 352)(589 352)(589 351)(588 348)(587 345)(585 342)
\cpath (583 340)(581 337)(579 334)(576 332)(573 329)
\cpath (569 326)(566 324)(562 322)(559 319)(555 317)
\cpath (551 315)(547 314)(543 312)(540 310)(535 309)
\cpath (531 307)(527 306)(523 304)(519 303)(519 303)
\cpath (511 295)(519 303)(504 305)(519 303)(515 302)
\cpath (511 301)(507 300)(503 299)(500 298)(496 297)
\cpath (492 296)(489 296)(485 295)(481 294)(478 294)
\cpath (474 293)(471 293)(468 292)(465 292)(463 292)
\cpath (460 292)(458 291)(456 291)(453 291)(451 291)
\cpath (449 291)(447 291)(446 291)(444 291)(442 291)
\cpath (441 291)(439 291)(438 291)(436 291)(435 292)
\cpath (434 292)(432 292)(431 292)(430 292)(429 292)
\cpath (428 293)(427 293)(426 293)(425 294)(424 294)
\cpath (423 294)(422 295)(422 295)(421 295)(420 296)
\cpath (419 296)(419 296)(418 297)(417 297)(417 298)
\cpath (416 298)(416 299)(415 299)(415 300)(414 300)
\cpath (414 301)(413 301)(413 302)(412 303)(412 303)
\cpath (411 304)(411 305)(411 305)(410 306)(410 307)
\cpath (410 307)(409 308)(409 309)(409 310)(408 310)
\cpath (408 311)(408 312)(408 313)(407 314)(407 315)
\cpath (407 316)(407 317)(407 318)(406 319)(406 320)
\cpath (406 321)(406 322)(406 323)(406 325)(405 326)
\cpath (405 327)(405 329)(405 330)(405 331)(405 333)
\cpath (405 334)(405 336)(405 337)(404 339)(404 340)
\cpath (404 342)(404 344)(404 345)(404 347)(404 348)
\cpath (404 350)(404 352)(404 354)(404 355)(404 357)
\cpath (404 359)(404 360)(404 362)(404 364)(404 366)
\cpath (404 368)(404 370)(404 371)(404 373)(404 375)
\cpath (404 377)(404 379)(404 381)(404 383)(404 385)
\cpath (404 387)(405 389)(405 391)(405 393)(405 395)
\cpath (405 397)(405 399)(405 402)(405 404)(406 406)
\cpath (406 409)(406 412)(406 414)(406 416)(407 418)
\cpath (407 420)(407 422)(407 424)(408 426)(408 428)
\cpath (408 429)(409 431)(409 432)(409 434)(410 435)
\cpath (410 437)(410 438)(411 439)(411 441)(411 442)
\cpath (412 443)(412 444)(413 446)(413 447)(414 448)
\cpath (414 449)(415 450)(415 451)(416 452)(416 453)
\cpath (417 454)(418 455)(418 456)(419 457)(420 458)
\cpath (421 459)(421 459)(422 460)(423 461)(424 462)
\cpath (425 462)(426 463)(427 464)(428 464)(429 465)
\cpath (430 465)(431 466)(432 466)(434 467)(435 467)
\cpath (436 467)(438 468)(439 468)(441 468)(442 468)
\cpath (444 468)(446 468)(447 468)(449 468)
\path (521 357)(521 357)(528 348)(521 357)(511 349)(521 357)
\cpath (521 357)(521 358)(521 360)(522 362)(521 364)
\cpath (521 366)(521 369)(520 371)(520 373)(519 375)
\cpath (518 378)(517 380)(515 382)(514 385)(512 387)
\cpath (510 390)(508 392)(506 394)(504 397)(501 399)
\cpath (499 401)(496 403)(493 406)(490 408)(487 410)
\cpath (484 412)(481 413)(478 415)(475 416)(472 418)
\cpath (469 419)(466 420)(464 421)(461 422)(459 422)
\cpath (456 423)(454 423)(452 424)(450 424)(448 424)
\cpath (446 424)(445 424)(443 424)(441 424)(440 423)
\cpath (438 423)(437 423)(436 422)(434 422)(433 421)
\cpath (432 421)(431 420)(430 420)(429 419)(428 418)
\cpath (427 417)(426 416)(425 415)(424 414)(423 413)
\cpath (422 412)(422 411)(421 410)(420 409)(420 407)
\cpath (419 406)(418 405)(418 403)(417 401)(416 400)
\cpath (416 398)(415 396)(415 393)(414 391)(414 389)
\cpath (413 387)(413 385)(413 383)(412 381)(412 379)
\cpath (412 377)(412 375)(412 374)(412 372)(412 370)
\cpath (412 368)(411 367)(411 365)(411 364)(411 362)
\cpath (411 360)(411 359)(412 357)(412 356)(412 354)
\cpath (412 353)(412 351)(412 350)(412 349)(412 347)
\cpath (412 346)(413 344)(413 343)(413 342)(413 341)
\cpath (414 339)(414 338)(414 337)(414 336)(415 334)
\cpath (415 333)(416 332)(416 331)(416 330)(417 329)
\cpath (417 328)(418 327)(418 326)(419 325)(420 325)
\cpath (420 324)(421 323)(422 322)(422 321)(423 321)
\cpath (424 320)(425 319)(426 319)(427 318)(428 318)
\cpath (429 317)(430 317)(431 316)(432 316)(433 316)
\cpath (434 315)(436 315)(437 315)(438 314)(440 314)
\cpath (441 314)(443 314)(445 314)(447 314)(448 314)
\cpath (450 314)(453 314)(455 315)(457 315)(459 315)
\cpath (462 316)(464 316)(467 317)(469 317)(471 318)
\cpath (474 318)(476 319)(479 320)(482 321)(484 322)
\cpath (487 323)(489 324)(492 326)(495 327)(497 328)
\cpath (500 330)(503 332)(505 334)
\path (521 356)(521 356)(521 356)(520 354)(520 352)(519 350)
\cpath (518 348)(517 346)(515 344)(514 342)(512 340)
\cpath (510 338)(508 336)(506 334)(503 332)(501 330)
\cpath (498 329)(495 327)(493 326)(490 324)(487 323)
\cpath (484 322)(482 321)(479 320)(476 319)(474 318)
\cpath (471 318)(469 317)(467 317)(464 316)(462 316)
\cpath (460 315)(457 315)(455 315)(453 314)(451 314)
\cpath (449 314)(447 314)(445 314)(443 314)(442 314)
\cpath (440 314)(438 314)(437 315)(436 315)(434 315)
\cpath (433 316)(432 316)(431 316)(430 317)(429 317)
\cpath (428 318)(427 318)(426 319)(425 319)(424 320)
\cpath (423 321)(422 321)(422 322)(421 323)(420 324)
\cpath (420 324)(419 325)(419 326)(418 327)(417 328)
\cpath (417 329)(416 330)(416 331)(416 332)(415 333)
\cpath (415 334)(415 335)(414 337)(414 338)(414 339)
\cpath (413 340)(413 342)(413 343)(413 344)(412 346)
\cpath (412 347)(412 348)(412 350)(412 351)(412 353)
\cpath (412 354)(412 356)(412 357)(411 359)(411 360)
\cpath (411 362)(411 363)(411 365)(411 366)(412 368)
\cpath (412 370)(412 371)(412 373)(412 375)(412 376)
\cpath (412 378)(412 380)(413 382)(413 384)(413 385)
\cpath (413 387)(414 389)(414 391)(415 393)(415 395)
\cpath (416 398)(417 400)(417 402)(418 404)(419 405)
\cpath (419 407)(420 408)(421 410)(422 411)(422 412)
\cpath (423 413)(424 415)(425 416)(426 416)(427 417)
\cpath (428 418)(429 419)(430 420)(431 420)(432 421)
\cpath (434 422)(435 422)(436 423)(438 423)(439 423)
\cpath (441 424)(442 424)(444 424)(445 424)(447 424)
\cpath (449 424)(451 424)(453 423)(455 423)(457 423)
\cpath (459 422)(461 422)(464 421)(466 420)(469 419)
\cpath (471 418)(474 417)(477 415)(480 414)(483 412)
\cpath (486 410)(489 408)(492 406)(495 404)(498 401)
\cpath (501 399)(504 397)(506 394)(508 392)(510 389)
\cpath (512 387)
\path (467 354)(467 354)(468 354)(468 355)(469 356)(469 357)
\cpath (469 358)(469 359)(470 360)(470 362)(470 363)
\cpath (470 364)(469 365)(469 366)(469 367)(469 368)
\cpath (468 369)(468 370)(467 371)(467 372)(466 373)
\cpath (465 375)(464 376)(463 377)(462 378)(461 379)
\cpath (460 379)(459 380)(458 381)(457 381)(456 382)
\cpath (455 382)(454 383)(452 383)(451 384)(451 384)
\cpath (450 384)(449 384)(448 384)(447 384)(446 384)
\cpath (445 384)(444 384)(443 384)(442 384)(441 384)
\cpath (441 383)(440 383)(439 383)(438 382)(438 382)
\cpath (437 382)(436 381)(436 381)(435 380)(435 379)
\cpath (434 379)(433 378)(433 377)(432 377)(432 376)
\cpath (432 375)(431 374)(431 373)(430 372)(430 371)
\cpath (430 370)(429 369)(429 368)(429 367)(429 365)
\cpath (429 364)(429 363)(428 362)(429 361)(429 360)
\cpath (429 359)(429 358)(429 357)(429 356)(429 355)
\cpath (430 354)(430 353)(430 353)(431 352)(431 351)
\cpath (431 350)(432 350)(432 349)(432 348)(433 348)
\cpath (433 347)(434 347)(435 346)(435 346)(436 345)
\cpath (436 345)(437 344)(438 344)(438 344)(439 343)
\cpath (440 343)(441 343)(442 343)(442 342)(443 342)
\cpath (444 342)(445 342)(446 342)(447 342)(448 342)
\cpath (449 342)(450 342)(451 343)(452 343)(453 343)
\cpath (455 344)(456 344)(457 344)(458 345)(459 346)
\cpath (460 346)(461 347)(462 348)(463 348)(464 349)
\cpath (465 350)(465 351)(466 352)(467 353)
\path (467 354)(467 354)(467 354)(467 353)(466 352)(465 351)
\cpath (464 350)(464 349)(463 348)(462 347)(461 347)
\cpath (460 346)(459 345)(458 345)(456 344)(455 344)
\cpath (454 343)(453 343)(452 343)(451 343)(450 342)
\cpath (449 342)(448 342)(447 342)(446 342)(445 342)
\cpath (444 342)(443 342)(442 342)(441 343)(440 343)
\cpath (440 343)(439 343)(438 344)(437 344)(437 345)
\cpath (436 345)(435 345)(435 346)(434 346)(434 347)
\cpath (433 348)(433 348)(432 349)(432 349)(431 350)
\cpath (431 351)(431 352)(430 352)(430 353)(430 354)
\cpath (430 355)(429 355)(429 356)(429 357)(429 358)
\cpath (429 359)(429 360)(429 361)(428 362)(429 363)
\cpath (429 364)(429 365)(429 366)(429 367)(429 368)
\cpath (429 370)(430 371)(430 372)(431 373)(431 374)
\cpath (432 375)(432 376)(433 377)(433 378)(434 379)
\cpath (434 379)(435 380)(436 380)(436 381)(437 381)
\cpath (438 382)(438 382)(439 383)(440 383)(441 383)
\cpath (441 384)(442 384)(443 384)(444 384)(445 384)
\cpath (446 384)(446 384)(447 384)(448 384)(449 384)
\cpath (450 384)(451 384)(452 383)(453 383)(454 383)
\cpath (455 382)(456 382)(457 381)(458 381)(459 380)
\cpath (460 379)(461 378)(462 378)(463 377)(464 376)
\cpath (465 374)(466 373)(467 372)(467 371)(468 370)
\cpath (468 369)(469 368)(469 366)(469 365)(470 364)
\cpath (470 363)(470 362)(470 361)(470 360)(469 359)
\cpath (469 358)(469 357)(468 356)(468 355)
\path (982 407)(982 407)(995 402)(982 407)(980 397)(982 407)
\cpath (980 409)(974 416)(967 423)(960 430)(952 438)
\cpath (943 446)(933 454)(922 462)(910 471)(910 471)
\cpath (925 468)(910 471)(912 461)(910 471)(896 480)
\cpath (882 489)(866 499)(850 509)(834 518)(818 527)
\cpath (818 527)(834 526)(818 527)(823 517)(818 527)
\cpath (802 536)(785 545)(768 553)(750 563)(730 572)
\cpath (730 572)(745 572)(730 572)(735 563)(730 572)
\cpath (713 580)(698 587)(683 594)(670 600)(658 605)
\cpath (646 610)(636 615)(636 615)(651 615)(636 615)
\cpath (642 605)(636 615)(626 619)(616 623)(607 626)
\cpath (599 630)(590 633)(583 636)(575 638)(568 641)
\cpath (562 643)(555 646)(549 648)(543 650)(543 650)
\cpath (559 651)(543 650)(551 641)(543 650)(538 652)
\cpath (532 653)(527 655)(522 657)(518 658)(513 659)
\cpath (509 661)(504 662)(500 663)(497 664)(493 665)
\cpath (489 666)(486 666)(483 667)(480 668)(477 668)
\cpath (474 669)(471 670)(468 670)(466 670)(463 671)
\cpath (461 671)(459 671)(456 671)(454 672)(452 672)
\cpath (450 672)(449 672)(447 672)(445 672)(443 672)
\cpath (442 672)(442 672)(454 678)(442 672)(455 667)
\cpath (442 672)(440 672)(439 672)(437 672)(436 671)
\cpath (435 671)(433 671)(432 671)(431 670)(430 670)
\cpath (429 670)(428 669)(427 669)(426 669)(425 668)
\cpath (424 668)(423 668)(422 667)(421 667)(421 666)
\cpath (420 666)(419 665)(419 665)(418 664)(417 664)
\cpath (417 663)(416 663)(416 662)(415 662)(414 661)
\cpath (414 660)(413 660)(413 659)(413 659)(412 658)
\cpath (412 657)(411 657)(411 656)(411 655)(410 655)
\cpath (410 654)(410 653)(409 653)(409 652)(409 651)
\cpath (408 651)(408 650)(408 649)(408 648)(407 648)
\cpath (407 647)(407 646)(407 645)(406 645)(406 644)
\cpath (406 643)(406 642)(406 642)(405 641)(405 640)
\cpath (405 639)(405 639)(405 638)(405 637)(405 636)
\cpath (404 635)(404 635)(404 634)(404 633)(404 632)
\cpath (404 631)(404 631)(404 630)(404 629)(403 628)
\cpath (403 627)(403 627)(403 626)(403 625)(403 624)
\cpath (403 623)(403 622)(403 622)(403 621)(403 620)
\cpath (403 619)(403 618)(402 617)(402 617)(402 616)
\cpath (402 615)(402 614)(402 613)(402 612)(402 611)
\cpath (402 611)(402 610)(402 609)(402 608)(402 607)
\cpath (402 606)(402 605)(402 604)(402 604)(402 603)
\cpath (402 602)(402 601)(402 600)(402 599)(402 598)
\cpath (402 597)(402 596)(402 596)(402 595)(402 594)
\cpath (401 593)(401 592)(401 591)(401 590)(401 589)
\cpath (401 588)(401 587)(401 586)(401 586)(401 585)
\cpath (401 584)(401 583)(401 582)(401 581)(401 580)
\cpath (401 579)(401 578)(401 577)(401 576)(401 575)
\cpath (401 574)(401 574)(401 573)(401 572)(401 571)
\cpath (401 570)(401 569)(401 568)(401 567)(401 566)
\cpath (401 565)(401 564)(401 563)(401 562)(401 561)
\cpath (401 560)(401 559)(401 558)(401 557)(401 556)
\cpath (401 555)(401 554)(401 553)(401 552)(401 551)
\cpath (401 550)(401 549)(401 548)(401 547)(401 546)
\cpath (401 545)(401 544)(401 543)(401 542)(401 541)
\cpath (401 540)(401 539)(401 538)(401 537)(401 536)
\cpath (401 535)(401 534)(401 533)(401 532)(401 531)
\cpath (401 530)(401 529)(401 528)(401 527)(401 526)
\cpath (401 525)(401 524)(401 523)(401 522)(401 521)
\cpath (401 520)(401 518)(401 517)(401 516)(401 515)
\cpath (401 514)(401 513)(401 512)(401 511)(401 510)
\cpath (401 509)(401 508)(401 506)(401 505)(401 504)
\cpath (401 503)(401 502)(401 501)(401 500)(401 498)
\cpath (401 497)(401 496)(401 495)(401 494)(401 493)
\cpath (401 491)(401 490)(401 489)(401 488)(401 487)
\cpath (401 485)(401 484)(401 483)(401 482)(401 480)
\cpath (401 479)(401 478)(401 477)(401 475)(401 474)
\cpath (401 473)(401 472)(401 470)(401 469)(401 468)
\cpath (401 466)(401 465)(401 464)(401 462)(401 461)
\cpath (401 459)(401 458)(401 457)(401 455)(401 454)
\cpath (401 452)(401 451)(401 449)(401 448)(401 446)
\cpath (401 445)(401 443)(401 442)(401 440)(401 438)
\cpath (401 437)(401 435)(401 433)(401 431)(401 430)
\cpath (401 428)(401 426)(401 424)(401 422)(401 420)
\cpath (401 417)(401 415)(401 412)(401 409)(401 407)
\cpath (401 404)(401 402)(401 399)(401 397)(401 395)
\cpath (401 393)(401 391)(401 389)(401 387)(401 385)
\cpath (401 383)(401 381)(401 379)(401 377)(401 375)
\cpath (401 373)(401 371)(401 369)(401 367)(401 365)
\cpath (401 363)(401 362)(401 360)(401 358)(401 356)
\cpath (401 354)(401 353)(401 351)(401 349)(401 347)
\cpath (401 346)(401 344)(401 342)(401 341)(401 339)
\cpath (401 337)(401 336)(401 334)(401 332)(401 331)
\cpath (401 329)(401 328)(401 326)(401 325)(401 324)
\cpath (401 322)(401 321)(401 320)(401 318)(401 317)
\cpath (401 316)(401 315)(401 314)(401 313)(401 312)
\cpath (401 311)(401 310)(401 309)(401 308)(401 307)
\cpath (401 306)(401 305)(401 304)(401 303)(401 302)
\cpath (401 302)(401 301)(401 300)(401 299)(401 299)
\cpath (401 298)(401 297)(401 296)(401 296)(401 295)
\cpath (401 294)(401 294)(401 293)(401 293)
\path (982 407)(982 407)(995 402)(982 407)(980 397)(982 407)
\cpath (983 404)(988 398)(992 392)(995 385)(997 379)
\cpath (999 374)(1000 368)(1000 362)(1000 357)(999 352)
\cpath (998 347)(995 342)(993 338)(993 338)(995 328)
\cpath (993 338)(980 333)(993 338)(990 333)(986 329)
\cpath (981 325)(976 321)(971 317)(965 313)(959 309)
\cpath (952 306)(944 303)(937 299)(929 296)(920 293)
\cpath (912 291)(912 291)(904 282)(912 291)(896 292)
\cpath (912 291)(903 288)(894 285)(884 283)(874 280)
\cpath (864 278)(854 276)(844 274)(833 272)(823 270)
\cpath (812 268)(801 266)(801 266)(790 259)(801 266)
\cpath (786 270)(801 266)(789 265)(778 263)(767 261)
\cpath (756 260)(745 259)(734 257)(724 256)(714 255)
\cpath (703 254)(693 253)(684 252)(674 251)(664 251)
\cpath (655 250)(646 249)(636 248)(627 248)(619 247)
\cpath (610 246)(601 246)(593 245)(585 245)(578 244)
\cpath (571 244)(564 243)(558 243)(552 243)(546 242)
\cpath (540 242)(535 242)(530 242)(525 241)(520 241)
\cpath (516 241)(511 241)(507 241)(503 241)(499 240)
\cpath (495 240)(492 240)(488 240)(485 240)(482 240)
\cpath (479 240)(476 240)(473 240)(470 240)(468 239)
\cpath (465 239)(463 239)(460 239)(458 239)(456 239)
\cpath (454 239)(452 239)(450 239)(448 239)(447 239)
\cpath (445 239)(443 239)(442 239)(440 239)(439 239)
\cpath (437 239)(436 239)(435 239)(433 239)(432 239)
\cpath (431 239)(430 239)(429 239)(428 240)(427 240)
\cpath (426 240)(425 240)(424 240)(423 240)(422 240)
\cpath (421 240)(421 240)(420 240)(419 240)(419 240)
\cpath (418 240)(417 240)(417 240)(416 240)(416 241)
\cpath (415 241)(414 241)(414 241)(414 241)(413 241)
\cpath (413 241)(412 241)(412 241)(411 241)(411 241)
\cpath (411 242)(410 242)(410 242)(410 242)(409 242)
\cpath (409 242)(409 242)(408 242)(408 242)(408 242)
\cpath (408 243)(407 243)(407 243)(407 243)(407 243)
\cpath (406 243)(406 243)(406 243)(406 243)(406 244)
\cpath (406 244)(405 244)(405 244)(405 244)(405 244)
\cpath (405 244)(405 244)(404 245)(404 245)(404 245)
\cpath (404 245)(404 245)(404 245)(404 245)(404 245)
\cpath (404 246)(403 246)(403 246)(403 246)(403 246)
\cpath (403 246)(403 246)(403 247)(403 247)(403 247)
\cpath (403 247)(403 247)(403 247)(403 248)(402 248)
\cpath (402 248)(402 248)(402 248)(402 248)(402 248)
\cpath (402 249)(402 249)(402 249)(402 249)(402 249)
\cpath (402 249)(402 250)(402 250)(402 250)(402 250)
\cpath (402 250)(402 250)(402 251)(402 251)(402 251)
\cpath (402 251)(402 251)(402 252)(402 252)(402 252)
\cpath (402 252)(402 252)(401 252)(401 253)(401 253)
\cpath (401 253)(401 253)(401 253)(401 254)(401 254)
\cpath (401 254)(401 254)(401 254)(401 255)(401 255)
\cpath (401 255)(401 255)(401 255)(401 256)(401 256)
\cpath (401 256)(401 256)(401 256)(401 257)(401 257)
\cpath (401 257)(401 257)(401 258)(401 258)(401 258)
\cpath (401 258)(401 258)(401 259)(401 259)(401 259)
\cpath (401 259)(401 260)(401 260)(401 260)(401 260)
\cpath (401 261)(401 261)(401 261)(401 261)(401 262)
\cpath (401 262)(401 262)(401 262)(401 263)(401 263)
\cpath (401 263)(401 264)(401 264)(401 264)(401 264)
\cpath (401 265)(401 265)(401 265)(401 265)(401 266)
\cpath (401 266)(401 266)(401 267)(401 267)(401 267)
\cpath (401 268)(401 268)(401 268)(401 269)(401 269)
\cpath (401 269)(401 269)(401 270)(401 270)(401 270)
\cpath (401 271)(401 271)(401 272)(401 272)(401 272)
\cpath (401 273)(401 273)(401 273)(401 274)(401 274)
\cpath (401 274)(401 275)(401 275)(401 276)(401 276)
\cpath (401 276)(401 277)(401 277)(401 278)(401 278)
\cpath (401 278)(401 279)(401 279)(401 280)(401 280)
\cpath (401 280)(401 281)(401 281)(401 282)(401 282)
\cpath (401 283)(401 283)(401 284)(401 284)(401 285)
\cpath (401 285)(401 286)(401 286)(401 287)(401 287)
\cpath (401 288)(401 288)(401 289)(401 289)(401 290)
\cpath (401 291)(401 291)(401 292)(401 292)(401 293)
\cpath (401 294)(401 294)(401 295)(401 296)(401 296)
\cpath (401 297)(401 298)(401 298)(401 299)(401 300)
\cpath (401 301)(401 301)(401 302)(401 303)(401 304)
\cpath (401 305)(401 305)(401 306)(401 307)(401 308)
\cpath (401 309)(401 310)(401 311)(401 312)(401 313)
\cpath (401 314)(401 315)(401 316)(401 317)(401 319)
\cpath (401 320)(401 321)
\path (1117 419)(1117 419)(1131 415)(1117 419)(1116 409)(1117 419)
\cpath (1114 422)(1106 430)(1097 438)(1088 446)(1077 455)
\cpath (1066 464)(1053 473)(1039 483)(1039 483)(1054 480)
\cpath (1039 483)(1042 473)(1039 483)(1024 493)(1007 503)
\cpath (989 514)(968 526)(947 538)(947 538)(963 537)
\cpath (947 538)(952 529)(947 538)(926 550)(905 561)
\cpath (883 572)(860 583)(860 583)(876 583)(860 583)
\cpath (866 574)(860 583)(837 595)(810 607)(790 616)
\cpath (772 625)(753 633)(753 633)(769 633)(753 633)
\cpath (759 624)(753 633)(737 640)(721 647)(707 653)
\cpath (694 658)(681 663)(670 668)(658 673)(658 673)
\cpath (674 673)(658 673)(665 664)(658 673)(648 677)
\cpath (638 681)(628 685)(619 688)(610 691)(602 694)
\cpath (594 697)(586 700)(579 703)(572 705)(565 707)
\cpath (565 707)(580 708)(565 707)(572 699)(565 707)
\cpath (559 709)(552 712)(546 713)(541 715)(535 717)
\cpath (530 718)(525 720)(520 721)(516 723)(511 724)
\cpath (507 725)(503 726)(499 727)(495 728)(492 729)
\cpath (488 730)(485 730)(482 731)(479 732)(476 732)
\cpath (473 733)(470 733)(468 734)(465 734)(465 734)
\cpath (479 738)(465 734)(476 727)(465 734)(463 734)
\cpath (460 734)(458 735)(456 735)(454 735)(452 735)
\cpath (450 735)(448 735)(446 735)(445 735)(443 735)
\cpath (441 735)(440 735)(438 735)(437 735)(436 735)
\cpath (434 734)(433 734)(432 734)(431 734)(430 734)
\cpath (429 733)(428 733)(427 733)(426 732)(425 732)
\cpath (424 731)(423 731)(422 731)(421 730)(421 730)
\cpath (420 729)(419 729)(418 728)(418 728)(417 727)
\cpath (417 727)(416 726)(415 726)(415 725)(414 725)
\cpath (414 724)(413 724)(413 723)(412 723)(412 722)
\cpath (412 721)(411 721)(411 720)(410 719)(410 719)
\cpath (410 718)(409 718)(409 717)(409 716)(409 716)
\cpath (408 715)(408 714)(408 714)(407 713)(407 712)
\cpath (407 712)(407 711)(407 710)(406 710)(406 709)
\cpath (406 708)(406 707)(406 707)(405 706)(405 705)
\cpath (405 705)(405 704)(405 703)(405 702)(405 702)
\cpath (404 701)(404 700)(404 699)(404 699)(404 698)
\cpath (404 697)(404 696)(404 696)(404 695)(403 694)
\cpath (403 693)(403 693)(403 692)(403 691)(403 690)
\cpath (403 690)(403 689)(403 688)(403 687)(403 686)
\cpath (403 686)(403 685)(402 684)(402 683)(402 682)
\cpath (402 682)(402 681)(402 680)(402 679)(402 679)
\cpath (402 678)(402 677)(402 676)(402 675)(402 675)
\cpath (402 674)(402 673)(402 672)(402 671)(402 670)
\cpath (402 670)(402 669)(402 668)(402 667)(402 666)
\cpath (402 666)(402 665)(402 664)(402 663)(402 662)
\cpath (401 662)(401 661)(401 660)(401 659)(401 658)
\cpath (401 657)(401 657)(401 656)(401 655)(401 654)
\cpath (401 653)(401 652)(401 652)(401 651)(401 650)
\cpath (401 649)(401 648)(401 647)(401 647)(401 646)
\cpath (401 645)(401 644)(401 643)(401 642)(401 642)
\cpath (401 641)(401 640)(401 639)(401 638)(401 637)
\cpath (401 636)(401 636)(401 635)(401 634)(401 633)
\cpath (401 632)(401 631)(401 630)(401 630)(401 629)
\cpath (401 628)(401 627)(401 626)(401 625)(401 624)
\cpath (401 624)(401 623)(401 622)(401 621)(401 620)
\cpath (401 619)(401 618)(401 617)(401 617)(401 616)
\cpath (401 615)(401 614)(401 613)(401 612)(401 611)
\cpath (401 610)(401 609)(401 609)(401 608)(401 607)
\cpath (401 606)(401 605)(401 604)(401 603)(401 602)
\cpath (401 601)(401 601)(401 600)(401 599)(401 598)
\cpath (401 597)(401 596)(401 595)(401 594)(401 593)
\cpath (401 592)(401 591)(401 591)(401 590)(401 589)
\cpath (401 588)(401 587)(401 586)(401 585)(401 584)
\cpath (401 583)(401 582)(401 581)(401 580)(401 579)
\cpath (401 579)(401 578)(401 577)(401 576)(401 575)
\cpath (401 574)(401 573)(401 572)(401 571)(401 570)
\cpath (401 569)(401 568)(401 567)(401 566)(401 565)
\cpath (401 564)(401 563)(401 562)(401 561)(401 560)
\cpath (401 560)(401 559)(401 558)(401 557)(401 556)
\cpath (401 555)(401 554)(401 553)(401 552)(401 551)
\cpath (401 550)(401 549)(401 548)(401 547)(401 546)
\cpath (401 545)(401 544)(401 543)(401 542)(401 541)
\cpath (401 540)(401 539)(401 538)(401 537)(401 536)
\cpath (401 535)(401 534)(401 532)(401 531)(401 530)
\cpath (401 529)(401 528)(401 527)(401 526)(401 525)
\cpath (401 524)(401 523)(401 522)(401 521)(401 520)
\cpath (401 519)(401 518)(401 517)(401 516)(401 514)
\cpath (401 513)(401 512)(401 511)(401 510)(401 509)
\cpath (401 508)(401 507)(401 506)(401 504)(401 503)
\cpath (401 502)(401 501)(401 500)(401 499)(401 498)
\cpath (401 496)(401 495)(401 494)(401 493)(401 492)
\cpath (401 491)(401 489)(401 488)(401 487)(401 486)
\cpath (401 484)(401 483)(401 482)(401 481)(401 480)
\cpath (401 478)(401 477)(401 476)(401 474)(401 473)
\cpath (401 472)(401 471)(401 469)(401 468)(401 467)
\cpath (401 465)(401 464)(401 463)(401 461)(401 460)
\cpath (401 458)(401 457)(401 455)(401 454)(401 453)
\cpath (401 451)(401 450)(401 448)(401 447)(401 445)
\cpath (401 443)(401 442)(401 440)(401 439)(401 437)
\cpath (401 435)(401 434)(401 432)(401 430)(401 428)
\cpath (401 426)(401 424)(401 422)(401 420)(401 418)
\cpath (401 416)(401 413)(401 410)(401 407)(401 404)
\cpath (401 402)(401 400)(401 397)(401 395)(401 393)
\cpath (401 391)(401 389)(401 387)(401 385)(401 383)
\cpath (401 381)(401 379)(401 377)(401 375)(401 373)
\cpath (401 371)(401 369)(401 367)(401 365)(401 364)
\cpath (401 362)(401 360)(401 358)(401 356)(401 355)
\cpath (401 353)(401 351)(401 349)(401 348)(401 346)
\cpath (401 344)(401 342)(401 341)(401 339)(401 337)
\cpath (401 336)(401 334)(401 333)(401 331)(401 330)
\cpath (401 328)(401 327)(401 325)(401 324)(401 322)
\cpath (401 321)(401 320)(401 319)(401 317)(401 316)
\cpath (401 315)(401 314)(401 313)(401 312)(401 311)
\cpath (401 310)(401 309)(401 308)(401 307)(401 306)
\cpath (401 305)(401 304)(401 303)(401 303)(401 302)
\cpath (401 301)
\path (1119 416)(1119 416)(1126 409)(1131 401)(1135 395)(1139 388)
\cpath (1142 381)(1144 375)(1145 369)(1145 364)(1145 358)
\cpath (1144 353)(1144 353)(1150 344)(1144 353)(1134 346)
\cpath (1144 353)(1143 348)(1141 343)(1138 338)(1135 333)
\cpath (1131 329)(1126 325)(1121 320)(1116 316)(1110 313)
\cpath (1103 309)(1096 305)(1089 302)(1081 298)(1081 298)
\cpath (1074 289)(1081 298)(1065 299)(1081 298)(1072 295)
\cpath (1063 292)(1053 289)(1044 286)(1033 283)(1023 281)
\cpath (1012 278)(1001 276)(989 273)(977 271)(965 269)
\cpath (953 267)(941 265)(928 263)(915 261)(902 259)
\cpath (889 258)(875 256)(862 255)(848 253)(834 252)
\cpath (820 250)(807 249)(795 248)(782 247)(770 246)
\cpath (758 245)(746 244)(735 244)(723 243)(712 242)
\cpath (701 242)(690 241)(680 240)(669 240)(659 239)
\cpath (649 239)(639 238)(630 238)(621 237)(612 237)
\cpath (612 237)(600 231)(612 237)(598 242)(612 237)
\cpath (604 236)(596 236)(588 236)(581 236)(574 235)
\cpath (567 235)(561 235)(555 235)(549 234)(543 234)
\cpath (538 234)(533 234)(528 234)(523 234)(518 233)
\cpath (514 233)(510 233)(505 233)(502 233)(502 233)
\cpath (489 227)(502 233)(488 238)(502 233)(498 233)
\cpath (494 233)(491 233)(487 233)(484 233)(481 232)
\cpath (478 232)(475 232)(472 232)(470 232)(467 232)
\cpath (465 232)(462 232)(460 232)(458 232)(456 232)
\cpath (453 232)(452 232)(450 232)(448 232)(446 232)
\cpath (444 232)(443 232)(441 232)(440 232)(438 232)
\cpath (437 232)(436 232)(434 232)(433 232)(432 232)
\cpath (431 232)(430 232)(428 232)(427 232)(426 232)
\cpath (426 232)(425 232)(424 232)(423 232)(422 233)
\cpath (421 233)(421 233)(420 233)(419 233)(418 233)
\cpath (418 233)(417 233)(417 233)(416 233)(415 233)
\cpath (415 233)(414 233)(414 233)(413 233)(413 233)
\cpath (412 233)(412 233)(412 233)(411 234)(411 234)
\cpath (410 234)(410 234)(410 234)(409 234)(409 234)
\cpath (409 234)(409 234)(408 234)(408 234)(408 234)
\cpath (407 234)(407 234)(407 234)(407 235)(407 235)
\cpath (406 235)(406 235)(406 235)(406 235)(406 235)
\cpath (405 235)(405 235)(405 235)(405 235)(405 235)
\cpath (405 235)(405 236)(404 236)(404 236)(404 236)
\cpath (404 236)(404 236)(404 236)(404 236)(404 236)
\cpath (404 236)(403 236)(403 237)(403 237)(403 237)
\cpath (403 237)(403 237)(403 237)(403 237)(403 237)
\cpath (403 237)(403 237)(403 237)(403 238)(402 238)
\cpath (402 238)(402 238)(402 238)(402 238)(402 238)
\cpath (402 238)(402 238)(402 238)(402 239)(402 239)
\cpath (402 239)(402 239)(402 239)(402 239)(402 239)
\cpath (402 239)(402 239)(402 240)(402 240)(402 240)
\cpath (402 240)(402 240)(402 240)(402 240)(402 240)
\cpath (402 240)(402 241)(401 241)(401 241)(401 241)
\cpath (401 241)(401 241)(401 241)(401 241)(401 241)
\cpath (401 242)(401 242)(401 242)(401 242)(401 242)
\cpath (401 242)(401 242)(401 242)(401 243)(401 243)
\cpath (401 243)(401 243)(401 243)(401 243)(401 243)
\cpath (401 243)(401 244)(401 244)(401 244)(401 244)
\cpath (401 244)(401 244)(401 244)(401 245)(401 245)
\cpath (401 245)(401 245)(401 245)(401 245)(401 245)
\cpath (401 246)(401 246)(401 246)(401 246)(401 246)
\cpath (401 246)(401 246)(401 247)(401 247)(401 247)
\cpath (401 247)(401 247)(401 247)(401 248)(401 248)
\cpath (401 248)(401 248)(401 248)(401 248)(401 248)
\cpath (401 249)(401 249)(401 249)(401 249)(401 249)
\cpath (401 249)(401 250)(401 250)(401 250)(401 250)
\cpath (401 250)(401 251)(401 251)(401 251)(401 251)
\cpath (401 251)(401 251)(401 252)(401 252)(401 252)
\cpath (401 252)(401 252)(401 253)(401 253)(401 253)
\cpath (401 253)(401 253)(401 254)(401 254)(401 254)
\cpath (401 254)(401 254)(401 255)(401 255)(401 255)
\cpath (401 255)(401 255)(401 256)(401 256)(401 256)
\cpath (401 256)(401 256)(401 257)(401 257)(401 257)
\cpath (401 257)(401 258)(401 258)(401 258)(401 258)
\cpath (401 258)(401 259)(401 259)(401 259)(401 259)
\cpath (401 260)(401 260)(401 260)(401 260)(401 261)
\cpath (401 261)(401 261)(401 261)(401 262)(401 262)
\cpath (401 262)(401 262)(401 263)(401 263)(401 263)
\cpath (401 263)(401 264)(401 264)(401 264)(401 265)
\cpath (401 265)(401 265)(401 265)(401 266)(401 266)
\cpath (401 266)(401 267)(401 267)(401 267)(401 268)
\cpath (401 268)(401 268)(401 268)(401 269)(401 269)
\cpath (401 269)(401 270)(401 270)(401 270)(401 271)
\cpath (401 271)(401 271)(401 272)(401 272)(401 273)
\cpath (401 273)(401 273)(401 274)(401 274)(401 274)
\cpath (401 275)(401 275)(401 275)(401 276)(401 276)
\cpath (401 277)(401 277)(401 277)(401 278)(401 278)
\cpath (401 279)(401 279)(401 280)(401 280)(401 280)
\cpath (401 281)(401 281)(401 282)(401 282)(401 283)
\cpath (401 283)(401 284)(401 284)(401 285)(401 285)
\cpath (401 286)(401 286)(401 287)(401 287)(401 288)
\cpath (401 288)(401 289)(401 289)(401 290)
\path (1228 433)(1228 433)(1243 429)(1228 433)(1228 423)(1228 433)
\cpath (1224 437)(1214 445)(1204 454)(1192 463)(1179 473)
\cpath (1165 482)(1150 493)(1150 493)(1165 491)(1150 493)
\cpath (1153 483)(1150 493)(1133 503)(1115 515)(1095 526)
\cpath (1073 539)(1048 552)(1048 552)(1064 551)(1048 552)
\cpath (1053 543)(1048 552)(1021 567)(995 580)(968 594)
\cpath (941 607)(941 607)(956 606)(941 607)(946 598)
\cpath (941 607)(911 621)(879 636)(856 646)(832 657)
\cpath (832 657)(847 657)(832 657)(838 648)(832 657)
\cpath (811 666)(792 674)(774 682)(757 689)(742 695)
\cpath (742 695)(757 695)(742 695)(748 686)(742 695)
\cpath (727 701)(714 706)(701 712)(688 716)(676 721)
\cpath (665 725)(654 729)(644 733)(644 733)(660 734)
\cpath (644 733)(651 724)(644 733)(635 737)(625 740)
\cpath (616 743)(608 747)(600 749)(592 752)(584 755)
\cpath (577 757)(570 759)(564 761)(557 763)(551 765)
\cpath (545 767)(545 767)(561 769)(545 767)(554 759)
\cpath (545 767)(540 769)(535 770)(529 772)(524 773)
\cpath (520 775)(515 776)(511 777)(507 778)(503 779)
\cpath (499 780)(495 781)(491 782)(488 783)(485 783)
\cpath (482 784)(478 784)(476 785)(473 785)(470 786)
\cpath (467 786)(465 787)(462 787)(460 787)(458 787)
\cpath (456 788)(454 788)(452 788)(450 788)(448 788)
\cpath (446 788)(445 788)(443 788)(443 788)(455 794)
\cpath (443 788)(456 783)(443 788)(441 788)(440 788)
\cpath (438 788)(437 788)(436 787)(434 787)(433 787)
\cpath (432 787)(431 787)(430 786)(428 786)(427 786)
\cpath (426 785)(426 785)(425 785)(424 784)(423 784)
\cpath (422 784)(421 783)(421 783)(420 782)(419 782)
\cpath (418 781)(418 781)(417 780)(417 780)(416 779)
\cpath (415 779)(415 778)(414 778)(414 777)(413 777)
\cpath (413 776)(412 776)(412 775)(412 774)(411 774)
\cpath (411 773)(410 773)(410 772)(410 772)(409 771)
\cpath (409 770)(409 770)(409 769)(408 768)(408 768)
\cpath (408 767)(407 766)(407 766)(407 765)(407 764)
\cpath (407 764)(406 763)(406 762)(406 762)(406 761)
\cpath (406 760)(405 760)(405 759)(405 758)(405 758)
\cpath (405 757)(405 756)(405 756)(404 755)(404 754)
\cpath (404 753)(404 753)(404 752)(404 751)(404 751)
\cpath (404 750)(404 749)(403 748)(403 748)(403 747)
\cpath (403 746)(403 745)(403 745)(403 744)(403 743)
\cpath (403 742)(403 742)(403 741)(403 740)(403 739)
\cpath (402 739)(402 738)(402 737)(402 736)(402 736)
\cpath (402 735)(402 734)(402 733)(402 733)(402 732)
\cpath (402 731)(402 730)(402 730)(402 729)(402 728)
\cpath (402 727)(402 727)(402 726)(402 725)(402 724)
\cpath (402 724)(402 723)(402 722)(402 721)(402 720)
\cpath (402 720)(402 719)(402 718)(401 717)(401 717)
\cpath (401 716)(401 715)(401 714)(401 713)(401 713)
\cpath (401 712)(401 711)(401 710)(401 709)(401 709)
\cpath (401 708)(401 707)(401 706)(401 706)(401 705)
\cpath (401 704)(401 703)(401 702)(401 702)(401 701)
\cpath (401 700)(401 699)(401 698)(401 698)(401 697)
\cpath (401 696)(401 695)(401 694)(401 694)(401 693)
\cpath (401 692)(401 691)(401 690)(401 690)(401 689)
\cpath (401 688)(401 687)(401 686)(401 686)(401 685)
\cpath (401 684)(401 683)(401 682)(401 681)(401 681)
\cpath (401 680)(401 679)(401 678)(401 677)(401 677)
\cpath (401 676)(401 675)(401 674)(401 673)(401 673)
\cpath (401 672)(401 671)(401 670)(401 669)(401 668)
\cpath (401 668)(401 667)(401 666)(401 665)(401 664)
\cpath (401 663)(401 663)(401 662)(401 661)(401 660)
\cpath (401 659)(401 658)(401 658)(401 657)(401 656)
\cpath (401 655)(401 654)(401 653)(401 653)(401 652)
\cpath (401 651)(401 650)(401 649)(401 648)(401 648)
\cpath (401 647)(401 646)(401 645)(401 644)(401 643)
\cpath (401 642)(401 642)(401 641)(401 640)(401 639)
\cpath (401 638)(401 637)(401 637)(401 636)(401 635)
\cpath (401 634)(401 633)(401 632)(401 631)(401 631)
\cpath (401 630)(401 629)(401 628)(401 627)(401 626)
\cpath (401 625)(401 624)(401 624)(401 623)(401 622)
\cpath (401 621)(401 620)(401 619)(401 618)(401 617)
\cpath (401 617)(401 616)(401 615)(401 614)(401 613)
\cpath (401 612)(401 611)(401 610)(401 610)(401 609)
\cpath (401 608)(401 607)(401 606)(401 605)(401 604)
\cpath (401 603)(401 602)(401 601)(401 601)(401 600)
\cpath (401 599)(401 598)(401 597)(401 596)(401 595)
\cpath (401 594)(401 593)(401 592)(401 592)(401 591)
\cpath (401 590)(401 589)(401 588)(401 587)(401 586)
\cpath (401 585)(401 584)(401 583)(401 582)(401 581)
\cpath (401 580)(401 580)(401 579)(401 578)(401 577)
\cpath (401 576)(401 575)(401 574)(401 573)(401 572)
\cpath (401 571)(401 570)(401 569)(401 568)(401 567)
\cpath (401 566)(401 565)(401 564)(401 563)(401 562)
\cpath (401 562)(401 561)(401 560)(401 559)(401 558)
\cpath (401 557)(401 556)(401 555)(401 554)(401 553)
\cpath (401 552)(401 551)(401 550)(401 549)(401 548)
\cpath (401 547)(401 546)(401 545)(401 544)(401 543)
\cpath (401 542)(401 541)(401 540)(401 539)(401 538)
\cpath (401 537)(401 536)(401 535)(401 534)(401 533)
\cpath (401 531)(401 530)(401 529)(401 528)(401 527)
\cpath (401 526)(401 525)(401 524)(401 523)(401 522)
\cpath (401 521)(401 520)(401 519)(401 518)(401 517)
\cpath (401 516)(401 514)(401 513)(401 512)(401 511)
\cpath (401 510)(401 509)(401 508)(401 507)(401 506)
\cpath (401 504)(401 503)(401 502)(401 501)(401 500)
\cpath (401 499)(401 498)(401 496)(401 495)(401 494)
\cpath (401 493)(401 492)(401 491)(401 489)(401 488)
\cpath (401 487)(401 486)(401 485)(401 483)(401 482)
\cpath (401 481)(401 480)(401 478)(401 477)(401 476)
\cpath (401 475)(401 473)(401 472)(401 471)(401 469)
\cpath (401 468)(401 467)(401 465)(401 464)(401 463)
\cpath (401 461)(401 460)(401 458)(401 457)(401 456)
\cpath (401 454)(401 453)(401 451)(401 450)(401 448)
\cpath (401 447)(401 445)(401 444)(401 442)(401 440)
\cpath (401 439)(401 437)(401 435)(401 434)(401 432)
\cpath (401 430)(401 428)(401 426)(401 424)(401 422)
\cpath (401 420)(401 418)(401 416)(401 413)(401 410)
\cpath (401 407)(401 405)(401 402)(401 400)(401 398)
\cpath (401 395)(401 393)(401 391)(401 389)(401 387)
\cpath (401 385)(401 383)(401 381)(401 379)(401 377)
\cpath (401 375)(401 373)(401 371)(401 369)(401 367)
\cpath (401 366)(401 364)(401 362)(401 360)(401 358)
\cpath (401 356)(401 355)(401 353)(401 351)(401 349)
\cpath (401 348)(401 346)(401 344)(401 343)(401 341)
\cpath (401 339)(401 338)(401 336)(401 334)(401 333)
\cpath (401 331)(401 330)(401 328)(401 327)(401 325)
\cpath (401 324)(401 323)(401 321)(401 320)(401 319)
\cpath (401 317)(401 316)(401 315)(401 314)(401 313)
\cpath (401 312)(401 311)(401 310)(401 309)(401 308)
\path (1232 429)(1232 429)(1240 421)(1247 413)(1253 406)(1258 398)
\cpath (1262 391)(1265 385)(1267 378)(1269 372)(1270 366)
\cpath (1270 366)(1279 358)(1270 366)(1262 358)(1270 366)
\cpath (1270 360)(1269 355)(1268 350)(1266 345)(1264 340)
\cpath (1261 335)(1257 330)(1253 326)(1248 322)(1243 318)
\cpath (1237 314)(1230 310)(1223 306)(1223 306)(1218 297)
\cpath (1223 306)(1208 306)(1223 306)(1216 303)(1208 299)
\cpath (1200 296)(1191 293)(1182 290)(1173 287)(1163 284)
\cpath (1152 281)(1142 279)(1130 276)(1119 274)(1119 274)
\cpath (1109 266)(1119 274)(1104 277)(1119 274)(1107 271)
\cpath (1094 269)(1081 267)(1068 265)(1054 263)(1040 261)
\cpath (1026 259)(1012 257)(997 255)(982 254)(967 252)
\cpath (952 251)(936 249)(921 248)(905 247)(889 245)
\cpath (873 244)(858 243)(843 242)(829 241)(815 240)
\cpath (801 239)(788 239)(775 238)(763 237)(750 237)
\cpath (750 237)(738 231)(750 237)(737 242)(750 237)
\cpath (738 236)(726 236)(714 235)(703 235)(691 234)
\cpath (680 234)(669 233)(659 233)(649 232)(639 232)
\cpath (630 232)(621 231)(612 231)(604 231)(596 231)
\cpath (589 231)(582 230)(575 230)(568 230)(562 230)
\cpath (556 230)(550 229)(544 229)(539 229)(534 229)
\cpath (529 229)(524 229)(519 229)(515 229)(510 229)
\cpath (506 229)(502 228)(499 228)(495 228)(491 228)
\cpath (488 228)(485 228)(482 228)(479 228)(476 228)
\cpath (473 228)(473 228)(460 222)(473 228)(460 233)
\cpath (473 228)(470 228)(468 228)(465 228)(463 228)
\cpath (460 228)(458 228)(456 228)(454 228)(452 228)
\cpath (450 228)(448 228)(447 228)(445 228)(443 228)
\cpath (442 228)(440 228)(439 228)(437 228)(436 228)
\cpath (435 228)(433 228)(432 228)(431 228)(430 228)
\cpath (429 228)(428 228)(427 228)(426 228)(425 228)
\cpath (424 228)(423 228)(422 228)(421 228)(421 228)
\cpath (420 228)(419 228)(419 228)(418 228)(417 228)
\cpath (417 228)(416 228)(416 228)(415 228)(415 229)
\cpath (414 229)(414 229)(413 229)(413 229)(412 229)
\cpath (412 229)(411 229)(411 229)(411 229)(410 229)
\cpath (410 229)(410 229)(409 229)(409 229)(409 229)
\cpath (408 229)(408 229)(408 229)(408 229)(407 229)
\cpath (407 229)(407 230)(407 230)(406 230)(406 230)
\cpath (406 230)(406 230)(406 230)(406 230)(405 230)
\cpath (405 230)(405 230)(405 230)(405 230)(405 230)
\cpath (404 230)(404 230)(404 230)(404 231)(404 231)
\cpath (404 231)(404 231)(404 231)(404 231)(403 231)
\cpath (403 231)(403 231)(403 231)(403 231)(403 231)
\cpath (403 231)(403 231)(403 231)(403 231)(403 232)
\cpath (403 232)(403 232)(402 232)(402 232)(402 232)
\cpath (402 232)(402 232)(402 232)(402 232)(402 232)
\cpath (402 232)(402 232)(402 232)(402 233)(402 233)
\cpath (402 233)(402 233)(402 233)(402 233)(402 233)
\cpath (402 233)(402 233)(402 233)(402 233)(402 233)
\cpath (402 233)(402 234)(402 234)(402 234)(402 234)
\cpath (401 234)(401 234)(401 234)(401 234)(401 234)
\cpath (401 234)(401 234)(401 234)(401 235)(401 235)
\cpath (401 235)(401 235)(401 235)(401 235)(401 235)
\cpath (401 235)(401 235)(401 235)(401 235)(401 235)
\cpath (401 236)(401 236)(401 236)(401 236)(401 236)
\cpath (401 236)(401 236)(401 236)(401 236)(401 236)
\cpath (401 236)(401 237)(401 237)(401 237)(401 237)
\cpath (401 237)(401 237)(401 237)(401 237)(401 237)
\cpath (401 237)(401 238)(401 238)(401 238)(401 238)
\cpath (401 238)(401 238)(401 238)(401 238)(401 238)
\cpath (401 238)(401 239)(401 239)(401 239)(401 239)
\cpath (401 239)(401 239)(401 239)(401 239)(401 239)
\cpath (401 240)(401 240)(401 240)(401 240)(401 240)
\cpath (401 240)(401 240)(401 240)(401 240)(401 241)
\cpath (401 241)(401 241)(401 241)(401 241)(401 241)
\cpath (401 241)(401 241)(401 242)(401 242)(401 242)
\cpath (401 242)(401 242)(401 242)(401 242)(401 242)
\cpath (401 243)(401 243)(401 243)(401 243)(401 243)
\cpath (401 243)(401 243)(401 243)(401 244)(401 244)
\cpath (401 244)(401 244)(401 244)(401 244)(401 244)
\cpath (401 245)(401 245)(401 245)(401 245)(401 245)
\cpath (401 245)(401 245)(401 246)(401 246)(401 246)
\cpath (401 246)(401 246)(401 246)(401 246)(401 247)
\cpath (401 247)(401 247)(401 247)(401 247)(401 247)
\cpath (401 247)(401 248)(401 248)(401 248)(401 248)
\cpath (401 248)(401 248)(401 249)(401 249)(401 249)
\cpath (401 249)(401 249)(401 249)(401 250)(401 250)
\cpath (401 250)(401 250)(401 250)(401 250)(401 251)
\cpath (401 251)(401 251)(401 251)(401 251)(401 252)
\cpath (401 252)(401 252)(401 252)(401 252)(401 253)
\cpath (401 253)(401 253)(401 253)(401 253)(401 253)
\cpath (401 254)(401 254)(401 254)(401 254)(401 254)
\cpath (401 255)(401 255)(401 255)(401 255)(401 256)
\cpath (401 256)(401 256)(401 256)(401 256)(401 257)
\cpath (401 257)(401 257)(401 257)(401 257)(401 258)
\cpath (401 258)(401 258)(401 258)(401 259)(401 259)
\cpath (401 259)(401 259)(401 260)(401 260)(401 260)
\cpath (401 260)(401 261)(401 261)(401 261)(401 261)
\cpath (401 262)(401 262)(401 262)(401 262)(401 263)
\cpath (401 263)(401 263)(401 263)(401 264)(401 264)
\cpath (401 264)(401 265)(401 265)(401 265)(401 265)
\cpath (401 266)(401 266)(401 266)(401 267)(401 267)
\cpath (401 267)(401 267)(401 268)(401 268)(401 268)
\cpath (401 269)(401 269)(401 269)(401 270)(401 270)
\cpath (401 270)(401 271)(401 271)(401 271)
\end{texdraw}

 \vspace{1cm}
\end{center}
}
{
\caption{\protect\capsize 
Faseportr{\ae}t for Volterra-Lotka lig\-ningerne for
f{\o}lgende valg af de indg{\aa}ende
hastigheds\-konstanter: $k_1$ = 10, $k_2$ = 1, $k_3$ = 1.
Samtlige l{\o}sningskurver beskriver en lukket
(oscillerende) bane omkring ligev{\ae}gstpunktet.
\label{fig:LotkaVolterraPlot}}
}

Divideres $\frac{dA}{dt}$ med $\frac{dB}{dt}$ i
lig\-ning~\ref{eq:LotkaRate} f{\aa}s 

\begin{equation}
  \frac{dA}{dB} = \frac{k_1 A - k_2 A B}{k_2 A B - k_3 B}
\end{equation}

der ved separation af variablene $A$ og $B$ integreres til

\begin{equation}
  F(A,B) = k_1 \ln B - k_2 B + k_3 \ln A - k_2 A = K
  \label{eq:MorseFunction}
\end{equation}

hvor $K$ er en reel konstant. Man kan vise, at $F(A,B)$ er
en s{\aa}kaldt Morse-funktion, hvoraf man kan slutte, at
l{\o}sningskurverne til lig\-ning~\ref{eq:MorseFunction}
vil v{\ae}re lukkede, svarende til at samtlige l{\o}sninger
til det oprindelige
differentiallig\-ningssystem~\ref{eq:LotkaRate} m{\aa}
v{\ae}re periodiske (for et bevis for Morses lemma samt en
anvendelse af dette p{\aa} Lotkas lig\-ninger, se
\cite{Verhulst}).

\vspace{4.0mm}
Lotkas kemiske model\-system beskriver en situation, hvor
en reaktion, via et reservoir af reaktanter, holdes v{\ae}k
fra ligev{\ae}gt resulterende i en periodisk bev{\ae}gelse
omkring ligev{\ae}gts\-punktet~\ref{eq:LotkaEqui}. Dette
r{\ae}sonnement forekommer ikke i Lotkas artikel, hvor han
ligeledes forkaster modellens fysiske og kemiske relevans,
idet han i stedet henviser til dennes anvendelighed
indenfor biologiske systemer\footnote{I 1925 publicerer
Lotka en artikel i hvilken
differentialligningerne~\protect\ref{eq:LotkaRate} benyttes
til at beskrive den indbyrdes kamp mellem bytte og rovdyr i
et {\o}kosystem \cite{LotkaBio}. Omtrent samtidig (1926)
beskrives dette system p{\aa} n{\ae}sten identisk vis af
den italienske matematiker \mbox{V.\ Volterra}
\cite{Volterra}. ``Lotka-Volterra'' modellen regnes for den
f{\o}rste s{\aa}kaldte ``predator-prey'' model. S{\aa}danne
modeller anvendes idag hyppigt til at beskrive biologiske
populationer.}, hvor oscillerende f{\ae}nomener p{\aa}
dav{\ae}rende tidspunkt var yderst vel\-kendte. Systemets
opf{\o}rsel i $(A,B)$-planet er illustreret i
figur~\ref{fig:LotkaVolterraPlot}.

\vspace{4.0mm}
Lotkas afstandtagen fra sin egen models kemiske
anvendelighed skal nok f{\o}rst og fremmest ses i et
naturvidenskabshistorisk lys. P{\aa} Lotkas tid forstod man
til fulde at beskrive kemisk ligev{\ae}gt, hvorimod teorien
for ikke-ligev{\ae}gts\-systemer var et fuldst{\ae}ndigt
ubeskrevet blad. Termodynamikkens 2.~hoved\-s{\ae}tning
fremstod i sin endelige form omkring 1847-51. Som bekendt
postulerer denne, at ethvert lukket fysisk system vil
bev{\ae}ge sig mod termodynamisk ligev{\ae}gt. Dette
bidrager ogs{\aa} med en forst{\aa}else for, hvorfor Lotka
v{\ae}grede sig ved at formulere en kemisk
reaktions\-mekanisme, der, set i et historisk perspektiv,
umiddelbart var i klar modstrid med termodynamikkens 2.\
hoveds{\ae}tning.

\vspace{4.0mm}
Til sidst b{\o}r det nok ogs{\aa} i al almindelighed
bem{\ae}rkes, at man n{\ae}ppe kan fritage forskere en vis
form for underdanighed og respekt i forhold til
autoriteterne indenfor en given videnskabelig disciplin.
N{\aa}r en s{\aa} respekteret og ber{\o}mmet person som
Nernst, der tilmed modtog Nobelprisen i 1920 siger, at en
kemisk reaktion ikke kan oscillere; ja, s{\aa} kan den den
simpelthen bare ikke oscillere.

\section{Kemisk kinetik idag}
\label{ModernKinetics}
Der skulle g{\aa} n{\ae}sten 50 {\aa}r f{\o}r man
inds{\aa}, at Volterra-Lotka lig\-ningerne faktisk ikke
strider mod termodynamikkens 2.\ hoveds{\ae}tning, idet
Volterra-Lotka lig\-ningerne jo {\em ikke\/} beskriver et
lukket system. Tilstedev{\ae}relsen af et uendeligt
reservoir af reaktanter sikrer netop, at systemet altid er
langt fra ligev{\ae}gt. Sammenligner man de egen\-skaber,
man idag finder i komplekse kemiske reaktioner, viser det
sig, at en stor del af disse i grove tr{\ae}k kan
genfindes i Volterra-Lotka lig\-ningerne. Vi kan nemlig
betragte den kemiske reaktion, der ligger til grund for
disse som findende sted i en {\aa}ben reaktor, hvortil der
tilf{\o}res en j{\ae}vn str{\o}m af reaktanter, samtidig
med at overskydende produkter pumpes ud.

\vspace{4.0mm}
Fortolkes Lotkas model s{\aa}ledes ses en oplagt analogi
til den eksperimentelle ops{\ae}tning, der kaldes for en
{\em CSTR-ops{\ae}tning\/} (Continously Flow Stirred Tank
Reactor). Skematisk best{\aa}r en CSTR-ops{\ae}tning af en
reaktions\-celle med en omr{\o}ringspropel, der sikrer
homogenitet af opl{\o}sningen. Reaktionscellen er
yder\-ligere udstyret med haner, gennem hvilke reaktanter
og produkter henholdsvis pum\-pes ind og ud.
CSTR-opstillingen har vist sig yderst anvendelig til at
studere ke\-miske reaktioner, idet denne netop udg{\o}r et
{\aa}bent system. Den studerede reaktion kan derfor i et
vilk{\aa}rligt tidsrum holdes langt fra ``ligev{\ae}gt'',
hvorved en ellers transient opf{\o}rsel kan bevares.
CSTR-opstillingen er skitseret i figur~\ref{fig:CSTRFigur}.

%%%%%%%%%%%%%%%%%%%%%%%%%%%%%%%%%%%%%%%%%%%%%%%%%%%%%%%%%%%%%%%%%%%%%%%%
%%
%% skematisk CSTR-opstilling
%%
%% fig43.tex
%%
%%%%%%%%%%%%%%%%%%%%%%%%%%%%%%%%%%%%%%%%%%%%%%%%%%%%%%%%%%%%%%%%%%%%%%%%
\boxfigure{t}{\textwidth}
{
 \vspace{3.0mm}
 \unitlength=0.50mm
\linethickness{0.8pt}
\begin{picture}(160.00,106.00)(-32.0,0)
\put(70.00,5.00){\line(1,0){70.00}}
\put(140.00,5.00){\line(0,1){30.00}}
\put(140.00,35.00){\line(-1,0){15.00}}
\put(125.00,35.00){\line(0,-1){20.00}}
\put(125.00,15.00){\line(-1,0){40.00}}
\put(85.00,15.00){\line(0,1){20.00}}
\put(85.00,35.00){\line(-1,0){15.00}}
\put(70.00,35.00){\line(0,-1){30.00}}
\put(125.00,37.00){\line(1,0){15.00}}
\put(140.00,37.00){\line(0,1){3.00}}
\put(140.00,40.00){\line(-1,0){15.00}}
\put(125.00,40.00){\line(0,-1){3.00}}
\put(85.00,37.00){\line(-1,0){15.00}}
\put(70.00,37.00){\line(0,1){3.00}}
\put(70.00,40.00){\line(1,0){15.00}}
\put(85.00,40.00){\line(0,-1){3.00}}
\put(125.00,40.00){\line(-6,5){17.00}}
\put(108.00,54.00){\line(0,1){6.00}}
\put(108.00,60.00){\line(1,0){32.00}}
\put(140.00,60.00){\line(0,-1){20.00}}
\put(140.00,62.00){\line(-1,0){32.00}}
\put(108.00,62.00){\line(0,1){3.00}}
\put(108.00,65.00){\line(1,0){32.00}}
\put(140.00,65.00){\line(0,-1){3.00}}
\put(85.00,40.00){\line(6,5){17.00}}
\put(102.00,54.00){\line(0,1){8.00}}
\put(102.00,62.00){\line(-1,0){32.00}}
\put(70.00,62.00){\line(0,-1){22.00}}
\put(70.00,63.00){\line(0,1){2.00}}
\put(70.00,65.00){\line(1,0){32.00}}
\put(102.00,65.00){\line(0,-1){2.00}}
\put(70.00,63.00){\line(1,0){32.00}}
\put(104.00,37.00){\line(0,1){69.00}}
\put(104.00,106.00){\line(1,0){2.00}}
\put(106.00,106.00){\line(0,-1){69.00}}
\put(90.00,65.00){\line(0,1){25.00}}
\put(90.00,90.00){\line(1,0){30.00}}
\put(120.00,90.00){\line(0,-1){25.00}}
\put(50.00,36.00){\vector(1,0){15.00}}
\put(160.00,36.00){\vector(-1,0){15.00}}
\put(145.00,61.00){\vector(1,0){15.00}}
\put(106.00,37.00){\line(1,0){8.00}}
\put(114.00,37.00){\line(0,-1){4.00}}
\put(114.00,33.00){\line(-1,0){5.00}}
\put(109.00,33.00){\line(0,1){2.00}}
\put(109.00,35.00){\line(-1,0){8.00}}
\put(101.00,35.00){\line(0,-1){2.00}}
\put(101.00,33.00){\line(-1,0){5.00}}
\put(96.00,33.00){\line(0,1){4.00}}
\put(96.00,37.00){\line(1,0){8.00}}
\put(58.00,40.00){\makebox(0,0)[cc]{A}}
\put(153.00,40.00){\makebox(0,0)[cc]{B}}
\put(153.00,65.00){\makebox(0,0)[cc]{O}}
\put(104.00,37.00){\rule{2.00\unitlength}{69.00\unitlength}}
\put(96.00,33.00){\rule{5.00\unitlength}{4.00\unitlength}}
\put(101.00,35.00){\rule{13.00\unitlength}{2.00\unitlength}}
\put(109.00,33.00){\rule{5.00\unitlength}{2.00\unitlength}}
\end{picture}

}
{
\caption{\protect\capsize Skematisk illustration af
CSTR-opstillingen. Reaktanter ledes ind gennem A og B og
overskydende produkter udpumpes gennem C. L{\ae}seren
henvises til \cite{CSTRfig} for en mere detaljeret
beskrivelse af CSTR-opstillingen og dennes eksperimentelle
realiseren.}
\label{fig:CSTRFigur}
}


{
\newcommand{\hso}        {\mbox{H$_2$SO$_4$}}
\newcommand{\malon}      {\mbox{CH$_2$(COOH)$_2$}}
\newcommand{\nabro}      {\mbox{NaBrO$_3$}}
\newcommand{\bromat}     {\mbox{BrO$_3^-$}}
\newcommand{\prot}       {\mbox{H$^+$}}
\newcommand{\brmalon}    {\mbox{BrCH(COOH)$_2$}}
\newcommand{\cotwo}      {\mbox{CO$_2$}}
\newcommand{\ho}         {\mbox{H$_2$O}}
\newcommand{\cesalt}     {\mbox{Ce(NO$_3$)$_6$(NH$_4$)$_2$}}

\vspace{4.0mm}
I 1950 foretager den russike kemiker B.\ P.\ Belousov en
unders{\o}gelse af katalyse i Krebs' cyklus med henblik
p{\aa} at modellere denne \cite{BZ-History}. Hvis
reaktionen i stedet for proteinbundne metalkomplekser
udskiftes med ceriumioner, opdagede Belousov til sin store
overraskelse, at reaktionsopl{\o}sningen periodisk skifter
mellem en farvel{\o}s og en gullig farve.

\vspace{4.0mm}
I 1951 fors{\o}ger Belousov efter en grundig
unders{\o}gelse af dette f{\ae}nomen at publicere disse
resultater, men hans artikel bliver forkastet. Hans arbejde
for\-taber sig dog ikke i fuldst{\ae}ndig glemsel, og i
1961 forts{\ae}tter kemikeren A.\ M.\ Zhabotinsky med at
studerer denne mystiske reaktion. I l{\o}bet af de
n{\ae}ste {\aa}r publiceres mindst ti artikler om
reaktionen i Sovjetunionen, men f{\o}rst i 1967 beskrives
denne for f{\o}rste gang i den vestlige
litteratur\footnote{Det ville nok v{\ae}re forkert at kalde
BZ-reaktionen for ``den f{\o}rste'' kemiske oscillator. I
stedet er betegnelsen ``den f{\o}rst accepterede'' nok mere
passende. Allerede i 1921 opdagede W.\ C.\ Bray den
f{\o}rste egentlige kemiske oscillator \protect\cite{Bray},
der idag kendes under navnet ``Bray-Liebhafsky reaktionen''
(se afsnit~\ref{sec:OscOversigt}). Bray beskyldtes
imidlertid af sin samtid for eksperimentel sjusk og
dovenskab \protect\cite{EpsteinHistory}, og hans arbejde
delte samme sk{\ae}bne som Lotkas og opn{\aa}ede f{\o}rst
anerkendelse efter BZ-reaktionens
opdagelse.}\cite{Degn1,Degn2}.

\vspace{4.0mm}
Siden har reaktionen, der af indlysende grunde har f{\aa}et
navnet Belou\-sov-Zhabotinsky reaktionen (BZ reaktionen),
v{\ae}ret genstand for en stor forsk\-nings\-aktivitet.
Denne reaktion har yderligere afsl{\o}ret et stort antal af
mere komplicerede former for opf{\o}rsel end de f{\o}rst
observerede simple sving\-ninger.

\vspace{4.0mm}
Normalt studerer man BZ-reaktionen ved at blande stofferne
\hso, \nabro, \malon~(ma\-lonsyre) og \cesalt~\-under
omr{\o}ring. En tidsr{\ae}kke svarende til det f{\ae}nomen,
som oprindeligt observeredes af Belousov, fremg{\aa}r af
figur~\ref{fig:BZTypes}a. Netto svarer selve reaktionen til
en oxidation af malonsyre med bromat i et surt vandigt
reaktions\-medium

\begin{eqnarray}
  2 \bromat + 3 \malon + 2 \prot \longrightarrow \nonumber\\
  2 \brmalon + 3 \cotwo + 4 \ho
  \label{eq:BZEquation}
\end{eqnarray}
}

%%%%%%%%%%%%%%%%%%%%%%%%%%%%%%%%%%%%%%%%%%%%%%%%%%%%%%%%%%%%%%%%%%%%%%%%
%%
%% figur med original BZ-tidsserie 
%% figur spiral dannelse
%%
%%%%%%%%%%%%%%%%%%%%%%%%%%%%%%%%%%%%%%%%%%%%%%%%%%%%%%%%%%%%%%%%%%%%%%%%
\boxfigure{t}{\textwidth}
{
\ \vspace{5cm} \ 
}
{
\caption{\protect\capsize
a) Potentiometriske m{\aa}linger af $\log\mbox{[Br$^-$]}$
og $\log\mbox{[Ce(IV)]/[Ce(III)]}$ udf{\o}rt p{\aa}
BZ-reaktionen (figuren stammer fra \protect\cite{FKNorig}).
b) Roterende spiralm{\o}nster i et tyndt lag opl{\o}sning
indeholdende reagenser fra BZ-reaktionen (figuren
stammer fra \protect\cite{WinfreeSpirals}).}
\label{fig:BZTypes}
}

\vspace{4.0mm}
Efter reaktionens ``opdagelse'' fremkom i 1972
den s{\aa}kaldte Field-K\"{o}r\"{o}s-Noyes model
(FKN-modellen) \cite{FKNorig}, der var det f{\o}rste
egentlige forslag til en reaktions\-mekanisme, der
kvalitativt kunne redeg{\o}re for tilstede\-v{\ae}relsen af
simple sving\-ninger i BZ-reaktionen.

\vspace{4.0mm}
Senere fandtes eksperimentelt, at BZ-reaktionen udover de
simple sving\-ninger ogs{\aa} kunne udvise en langt mere
kompleks opf{\o}rsel i form af periodefordoblede,
kvasiperiodiske og kaotiske sving\-ninger. Ogs{\aa} under
heterogene forhold udviser BZ-reaktionen
bem{\ae}rkelses\-v{\ae}rdige egen\-skaber, idet man under
s{\aa}danne forhold bl.a.\ har observeret m{\o}n\-ster- og
spiraldannelse (figur~\ref{fig:BZTypes}b), samt s{\aa}\-kaldt
kemisk tubulens.

\vspace{4.0mm}
BZ-reaktionen er dog ikke l{\ae}ngere den eneste kendte
reaktion med de n{\ae}vnte egen\-skaber. Idag kendes mange
reaktioner, der alle udviser komplekse sving\-ninger. Uden
at g{\aa} helt i detaljer vil vi i det n{\ae}ste afsnit
give en kort oversigt og diskussion af de vigtigste
oscillerende reaktioner, der kendes idag.

\section{Oversigt over kemiske oscillatorer}
\label{sec:OscOversigt}
\subsection{Bray-Liebhafsky-reaktionen}
{
\newcommand{\htooto} {\mbox{H$_2$O$_2$ }}
\newcommand{\iodat}  {\mbox{IO$_3^-$ }}
\newcommand{\hiodat} {\mbox{HIO$_3$ }}
\newcommand{\iod}    {\mbox{I$_2$ }}
\newcommand{\oxy}    {\mbox{O$_2$ }}
\newcommand{\htoo}   {\mbox{H$_2$O }}

I 1911 studerede Auger \cite{Auger} den iodat katalyserede
dekomposition af hydrogenperoxid H$_2$O$_2$, der senere af
Bray i 1921 vistes at oscillere \cite{Bray}. Auger
opstillede de to reaktioner

\begin{eqnarray}
  &&5~\htooto +~\iod \rightleftharpoons 2~\hiodat +~4~\htoo\\
  &&5~\htooto +~2~\hiodat \rightleftharpoons \iod +~5~\oxy +~6~\htoo
\end{eqnarray}

der netto svarer til den vel\-kendte dekomposition

\begin{equation}
  2~\htooto \rightleftharpoons 2~\htoo +~\oxy
\end{equation}

Oscillationer kan f.eks.\ observeres ved hj{\ae}lp af en
iodid f{\o}lsom elektrode, idet I$^-$-ioner indg{\aa}r som
intermediat i Bray-Liebhafsky-reaktionen. En detaljeret
mekanisme blev i 1976 foresl{\aa}et for
Bray-Liebhafsky-reaktionen \cite{BrayModel}. 
}

\subsection{Briggs-Rauscher-reaktionen}
{
\newcommand{\htooto} {\mbox{H$_2$O$_2$ }}
\newcommand{\iodat}  {\mbox{IO$_3^-$ }}
\newcommand{\malon}  {\mbox{CH$_2$(COOH)$_2$ }}
\newcommand{\maloni} {\mbox{CHI(COOH)$_2$ }}
\newcommand{\oxy}    {\mbox{O$_2$ }}
\newcommand{\htoo}   {\mbox{H$_2$O }}
\newcommand{\prot}   {\mbox{H$^+$ }}
Denne reaktion blev opdaget af Briggs og Rauscher i 1972
\cite{BriggsRauscher} og kan betragtes som en
``mellemting'' mellem Bray-Liebhafsky-reaktionen og
BZ-reaktionen, idet den prepareres ved at blande de
reagenser, der deltager i disse to reaktioner. Den totale
st{\o}kiometri for Briggs-Rauscher-reaktionen er beskrevet
ved

\begin{eqnarray}
&\iodat +~2~\htooto +~\malon +~\prot \rightleftharpoons&\nonumber\\
&\maloni +~2~\oxy +~3~\htoo&
\end{eqnarray}
}

Reaktionen finder alts{\aa} sted i sur opl{\o}sning med
Mn(II)-ioner som katalysator. Briggs-Rauscher-reaktionen
kan ogs{\aa} fungere med andre substrater, idet malonsyre
kan udskiftes med 2,4-pentadion. Ligeledes kan Ce(IV)-ioner
ogs{\aa} anvendes som katalysator. To n{\ae}sten identiske
reaktionsmekanismer for Briggs-Rauscher-reaktionen blev
uafh{\ae}ngigt af hinanden foresl{\aa}et i 1982
\cite{BriggsModel1,BriggsModel2}.

\subsection{Oxidation af carbonmonooxid}
{
\newcommand{\oxy}    {\mbox{O$_2$ }}
\newcommand{\coto}    {\mbox{CO$_2$ }}
\newcommand{\co}    {\mbox{CO }}
Oscillerende kemiske reaktioner kendes ogs{\aa} i
heterogene medier, hvor et af de mest velstuderede
tilf{\ae}lde er oxidationen af \co p{\aa} en Pt-overflade
\cite{CO-Oscillator}.

\begin{equation}
  2~\co +~\oxy \stackrel{{\rm Pt}}{\rightleftharpoons} 2~\coto
\end{equation}

Ved hj{\ae}lp af lav energi elektron diffraktion (LEED) har
man i dette system yderligere kunnet observere dannelse af
b{\o}lgefronter og spiralm{\o}nstrer p{\aa} den
p{\aa}g{\ae}ldende Pt-overflade \cite{CO-Oscillator}.

\vspace{4.0mm}
Denne reaktion er dog ikke den eneste kendte kemiske
oscillator, der finder sted under heterogene forhold, idet
man ogs{\aa} tidligere har observeret svingninger i
dekompositionen af gasformig N$_2$O p{\aa} en CuO overflade
\cite{Hugo}.
}

\subsection{Forbr{\ae}nding af hydrogen}
{
\newcommand{\oxy}    {\mbox{O$_2$ }}
\newcommand{\htoo}  {\mbox{H$_2$O }}
\newcommand{\brint}  {\mbox{H$_2$ }}
Den klasssike gasfaseforbr{\ae}ndingsproces 

\begin{equation}
  2~\brint +~\oxy \rightleftharpoons 2~\htoo
\end{equation}

har vist sig at oscillere, hvis denne studeres i et
CSTR-kammer under ikke-konstante temperaturforhold
\cite{HydrogenOxygen1}. En r{\ae}kke af de intermedi{\ae}re
reaktioner, der indg{\aa}r i forbr{\ae}ndingprocessen, er
exoterme: Da temperaturen ikke er konstant under
fors{\o}gsbetingelserne vil de intermedi{\ae}re
reaktioner v{\ae}re autokatalytiske, idet en for{\o}gelse
af temperaturen svarer til en yderligere for{\o}gelse af
reaktionshastigheden. Simuleringer af dette system er
blandt andet foretaget ved hj{\ae}lp af den s{\aa}kaldte
Baldwin-Walker model, der indeholder ikke mindre end 35
forskellige elementarreaktioner \cite{HydrogenOxygen1}.

\vspace{4.0mm}
Eksperimentelt har reaktionen vist sig at v{\ae}re yderst
kontrolabel og har blandt andet givet anledning til nogle
af de mest pr{\ae}cise studier af Hopfbifurkationer i
kemiske reaktioner, der frem til idag er foretaget
\cite{HydrogenOxygen2}.
}

\subsection{Oxidation af benzaldehyd}
{
\newcommand{\benzal} {\mbox{BzH }}
\newcommand{\benzso} {\mbox{BzOH }}
\newcommand{\oxy}    {\mbox{O$_{2_{{\rm (g)}}}$ }}
Luftens oxygen kan oxidere benzaldehyd til benzosyre efter
reaktionsligningen

\begin{equation}
 2~\benzal +~\oxy \rightleftharpoons 2~\benzso
\end{equation}

Under normale omst{\ae}ndigheder er reaktionen uhyre
langsom, men til\-s{\ae}ttes Co(II)-ioner eller
Br$^-$-ioner som katalysator i koncentreret eddikkesyre ved
70$^\circ$C observeres oscillationer mellem en lyser{\o}d
til en grumset gr{\o}n farve svarende til de to forskellige
oxidationstrin Co(II) og Co(III) for cobalt
\cite{BenzAldehyd}. }

\subsection{Svovloscillatorer}
{
\newcommand{\iodat}  {\mbox{IO$_3^-$ }}
\newcommand{\bromat} {\mbox{BrO$_3^-$ }}
\newcommand{\oxy}    {\mbox{O$_2$ }}
\newcommand{\htoo}   {\mbox{H$_2$O }}
\newcommand{\prot}   {\mbox{H$^+$ }}
\newcommand{\thio}   {\mbox{SCN$^-$ }}
\newcommand{\iodid}  {\mbox{I$^-$ }}
\newcommand{\sofire} {\mbox{SO$_4^{2-}$ }}
\newcommand{\cyanid} {\mbox{CN$^-$ }}
\newcommand{\brom}   {\mbox{Br$_2$ }}
\newcommand{\hs}     {\mbox{HS$^-$ }}
De bedst unders{\o}gte kemiske oscillatorer involverer
prim{\ae}rt forskellige oxyhalogener, men idag kendes
ogs{\aa} en del oscillerende reaktioner, hvor svovl spiller
en v{\ae}sentlig rolle. De fleste af disse oscillatorer har
vist sig at v{\ae}re baseret p{\aa} en kompliceret kinetik,
der idag endnu ikke er fuldt ud forst{\aa}et. Eksempelvis
kan n{\ae}vnes iodat-thiocyanat \cite{ThioCyanat} og
bromat-sulfid \cite{BromatSulfid} oscillatoren, hvis
st{\o}kiometri respeketivt angives som

\begin{eqnarray}
  \iodat +~\thio +~\htoo & \rightleftharpoons & 
  \iodid +~\sofire +~\cyanid + 2~\prot \\
  5~\hs + 8~\bromat + 3~\prot & \rightleftharpoons &
  5~\sofire +~4~\brom + 4~\htoo
\end{eqnarray}

I iodat-thiocyanat oscillatoren kan iodat og thiocyanat
respektivt udskiftes med bromat og thiourinstof. 

\vspace{4.0mm}
Udover disse har methylenbl{\aa}t-O$_2$-HS$^-$ oscillatoren
ogs{\aa} v{\ae}ret genstand for megen opm{\ae}rksomhed i de
seneste {\aa}r. Den totale st{\o}kiometri for denne
reaktion beskriver den methylenbl{\aa}t katalyserede
oxidation af hydrogensulfid, hvorunder forskellige
svovlholdige produkter dannes. Oxygen omdannes bl.a.\
til vand og hydrogenperoxid, hvor specielt sidstn{\ae}vnte
spiller en vigtig rolle for selve reaktionsforl{\o}bet,
idet denne deltager i en autokatalytisk intermedi{\ae}r
reaktion \cite{KemiskProjekt}.

\vspace{4.0mm}
Til sidst b{\o}r ogs{\aa} n{\ae}vnes den s{\aa}kaldte Mason
reaktion, der beskriver den autokatalytiske dekomposition
af en vandig opl{\o}sning af dithionit, S$_2$O$_4^{2-}$,
til bisulfitioner, HSO$_3^{2-}$
\cite{Dithionit1,Dithionit2}.
}

\subsection{Biokemiske oscillatorer}
{
\newcommand{\oxy}    {\mbox{O$_2$ }}
\newcommand{\htoo}   {\mbox{H$_2$O }}
\newcommand{\prot}   {\mbox{H$^+$ }}
Der kendes idag to biokemiske reaktioner, der begge kan
udvise forskellige former for oscillationer: {\em
peroxidase reaktionen\/} og {\em glycolysen\/}
\cite{BioChaos}. Peroxidase reaktionen beskriver
reduktionen af oxygen til vand katalyseret af enzymet
peroxidase med NADH som elektron donor

\begin{equation}
 \oxy +~2~{\rm NADH} +~2~\prot \rightleftharpoons
 2~\htoo +~2~{\rm NAD}^+
\end{equation}

Peroxidase reaktionen har v{\ae}ret unders{\o}gt i
s{\aa}vel en CSTR-opstilling \cite{Per1}, som i en
eksperimentel ops{\ae}tning, hvor kun substratet NADH
tilf{\o}res \cite{Per2}. Oscillationer kan observeres dels
spektrofotometrisk ved m{\aa}ling af
koncentrations{\ae}ndringer i NADH eller en form af
enzymet, og dels med en oxygen selektiv elektrode til
bestemmelse af oxygenkoncentrationen.

\vspace{4.0mm}
Glycolysen, der er navnet for den anaerobe del af den
metabolistiske nedbrydelse af glycose i bl.a.\ den
menneskelige organisme, er den eneste kendte biokemiske
reaktion, der udviser oscillationer {\em in vivo\/} (dette
f{\ae}nomen kendes f.eks.\ ikke fra peroxidase reaktionen).
Skematisk kan glycolysen bekrives som

\vspace{4.0mm}
%%%%%%%%%%%%%%%%%%%%%%%%%%%%%%%%%%%%%%%%%%%%%%%%%%%%%%%%%%%%%%%%%%%%%%%%
%% figur
%%
%% beskrivelse : Skematisk fremstilling af glycolysen
%% makroer     : PSTricks, PST-Node, FancyBox
%%%%%%%%%%%%%%%%%%%%%%%%%%%%%%%%%%%%%%%%%%%%%%%%%%%%%%%%%%%%%%%%%%%%%%%%
\begin{center}
 \begin{pspicture}(0,2)(14,9)
%  \psgrid[](0,2)(0,2)(14,9)
  \psset{arrowinset=0}
  \rput[cl]{*0}( 2.0,8.0){\ovalnode{A1}{Glycogen}}
  \rput[cr]{*0}(12.0,8.0){\ovalnode{A2}{Glucose}}
  \rput[cc]{*0}( 7.0,7.0){\ovalnode{A3}{Glucose-6-P}}
  \rput[cc]{*0}(12.0,7.0){\circlenode{A4}{\scriptsize ATP}}
  \rput[cc]{*0}( 9.0,6.2){\circlenode{A5}{\scriptsize ATP}}
  \rput[cc]{*0}(10.5,5.0){\circlenode{A6}{\scriptsize 4 ATP}}
  \rput[cc]{*0}( 3.0,4.0){\ovalnode{A7}{2 Pyruvat}}
  \rput[cc]{*0}( 7.0,3.8){\ovalnode{A7}{2 M{\ae}lkesyre}}
  \rput[cc]{*0}(11.0,3.6){\ovalnode{A7}{2 Lactat}}
  \psline[arrowsize=1.5pt 2.25,linearc=0.25,linewidth=1.2pt]{->}
  (3.1,7.6)(3.1,7.0)(5.4,7.0)
  \psline[arrowsize=1.5pt 2.25,linearc=0.25,linewidth=1.2pt]{->}
  (11.0,7.7)(11.0,7.0)(8.6,7.0)
  \psline[arrowsize=1.5pt 2.25,linewidth=1.2pt]{->}(11.5,7.0)(11.0,7.0)
  \psline[arrowsize=1.5pt 2.25,linearc=0.25,linewidth=1.2pt]{->}
  (7.0,6.7)(7.0,6.2)(3.0,6.2)(3.0,4.6)
  \psline[arrowsize=1.5pt 2.25,linearc=0.25,linewidth=1.2pt]{->}
  (3.0,5.2)(3.0,5.0)(9.8,5.0)
  \psline[arrowsize=1.5pt 2.25,linewidth=1.2pt]{->}(8.5,6.2)(7.0,6.2)
  \psline[arrowsize=1.5pt 2.25,linewidth=1.2pt]{->}(4.35,4.0)(5.4,4.0)
  \psline[arrowsize=1.5pt 2.25,linewidth=1.2pt]{->}(8.65,3.83)(9.8,3.83)
  \psline[arrowsize=1.5pt 2.25,linewidth=1.2pt]{->}(3.0,3.6)(3.0,3.0)
  \rput[tc]{*0}(3.0,2.8){\em til citronsyrecyklus}
 \end{pspicture}
\end{center}

\vspace{4.0mm}
Oscillationer i glycolysen studeres analogt med peroxidase
systemet ved spektrofotometrisk at m{\aa}le NADH
koncentrationen. Udtages pr{\o}ver under
oscillationsforl{\o}bet kan man udfra biokemiske
analysemetoder rekonstruere koncentrationsforl{\o}bet for
en r{\ae}kke stoffer, der deltager i glycolysen, f.eks.\
ATP og glucose-6-P.

\vspace{4.0mm}
Der har i litteraturen forekommet en del spekulationer
over, hvilken rolle oscillationerne i glycolysen spiller
for den levende organisme. Idag hersker der dog en vis
enighed om, at oscillationerne ikke i sig selv tjener nogen
funktion, men derimod er et resultat af samspillet mellem
den feedbackregulering, der afbalancerer dannelse af
forskelige intermediater og glykolytisk produceret ATP
\cite{BioChaos}. }

\subsection{Klassifikation af kemiske oscillatorer}
Kemiske oscillationer kan til en vis grad opdeles i et
hierarki, der ofte kaldes en {\em taxonomi\/}. En s{\aa}dan
ordning er eksempelvis beskrevet i \cite{Taxonomy}, hvorfra
figur~\ref{fig:Taxonomi} ogs{\aa} stammer. Figuren
opsummerer alle de reaktioner, der er gennemg{\aa}et i
dette afsnit og inkluderer ogs{\aa} enkelte reaktioner, der
ikke er gennemg{\aa}et. Disse er medtaget for helhedens
skyld.

\vspace{4.0mm}
Vi vil i l{\o}bet af de n{\ae}ste kapitler redeg{\o}re for
et udvalg af de vigtigste metoder og det teoretiske
grundlag, der idag danner rammen for at skabe en bredere
forst{\aa}else og indsigt i den dynamiske natur, der
karakteriserer s{\aa}danne reaktioner.

%%%%%%%%%%%%%%%%%%%%%%%%%%%%%%%%%%%%%%%%%%%%%%%%%%%%%%%%%%%%%%%%%%%%%%%%
%% figur
%%
%% beskrivelse : Taxonomisk oversigt over oscillerende
%%               kemiske reaktioner
%% makroer     : PSTricks, PST-Node, FancyBox
%%%%%%%%%%%%%%%%%%%%%%%%%%%%%%%%%%%%%%%%%%%%%%%%%%%%%%%%%%%%%%%%%%%%%%%%
\renewcommand{\capfont}{\bf}
\begin{landfloat}{figure}{\rotateright}
\footnotesize
\newgray{middlegray}{0.6}
\begin{center}
 \begin{pspicture}(1,0)(21,12)
%  \psgrid[](1,0)(1,0)(21,12)
  %%%%%%%%%%%%%%%%%%%%%%%%% bromat %%%%%%%%%%%%%%%%%%%%%%%%%%%%
  \psset{fillcolor=gray}
  \rput[cc]{*0}( 6.7,11.2){\ovalnode*{A1}{Bromat (BrO$_3^-$)}}
  \rput[cc]{*0}( 3.9,10.1){\rnode{A2}{\psframebox*{$+$ katalysator}}}
  \rput[cc]{*0}( 8.7,10.1){\rnode{A3}{\psframebox*{$\div$ katalysator}}}
  \rput[cc]{*0}( 3.4, 8.7){\rnode{A4}{\psframebox*{
    \parbox[c]{1.4cm}{BrO$_3^-$ $+$ \\ Br$^-$ $+$ \\ M$^+$}}}}
  \rput[cc]{*0}(10.65,8.7){\rnode{A7}{\psframebox*{
    \parbox[c]{1.4cm}{BrO$_3^-$ $+$ \\ Br$^-$ $+$ \\ ClO$_2^-$}}}}
  \rput[cc]{*0}(5.9,8.9){\rnode{A5}{\psframebox*{
    \parbox[c]{1.4cm}{BrO$_3^-$ $+$ \\ R(Ar)}}}}
  \rput[cc]{*0}(8.4,8.9){\rnode{A6}{\psframebox*{
    \parbox[c]{1.4cm}{BrO$_3^-$ $+$ \\ I$^-$}}}}
  \rput[cc]{*0}(3.4,6.9){\rnode{A8}{\psframebox*{
    \parbox[c]{1.4cm}{BrO$_3^-$ $+$ \\ M$^+$ $+$ \\ R(Uorg)}}}}
  \rput[cc]{*0}(5.9,6.9){\rnode{A9}{\psframebox*{
    \parbox[c]{1.4cm}{BrO$_3^-$ $+$ \\ M$^+$ $+$ \\ R(Org)}}}}
  \rput[cc]{*0}(8.4,6.9){\rnode{A10}{\psframebox*{
    \parbox[c]{1.4cm}{BrO$_3^-$ $+$ \\ ClO$_2^-$ $+$ \\ R}}}}
  %%%%%%%%%%%%%%%%%%%%%%%%% chlorit %%%%%%%%%%%%%%%%%%%%%%%%%%%%
  \psset{fillcolor=lightgray}
  \rput[cc]{*0}(15.3,11.2){\ovalnode*{B1}{Chlorit (ClO$_2^-$)}}
  \rput[cc]{*0}(16.4, 8.7){\rnode{B3}{\psframebox*{
    \parbox[c]{1.4cm}{ClO$_2^-$ $+$ \\ svovlfor- \\ bindelser}}}}
  \rput[cc]{*0}(14.1,8.9){\rnode{B2}{\psframebox*{
    \parbox[c]{1.4cm}{ClO$_2^-$ $+$ \\ I$^-$}}}}
  \rput[cc]{*0}(11.8, 6.9){\rnode{B4}{\psframebox*{
    \parbox[c]{1.4cm}{ClO$_2^-$ $+$ \\ IO$_3^-$ $+$ \\ R(Uorg)}}}}
  \rput[cc]{*0}(14.1, 6.9){\rnode{B5}{\psframebox*{
    \parbox[c]{1.4cm}{ClO$_2^-$ $+$ \\ I$^-$ $+$ \\ R(Org)}}}}
  \rput[cc]{*0}(16.4, 6.9){\rnode{B6}{\psframebox*{
    \parbox[c]{1.4cm}{ClO$_2^-$ $+$ \\ I$^-$ $+$ \\ Ox}}}}
  \rput[cc]{*0}(18.9, 6.9){\rnode{B7}{\psframebox*{
    \parbox[c]{1.4cm}{ClO$_2^-$ $+$ \\ S$_2$O$_3^{2-}$ $+$ \\ Ox}}}}
  %%%%%%%%%%%%%%%%%%%%%%%%% carbon %%%%%%%%%%%%%%%%%%%%%%%%%%%%
  \psset{fillcolor=white}
  \rput[cc]{*0}( 6.1,5.0){\ovalnode{C1}{Carbon}}
  \rput[tc]{*0}( 3.4,4.1){\rnode{C2}{\psframebox{
    \parbox[c]{1.6cm}{R(Org) $+$\\O$_2$ $+$\\Co$^{2+}$$+$\\ Br$^-$}}}}
  \rput[tc]{*0}( 5.9,4.1){\rnode{C3}{\psframebox{
    \parbox[c]{1.9cm}{Biokemiske \\ reaktioner}}}}
  %%%%%%%%%%%%%%%%%%%%%%%%% iodat %%%%%%%%%%%%%%%%%%%%%%%%%%%%
  \psset{fillcolor=middlegray}
  \rput[cc]{*0}(10.3,5.0){\ovalnode*{D1}{Iodat (IO$_3^-$)}}
  \rput[tc]{*0}( 9.0,4.1){\rnode{D2}{\psframebox*{
    \parbox[c]{1.4cm}{IO$_3^-$ $+$\\H$_2$O$_2$}}}}
  \rput[tc]{*0}(11.5,4.1){\rnode{D3}{\psframebox*{
    \parbox[c]{1.6cm}{IO$_3^-$ $+$\\SO$_3^{2-}$ $+$\\Fe(CN)$_6^{4-}$}}}}
  \rput[cc]{*0}(11.5,1.7){\ovalnode*{D4}{Overgangsmetal}}
  \rput[bc]{*0}(11.5,0.0){\rnode{D5}{\psframebox*{
    \parbox[c]{1.6cm}{MnO$_4^-$ $+$\\H$_2$O$_2$}}}}
  %%%%%%%%%%%%%%%%%%%%%%%%% svovl %%%%%%%%%%%%%%%%%%%%%%%%%%%%
  \psset{fillcolor=white}
  \rput[cc]{*0}(17.0,5.0){\ovalnode{E1}{Svovl}}
  \rput[tc]{*0}(14.5,4.1){\rnode{E2}{\psframebox{
    \parbox[c]{1.4cm}{H$_2$O$_2$ $+$\\S$^{2-}$}}}}
  \rput[tc]{*0}(17.0,4.1){\rnode{E3}{\psframebox{
    \parbox[c]{1.4cm}{H$_2$O$_2$ $+$\\S$_2$O$_3^{2-}$
                      $+$\\eller \\SNC$^-$ $+$\\Cu$^{2+}$}}}}
  \rput[tc]{*0}(19.5,4.1){\rnode{E4}{\psframebox{
    \parbox[c]{1.4cm}{MB$^+$ $+$\\O$_2$ $+$\\S$^{2-}$ $+$\\SO$_3^{2-}$}}}}
  %%%%%%%%%%%%%%%%%%%%%%%%% A nodes %%%%%%%%%%%%%%%%%%%%%%%%%%%%
  \ncline{A4}{A8}
  \ncline{A4}{A9}
  \ncline{A7}{A10}
  \psline(3.4,9.38)(3.4,9.82) % trouble with the nodes
  \psline(8.4,9.38)(8.4,9.82) % trouble with the nodes
  \ncdiag[angleA=-170,angleB=90,arm=0]{A1}{A2}
  \ncdiag[angleA= -10,angleB=90,arm=0]{A1}{A3}
  \ncdiag[angleA= 180,angleB=90,arm=0]{A3}{A5}
  \ncdiag[angleA=   0,angleB=90,arm=0]{A3}{A7}
  %%%%%%%%%%%%%%%%%%%%%%%%% B nodes %%%%%%%%%%%%%%%%%%%%%%%%%%%%
  \ncline{B2}{B4}
  \ncline{B2}{B5}
  \ncline{B2}{B6}
  \ncdiag[angleA=   0,angleB=90,arm=0]{B3}{B7}
  \ncdiag[angleA=-140,angleB=90,arm=0]{B1}{B2}
  \ncdiag[angleA= -40,angleB=90,arm=0]{B1}{B3}
  %%%%%%%%%%%%%%%%%%%%%%%%% C nodes %%%%%%%%%%%%%%%%%%%%%%%%%%%%
  \ncline{C1}{C2}
  \psline(6.1,4.108)(6.1,4.8) % trouble with the nodes
  %%%%%%%%%%%%%%%%%%%%%%%%% D nodes %%%%%%%%%%%%%%%%%%%%%%%%%%%%
  \ncline{D4}{D5}
  \ncline[linestyle=dashed,dash=2.5pt 0.7pt]{D3}{D4}
  \ncdiag[angleA=-140,angleB=90,arm=0]{D1}{D2}
  \ncdiag[angleA= -40,angleB=90,arm=0]{D1}{D3}
  %%%%%%%%%%%%%%%%%%%%%%%%% E nodes %%%%%%%%%%%%%%%%%%%%%%%%%%%%
  \ncdiag[angleA=-140,angleB=90,arm=0]{E1}{E2}
  \ncdiag[angleA= -90,angleB=90,arm=0]{E1}{E3}
  \ncdiag[angleA= -40,angleB=90,arm=0]{E1}{E4}
  %%%%%%%%%%%%%%%%%%%%%% mixed nodes %%%%%%%%%%%%%%%%%%%%%%%%%%%
  \ncangle[angleA=180,angleB=180,
           arm=7.9,linestyle=dashed,dash=2.5pt 0.7pt]{A2}{D4}
  \ncbar[angleA=90,
         arm=0.2,linestyle=dashed,dash=2.5pt 0.7pt]{A10}{B6}
  \ncangle[angleA=0,angleB=180,
           arm=0,linestyle=dashed,dash=2.5pt 0.7pt]{A10}{B4}
  \ncdiag[angleA=-90,angleB= 90,
          linestyle=dashed,dash=2.5pt 0.7pt,arm=0]{B4}{D1}
  \ncdiag[angleA=  0,angleB=170,
          linestyle=dashed,dash=2.5pt 0.7pt,arm=0]{D4}{E3}
  \ncdiag[angleA=-90,angleB= 90,
          linestyle=dashed,dash=2.5pt 0.7pt,arm=0]{A9}{C2}
  %%%%%%%%%%%%%%%%%%%%%% text %%%%%%%%%%%%%%%%%%%%%%%%%%%
  \rput[tl]{*0}(1.4,1.4){\scriptsize\begin{minipage}{8cm}
   R = reduktant, R(Uorg) = uorganisk reduktant, R(or) =
   organisk reduktant, R(Ar) = aromatisk reduktant, Ox =
   oxidant, MB = methylenbl{\aa}t, M$^+$ = metalion
   katalysator, Svovl = S$_2$O$_3^{2-}$, SNC$^-$, ell.\
   thiourinstof.

   \vspace{2.0mm}
   {\bf Note}: Fuldt optrukne linier forbinder kemiske
   systemer inden for den samme gruppe af oscillatorer.
   Punkterede linier forbinder grupper med visse f{\ae}lles
   egenskaber.
  \end{minipage}}
 \end{pspicture}
\end{center}
\caption{\protect\capsize}
\label{fig:Taxonomi}
\end{landfloat}
\renewcommand{\capfont}{\rm}









