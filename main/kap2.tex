\chapter{Quenching af kemiske oscillatorer}
\label{cha:Quench}
For at unders{\o}ge og n{\ae}rme sig en dybere
forst{\aa}else af de underliggende meka\-nis\-mer, der
giver anledning til en manifestation af kompleks dynamik i
kemiske reaktioner, er det fundamentalt at finde et s{\ae}t
elementar reaktioner, der kan udg{\o}re en forklaringsramme
for de egenskaber, den kemiske reaktion udviser i den
eksperimentelle situation. Et s{\aa}dant s{\ae}t elementar
reaktioner kaldes en reaktionsmekanisme.

\vspace{4.0mm}
Under en r{\ae}kke fors{\o}gsbetingelser, der tillader, at
den enkelte elementarreaktion kan unders{\o}ges isoleret,
analyseres dennes dynamiske egenskaber. Dette giver ofte
anledning til, at hastighedskonstanten, der er associeret
med den p{\aa}g{\ae}ldende elementarreaktion, kan
estimeres. Netto ender man med et s{\ae}t af reaktioner,
der udg{\o}r en model over det forl{\o}b af delreaktioner
og kemiske omdannelser, som reaktanterne i bruttoreaktionen
underg{\aa}r, inden bruttoreaktionens endelige produkter
dannes.

\vspace{4.0mm}
I den forstand, kan vi tillade os at betragte en model for
en kemisk reaktion som en ``funktion'' af de fysiske
st{\o}rrelser, der definerer den p{\aa}g{\ae}ldende model
\cite{HopfQuench}. Skematisk kan denne betragtning
opstilles som f{\o}lgende

\begin{itemize}
 \item Det s{\ae}t af elementarreaktioner, der udg{\o}r
  selve modellens ``skelet''.
 \item Hastighedskonstanterne, der bestemmer det dynamiske
  forl{\o}b af s{\aa}vel de enkelte delreaktioner som selve
  bruttoreaktionen.
 \item Koncentrationen af de kemiske stoffer, der
  indg{\aa}r i reaktionen.
\end{itemize}

I unders{\o}gelsen af en kemisk reaktions dynamiske
egenskaber vil dette trin, ``udarbejdelse og bestemmelse af
grundlaget for en realistisk model\-beskrivelse'', ofte
v{\ae}re det f{\o}rste led i en s{\aa}dan analyse. For at
afpr{\o}ve v{\ae}rdien af en s{\aa}dan model kan dennes
evne til at reproducere eksperimentelle tidsr{\ae}kke for
eksempel afpr{\o}ves. Ved integration af de
hastighedsudtryk model\-len definerer, kan man numerisk
fors{\o}ge at simulere en r{\ae}kke af de egenskaber, den
kemiske reaktion udviser eksperimentelt. Disse
sammenligninger forbliver dog let af kvalitativ karakter og
har ofte en yderligere begr{\ae}nsning, idet man i de
f{\ae}rreste situationer er i stand til at m{\aa}le
koncentrationen af samtlige dynamiske stoffer.

\vspace{4.0mm}
Sammenligningen mellem model og eksperiment vil derfor
hyppigt v{\ae}re forbeholdt et eller to stoffer. Det er
derfor v{\ae}sentligt at unders{\o}ge, hvorvidt en
kvantitativ sammenligning af en r{\ae}kke af model\-lens og
den ``virkelige'' reaktions dynamiske karakteristika kan
foretages. Endvidere er en teori, der tillader en
beskrivelse af de dynamiske egenskaber for samtlige stoffer
i reaktionen, n{\o}dvendig.

\vspace{4.0mm}
Her indtager teorien om quenching af kemiske oscillationer
en v{\ae}sentlig rolle, idet den for s{\aa}vel model som
eksperiment tillader en kvantitativ bestemmelse af en
r{\ae}kke af den oscillerende reaktions karakteristiske
egenskaber. I korte tr{\ae}k bygger denne p{\aa}
eksistensen af en opsplitning af faserummet i en stabil og
en ustabil mangfoldighed. Ved tils{\ae}tning af de stoffer,
der deltager i den kemiske reaktion, kan den oscillerende
tidsopf{\o}rsel midlertidig stoppes, idet den tilsatte
stofm{\ae}ngde skifter systemets position i faserummet fra
den stabile mangfoldighed til den ustabile mangfoldighed.

\vspace{4.0mm}
Vi vil her g{\o}re rede for det teoretiske grundlag bag
denne id\'{e} for til sidst at give en r{\ae}kke eksempler
p{\aa} dennes anvendelser i det eksperimentelle arbejde. En
grundig og generel introduktion til emnet findes ogs{\aa} i
\cite{GenQuench}.

\section{Quenching teori}
Helt generelt lader vi den kemiske reaktions dynamik
v{\ae}re beskrevet ved den kintiske ligning

\begin{equation}
 \dot{\bf c} = {\bf f}({\bf c},j_0)
 \label{eq:GeneralKinetic}
\end{equation}

hvor ${\bf c}$ og $j_0$ henholdsvis symboliserer
koncentrationen af de indg{\aa}ende stoffer og flowet
gennem CSTR-kammeret.

\vfill
\vspace{4.0mm} 
Vi antager nu, at der for denne differentialligning
eksisterer en v{\ae}rdi $j_0^{\rm Hopf}$ for flowet $j_0$
s{\aa}ledes, at systemet underg{\aa}r en superkritisk
Hopfbifurkation. Matematisk betyder dette, at et par af
kompleks konjugerede egenv{\ae}rdier for Jacobimatricen
tilfredstiller f{\o}lgende to kriterier

\begin{itemize}
\item De to kompleks konjugerede egenv{\ae}rdier 
      $\lambda_{1,2}(j_0)=\alpha\pm i\omega$ opfylder for 
      $j_0=j_0^{\rm Hopf}$ 
      $$
       \lambda_{1,2}(j_0^{\rm Hopf}) = \pm i\omega
      $$
\item Den afledede af $\lambda_{1,2}(j_0)$ skal opfylde
      $$
       \left.\frac{d\lambda_{1,2}(j_0)}{dj_0}\right|_{j_0^{\rm Hopf}} 
       \neq 0
      $$
\end{itemize}

\boxfigure{t}{\textwidth}{
 \vspace{-0.25cm}
 \begin{minipage}{5.5cm}
  % GNUPLOT: LaTeX picture
\setlength{\unitlength}{0.240900pt}
\ifx\plotpoint\undefined\newsavebox{\plotpoint}\fi
\sbox{\plotpoint}{\rule[-0.175pt]{0.350pt}{0.350pt}}%
\begin{picture}(600,765)(175,0)
\tenrm
\sbox{\plotpoint}{\rule[-0.175pt]{0.350pt}{0.350pt}}%
\put(264,405){\rule[-0.175pt]{141.408pt}{0.350pt}}
\put(557,158){\rule[-0.175pt]{0.350pt}{119.005pt}}
\put(861,405){\makebox(0,0)[l]{$Re \lambda$}}
\put(557,693){\makebox(0,0){$Im \lambda$}}
\put(753,560){\makebox(0,0)[l]{$\lambda_1$}}
\put(753,247){\makebox(0,0)[l]{$\lambda_2$}}
\put(850,405){\vector(1,0){1}}
\put(557,651){\vector(0,1){1}}
\put(749,555){\usebox{\plotpoint}}
\put(750,556){\usebox{\plotpoint}}
\put(751,557){\usebox{\plotpoint}}
\put(753,560){\vector(3,4){0}}
\put(752,558){\usebox{\plotpoint}}
\put(749,253){\usebox{\plotpoint}}
\put(750,252){\usebox{\plotpoint}}
\put(751,251){\usebox{\plotpoint}}
\put(753,250){\vector(3,-4){0}}
\put(752,250){\usebox{\plotpoint}}
\sbox{\plotpoint}{\rule[-0.350pt]{0.700pt}{0.700pt}}%
\put(362,440){\usebox{\plotpoint}}
\put(362,440){\rule[-0.350pt]{3.854pt}{0.700pt}}
\put(378,441){\rule[-0.350pt]{7.468pt}{0.700pt}}
\put(409,442){\rule[-0.350pt]{5.782pt}{0.700pt}}
\put(433,443){\rule[-0.350pt]{4.818pt}{0.700pt}}
\put(453,444){\rule[-0.350pt]{3.854pt}{0.700pt}}
\put(469,445){\rule[-0.350pt]{3.613pt}{0.700pt}}
\put(484,446){\rule[-0.350pt]{2.891pt}{0.700pt}}
\put(496,447){\rule[-0.350pt]{2.891pt}{0.700pt}}
\put(508,448){\rule[-0.350pt]{1.927pt}{0.700pt}}
\put(516,449){\rule[-0.350pt]{1.927pt}{0.700pt}}
\put(524,450){\rule[-0.350pt]{1.927pt}{0.700pt}}
\put(532,451){\rule[-0.350pt]{1.927pt}{0.700pt}}
\put(540,452){\rule[-0.350pt]{1.927pt}{0.700pt}}
\put(548,453){\rule[-0.350pt]{1.927pt}{0.700pt}}
\put(556,454){\rule[-0.350pt]{0.723pt}{0.700pt}}
\put(559,455){\rule[-0.350pt]{1.927pt}{0.700pt}}
\put(567,456){\rule[-0.350pt]{0.964pt}{0.700pt}}
\put(571,457){\rule[-0.350pt]{0.964pt}{0.700pt}}
\put(575,458){\rule[-0.350pt]{1.927pt}{0.700pt}}
\put(583,459){\rule[-0.350pt]{0.964pt}{0.700pt}}
\put(587,460){\rule[-0.350pt]{0.964pt}{0.700pt}}
\put(591,461){\rule[-0.350pt]{0.964pt}{0.700pt}}
\put(595,462){\rule[-0.350pt]{0.964pt}{0.700pt}}
\put(599,463){\rule[-0.350pt]{0.964pt}{0.700pt}}
\put(603,464){\rule[-0.350pt]{0.964pt}{0.700pt}}
\put(607,465){\rule[-0.350pt]{0.964pt}{0.700pt}}
\put(611,466){\usebox{\plotpoint}}
\put(613,467){\usebox{\plotpoint}}
\put(615,468){\rule[-0.350pt]{0.964pt}{0.700pt}}
\put(619,469){\rule[-0.350pt]{0.964pt}{0.700pt}}
\put(623,470){\rule[-0.350pt]{0.964pt}{0.700pt}}
\put(627,471){\usebox{\plotpoint}}
\put(629,472){\usebox{\plotpoint}}
\put(631,473){\rule[-0.350pt]{0.964pt}{0.700pt}}
\put(635,474){\usebox{\plotpoint}}
\put(637,475){\usebox{\plotpoint}}
\put(639,476){\rule[-0.350pt]{0.723pt}{0.700pt}}
\put(642,477){\usebox{\plotpoint}}
\put(644,478){\usebox{\plotpoint}}
\put(646,479){\rule[-0.350pt]{0.964pt}{0.700pt}}
\put(650,480){\usebox{\plotpoint}}
\put(652,481){\usebox{\plotpoint}}
\put(654,482){\usebox{\plotpoint}}
\put(656,483){\usebox{\plotpoint}}
\put(658,484){\usebox{\plotpoint}}
\put(660,485){\usebox{\plotpoint}}
\put(662,486){\usebox{\plotpoint}}
\put(664,487){\usebox{\plotpoint}}
\put(666,488){\usebox{\plotpoint}}
\put(668,489){\usebox{\plotpoint}}
\put(670,490){\usebox{\plotpoint}}
\put(672,491){\usebox{\plotpoint}}
\put(674,492){\usebox{\plotpoint}}
\put(676,493){\usebox{\plotpoint}}
\put(678,494){\usebox{\plotpoint}}
\put(679,495){\usebox{\plotpoint}}
\put(680,496){\usebox{\plotpoint}}
\put(681,497){\usebox{\plotpoint}}
\put(684,498){\usebox{\plotpoint}}
\put(686,499){\usebox{\plotpoint}}
\put(687,500){\usebox{\plotpoint}}
\put(688,501){\usebox{\plotpoint}}
\put(689,502){\usebox{\plotpoint}}
\put(692,503){\usebox{\plotpoint}}
\put(694,504){\usebox{\plotpoint}}
\put(695,505){\usebox{\plotpoint}}
\put(696,506){\usebox{\plotpoint}}
\put(697,507){\usebox{\plotpoint}}
\put(699,508){\usebox{\plotpoint}}
\put(700,509){\usebox{\plotpoint}}
\put(701,510){\usebox{\plotpoint}}
\put(703,511){\usebox{\plotpoint}}
\put(704,512){\usebox{\plotpoint}}
\put(705,513){\usebox{\plotpoint}}
\put(707,514){\usebox{\plotpoint}}
\put(708,515){\usebox{\plotpoint}}
\put(709,516){\usebox{\plotpoint}}
\put(711,517){\usebox{\plotpoint}}
\put(712,518){\usebox{\plotpoint}}
\put(713,519){\usebox{\plotpoint}}
\put(715,520){\usebox{\plotpoint}}
\put(716,521){\usebox{\plotpoint}}
\put(717,522){\usebox{\plotpoint}}
\put(718,523){\usebox{\plotpoint}}
\put(719,524){\usebox{\plotpoint}}
\put(720,525){\usebox{\plotpoint}}
\put(721,526){\usebox{\plotpoint}}
\put(722,526){\usebox{\plotpoint}}
\put(723,527){\usebox{\plotpoint}}
\put(724,528){\usebox{\plotpoint}}
\put(725,529){\usebox{\plotpoint}}
\put(725,530){\usebox{\plotpoint}}
\put(726,531){\usebox{\plotpoint}}
\put(727,532){\usebox{\plotpoint}}
\put(728,533){\usebox{\plotpoint}}
\put(730,534){\usebox{\plotpoint}}
\put(731,535){\usebox{\plotpoint}}
\put(732,536){\usebox{\plotpoint}}
\put(733,537){\usebox{\plotpoint}}
\put(734,538){\usebox{\plotpoint}}
\put(735,539){\usebox{\plotpoint}}
\put(736,540){\usebox{\plotpoint}}
\put(737,541){\usebox{\plotpoint}}
\put(738,542){\usebox{\plotpoint}}
\put(739,543){\usebox{\plotpoint}}
\put(740,544){\usebox{\plotpoint}}
\put(741,546){\usebox{\plotpoint}}
\put(742,547){\usebox{\plotpoint}}
\put(743,548){\usebox{\plotpoint}}
\put(744,549){\usebox{\plotpoint}}
\put(745,550){\usebox{\plotpoint}}
\put(746,551){\usebox{\plotpoint}}
\put(747,552){\usebox{\plotpoint}}
\put(748,553){\usebox{\plotpoint}}
\put(749,555){\usebox{\plotpoint}}
\put(750,556){\usebox{\plotpoint}}
\put(751,557){\usebox{\plotpoint}}
\put(752,558){\usebox{\plotpoint}}
\put(362,370){\usebox{\plotpoint}}
\put(362,370){\rule[-0.350pt]{3.854pt}{0.700pt}}
\put(378,369){\rule[-0.350pt]{7.468pt}{0.700pt}}
\put(409,368){\rule[-0.350pt]{5.782pt}{0.700pt}}
\put(433,367){\rule[-0.350pt]{4.818pt}{0.700pt}}
\put(453,366){\rule[-0.350pt]{3.854pt}{0.700pt}}
\put(469,365){\rule[-0.350pt]{3.613pt}{0.700pt}}
\put(484,364){\rule[-0.350pt]{2.891pt}{0.700pt}}
\put(496,363){\rule[-0.350pt]{2.891pt}{0.700pt}}
\put(508,362){\rule[-0.350pt]{1.927pt}{0.700pt}}
\put(516,361){\rule[-0.350pt]{1.927pt}{0.700pt}}
\put(524,360){\rule[-0.350pt]{1.927pt}{0.700pt}}
\put(532,359){\rule[-0.350pt]{1.927pt}{0.700pt}}
\put(540,358){\rule[-0.350pt]{1.927pt}{0.700pt}}
\put(548,357){\rule[-0.350pt]{1.927pt}{0.700pt}}
\put(556,356){\rule[-0.350pt]{0.723pt}{0.700pt}}
\put(559,355){\rule[-0.350pt]{1.927pt}{0.700pt}}
\put(567,354){\rule[-0.350pt]{0.964pt}{0.700pt}}
\put(571,353){\rule[-0.350pt]{0.964pt}{0.700pt}}
\put(575,352){\rule[-0.350pt]{1.927pt}{0.700pt}}
\put(583,351){\rule[-0.350pt]{0.964pt}{0.700pt}}
\put(587,350){\rule[-0.350pt]{0.964pt}{0.700pt}}
\put(591,349){\rule[-0.350pt]{0.964pt}{0.700pt}}
\put(595,348){\rule[-0.350pt]{0.964pt}{0.700pt}}
\put(599,347){\rule[-0.350pt]{0.964pt}{0.700pt}}
\put(603,346){\rule[-0.350pt]{0.964pt}{0.700pt}}
\put(607,345){\rule[-0.350pt]{0.964pt}{0.700pt}}
\put(611,344){\usebox{\plotpoint}}
\put(613,343){\usebox{\plotpoint}}
\put(615,342){\rule[-0.350pt]{0.964pt}{0.700pt}}
\put(619,341){\rule[-0.350pt]{0.964pt}{0.700pt}}
\put(623,340){\rule[-0.350pt]{0.964pt}{0.700pt}}
\put(627,339){\usebox{\plotpoint}}
\put(629,338){\usebox{\plotpoint}}
\put(631,337){\rule[-0.350pt]{0.964pt}{0.700pt}}
\put(635,336){\usebox{\plotpoint}}
\put(637,335){\usebox{\plotpoint}}
\put(639,334){\rule[-0.350pt]{0.723pt}{0.700pt}}
\put(642,333){\usebox{\plotpoint}}
\put(644,332){\usebox{\plotpoint}}
\put(646,331){\rule[-0.350pt]{0.964pt}{0.700pt}}
\put(650,330){\usebox{\plotpoint}}
\put(652,329){\usebox{\plotpoint}}
\put(654,328){\usebox{\plotpoint}}
\put(656,327){\usebox{\plotpoint}}
\put(658,326){\usebox{\plotpoint}}
\put(660,325){\usebox{\plotpoint}}
\put(662,324){\usebox{\plotpoint}}
\put(664,323){\usebox{\plotpoint}}
\put(666,322){\usebox{\plotpoint}}
\put(668,321){\usebox{\plotpoint}}
\put(670,320){\usebox{\plotpoint}}
\put(672,319){\usebox{\plotpoint}}
\put(674,318){\usebox{\plotpoint}}
\put(676,317){\usebox{\plotpoint}}
\put(678,316){\usebox{\plotpoint}}
\put(679,315){\usebox{\plotpoint}}
\put(680,314){\usebox{\plotpoint}}
\put(681,313){\usebox{\plotpoint}}
\put(684,312){\usebox{\plotpoint}}
\put(686,311){\usebox{\plotpoint}}
\put(687,310){\usebox{\plotpoint}}
\put(688,309){\usebox{\plotpoint}}
\put(689,308){\usebox{\plotpoint}}
\put(692,307){\usebox{\plotpoint}}
\put(694,306){\usebox{\plotpoint}}
\put(695,305){\usebox{\plotpoint}}
\put(696,304){\usebox{\plotpoint}}
\put(697,303){\usebox{\plotpoint}}
\put(699,302){\usebox{\plotpoint}}
\put(700,301){\usebox{\plotpoint}}
\put(701,300){\usebox{\plotpoint}}
\put(703,299){\usebox{\plotpoint}}
\put(704,298){\usebox{\plotpoint}}
\put(705,297){\usebox{\plotpoint}}
\put(707,296){\usebox{\plotpoint}}
\put(708,295){\usebox{\plotpoint}}
\put(709,294){\usebox{\plotpoint}}
\put(711,293){\usebox{\plotpoint}}
\put(712,292){\usebox{\plotpoint}}
\put(713,291){\usebox{\plotpoint}}
\put(715,290){\usebox{\plotpoint}}
\put(716,289){\usebox{\plotpoint}}
\put(717,288){\usebox{\plotpoint}}
\put(718,287){\usebox{\plotpoint}}
\put(719,286){\usebox{\plotpoint}}
\put(720,285){\usebox{\plotpoint}}
\put(721,284){\usebox{\plotpoint}}
\put(722,282){\usebox{\plotpoint}}
\put(723,281){\usebox{\plotpoint}}
\put(724,280){\usebox{\plotpoint}}
\put(725,280){\usebox{\plotpoint}}
\put(726,279){\usebox{\plotpoint}}
\put(727,278){\usebox{\plotpoint}}
\put(728,277){\usebox{\plotpoint}}
\put(730,276){\usebox{\plotpoint}}
\put(731,275){\usebox{\plotpoint}}
\put(732,274){\usebox{\plotpoint}}
\put(733,273){\usebox{\plotpoint}}
\put(734,272){\usebox{\plotpoint}}
\put(735,271){\usebox{\plotpoint}}
\put(736,270){\usebox{\plotpoint}}
\put(737,267){\usebox{\plotpoint}}
\put(738,266){\usebox{\plotpoint}}
\put(739,265){\usebox{\plotpoint}}
\put(740,264){\usebox{\plotpoint}}
\put(741,264){\usebox{\plotpoint}}
\put(742,263){\usebox{\plotpoint}}
\put(743,262){\usebox{\plotpoint}}
\put(744,261){\usebox{\plotpoint}}
\put(745,258){\usebox{\plotpoint}}
\put(746,257){\usebox{\plotpoint}}
\put(747,256){\usebox{\plotpoint}}
\put(748,255){\usebox{\plotpoint}}
\put(749,253){\usebox{\plotpoint}}
\put(750,252){\usebox{\plotpoint}}
\put(751,251){\usebox{\plotpoint}}
\put(752,250){\usebox{\plotpoint}}
\end{picture}

 \end{minipage}
 \ \hfill \
 \begin{minipage}{6cm}
  % GNUPLOT: LaTeX picture
\setlength{\unitlength}{0.240900pt}
\ifx\plotpoint\undefined\newsavebox{\plotpoint}\fi
\sbox{\plotpoint}{\rule[-0.175pt]{0.350pt}{0.350pt}}%
\begin{picture}(600,765)(250,0)
\tenrm
\sbox{\plotpoint}{\rule[-0.175pt]{0.350pt}{0.350pt}}%
\put(558,158){\rule[-0.175pt]{0.350pt}{119.005pt}}
\put(858,405){\makebox(0,0)[l]{$j_0$}}
\put(558,677){\makebox(0,0){amplitude}}
\put(850,405){\vector(1,0){1}}
\put(558,651){\vector(0,1){1}}
\sbox{\plotpoint}{\rule[-0.350pt]{0.700pt}{0.700pt}}%
\put(558,405){\usebox{\plotpoint}}
\put(558,405){\rule[-0.350pt]{0.700pt}{2.048pt}}
\put(559,413){\rule[-0.350pt]{0.700pt}{2.048pt}}
\put(560,422){\usebox{\plotpoint}}
\put(561,424){\usebox{\plotpoint}}
\put(562,426){\usebox{\plotpoint}}
\put(563,428){\usebox{\plotpoint}}
\put(564,430){\usebox{\plotpoint}}
\put(565,432){\usebox{\plotpoint}}
\put(566,434){\usebox{\plotpoint}}
\put(567,435){\usebox{\plotpoint}}
\put(568,436){\usebox{\plotpoint}}
\put(569,438){\usebox{\plotpoint}}
\put(570,439){\usebox{\plotpoint}}
\put(571,440){\usebox{\plotpoint}}
\put(572,442){\usebox{\plotpoint}}
\put(573,443){\usebox{\plotpoint}}
\put(574,444){\usebox{\plotpoint}}
\put(575,446){\usebox{\plotpoint}}
\put(576,447){\usebox{\plotpoint}}
\put(577,448){\usebox{\plotpoint}}
\put(578,449){\usebox{\plotpoint}}
\put(579,450){\usebox{\plotpoint}}
\put(580,451){\usebox{\plotpoint}}
\put(581,452){\usebox{\plotpoint}}
\put(582,453){\usebox{\plotpoint}}
\put(583,454){\usebox{\plotpoint}}
\put(584,455){\usebox{\plotpoint}}
\put(585,456){\usebox{\plotpoint}}
\put(587,457){\usebox{\plotpoint}}
\put(588,458){\usebox{\plotpoint}}
\put(589,459){\usebox{\plotpoint}}
\put(590,460){\usebox{\plotpoint}}
\put(591,461){\usebox{\plotpoint}}
\put(593,462){\usebox{\plotpoint}}
\put(594,463){\usebox{\plotpoint}}
\put(595,464){\usebox{\plotpoint}}
\put(596,465){\usebox{\plotpoint}}
\put(597,466){\usebox{\plotpoint}}
\put(599,467){\usebox{\plotpoint}}
\put(600,468){\usebox{\plotpoint}}
\put(602,469){\usebox{\plotpoint}}
\put(603,470){\usebox{\plotpoint}}
\put(605,471){\usebox{\plotpoint}}
\put(606,472){\usebox{\plotpoint}}
\put(608,473){\usebox{\plotpoint}}
\put(609,474){\usebox{\plotpoint}}
\put(611,475){\usebox{\plotpoint}}
\put(612,476){\usebox{\plotpoint}}
\put(614,477){\usebox{\plotpoint}}
\put(615,478){\usebox{\plotpoint}}
\put(617,479){\usebox{\plotpoint}}
\put(618,480){\usebox{\plotpoint}}
\put(620,481){\usebox{\plotpoint}}
\put(621,482){\usebox{\plotpoint}}
\put(623,483){\rule[-0.350pt]{0.723pt}{0.700pt}}
\put(626,484){\usebox{\plotpoint}}
\put(627,485){\usebox{\plotpoint}}
\put(629,486){\usebox{\plotpoint}}
\put(630,487){\usebox{\plotpoint}}
\put(632,488){\rule[-0.350pt]{0.723pt}{0.700pt}}
\put(635,489){\usebox{\plotpoint}}
\put(636,490){\usebox{\plotpoint}}
\put(638,491){\usebox{\plotpoint}}
\put(639,492){\usebox{\plotpoint}}
\put(641,493){\usebox{\plotpoint}}
\put(643,494){\usebox{\plotpoint}}
\put(644,495){\usebox{\plotpoint}}
\put(646,496){\rule[-0.350pt]{0.723pt}{0.700pt}}
\put(649,497){\usebox{\plotpoint}}
\put(650,498){\usebox{\plotpoint}}
\put(652,499){\rule[-0.350pt]{0.723pt}{0.700pt}}
\put(655,500){\usebox{\plotpoint}}
\put(656,501){\usebox{\plotpoint}}
\put(658,502){\rule[-0.350pt]{0.723pt}{0.700pt}}
\put(661,503){\rule[-0.350pt]{0.723pt}{0.700pt}}
\put(664,504){\usebox{\plotpoint}}
\put(665,505){\usebox{\plotpoint}}
\put(667,506){\rule[-0.350pt]{0.723pt}{0.700pt}}
\put(670,507){\rule[-0.350pt]{0.723pt}{0.700pt}}
\put(673,508){\usebox{\plotpoint}}
\put(674,509){\usebox{\plotpoint}}
\put(676,510){\rule[-0.350pt]{0.723pt}{0.700pt}}
\put(679,511){\rule[-0.350pt]{0.723pt}{0.700pt}}
\put(682,512){\usebox{\plotpoint}}
\put(683,513){\usebox{\plotpoint}}
\put(685,514){\rule[-0.350pt]{0.723pt}{0.700pt}}
\put(688,515){\rule[-0.350pt]{0.723pt}{0.700pt}}
\put(691,516){\rule[-0.350pt]{0.723pt}{0.700pt}}
\put(694,517){\rule[-0.350pt]{0.723pt}{0.700pt}}
\put(697,518){\usebox{\plotpoint}}
\put(698,519){\usebox{\plotpoint}}
\put(700,520){\rule[-0.350pt]{0.723pt}{0.700pt}}
\put(703,521){\rule[-0.350pt]{0.723pt}{0.700pt}}
\put(706,522){\rule[-0.350pt]{0.723pt}{0.700pt}}
\put(709,523){\rule[-0.350pt]{0.723pt}{0.700pt}}
\put(712,524){\rule[-0.350pt]{0.723pt}{0.700pt}}
\put(715,525){\usebox{\plotpoint}}
\put(716,526){\usebox{\plotpoint}}
\put(718,527){\rule[-0.350pt]{0.723pt}{0.700pt}}
\put(721,528){\rule[-0.350pt]{0.723pt}{0.700pt}}
\put(724,529){\usebox{\plotpoint}}
\put(726,530){\rule[-0.350pt]{0.723pt}{0.700pt}}
\put(729,531){\rule[-0.350pt]{0.723pt}{0.700pt}}
\put(732,532){\rule[-0.350pt]{0.723pt}{0.700pt}}
\put(735,533){\rule[-0.350pt]{0.723pt}{0.700pt}}
\put(738,534){\rule[-0.350pt]{0.723pt}{0.700pt}}
\put(741,535){\rule[-0.350pt]{0.723pt}{0.700pt}}
\put(744,536){\rule[-0.350pt]{0.723pt}{0.700pt}}
\put(747,537){\rule[-0.350pt]{0.723pt}{0.700pt}}
\put(750,538){\rule[-0.350pt]{0.723pt}{0.700pt}}
\put(753,539){\rule[-0.350pt]{0.723pt}{0.700pt}}
\put(756,540){\rule[-0.350pt]{0.723pt}{0.700pt}}
\put(759,541){\rule[-0.350pt]{0.723pt}{0.700pt}}
\put(762,542){\rule[-0.350pt]{0.723pt}{0.700pt}}
\put(765,543){\rule[-0.350pt]{0.723pt}{0.700pt}}
\put(768,544){\rule[-0.350pt]{0.723pt}{0.700pt}}
\put(771,545){\rule[-0.350pt]{0.723pt}{0.700pt}}
\put(774,546){\rule[-0.350pt]{0.723pt}{0.700pt}}
\put(777,547){\rule[-0.350pt]{0.723pt}{0.700pt}}
\put(780,548){\rule[-0.350pt]{0.723pt}{0.700pt}}
\put(783,549){\rule[-0.350pt]{0.723pt}{0.700pt}}
\put(786,550){\rule[-0.350pt]{0.723pt}{0.700pt}}
\put(789,551){\rule[-0.350pt]{0.723pt}{0.700pt}}
\put(792,552){\rule[-0.350pt]{0.723pt}{0.700pt}}
\put(795,553){\rule[-0.350pt]{0.723pt}{0.700pt}}
\put(798,554){\rule[-0.350pt]{0.723pt}{0.700pt}}
\put(801,555){\rule[-0.350pt]{0.723pt}{0.700pt}}
\put(804,556){\rule[-0.350pt]{0.723pt}{0.700pt}}
\put(807,557){\usebox{\plotpoint}}
\put(809,558){\rule[-0.350pt]{1.445pt}{0.700pt}}
\put(815,559){\rule[-0.350pt]{0.723pt}{0.700pt}}
\put(818,560){\rule[-0.350pt]{0.723pt}{0.700pt}}
\put(821,561){\rule[-0.350pt]{0.723pt}{0.700pt}}
\put(824,562){\rule[-0.350pt]{0.723pt}{0.700pt}}
\put(827,563){\rule[-0.350pt]{0.723pt}{0.700pt}}
\put(830,564){\rule[-0.350pt]{0.723pt}{0.700pt}}
\put(833,565){\rule[-0.350pt]{1.445pt}{0.700pt}}
\put(839,566){\rule[-0.350pt]{0.723pt}{0.700pt}}
\put(842,567){\rule[-0.350pt]{0.723pt}{0.700pt}}
\put(845,568){\rule[-0.350pt]{0.723pt}{0.700pt}}
\put(848,569){\rule[-0.350pt]{0.723pt}{0.700pt}}
\put(264,405){\usebox{\plotpoint}}
\put(264,405){\rule[-0.350pt]{70.825pt}{0.700pt}}
\sbox{\plotpoint}{\rule[-0.250pt]{0.500pt}{0.500pt}}%
\put(558,405){\usebox{\plotpoint}}
\put(558,405){\usebox{\plotpoint}}
\put(578,405){\usebox{\plotpoint}}
\put(599,405){\usebox{\plotpoint}}
\put(620,405){\usebox{\plotpoint}}
\put(641,405){\usebox{\plotpoint}}
\put(661,405){\usebox{\plotpoint}}
\put(682,405){\usebox{\plotpoint}}
\put(703,405){\usebox{\plotpoint}}
\put(724,405){\usebox{\plotpoint}}
\put(744,405){\usebox{\plotpoint}}
\put(765,405){\usebox{\plotpoint}}
\put(786,405){\usebox{\plotpoint}}
\put(807,405){\usebox{\plotpoint}}
\put(827,405){\usebox{\plotpoint}}
\put(848,405){\usebox{\plotpoint}}
\put(851,405){\usebox{\plotpoint}}
\end{picture}

 \end{minipage}
 \vspace{-1cm}
}
{
\caption{\protect\capsize
Den kvalitative opf{\o}rsel af det kompleks konjugerede par
af egenv{\ae}rdier $\lambda_{1,2}=\alpha\pm i\omega$.
Pilene p{\aa} de to kurver angiver den retning, i hvilken
$\lambda_{1,2}$ {\ae}ndrer sig for voksende v{\ae}rdier af
flowparameteren $j_0$. For $j_0 < j_0^{\rm Hopf}$ er
$\alpha<0$ og fikspunktet ${\bf c}_f$ er stabilt. I punktet
$j_0 = j_0^{\rm Hopf}$ krydser $\lambda_{1,2}$ $Im$-aksen,
fikspunktet mister sin stabilitet og en stabil
gr{\ae}nsecyklus er dannet svarende til $j_0 > j_0^{\rm
Hopf}$.}
\label{fig:eigval}
}

Fysisk svarer dette til en situation, hvor systemets
tidsopf{\o}rsel for $j_0 < j_0^{\rm Hopf}$ er beskrevet ved
et simpelt stabilt fikspunkt ${\bf c}_f$. For $j_0 >
j_0^{\rm Hopf}$ bliver dette fikspunkt ustabilt, samtidig
med en stabil gr{\ae}nsecyklus opst{\aa}r. Den kvalitative
opf{\o}rsel af de to kompleks konjugerede egenv{\ae}rdier
er illustreret i figur~\ref{fig:eigval}. Vi indf{\o}rer nu
f{\o}lgende notation for de egenvektorer og
egenv{\ae}rdier, der er til\-knyt\-tet Jacobimatricen. Lad
$\lambda_1,\ldots,\lambda_n$ v{\ae}re de $n$
egenv{\ae}rdier og lad ydermere disse v{\ae}re ordnet efter


\begin{equation}
 Re \lambda_n \leq \ldots \leq Re \lambda_1
\end{equation}

Specielt ser vi, at t{\ae}t p{\aa} Hopfbifurkationen vil 
$\lambda_{1,2}$ v{\ae}re af formen

\begin{equation}
 \lambda_{1,2} = \alpha \pm i\omega
\end{equation}

De egenvektorer, der er tilknyttet egenv{\ae}rdierne,
symboliseres ved ${\bf e}_1,\ldots,{\bf e}_n$. Her svarer
vektorerne ${\bf e}_1$ og ${\bf e}_2$ til henholdsvis real-
og imagin{\ae}rdel af de to komplekst konjugerede
egenvektorer ${\bf e}_{\pm}$ h{\o}rende til
egenv{\ae}rdierne $\lambda_{1,2}$. Idet den oscillerende
dynamik for{\aa}rsages af disse komplekst konjugerede
egenv{\ae}rdier $\lambda_{1,2} = \alpha \pm i\omega$,
indf{\o}res endvidere f{\o}lgende notation for ${\bf
e}_{\pm}$

\begin{equation}
 {\bf e}_{\pm} = {\bf e}_1 \mp i{\bf e}_2 = {\bf u} \mp i{\bf v}
\end{equation}

For $|j_0-j_0^{\rm Hopf}|<\epsilon$, hvor $\epsilon$ er
lille, vil systemets tidsopf{\o}rsel ${\bf c}(t)$ med god
tiln{\ae}rmelse kunne approksimeres ved udtrykket

\begin{equation}
 {\bf c}(t) = {\bf c}_f + a \left[ {\bf u}\cos \omega t + {\bf v}\sin \omega t \right]
            = {\bf c}_f + {\bf r}(t)
 \label{eq:oscdyn}
\end{equation}

V{\ae}lges nu et referencestof karakteriseret ved indeks
{\em ref\/} s{\aa}ledes, at ${\bf u}$ og ${\bf v}$ bliver
nomeret efter

\begin{equation}
 u_{\em \/} = 1  \mbox{\ \ og\ \ } v_{\em \/} = 0
\end{equation}

ser vi, at ligning~\ref{eq:oscdyn} netop angiver
tidsudviklingen for koncentrationen af de $n$ kemiske
stoffer s{\aa}ledes, at disses faser alle er beskrevet
relativt til reference\-stoffets fase. Endvidere
medf{\o}rer den valgte normering, at skalaren $a$ svarer
til sving\-ningsamplituden for det valgte referencestof.
V{\ae}lger vi nu at m{\aa}le samtlige af de $n$
koncentrationer i enheder af denne referenceamplitude,
finder vi slutteligt, at oscillationerne for de
indg{\aa}ende koncentrationer er beskrevet ved udtrykket,

\begin{equation}
 {\bf r}(t) = \left[ {\bf u}\cos \omega t + {\bf v}\sin \omega t \right]
 \label{eq:oscdynnom}
\end{equation}

hvor udtrykket er yderligere simplificeret, idet vi har valgt 
fikspunktet ${\bf c}_f$ som koncentrationrummets origo. Omvendt
kan tidsudviklingen for hver af de $n$ stoffer ogs{\aa} beskrives
ved udtrykket 

\begin{equation}
 r_k(t) = a_k \cos (\omega t-\theta_k)
 \label{eq:oscdynfase}
\end{equation}

hvor $a_k$ og $\theta_k$ beskriver henholdsvis amplituden af 
det $k$'te stof m{\aa}lt i enheder af referencestoffets amplitude
og sving\-ningsfasen af det $k$'te stof i forhold til fasen af
referencestoffet. Sammenholdes ligning~\ref{eq:oscdynnom} med 
ligning~\ref{eq:oscdynfase} ses, at vektorerne ${\bf u}$ og 
${\bf v}$ entydigt fastl{\ae}gger de skal{\ae}re st{\o}rrelser $a_k$ og 
$\theta_k$ som

\begin{subequations}
 \begin{eqalignno}
  a_k       &= {\rm abs}(u_k+iv_k) \\
  \theta_k  &= \arg (u_k+iv_k)
 \end{eqalignno}
 der yderligere kan reduceres til det simple udtryk
 \begin{equation}
  u_k+iv_k = a_k e^{i\theta_k}
 \end{equation}
\label{eq:uv_atheta}
\end{subequations}

Disse ligninger viser, at vektorerne ${\bf u}$ og ${\bf v}$
entydigt fastl{\ae}gger tidsvariationen af de $n$ kemiske stoffer.

%%%%%%%%%%%%%%%%%%%%%%%%%%%%%%%%%%%%%%%%%%%%%%%%%%%%%%%%%%%%%%%%%%%%%%%%
%% figur
%%
%% beskrivelse : Skematisk quenching
%% type        : PSTricks
%%%%%%%%%%%%%%%%%%%%%%%%%%%%%%%%%%%%%%%%%%%%%%%%%%%%%%%%%%%%%%%%%%%%%%%%
\boxfigure{t}{\textwidth}
{
\newlength{\mylength}
\begin{center}
  \begin{pspicture}(0,0)(14,7)
   \psellipse[linewidth=1.2pt,
              border=1pt,
              fillstyle=solid,
              fillcolor=lightgray](7.0,3.5)(2.5,1.0)
%   \psgrid[](0,0)(0,0)(14,7)
   \setlength{\mylength}{-0.155cm}
   \psline[origin={-7,-3.5},
           linewidth=1.2pt,
           linestyle=dashed,
           dash=3pt 2pt]{-}(8\mylength,5\mylength)
   \setlength{\mylength}{0.324cm}
   \psline[linewidth=1.2pt,origin={-3,-1}]{-}(8\mylength,5\mylength)
   \psline[linewidth=1.2pt]{-}(7.0,3.5)(11.0,6.0)
   \psline[linewidth=0.8pt,arrowinset=0]{->}(9.49,3.5)(9.49,5.05)
   \psline[linewidth=0.8pt,arrowinset=0]{->}(7.0,3.5)(8.0,2.6)
   \psline[linewidth=0.8pt,arrowinset=0]{->}(7.0,3.5)(4.6,3.2)
   \pscircle*[](7,3.5){0.075}
   \rput[lc]{*0}(9.6,4.25){\footnotesize ${\bf q}$}
   \rput[bc]{*0}(7.6,3.10){\footnotesize ${\bf u}$}
   \rput[bc]{*0}(5.8,3.42){\footnotesize ${\bf v}$}
   \rput[bl]{*0}(6.7,3.56){\footnotesize ${\bf c}_f$}
   \rput[bl]{*0}(11.1,6.1){\footnotesize $E^s$}
  \end{pspicture}
\end{center}
}
{
\caption{\protect\capsize
Figuren illustrere skematisk det teoretiske grundlag bag en
quenching. Gr{\ae}nsecyklusen, der er genereret via en
subkritisk Hopfbifurkation, udsp{\ae}ndes af real- og
imagin{\ae}rdelen ${\bf u}$ og ${\bf v}$ af den komplekse
egenvektor. Ved tils{\ae}tning af en stof\-blanding,
karakteriseret ved vektoren ${\bf q}$, flyttes sytstemet i
faserummet fra gr{\ae}nsecyklusen til det stabile underrum
$E^s$. I faserummet vil systemet nu bev{\ae}ge sig langs
$E^s$ mod det station{\ae}re punkt ${\bf c}_f$, men da
$E^s$ aldrig kan rammes n{\o}jagtigt, vil systemmet atter
spiralerer ud til gr{\ae}nsecyklusen. Vi betegner et
s{\aa}dant midlertidigt stop af oscillationerne som {\em en
quenching\/}. }
\label{fig:skemaquenching}
}
\normalsize

\vspace{4.0mm}
De oscillationer, vi observerer t{\ae}t p{\aa} den
superkritiske Hopfbifurkation, svarer i
koncentrationsrummet til en elliptisk gr{\ae}nsecyklus
liggende i et sving\-ningsplan udsp{\ae}ndt af vektorerne
${\bf u}$ og ${\bf v}$. T{\ae}t p{\aa} bifurkationspunktet
vil dette sving\-ningsplan tiln{\ae}rme den ustabile
mangfoldighed $W^u$ til det nu ustabile station{\ae}re
punkt ${\bf c}_f$. Ydermere vil dette fikspunkts stabile
mangfoldighed $W^s$ kunne tiln{\ae}rmes ved spannet af de
reelle egenvektorer ${\bf e}_3,\ldots,{\bf e}_n$. Det
line{\ae}re underrum, der udsp{\ae}ndes af disse reelle
egenvektorer, kaldes det stabile underrum og betegnes ofte
$E^s$.

\vspace{4.0mm}
Lad os nu antage, at en blanding af stofferne i en given
fase $\phi_k$ med en given m{\ae}ngde ${\bf q}_k$
tils{\ae}ttes s{\aa}ledes, at systemets position i
faserummet flyt\-tes fra svigningsplanet til et punkt i det
stabile underrum $E^s$. Eftersom enhver bev{\ae}gelse i
$E^s$ er karakteriseret ved en linearkombination af
eksponentielle henfald langs de reelle egenvektorer ${\bf
e}_3,\ldots,{\bf e}_n$, vil denne situation betyde, at
koncentrations sving\-ningerne stoppes. Vi kalder et
s{\aa}dant stop for en {\em quenching}\footnote{{\em
Quench}: stoppe, slukke, stille, d{\ae}mpe ell.
undertrykke}, karakteriseret ved henholdsvis
quenchingvektoren ${\bf q}_k$ og quenching\-fasen $\phi_k$.

\vspace{4.0mm}
Da enhver vektor i $E^s$ aldrig vil kunne rammes eksakt i
et s{\aa}dant quenchingeks\-peri\-ment, vil quenchingen
v{\ae}re af midlertidig karakter, hvorfor systemet efter et
vist tidsrum atter spiralerer tilbage i sving\-ningsplanet.
Den teoretiske id\'{e} bag quenching eksperimentet samt
quenchingens karakteristiske eksperimentelle manifestation
er illustreret henholdsvis i figur~\ref{fig:skemaquenching}
og \ref{fig:eksquenching}.

\boxfigure{t}{\textwidth}{
 \vspace{5cm}
}
{
\caption{\protect\capsize
         Eksperimentel quenching udf{\o}rt i BZ-reaktionen
         udf{\o}rt ved tils{\ae}tning af HBrO$_2$.}
\label{fig:eksquenching}
}

\vspace{4.0mm}
Matematisk kan kriteriet for en succesfuld quenching
formuleres pr{\ae}cist. Lad ${\bf q}_k$ og $\phi_k$
v{\ae}re henholdsvis quenchingvektor og -fase i en succesiv
serie af eksperimenter. Lad ydermere ${\bf t}$ v{\ae}re en
vektor i det stabile underrum $E^s$. Quenchingbetingelsen
kan da udtrykkes som

\begin{equation}
 {\bf q}_k + {\bf r}(\phi_k) = {\bf t},
 \mbox{\ \ hvor\ \ } {\bf t} \in E^s.
 \label{eq:quenchkrit}
\end{equation}

Antag nu at $m$ quenchingeksperimenter er udf{\o}rt. Hertil
defineres matricen ${\bf Q}$, hvis s{\o}jler svarer til de
$m$ quenchingvektorer ${\bf q}_k$. Elementet $q_{jk}$ i
${\bf Q}$ beskriver dermed den m{\ae}ngde stof, der
benyttedes af det $j$'te stof ved den $k$'te quenching. Det
viser sig at v{\ae}re mere fordelagtigt at beskrive
quenchingvektorerne i den basis, der udsp{\ae}ndes af
egenvektorerne ${\bf e}_1,\ldots,{\bf e}_n$. Definerer vi
matricerne ${\bf P}$ og ${\bf \Lambda}$

\begin{equation}
 {\bf P} = \left[{\bf e}_1\ldots{\bf e}_n\right]
 \mbox{\ \ og\ \ }
 {\bf \Lambda} = 
 \left[
 \begin{array}{ccc}
  \lambda_1 & & \\
  & \ddots &    \\
  & & \lambda_n 
  \end{array}
 \right] 
\end{equation}

ser vi udfra overvejelserne

\begin{equation}
 \begin{array}{rcl}
             {\bf J}{\bf P} & = &             {\bf P}{\bf \Lambda}\\
 {\bf P}^{-1}{\bf J}{\bf P} & = &                    {\bf \Lambda}\\
 {\bf P}^{-1}{\bf J}        & = & {\bf \Lambda}{\bf P}^{-1}
 \end{array}
\end{equation}

at r{\ae}kkerne ${\bf e}^k$ i matricen ${\bf P}^{-1}$ er 
venstreegenvektorer til Jacobimatricen $\bf J$. Disse $n$ r{\ae}kker 
udg{\o}r en s{\aa}kaldt biorthogonal basis til basen af reelle
egenvektorer, idet f{\o}lgende egenskab oplagt er opfyldt,

\begin{equation}
{\bf e}^j \cdot {\bf e}_k = \delta_{jk}
\label{eq:kronecker}
\end{equation}

hvor $\delta_{jk}$ er det s{\ae}dvanlige Kronecker delta.
Vi ser, at hver af disse venstre\-egenvektorer ${\bf e}^k$
definerer en projektion\footnote{Denne egenskab ved ${\bf
e}^j$ indses umiddelbart ved f{\o}lgende overvejelse. Lad
${\bf x}$ v{\ae}re en vektor i koncentrationsrummet, dvs.\
${\bf x} = \sum_{k=1}^n a_k{\bf e}_k$. Vi har da, at ${\bf
e}^j \cdot {\bf x} = \sum_{j=1}^n a_k {\bf e}^j \cdot {\bf
e}_k = a_j$, hvilket netop er lig komponenten af ${\bf x}$
langs med vektoren ${\bf e}_j$.} af enhver vektor i
koncentrationsrummet p{\aa} h{\o}jreegenvektoren ${\bf
e}_k$. Generelt finder vi, at komponenten $R_{jk}$ af den
$k$'te quenchingvektor langs den $j$'te h{\o}jreegenvektor
bliver

\begin{equation}
 R_{jk} = {\bf e}^j \cdot {\bf q}_k 
\end{equation}

Specielt findes quenchingvektorernes komponenter i 
svigningsplanet som,

\begin{subequations}
 \begin{eqalignno}
  R_{1k}  &= {\bf e}^1 \cdot {\bf q}_k = 
             {\bf e}^1 \cdot \left[ {\bf t} - {\bf r}(\phi_k) \right]
             = -\cos \phi_k\\
  R_{2k}  &= {\bf e}^2 \cdot {\bf q}_k = 
             {\bf e}^2 \cdot \left[ {\bf t} - {\bf r}(\phi_k) \right]
             = -\sin \phi_k
 \end{eqalignno}
 \label{eq:oscplankomp}
\end{subequations}

hvor vi har udnyttet venstreegenvektorernes biorthogonale egenskab.
Af lig\-ning\-erne

\begin{subequations}
 \begin{eqalignno}
  \left[ e^{11}\ldots e^{1n} \right]{\bf Q} &=
  \left[ \cos\phi_1\ldots\cos\phi_n \right] \\
  \left[ e^{21}\ldots e^{2n} \right]{\bf Q} &=
  \left[ \sin\phi_1\ldots\sin\phi_n \right] 
 \end{eqalignno}
 \label{eq:genquench}
\end{subequations}

ses, at $n$ line{\ae}rt uafh{\ae}ngige quenchinger entydigt
fastl{\ae}gger vektorerne ${\bf e}^1$ og ${\bf e}^2$. Dette
resultat er generelt, men tillader ofte anvendelsen af en
v{\ae}sentlig simplificering, idet quenchingfors{\o}g af
praktiske grunde ofte udf{\o}res ved tils{\ae}tning af
opl{\o}sninger, der kun indeholder et enkelt af de $n$
stoffer. I denne sammenh{\ae}ng taler man om quenching
langs med koncentrationsakserne. I dette tilf{\ae}lde
reduceres indholdet af quenchingvektoren til en skalar
$q_k$, hvorfor lig\-ning~\ref{eq:genquench} kan reduceres til

\begin{subequations}
 \begin{eqalignno}
  e^{1k} &= -\frac{\cos\phi_1}{q_k} \\
  e^{2k} &= -\frac{\sin\phi_1}{q_k} 
 \end{eqalignno}
 \label{eq:simquench}
\end{subequations}

I analogi med vektorerne ${\bf u}$ og ${\bf v}$, der
fastlagde amplituderne og faserne af stofferne, ses en
lignende sammenh{\ae}ng at g{\o}re sig g{\ae}ldende mellem
venstre\-egenvektorerne ${\bf e}^1$, ${\bf e}^2$ og
quenchingvektorene og quenching\-faserne. Vi finder nemlig
f{\o}lgende sammenh{\ae}ng

\begin{subequations}
 \begin{eqalignno}
  q_k       &= {\rm abs}(-e^{1k}-ie^{2k}) \\
  \phi_k    &= \arg (-e^{1k}-ie^{2k})
 \end{eqalignno}

  der analogt med lig\-ning~\ref{eq:uv_atheta} yderligere
  kan reduceres til det simplere udtryk

 \begin{equation}
  -e^{1k}-ie^{2k} = q_k e^{i\phi_k}
 \end{equation}
\label{eq:e1e2_qphi}
\end{subequations}

%%%%%%%%%%%%%%%%%%%%%%%%%%%%%%%%%%%%%%%%%%%%%%%%%%%%%%%%%%%%%%%%%%%%%%%%
%% figur
%%
%% beskrivelse : Skematisk quenching ved fortynding
%% type        : PSTricks
%%%%%%%%%%%%%%%%%%%%%%%%%%%%%%%%%%%%%%%%%%%%%%%%%%%%%%%%%%%%%%%%%%%%%%%%
\newsavebox{\psfigure}
\sbox{\psfigure}{
 \psset{xunit=0.7cm,yunit=0.8cm}
 \newlength{\x} \newlength{\y}
 \begin{pspicture}(0,0)(14,7)
   \psellipse[linewidth=1.2pt,
              border=1pt,
              fillstyle=solid,
              fillcolor=lightgray](7.0,3.5)(2.5,1.0)
%   \psgrid[](0,0)(0,0)(14,7)
   \setlength{\mylength}{-0.155cm}
   \setlength{\x}{0.7\mylength}
   \setlength{\y}{0.8\mylength}
   \psline[origin={-7,-3.5},
           linewidth=1.2pt,
           linestyle=dashed,
           dash=3pt 2pt]{-}( 8\x,5\y)
   \setlength{\mylength}{0.324cm}
   \setlength{\x}{0.7\mylength}
   \setlength{\y}{0.8\mylength}
   \psline[linewidth=1.2pt,origin={-3,-1}]{-}(8\x,5\y)
   \psline[linewidth=1.2pt]{-}(7.0,3.5)(11.0,6.0)
   \psline[linewidth=0.8pt,arrowinset=0]{->}(7.0,3.5)(7.0,4.5)
   \psline[linewidth=0.8pt,arrowinset=0]{->}(7.0,3.5)(4.6,3.2)
   \pscircle*[](7,3.5){0.075}
   \rput[lc]{*0}(7.1,4.00){\footnotesize ${\bf u}$}
   \rput[bc]{*0}(5.8,3.42){\footnotesize ${\bf v}$}
   \rput[bl]{*0}(11.1,6.1){\footnotesize $E^s$}
  \end{pspicture}
}
\boxfigure{t}{\textwidth}
{
\begin{center}
  \begin{pspicture}(0,0.4)(14,7.4)
   \rput[cc]{*0}(7.48,4.5){\usebox{\psfigure}}
   \psset{xunit=1cm,yunit=1cm}
%   \psgrid[](0,0)(0,0)(14,7)
   \psline[linewidth=0.8pt,arrowinset=0]{->}(2,2)(8,1)
   \psline[linewidth=0.8pt,arrowinset=0]{->}(2,2)(2,6)
   \psline[linewidth=0.8pt,arrowinset=0]{->}(2,2)(5,5)
   \setlength{\mylength}{-0.3cm}
   \psline[linewidth=0.8pt,
           linestyle=dotted,
           dotsep=2pt,
           origin={-9.4,-4.5}]{-}(7.4\mylength,2.5\mylength)
   \setlength{\mylength}{-0.12cm}
   \psline[linewidth=0.8pt,
           arrowinset=0,
           origin={-7.106,-3.725}]{->}(7.4\mylength,2.5\mylength)
   \psline[linewidth=0.8pt,
           linestyle=dotted,
           dotsep=2pt]{-}(2,2)(6.218,3.425)
   \rput[bc]{*0}(8.0,4.1){\footnotesize ${\bf q}$}
   \rput[cl]{*0}(8.1,1.0){\tiny [HBrO$_2$]}
   \rput[bc]{*0}(2.0,6.1){\tiny [Ce$^{4+}$]}
   \rput[bl]{*0}(5.1,5.1){\tiny [Br$^-$]}
  \end{pspicture}
\end{center}
}
{
\caption{\protect\capsize
Skematisk illustration af grundlaget for at udf{\o}\-re
quenchingeksperimenter ved fortynding af
reaktionsblandingen. Ved en tils{\ae}tning af
opl{\o}sningsmidlet (ofte vand) flyttes sytstemet fra et
givet punkt p{\aa} gr{\ae}nsecyklusen mod origo i
faserummet. For en bestemt fase og st{\o}rrelse af denne
fortyndingsperturbation vil systemet ramme det stabile
underum $E^s$ svarende til, at oscillationerne midlertidigt
stoppes. Vi betegner denne situation som {\em quenching ved
fortynding\/}.}
\label{fig:dilquenching}
}

Disse hovedresultater foruds{\ae}tter alle, at quenchingen
af den kemiske oscillator foretages ved en tils{\ae}tning
af enten et eller flere af de stoffer, der deltager i
reaktionsforl{\o}bet. Der eksisterer dog ogs{\aa} en
mulighed for at quenche de kemiske sving\-ninger ved en
simpel fortynding med reaktionssolventet. Geo\-metrisk
svarer dette til en bev{\ae}gelse i koncentrationsrummet
langs med vektoren ${\bf c}_f+{\bf r}(t)$ med retning mod
koncentrationsrummets nulvektor, som illustreret i
figur~\ref{fig:dilquenching}. Betingelsen for at quenche
ved fortynding kan udtrykkes som

\begin{equation}
 {\bf q}_d = -d({\bf c}_f+{\bf r}(\phi_d)) = {\bf t}-{\bf r}(\phi_d)
\end{equation}

hvor $d$ er den relative fortynding. For at simplificere 
notationen s{\ae}ttes 

\begin{eqnarray}
g_d = -{\bf e}^1 \cdot {\bf c}_f \nonumber\\
h_d = -{\bf e}^2 \cdot {\bf c}_f \nonumber
\end{eqnarray}

Vi kan nu, helt analogt med den forrige fremgansm{\aa}de, beskrive
quenchingvektorens komponenter i sving\-ningsplanet som

\begin{subequations}
 \begin{eqalignno}
  {\bf e}^1 \cdot {\bf q}_d &= 
  -d{\bf e}^1 \cdot ({\bf c}_f+{\bf r}(\phi_d)) = -\cos\phi_d \\
  {\bf e}^2 \cdot {\bf q}_d &= 
  -d{\bf e}^2 \cdot ({\bf c}_f+{\bf r}(\phi_d)) = -\sin\phi_d 
 \end{eqalignno}
\end{subequations}

Disse lig\-ninger kan ved hj{\ae}lp af en ekstra udnyttelse
af venstreegenvektorernes bi\-orthogonalitet samt nogle
f{\aa} simple aritmetriske manipulationer yderligere
omskrives til

\begin{subequations}
 \begin{eqalignno}
  d g_d = (d-1)\cos\phi_d \\
  d h_d = (d-1)\sin\phi_d 
 \end{eqalignno}
\end{subequations}

Heraf indser man f{\o}rst og fremmest

\begin{equation}
 d = \frac{1}{1-\sqrt{g_d^2+h_d^2}}
\end{equation}

hvorfor vi som endeligt resultat finder

\begin{subequations}
 \begin{eqalignno}
  {\bf q}_d = -\frac{{\bf c}_f+{\bf r}(\phi_d)}{1-\sqrt{g_d^2+h_d^2}}\\
  \phi_d    =  \arg (-g_d-ih_d)  
 \end{eqalignno}
 \label{eq:dilquench}
\end{subequations}

Afslutningsvis runder vi dette teoretiske afsnit af med at
beskrive, hvilken yderligere information der kan opn{\aa}s
ved at udf{\o}re ekstra quenching\-eks\-peri\-menter, efter
at $n$ line{\ae}rt uafh{\ae}ngige quenchingvektorer og
dertil h{\o}r\-ende faser m{\aa}tte v{\ae}re bestemt. Lad
os derfor antage, at vi eksperimentelt har udf{\o}rt en
s{\aa}dan ekstra quenching svarende til quenchingvektoren
${\bf q}_{ex}$ og -fasen $\phi_{ex}$. Da $n$ line{\ae}rt
uafh{\ae}ngige allerede er bestemt, kan ${\bf q}_{ex}$
skrives som en linearkombination af disse, dvs.

\begin{equation}
 {\bf q}_{ex} = \sum_{k=1}^n a_k{\bf q}_k
\end{equation}

I egenbasen ${\bf e}_1,\ldots,{\bf e}_n$ beskrives
den ekstra quenching som 

\begin{equation}
 {\bf R}_{ex} = 
 {\bf P}{\bf q}_{ex} = 
 {\bf P}\sum_{k=1}^n a_k{\bf q}_k =
 \sum_{k=1}^n a_k{\bf R}_k
\end{equation}

Idet $R_{ex}^1$ og $R_{ex}^2$ skal opfylde kravet
$(R_{ex}^1)^2+(R_{ex}^2)^2 = 1$, fastl{\ae}gges
koefficienterne $a_1,\ldots,a_n$ udfra normeringskravet

\begin{equation}
 \left[ \sum_{k=1}^n a_k R_{1k} \right]^2 +
 \left[ \sum_{k=1}^n a_k R_{2k} \right]^2 = 1
\end{equation}

Ydermere finder vi udfra lig\-ning~\ref{eq:oscplankomp} f{\o}lgende
sammenh{\ae}ng mellem quenching\-fasen $\phi_{ex}$ og de $n$ andre
faser $\phi_k$

\begin{subequations}
 \begin{eqalignno}
  \cos\phi_{ex} &= \sum_{k=1}^n a_k \cos\phi_k\\
  \sin\phi_{ex} &= \sum_{k=1}^n a_k \sin\phi_k
 \end{eqalignno}
 \label{eq:addquench}
\end{subequations}

Disse overvejelser viser, at en ekstra quenching ikke
indeholder nogen yderligere information, idet dennes
st{\o}rrelse og fase p{\aa} forh{\aa}nd er bestemt udfra de
$n$ line{\ae}rt uafh{\ae}ngige quenchinger. P{\aa} trods af
dette faktum, har resultatet dog en v{\ae}sentlig og meget
praktisk anvendelse. {\O}nsker vi for eksempel, i det
tilf{\ae}lde hvor de $n$ forrige quenchinger er udf{\o}rt
langs koordinatakserne, at udf{\o}re en quenching ved
samtidig tils{\ae}tning af lige store m{\ae}ngder af to
stoffer ($a_1=a_2$), ses lig\-ning~\ref{eq:addquench} at
reducere til

\begin{subequations}
 \begin{eqalignno}
  \cos\phi_{ex} &= a_1 (\cos\phi_1 + \cos\phi_2)\\
  \sin\phi_{ex} &= a_1 (\sin\phi_1 + \sin\phi_2)
 \end{eqalignno}
\end{subequations}

hvorfor $\phi_{ex}$ ved division af $\sin\phi_{ex}$ med
$\cos\phi_{ex}$ og anvendelse af de tri\-go\-no\-me\-triske
additionsformler findes som

\begin{eqalignno}
 \phi_{ex} = \frac{\phi_1 + \phi_2}{2}+p\pi, 
 \mbox{\ \ hvor\ \ } p\in \mbox{\Bb Z}
\end{eqalignno}

Heraf ses f{\o}lgende: Hvis to stoffer quencher med
m{\ae}ngderne ${\bf q}_1$ og ${\bf q}_2$ med tilh{\o}rende
faser $\phi_1$ og $\phi_2$, da findes et reelt tal $a$,
s{\aa}ledes at tils{\ae}tningen $a({\bf q}_1+{\bf q}_2)$
vil quenche i en fase svarende til middelv{\ae}rdien af
$\phi_1$ og $\phi_2$.

\vspace{4.0mm}
Da enhver eksperi\-mentel anvendelse af quenchingteorien
foruds{\ae}tter en antag\-else om koncentrationsrummets
dimension, angiver dette resultat en vigtig m{\aa}de,
hvorp{\aa} rigtigheden af denne antagelse kan afpr{\o}ves i
praksis. Ofte vil man her finde en d{\aa}rlig
korrespondance mellem teori og praksis svarende til et alt
for lavt esti\-mat af systemets dimension. For
model\-beregninger vil s{\aa}danne udregninger
selvf{\o}lgelig ikke tilvejebringe nogen interessant
information, da dimensionen af model\-len i sagens natur er
kendt p{\aa} forh{\aa}nd. Vi skal senere se en r{\ae}kke
eksempler p{\aa}, hvorledes denne metode har v{\ae}ret
anvendt i eksperimentelt arbejde.

\section{Eksperimentel rekonstruktion af kemiske oscillationer}
Som n{\ae}vnt tidligere, tilbyder
quenching\-eksperi\-menter umiddelbart et
sam\-men\-lig\-ningsgrundlag mellem model\-ler for den
kemiske reaktion og de eksperimentelle data. Teorien
tillader derudover en anden v{\ae}sentlig anvendelse, idet
quenchingdata kan udnyttes til at rekonstruere en r{\ae}kke
af de geo\-metriske st{\o}rrelser, der karakteriserer
systemets indlejring og opf{\o}rsel i
kon\-cen\-tra\-tions\-rummet.

\vspace{4.0mm}
Lad os derfor antage, at den unders{\o}gte kemiske reaktion
har dimensionen $n$. Yderligere antages det, at
koncentrationen af $n-2$ af disse $n$ stoffer kan m{\aa}les
eksperimentelt. Med andre ord kan et passende referencestof
udv{\ae}lges blandt disse $n-2$ stoffer, hvorved amplituden
og fasen af de resterende $n-3$ stoffer kan m{\aa}les i
forhold til referencestoffet efter de tidligere beskrevne
retningslinier. Udfra lig\-ning~\ref{eq:uv_atheta} ser vi
alts{\aa}, at opfyldelsen af de n{\ae}vnte kriterier
medf{\o}rer, at komponenterne i vektorerne ${\bf u}$ og
${\bf v}$ kan bestemmes bortset fra to komponenter,
svarende til de to restende stoffer, hvis koncentration
eksperimentelt ikke er m{\aa}lelig.

\vspace{4.0mm}
Udf{\o}res nu $n$ quenchingeksperimenter, svarende til en
bestemmelse af $n$ quen\-ching\-amplituder og $n$
tilh{\o}rende faser, giver dette anledning til, at
venstreegenvektorerne ${\bf e}^1$ og ${\bf e}^2$ kan
bestemmes. Sammenholder vi den opn{\aa}ede information med
lig\-ning~\ref{eq:kronecker}, ser vi, at betingelserne

\begin{equation}
 \begin{array}{rcrcl}
  {\bf e}^1 \cdot {\bf u} & = & {\bf e}^1 \cdot {\bf e}_1 & = & 1\\
  {\bf e}^1 \cdot {\bf v} & = & {\bf e}^1 \cdot {\bf e}_2 & = & 0\\
  {\bf e}^2 \cdot {\bf u} & = & {\bf e}^2 \cdot {\bf e}_1 & = & 0\\
  {\bf e}^2 \cdot {\bf v} & = & {\bf e}^2 \cdot {\bf e}_2 & = & 1
 \end{array}
 \label{eq:uvdet}
\end{equation}

pr{\ae}cist fastl{\ae}gger de ubekendte komponenter i
vektorerne ${\bf u}$ og ${\bf v}$. Disse resultater viser
alts{\aa}, at hvis koncentrationen af $n-2$ stoffer er
eksperimentel m{\aa}lelig, da kan sving\-ningsplanet ${\rm
sp}\{{\bf u},{\bf v}\}$ i koncentrationsrummet bestemmes
udfra $n$ quenchingeksperimenter. Endvidere fastl{\ae}gger
de m{\aa}lte quenchingdata venstreegenvektorerne, hvorfor
transientrummet $E^s$, udsp{\ae}ndt af de reelle
egenvektorer, er bestemt udfra
lig\-ning~\ref{eq:kronecker}.

\vspace{4.0mm}
Vi b{\o}r dog bem{\ae}rke, at basen ${\bf e}_3,\ldots,{\bf
e}_n$, der her er benyttet til at beskrive $E^s$, i det
generelle tilf{\ae}lde ikke er tilg{\ae}ngelig udfra
quenchingdata\footnote{I tilf{\ae}ldet $n=3$ kan ${\bf
e}_3$ dog godt bestemmes, idet transientrummet i dette
tilf{\ae}lde opfylder $\dim E^s=1$.}. Ydermere ser vi, at
\'{e}n ekstra quenching, udf{\o}rt ved fortynding af
opl{\o}sningsmidlet, vil bidrage til at fastl{\ae}gge
v{\ae}rdien for det station{\ae}re punkt ${\bf c}_f$,
hvilket fremg{\aa}r af lig\-ning~\ref{eq:dilquench}.

\vspace{4.0mm}
Da rekonstruktionen tillader en beskrivelse af variationen
af koncentrationen af de $n$ stoffer p{\aa}
gr{\ae}nsecyklusen ${\bf r}(t)$, vil samtlige komponenter i
fortyndingsvektoren ${\bf q}_d$ v{\ae}re kendt.
Ligning~\ref{eq:dilquench} reducerer hermed til to
lig\-ninger med to ube\-kend\-te, svarende til de
station{\ae}re koncentrationer af de to stoffer, der
eksperimentelt ikke er m{\aa}lelige. Hermed er det
station{\ae}re punkt ${\bf c}_f$ fuldst{\ae}ndigt bestemt.

\vspace{4.0mm}
Vi afslutter disse to forholdsvis lange afsnit med en kort
opsummering af resultaterne i form af to skematiske
strategier for henholdsvis model\-beregniger af
quenchingdata og eksperimentel rekonstruktion udfra
quenchingdata.

\subsubsection{Model beregninger}
\begin{itemize}
\begin{enumerate}
 \item Udregn v{\ae}rdien af $j_0^{\rm Hopf}$ og bestem
  Jacobimatricen ${\bf J}$.
 \item Udregn ${\bf e}_1,\ldots,{\bf e}_n$ og norm\'{e}r
  disse, idet et passende referencestof udv{\ae}lges.
 \item Bestem ${\bf P}^{-1}$ og dermed venstreegenvektorerne
  ${\bf e}_1,\ldots,{\bf e}_n$. Quenchingvektorerne ${\bf
  q}_k$ og -faserne $\phi_k$ kan dermed bestemmes udfra
  lig\-ning~\ref{eq:simquench}.
 \item For at bestemme data for fortyndingseksperimenter udregnes
  det station{\ae}re punkts v{\ae}rdi ${\bf c}_f(j_0^{\rm
  Hopf})$ i Hopfbifurkationspunktet, hvor\-ved ${\bf q}_d$ og
  $\phi_d$ kan beregnes udfra lig\-ning~\ref{eq:dilquench}.
\end{enumerate}
\end{itemize}

\subsubsection{Eksperimentel rekonstruktion}
\begin{itemize}
\begin{enumerate}
 \item Udv{\ae}lg et passende referencestof. $n-2$ af komponenterne
       i ${\bf u}$ og ${\bf v}$ kan nu bestemmes.
 \item Udfra $n$ quenchingeksperimenter udregnes nu 
       ventreegenvektorerne ${\bf e}^1$ og ${\bf e}^2$ 
       ved hj{\ae}lp af lig\-ning~\ref{eq:simquench}.
 \item De resterende to komponenter i ${\bf u}$ og ${\bf v}$
       udregnes udfra lig\-ning~\ref{eq:uvdet}, hvilket 
       fastl{\ae}gger sving\-ningsplanet.
 \item Udf{\o}res en ekstra quenching ved fortynding, 
       kan ${\bf c}_f$ udregnes ved udnyttelse af 
       lig\-ning~\ref{eq:dilquench}.
\end{enumerate}
\end{itemize}

\section{Anvendelse af quenchingteorien}
Efter denne teoretiske diskussion, vil vi nu pr{\ae}sentere
nogle resultater fra den fysiske og kemiske litteratur, der
illustrerer nogle af de mange anvendelser quenchingteorien
kan indg{\aa} i.

\vspace{4.0mm}
En af de v{\ae}sentligste forskningsm{\ae}ssige indsatser i
beskrivelsen af oscillerende kemiske reaktioner har
v{\ae}ret fors{\o}get p{\aa} at opstille en realistisk
model samt at bestemme et s{\ae}t hastighedskonstanter, der
kan redeg{\o}re for det brede spektrum af egenskaber som
BZ-reaktionen udviser. Et af de v{\ae}sentligste og
f{\o}rste skridt i denne retning skyldes Field,
K\"{o}r\"{o}s og Noyes, der i 1972 formulerede den
s{\aa}kaldte FKN-mekanisme \cite{FKNorig}. Det s{\ae}t
hastighedskonstanter, der er tilknyttet FKN-mekanismen, er
sidenhen blevet revurderet af Field og F\"{o}rsterling i
1986 \cite{ffrate}, hvor et s{\ae}t modificerede
hastighedskonstanter foresl{\aa}s (disse to s{\ae}t
hastighedskonstanter kaldes i det f{\o}lgende for
henholdsvis FKN-s{\ae}ttet og FF-s{\ae}ttet). Ved en
r{\ae}kke passende reduktioner kan FKN-mekanismen omskrives
til den 3-dimensionale Oregonator model, der var den
f{\o}rste model, der kvalitativt kunne gengive en r{\ae}kke
af de f{\ae}nomener, der eksperimentelt observeres i
BZ-reaktionen. Under anvendelse af en simplificerende
notation kan Oregonator model\-len skrives som
 
\begin{eqnarray} 
 \label{eq:orgdiff}
 A + Y + 2H & \stackrel{k_1}{\lr} & X + U     \nonumber \\
 X + Y + H  & \stackrel{k_2}{\lr} & 2U        \nonumber \\
 2X         & \stackrel{k_5}{\lr} & U + A + H           \\ 
 A + X + H  & \stackrel{k_6}{\lr} & 2X + 2Z   \nonumber \\ 
 Z          & \stackrel{k_7}{\lr} & gY + C    \nonumber  
\end{eqnarray}  

hvor

$$
 A = \mbox{BrO$_3^-$, }\;
 H = \mbox{H$^+$, }\;
 U = \mbox{HOBr, }\;
 X = \mbox{HBrO$_2$, }\;
 Y = \mbox{Br$^-$ og }
 Z = \mbox{Ce$^{4+}$.}
$$

Konstanterne $k_1$, $k_2$, $k_5$ og $k_6$ er forholdsvist
velbestemte, hvorimod v{\ae}rdien af den st{\o}kiometriske
faktor $g$ og hastighedskonstanten $k_7$ stadig er genstand
for megen diskussion. I den situation, hvor man {\o}nsker
at optimere eller tilpasse egenskaberne ved Oregonator
model\-len til eksperimentel data, vil disse to parametre
v{\ae}re oplagte at v{\ae}lge som ``justeringsknapper''.
Det er blandt andet i denne situation, at quenchingteorien
viser sin styrke, idet den muligg{\o}r en sammenlig\-ning
mellem en r{\ae}kke eksakte st{\o}rrelser (quenching\-faser,
quenchingamplituder, sving\-ningsfrekvenser, etc.\ ), der er
tilg{\ae}ngelige fra b{\aa}de eksperi\-menter og
model\-simuleringer. Med denne strategi for {\o}je, {\o}nsker
vi nu med udgangpunkt i en r{\ae}kke resultater fra
\cite{HopfQuench} at diskutere fordele s{\aa}vel som
ulemper ved tre forskellige model\-ler for BZ-reaktionen.
 
\vspace{4.0mm}
I \cite{HopfQuench} er egenskaberne ved f{\o}lgende tre model\-ler 
for BZ-reaktionen blevet unders{\o}gt

\begin{center}
  \begin{description}
    \item[M3] Oregonator model\-len
    \item[M4] Showalter, Noyes og Bar-Eli model\-len
    \item[M5] En udvidet Oregonator model
  \end{description}
\end{center}

Disse tre model\-ler fremg{\aa}r alle af
tabel~\ref{tab:modeller}, hvor v{\ae}rdien af
hastighedskonstanterne for de enkelte elementarreaktioner
er angivet for b{\aa}de FKN-s{\ae}ttet og FF-s{\ae}ttet.

%%%%%%%%%%%%%%%%%%%%%%%%%%%%%%%%%%%%%%%%%%%%%%%%%%%%%%%%%%%%%%%%%%%%%%%%
%% tabel
%%
%% beskrivelse : Tabel over modeldata for M3, M4 og M5
%% tex         : tab11.tex
%% type        : PSTricks
%%%%%%%%%%%%%%%%%%%%%%%%%%%%%%%%%%%%%%%%%%%%%%%%%%%%%%%%%%%%%%%%%%%%%%%%
\begin{landfloat}{table}{\rotateright}
\renewcommand{\capfont}{\bf}
\begin{center}
 %%%%%%%%%%%%%%%%%%%%%%%%%%%%%%%%%%%%%%%%%%  
%     forward FKN-rateconstants          %
%%%%%%%%%%%%%%%%%%%%%%%%%%%%%%%%%%%%%%%%%%  
  \newcommand{\fkar}{$2.1$}
  \newcommand{\fkbr}{$2   \times 10^{9} $}
  \newcommand{\fkcr}{$1   \times 10^{4} $}
  \newcommand{\fkdr}{$6.5 \times 10^{5} $}
  \newcommand{\fker}{$4   \times 10^{7} $}
  \newcommand{\fkfr}{$1   \times 10^{4} $}

%%%%%%%%%%%%%%%%%%%%%%%%%%%%%%%%%%%%%%%%%%  
%     backward FKN-rateconstants         %
%%%%%%%%%%%%%%%%%%%%%%%%%%%%%%%%%%%%%%%%%%  
  \newcommand{\fkal}{$1   \times 10^{4}  $}
  \newcommand{\fkbl}{$5.0 \times 10^{-5} $}
  \newcommand{\fkcl}{$2   \times 10^{7}  $}
  \newcommand{\fkdl}{$2.4 \times 10^{7}  $}
  
%%%%%%%%%%%%%%%%%%%%%%%%%%%%%%%%%%%%%%%%%%  
%     forward FF-rateconstants           %
%%%%%%%%%%%%%%%%%%%%%%%%%%%%%%%%%%%%%%%%%%  
  \newcommand{\ffar}{$2                  $}
  \newcommand{\ffbr}{$3    \times 10^{6} $}
  \newcommand{\ffcr}{$42                 $}
  \newcommand{\ffdr}{$8    \times 10^{4} $}
  \newcommand{\ffer}{$3    \times 10^{3} $}
  \newcommand{\fffr}{$42   \times 10^{4} $}
  
%%%%%%%%%%%%%%%%%%%%%%%%%%%%%%%%%%%%%%%%%%  
%     backward FF-rateconstants          %
%%%%%%%%%%%%%%%%%%%%%%%%%%%%%%%%%%%%%%%%%%  
  \newcommand{\ffal}{$3.2                 $}
  \newcommand{\ffbl}{$2    \times 10^{-5} $}
  \newcommand{\ffcl}{$4.2  \times 10^{7}  $}
  \newcommand{\ffdl}{$8.9  \times 10^{3}  $}
  \newcommand{\ffel}{$1    \times 10^{-8} $}
  
  \newcommand{\lra}{\longrightarrow}
  \newcommand{\lla}{\longleftarrow}
  \newcommand{\rlh}{\rightleftharpoons}
  \newcommand{\arrx}{\multicolumn{1}{|c|}{$\lra$}}
  \newcommand{\arry}{\multicolumn{1}{|c|}{$\lla$}}
  \footnotesize
  
  \begin{tabular}{|c|rcl|l|l|l|l|c|c|c|}     \cline{5-8}\cline{5-8}
   \multicolumn{4}{c}{} & \multicolumn{4}{|c|}{Hastighedskonstanter} & \multicolumn{3}{c}{}                  \\ \cline{5-11}
   \multicolumn{4}{c}{} & \multicolumn{2}{|c|}{FKN}  & \multicolumn{2}{|c|}{FF} & \multicolumn{3}{|c|}{Model}\\ \hline
   Nr.& \multicolumn{3}{|c|}{Reaktion}  & \arrx & \arry & \arrx & \arry & M3    & M4    & M5       \\ \hline
   R1 & A + Y + 2H & $\rlh$ & X + U     & \fkar & \fkal & \ffar & \ffal & $\lra$ & $\lra$ & $\rlh$ \\ \hline
   R2 & X + Y + H  & $\rlh$ & 2U        & \fkbr & \fkbl & \ffbr & \ffbl & $\lra$ & $\lra$ & $\rlh$ \\ \hline
   R3 & A + X + H  & $\rlh$ & 2W + K    & \fkcr & \fkcl & \ffcr & \ffcl &       &       & $\rlh$   \\ \hline
   R4 & W + C + H  & $\rlh$ & X + Z     & \fkdr & \fkdl & \ffdr & \ffdl &       &       & $\rlh$   \\ \hline
   R5 & 2X         & $\rlh$ & U + A + H & \fker &       & \ffer & \ffel & $\lra$ & $\lra$ & $\rlh$ \\ \hline
   R6 & A + X + H  & $\rlh$ & 2X + 2Z   & \fkfr &       & \fffr &       & $\lra$ & $\lra$ &        \\ \hline
   R7 & Z          & $\rlh$ & gY + C    &       &       &       &       & $\lra$ & $\lra$ & $\lra$ \\ \hline
   R8 & U          & $\rlh$ & Y         &       &       &       &       &       & $\lra$ &         \\ \hline
   R9 & U          & $\rlh$ & P         &       &       &       &       &       & $\lra$ &         \\ \hline\hline
  \end{tabular}
 \normalsize

\end{center}
\caption{\protect\capsize
  Tabellen stammer fra \protect\cite{HopfQuench} og angiver
  data for de tre model\-ler M3, M4 og M5. F{\o}lgende
  forkortelser er anvendt for de forskellige kemiske
  stoffer: A = BrO$_3^-$, C = Ce$^{3+}$, H = H$^+$, K =
  H$_2$O, P = produkt uden indflydelse p{\aa} selve
  reaktionerne, U = HBrO, W = BrO$_2$, X = HBrO$_2$, Y =
  Br$^-$, Z = Ce$^{4+}$. FKN og FF henviser til de to
  forskellige s{\ae}t hastighedskonstanter, der er
  foresl{\aa}et af Field, K\"{o}r\"{o}s og Noyes
  \protect\cite{FKNorig}, og Field og F\"{o}rsterling
  \protect\cite{ffrate}. Enhederne for
  hastighedskonstanterne er afledt af mol/l og tid
  \mbox{pr.\ sekund} og f{\o}lger af de enkelte hastighedsudtryk.}
  \label{tab:modeller}
  \renewcommand{\capfont}{\rm}
\end{landfloat}

\vspace{4.0mm}
Som det fremg{\aa}r af tabel~\ref{tab:modeller}, s{\aa}
indeholder model\-lerne M3, M4 og M5 alle konstanter, hvis
numeriske v{\ae}rdi endnu ikke har v{\ae}ret
tilg{\ae}ngelig via eksperimentelle m{\aa}linger. Som
udgangpunkt er v{\ae}rdierne for disse derfor valgt, s{\aa}
v{\ae}rdien for $j_t$ i de respektive model\-ler giver
anledning til den bedst mulige overensstemmelse mellem
model\-lernes forudsigelse af quenching\-faser og amplituder og
de {\ae}kvivalente eksperimentelle data. Herved f{\aa}s det
datas{\ae}t, der er angivet i tabel~\ref{tab:initdata}, der
i \cite{HopfQuench} har tjent som udgangspunkt for
udregning af Hopfbifurkationsdiagrammer.

\begin{table}[t]
 \renewcommand{\capfont}{\bf}
 \capsize
 \begin{minipage}{8cm}
 \begin{center}
  \begin{tabular}{|l|l|l|l|}                  \hline\hline
         & \multicolumn{1}{|c}{M3} 
         & \multicolumn{1}{|c|}{M4} 
         & \multicolumn{1}{c|}{M5}                    \\ \hline
   $k_7$ & $0.167$               & $0.158$   & $0.29$ \\ \hline
   $k_8$ &                       & $0.1229$  &        \\ \hline
   $k_9$ &                       & $0.226$   &        \\ \hline
   $g  $ & $0.7905$              & $0.28$    & $0.59$ \\ \hline
   $j_t$ & $2.99 \times 10^{-5}$ & $3.56 \times 10^{-5}$ & $3.40 \times 10^{-5}$ \\ \hline
   $j_f$ & $0.141$               & $0.168$   & $0.030$\\ \hline
   $j_c$ & $0.513$               & $0.507$   & $0.479$\\ \hline\hline
  \end{tabular}
 \end{center} 
 \end{minipage}
 \ \hfill \
 \begin{minipage}{5cm}
 \caption{\protect\capsize
  Parametervalg for \protect\\model\-lerne M3-M5 for det s{\ae}t 
  af konstanter for hvilket kendte v{\ae}rdier ikke 
  er tilg{\ae}ngelige. Parameters{\ae}ttet for hver af
  model\-lerne er valgt s{\aa} quen\-ching\-fa\-ser
  og -amplituder er optimale i forhold til de
  eksperimentelt {\ae}kvivalente data.
 }
 \end{minipage}
 \label{tab:initdata}
 \normalsize
 \renewcommand{\capfont}{\rm}
\end{table} 
 
\vspace{4.0mm}
De eksperimentelle data for BZ-reaktionen og analoge
numeriske data for model\-lerne M3-M5 fremg{\aa}r af
tabel~\ref{tab:data}. Ved sammenlig\-ning med de
eksperimentelle data ser vi, at model\-lerne kvalitativt
giver anledning til det samme m{\o}nster i de udregnede
quenchingdata. HBrO$_2$ quencher med fasen $92^\circ$,
hvor\-imod de andre stoffer alle har en fase indenfor
intervallet $[-128^\circ;-104^\circ]$. Indenfor en
afvigelse p{\aa} en enkelt st{\o}rrelsesorden ser vi
endvidere, at quen\-ching\-amplituderne for de tre
model\-ler er i rimelig overensstemmelse med de
eksperimentelt fundne. Dog er v{\ae}rdien for $q($HBrO$_2)$
i M5 urealistisk h{\o}j, hvilket skyldes en form for
udartning eller degeneration i den p{\aa}g{\ae}ldende
model. Vi vil senere i afsnit~\ref{sec:QuenchUdart}
redeg{\o}re teoretisk for dette f{\ae}nomen. Ydermere er
det v{\ae}sentligt at notere den ringe lighed mellem den
eksperimentelle sving\-ningsfrekvens $\omega_0$ og de
numerisk bestemte frekvenser.

\vspace{4.0mm}
Med udgangspunkt i data fra tabel~\ref{tab:initdata} har
man i \cite{HopfQuench} udregnet en r{\ae}kke
Hopfbifurkationsdiagrammer for model\-lerne M3-M5. Idet
Hopfpunktet kontinueres som funktion af parameters{\ae}ttet
$(j_t,\mu)$ (hvor $\mu = k_7, k_8, k_9, g $), kan
resultaterne kort opsummeres som f{\o}lgende

\begin{itemize}
 \item {\bf M3 }($\mu = k_7, g $)\\
 Meget lille {\ae}ndring af quenchingdata under variation
 af $g$. Hopfbifurkation ved den termodynamiske gren
 er altid superkritisk ved valg af FF-s{\ae}ttet, hvorimod
 denne altid er subkritisk med FKN-s{\ae}ttet.
 \item {\bf M4 }($\mu = k_7, k_8, k_9, g $)\\
 Det samme som ved M3, dog med den forskel at quenchingdata
 varierer kraftig for h{\o}je v{\ae}rdier af $j_t$.
 \item {\bf M5 }($\mu = k_7, g $)\\
 Hopfkurven for parametervalget $(j_t,g)$ danner en lukket
 kurve, hvorfor der for samme v{\ae}rdi af $j_t$ eksisterer
 to Hopfpunkter svarende til to forskellige v{\ae}rdier af
 $g$. Endvidere udvises der store forskelle i quenchingdata
 for de to v{\ae}rdier af $g$, der til samme v{\ae}rdi af
 $j_t$ giver anledning til to Hopfbifurkationer fra den
 termodynamiske gren. 
\end{itemize}

\begin{table}[t]
 \renewcommand{\capfont}{\bf}
 \begin{center}
  %%%%%%%%%%%%%%%%%%%%%%%%%%%%%%%%%%%%%%%%%%  
%           Left Column Entries          %
%%%%%%%%%%%%%%%%%%%%%%%%%%%%%%%%%%%%%%%%%%  
  \newcommand{\lca}{$j_t/s^{-1}     $}
  \newcommand{\lcb}{$\omega_0/s^{-1}$}
  \newcommand{\lcc}{$q($HBrO$_2)    $}
  \newcommand{\lcd}{$q($Br$^{-1})   $}
  \newcommand{\lce}{$q($Ce$^{4+})   $}
  \newcommand{\lcf}{$q($HBrO$)      $}
  \newcommand{\lcg}{$q($BrO$_2)     $}
  \newcommand{\lch}{$\phi($HBrO$_2)    $}
  \newcommand{\lci}{$\phi($Br$^{-1})   $}
  \newcommand{\lcj}{$\phi($Ce$^{4+})   $}
  \newcommand{\lck}{$\phi($HBrO$)      $}
  \newcommand{\lcl}{$\phi($BrO$_2)     $}

%%%%%%%%%%%%%%%%%%%%%%%%%%%%%%%%%%%%%%%%%%  
%           Experimental results         %
%%%%%%%%%%%%%%%%%%%%%%%%%%%%%%%%%%%%%%%%%%  
  \newcommand{\expa}{$3.39   \times 10^{-5} $}
  \newcommand{\expb}{$0.20    $}
  \newcommand{\expc}{$0.20    $}
  \newcommand{\expd}{$0.11    $}
  \newcommand{\expe}{$0.43    $}
  \newcommand{\expf}{$1.58    $}
  \newcommand{\expg}{$92      $}
  \newcommand{\exph}{$-104    $}
  \newcommand{\expi}{$-128    $}
  \newcommand{\expj}{$-120    $}

%%%%%%%%%%%%%%%%%%%%%%%%%%%%%%%%%%%%%%%%%%  
%               M3 results               %
%%%%%%%%%%%%%%%%%%%%%%%%%%%%%%%%%%%%%%%%%%  
  \newcommand{\mthreea}{$2.99   \times 10^{-5} $}
  \newcommand{\mthreeb}{$0.108   $}
  \newcommand{\mthreec}{$0.290   $}
  \newcommand{\mthreed}{$0.595   $}
  \newcommand{\mthreee}{$0.898   $}
  \newcommand{\mthreef}{$116     $}
  \newcommand{\mthreeg}{$-103.7  $}
  \newcommand{\mthreeh}{$-136.7  $}

%%%%%%%%%%%%%%%%%%%%%%%%%%%%%%%%%%%%%%%%%%  
%               M4 results               %
%%%%%%%%%%%%%%%%%%%%%%%%%%%%%%%%%%%%%%%%%%  
  \newcommand{\mfoura}{$3.56   \times 10^{-5} $}
  \newcommand{\mfourb}{$0.081   $}
  \newcommand{\mfourc}{$0.262   $}
  \newcommand{\mfourd}{$0.346   $}
  \newcommand{\mfoure}{$1.391   $}
  \newcommand{\mfourf}{$1.009   $}
  \newcommand{\mfourg}{$136.6   $}
  \newcommand{\mfourh}{$-105.8  $}
  \newcommand{\mfouri}{$-133.1  $}
  \newcommand{\mfourj}{$-119     $}

%%%%%%%%%%%%%%%%%%%%%%%%%%%%%%%%%%%%%%%%%%  
%               M5 results               %
%%%%%%%%%%%%%%%%%%%%%%%%%%%%%%%%%%%%%%%%%%  
  \newcommand{\mfivea}{$3.40   \times 10^{-5} $}
  \newcommand{\mfiveb}{$0.077  $}
  \newcommand{\mfivec}{$0.278  $}
  \newcommand{\mfived}{$0.365  $}
  \newcommand{\mfivee}{$0.654  $}
  \newcommand{\mfivef}{$20900  $}
  \newcommand{\mfiveg}{$0.388  $}
  \newcommand{\mfiveh}{$102.1  $}
  \newcommand{\mfivei}{$-99.9  $}
  \newcommand{\mfivej}{$-118.8 $}
  \newcommand{\mfivek}{$-139.0 $}
  \newcommand{\mfivel}{$123.1  $}

  \capsize
  \begin{tabular}{|l|l|l|l|l|}\hline\hline
   \multicolumn{1}{|c}{}              & 
   \multicolumn{1}{c|}{Eksperiment}   & 
   \multicolumn{1}{|c|}{M3}           & 
   \multicolumn{1}{|c|}{M4}           &
   \multicolumn{1}{|c|}{M5} \\ \hline 
   \lca & \expa & \mthreea & \mfoura & \mfivea \\ \hline 
   \lcb & \expb & \mthreeb & \mfourb & \mfiveb \\ \hline 
   \lcc & \expc & \mthreec & \mfourc & \mfivec \\ \hline 
   \lcd & \expd & \mthreed & \mfourd & \mfived \\ \hline 
   \lce & \expe & \mthreee & \mfoure & \mfivee \\ \hline 
   \lcf & \expf &          & \mfourf & \mfivef \\ \hline 
   \lcg &       &          &         & \mfiveg \\ \hline 
   \lch & \expg & \mthreef & \mfourg & \mfiveh \\ \hline 
   \lci & \exph & \mthreeg & \mfourh & \mfivei \\ \hline 
   \lcj & \expi & \mthreeh & \mfouri & \mfivej \\ \hline 
   \lck & \expj &          & \mfourj & \mfivek \\ \hline 
   \lcl &       &          &         & \mfivel \\ \hline\hline 
  \end{tabular}
 \normalsize

 \end{center}
 \caption{\protect\capsize
  Tabellen sammenholder eksperimentelle v{\ae}rdier for
  quenchingdata med {\ae}kvivalente v{\ae}rdier for de tre
  model\-ler {\bf M3}, {\bf M4} og {\bf M5}. Specielt
  bem{\ae}rkes den ekstremt h{\o}je quenchingamplitude for
  HBrO i model {\bf M5}. Dette hidr{\o}rer fra en
  s{\aa}kaldt udartning af den p{\aa}g{\ae}ldende
  quenchingvektor, hvilket vil blive diskuteret i detaljer
  i afsnit~\protect\ref{sec:QuenchUdart}. Der henvises til
  teksten for en uddybende diskussion og sammenlig\-ning af
  de {\o}vrige data. }
 \label{tab:data}
 \renewcommand{\capfont}{\rm}
\end{table} 

Vi har set, hvorledes disse model\-ler kan gengive eller
reproducere en r{\ae}kke eksperimentelle egenskaber ved
BZ-reaktionen. Som det da ogs{\aa} understreges i
\cite{HopfQuench}, er disse overensstemmelser udelukkende
semikvantitative, da en kvantitativ korrelation stadig
mangler mellem model og eksperiment. Endvidere er
model\-lerne langt bedre til at reproducere eksperimentelle
data ved benyttelse af FF-hastighedskonstanter, hvorfor
dette valg af hastighedskonstanter m{\aa} anses for at
v{\ae}re bedre end FKN-s{\ae}ttet. De n{\ae}vnte resultater
peger endvidere imod, at dimensionen af de unders{\o}gte
model\-ler ($\dim = 3,4,5$) er for lille. Med andre ord er
det temmelig plausibelt, at et st{\o}rre antal dynamiske
variable b{\o}r indrages i en model\-lering af
BZ-reaktionen, f{\o}r mere realistiske resultater kan
opn{\aa}s.

\section{Udartning af quenchingvektorer}
\label{sec:QuenchUdart}
Afslutningsvis vil vi nu diskutere {\aa}rsagen til, at
model M5 indeholder en v{\ae}sentlig mangel. Som det
fremg{\aa}r af tabel~\ref{tab:data}, er quenchingamplituden
for HBrO enorm ($q($HBrO$)$ $= 20900$), hvilket skyldes, at
quenchingvektoren $q($HBrO$)$ stort set er indeholdt i
transientrummet $E^s$. I denne forbindelse taler man om en
udartning af quenchingvektoren. Det viser sig, at dette
f{\ae}nomen kan formuleres og forklares meget pr{\ae}cist
udfra nogle simple teoretiske over\-vejelser. Da denne
teoretiske diskussion endnu ikke har v{\ae}ret
pr{\ae}senteret i den fysiske eller kemiske litteratur,
helliger vi derfor de n{\ae}ste par linier til en mere
abstrakt pr{\ae}sentation af problemet. Dern{\ae}st
illustreres, hvorledes udartningen manifesteres i M5, samt
hvorledes M5 forholdsvist simpelt kan modificeres
s{\aa}ledes, at udartningen forsvinder.

\vspace{4.0mm}
Vi {\o}nsker nu at udlede et tilstr{\ae}kkeligt kriterium,
der, hvis det er opfyldt, medf{\o}rer, at et stof ikke kan
quenche en kemisk oscillation. Hertil betragter vi en
kemisk reaktion, hvis kinetik er beskrevet ved
lig\-ning~\ref{eq:GeneralKinetic} s{\aa}ledes, at denne
net\-op har gennemg{\aa}et en superkritisk Hopfbifurkation.
Formel\-udtrykkene og koncentrationerne af de $n$ stoffer,
der deltager i reaktionen, betegnes med $s_1,\ldots,s_n$
henholdsvis $c_1,\ldots,c_n$. Lad os nu antage, at den
tilh{\o}rende Jacobimatrix er af formen

\begin{equation}
  {\bf J} =
  \left[
  \begin{array}{cccc}
   x & \cdots & x & 0 \\
   \vdots & & & \vdots\\
   x & \cdots & x & 0 \\
   x & \cdots & x & x 
  \end{array}
  \right]
  \label{eq:JacobiDegen}
\end{equation}

hvor $x$ angiver et vilk{\aa}rligt element forskelligt fra
nul. Heraf ses at den $n$'te enhedsvektor ${\bf e}_n$ er
egenvektor til ${\bf J}$, idet 

\begin{equation}
  {\bf J} {\bf e}_n = x {\bf e}_n
\end{equation}

${\bf e}_n$ ligger alts{\aa} i det stabile underrum $E^s$.
Men da ${\bf e_n}$ jo endvidere udsp{\ae}nder
koncentrationskoordinaten for det $n$'te stof $s_n$, vil
enhver tils{\ae}tning af $s_n$ i faserummet v{\ae}re
beskrevet ved en bev{\ae}gelse parallel med $E^s$. Heraf
kan vi slutte, at det til en vilk{\aa}rlig fase og
amplitude vil v{\ae}re umuligt at quenche med $s_n$, da vi
aldrig vil kunne ramme $E^s$ ved en s{\aa}dan
tils{\ae}tning. Denne situation er illustreret geometrisk i
figur~\ref{fig:QuenchUdart}.

%%%%%%%%%%%%%%%%%%%%%%%%%%%%%%%%%%%%%%%%%%%%%%%%%%%%%%%%%%%%%%%%%%%%%%%%
%% figur
%%
%% beskrivelse : Skematisk illustration af udartning af
%%               quenchingvektor
%% makroer     : PSTricks
%%%%%%%%%%%%%%%%%%%%%%%%%%%%%%%%%%%%%%%%%%%%%%%%%%%%%%%%%%%%%%%%%%%%%%%%
\boxfigure{t}{\textwidth}
{
\begin{center}
 \begin{pspicture}(0,0)(11,7)
%  \psgrid(0,0)(0,0)(11,8)
  \psset{fillstyle=solid,fillcolor=lightgray}
  \psset{shadowcolor=gray}
  \psline[linewidth=0.8pt,arrowinset=0]{->}(2,2)(8,1)
  \psline[linewidth=0.8pt,arrowinset=0]{->}(2,2)(2,6)
  \psline[linewidth=0.8pt,arrowinset=0]{->}(2,2)(5,5)
  \psline[linewidth=0.8pt](7,4)(9,6)
  \rput[cc]{-25}(7.0,4.0){\psellipse[shadow=true,shadowangle=45](0.0,0.0)(1.50,0.40)}
  \psline[dotsep=1.5pt,linestyle=dotted](7.3,4.3)(7,4)
  \psline[linewidth=0.8pt](5.5,2.5)(7,4)
  \psline[linewidth=0.8pt,arrowinset=0]{->}(5.641,4.634)(7.141,6.134)
  \rput[cl]{*0}(8.1,1.0){\tiny [HBrO$_2$]}
  \rput[bc]{*0}(2.0,6.1){\tiny [Ce$^{4+}$]}
  \rput[bl]{*0}(5.1,5.1){\tiny [Br$^-$]}
  \rput[tc]{*0}(7.8,3.2){\footnotesize $\gamma$}
  \rput[tl]{*0}(6.45,5.35){\footnotesize ${\bf q}$}
  \rput[bl]{*0}(9.05,6.0){\footnotesize $E^s$}
 \end{pspicture}
\end{center}
}
{
\caption{\protect\capsize
Skematisk illustrering af udartning af en quenchingvektor i
en kemisk oscillation. Det stabile underrum $E^s$
sk{\ae}rer gr{\ae}nsecyklusens plan nedefra parallelt med
en af koncentrationsakserne (Br$^-$ er her valgt som
eksempel). Ligegyldigt hvilken fase og amplitude, der
v{\ae}lges ved tils{\ae}tningen ${\bf q}$ af det
p{\aa}g{\ae}ldende stof, vil tils{\ae}tningen aldrig kunne
ramme $E^s$.}
\label{fig:QuenchUdart}
}

\vspace{4.0mm}
Hvilken rolle spiller stoffet $s_n$ i den kemiske
reaktion, hvis Jacobimatricen har det n{\ae}vnte
udseende? Lad os besvare dette sp{\o}rgsm{\aa}l ved at
betragte hvilke typer af reaktioner $s_n$ ikke kan deltage
i. F{\o}rst forestiller vi os, at $s_n$ er reaktant i en
reaktion s{\aa}ledes, at et af de $n-1$ andre stoffer
produceres i denne reaktion. Vi har alts{\aa}

\begin{equation}
  \ldots + s_n + \ldots \stackrel{k}{\longrightarrow}
  \ldots + s_j + \ldots, \;\;\; j \in \{1,\ldots,n-1 \}
\end{equation}

I hastighedsudtrykket for $s_j$ forekommer alts{\aa} et
led af typen $k c_n$ svarende til f{\o}lgende bidrag til
Jacobimatricen

\begin{equation}
  \frac{\partial \dot{c}_j}{\partial c_n} = \ldots + k + \ldots
\end{equation}

Dette er i modstrid med den form vi antog Jacobimatricen
havde. Kan man ikke quenche med en given forbindelse, da
kan denne forbindelse ikke v{\ae}re reaktant i en reaktion,
i hvilken andre af de dynamiske stoffer produceres. Med
helt samme teknik kan man yderligere vise, at $s_n$ heller
ikke kan deltage i nogen reaktion, s{\aa}ledes at de andre
dynamiske stoffer ogs{\aa} indg{\aa}r som reaktanter. Dette
giver f{\o}lgende resultat

\begin{itemize}
  \item Et stof, der deltager i en kemisk oscillation, kan
  ikke quenche, hvis dette ikke ogs{\aa} forbruges i en
  reaktion, i hvilken mindst et af de andre stoffer
  indg{\aa}r enten som reaktant eller produkt.
\end{itemize}

F.eks.\ ser vi, at et flow af et stof igennem et
CSTR-kammer ikke er nok til, at det p{\aa}g{\ae}ldende stof
kan quenche. Vi m{\aa} kr{\ae}ve, at stoffet som reaktant
er involveret i mindst en af de reaktioner, de andre
stoffer deltager i.

\vspace{4.0mm}
I model M5 s{\aa} vi tidligere, hvorledes
quenchingvektorens amplitude for HBrO var
uforholdsm{\ae}ssig stor. Dette faktum kan forklares ved
hj{\ae}lp af den netop gennemg{\aa}ede teori. Lad os
betragte de tre reaktioner i hvilke HBrO indg{\aa}r

\begin{eqnarray}
  \mbox{A + Y + 2H} & \rightleftharpoons & \mbox{X + U  } \nonumber\\
  \mbox{X + Y + H } & \rightleftharpoons & \mbox{2U   }   \\
  \mbox{2X        } & \rightleftharpoons & \mbox{U + A + H}\nonumber
\end{eqnarray}

hvor U = [HBrO], A = [BrO$_3^-$], X = [HBrO$_2$] og Y =
[Br$^-$] (for yderligere detaljer se
tabel~\ref{tab:modeller}). Umiddelbart synes de
tilbageg{\aa}ende reaktioner i disse tre reaktioner at
tilfredsstille de n{\ae}vnte krav for at kunne quenche med
HBrO, idet alle disse forbruger HBrO under indflydelse af
andre dynamiske stoffer (X og Y). Problemet er her, at alle
tre reaktioner er s{\aa} langsomme, at den
p{\aa}g{\ae}ldende egenvektor ${\bf e}_{\mbox{\tiny HBrO}}$
kun f{\aa}r en meget lille komponent i komplimentet til
$E^s$. Teoretisk betyder dette, at vi skal bev{\ae}ge os
``meget'' langt ud langs ${\bf e}_{\mbox{\tiny HBrO}}$,
f{\o}rend denne rammer det stabile underum $E^s$.

\vspace{4.0mm}
Da HBrO kan quenche eksperimentelt, og da
hastighedskonstanterne for de tre tilbageg{\aa}ende
reaktioner alle er velbestemte eksperimentelt, antyder
dette, at der i BZ-reaktionen n{\o}dvendigvis m{\aa}
forekomme en hurtig reaktion, der forbruger HBrO. Dette
faktum er udnyttet i model M4, hvor to s{\aa}danne
reaktioner, R8 og R9, er tilf{\o}jet. Derved fjernes den
n{\ae}vnte udartning, og en langt bedre overensstemmelse
mellem den eksperimentelle og beregnede st{\o}rrelse af
quenchingamplituden opn{\aa}s (1.58 vs.\ 1.009).







