% Sidst opdateret: 13/1 1994
\chapter{Perturbation af station{\ae}re tilstande}
\label{cha:PertStat}
I kapitel \ref{cha:Quench} har vi beskrevet en succesrig
metode til at f{\aa} information ud af periodiske
tilstande. I dette kapitel vil vi se p{\aa} station{\ae}re
tilstande og den information, som vi kan f{\aa} ud af dem
[\citen{pertstat1,pertstat2}].

\section{Bestemmelse af Jacobimatricen}
\label{Pert:jac}
Kinetikken for et kemisk system er givet ved den kinetiske
ligning

\begin{equation}
  \frac{d {\vec{c}} }{dt} = \vec{f}( \vec{c} ).
  \label{Pert:KinEq}
\end{equation}

Vi vil se p{\aa} en station{\ae}r tilstand for et kemisk
system, der beskrives af ligning~\ref{Pert:KinEq}. Den
station{\ae}re tilstand kalder vi $\vec{c}^S$ og der g{\ae}lder
$\vec{f}( \vec{c} ^S) = 0$. Vi vil foretage en perturbation af
systemet, dvs.\ vi {\o}nsker at fjerne os fra det
station{\ae}re punkt. Men perturbationen skal v{\ae}re
lille, s{\aa} vi kan linearisere ligning~\ref{Pert:KinEq}. Vi
definerer nu $\delta \vec{c}$ som afvigelsen fra det
station{\ae}re punkt, dvs.\

\begin{equation}
\delta \vec{c} \equiv \vec{c} - \vec{c}^S.
\end{equation}

Vores linearisering bliver da (ved en Taylorudvikling til
1.\ orden)

\begin{equation}
\label{Pert:Lin}
\frac{d\delta c_i}{dt} = \sum_{j=1}^n \left.\frac{\partial
f_i}{\partial c_j}\right|_{\vec{c}^S} \delta c_j.
\end{equation}

De partielle afledede af $\vec{f}$ mht.\ $\vec{c}$ er
indeholdt i Jacobimatricen, $\matrix{J}$. Vi kan ogs{\aa}
skrive ligning \ref{Pert:KinEq} p{\aa} matrixform

\[
\frac{d\delta\vec{c}}{dt} = \matrix{J}(\vec{c}^S)\delta\vec{c}.
\]

Som vi skal se,  er det muligt at bestemme $\matrix{J}$
eksperimentelt. Vi er dog n{\o}dsaget at g{\o}re en
antagelse, som n\ae sten aldrig vil holde i en
eksperimentel sammenh{\ae}ng. Vi antager, at vi er i stand
til at m{\aa}le koncentrationen af alle de $n$
indg{\aa}ende stoffer. Der er flere grunde til at denne
antagelse ikke altid vil v{\ae}re opfyldt. For det f{\o}rste
kan et stof v{\ae}re til stede i meget lave
koncentrationer, som er under detektionsgr{\ae}nsen for det
eksperimentelle udstyr. For det andet er det ikke altid
muligt at m{\aa}le koncentrationen af et stof: Radikaler
vil give problemer, og ved en fotometrisk m{\aa}ling kan to
stoffer absorbere ved samme b{\o}lgel{\ae}ngde.

\vspace{4.0mm}
Den eksperimentelt fundne Jacobimatrix kan bruges til to
ting. Som vi skal se i et senere afsnit, kan mekanismen for
reaktionen findes vha.\ denne matrix. For det andet kan
modeller sammenlignes med eksperimenterne ved at udregne
Jacobimatricen for dem i det station{\ae}re punkt.

\vspace{4.0mm}
Der findes to metoder til at bestemme Jacobimatricen, og vi
vil i de n{\ae}ste to afsnit beskrive dem.

\subsection{Initielle h{\ae}ldninger}
Vi antager, at systemet er i en station{\ae}r tilstand, og
vi adderer nu en m{\ae}ngde af det $k$'te stof. Henfaldet
tilbage til den station{\ae}re tilstand er givet ved
ligningen

\begin{equation}
  \frac{\delta c_i}{dt} = \frac{\partial f_i}{\partial c_k}\delta c_k.
\end{equation}

Men ligningen g{\ae}lder kun for tidspunktet for
pertubationen, dvs.\ $t=0$. H{\ae}ldningen i $t=0$ giver
derved den $k$'te s{\o}jle i Jacobimatricen. Vi skal
alts{\aa} lave $n$ af disse pertubationsfors{\o}g (et for
hvert stof) for at f{\aa} hele Jacobimatricen.

\subsection{Mindste-kvadrater}
Hvis vi laver en perturbation, som er lille, vil systemet
vende tilbage til den oprindelige tilstand, hvis der er
tale om et stabilt station{\ae}rt punkt. Vi vil derfor
foretage m{\aa}linger af koncentrationen efter
perturbationen, indtil systemet igen er i det
station{\ae}re punkt. Vi udf{\o}rer et endeligt antal
m{\aa}linger, $N$, og nummererer dem ved $l = 1, \ldots,
N$. Vi vil antage, at vi udf{\o}rer mange m{\aa}linger i
forhold til dimensionen af koncentrationsrummet ($N \gg
n$). Vi har med andre ord tidsr{\ae}kken $\vec{c}^{(l)}$.
M{\aa}lingerne foretages med afstandene $\Delta t^{(l)}$.
Vi kan samle tidsr{\ae}kken til en matrix $\matrix{U}$ ved
$U_{il} = c_i^{(l)}$. I stedet for at se p{\aa} de
``absolutte'' koncentrationer, kan vi se p{\aa} afvigelsen
fra det station{\ae}re punkt. Disse afvigelser kan ligesom
de ``absolutte'' koncentrationer samles i en matrix, dvs.\
$\delta U_{il} = U_{il} - c_i^S$. Ved lineariseringen i
ligning \ref{Pert:Lin} approksimeres de tidsafledede
$\frac{d\delta c_i(t)}{dt}$ med $\frac{d\delta
\matrix{U}}{dt}$, der kan findes ved en differenskvotient

\[
\frac{d\delta U_{il}}{dt} \approx \frac{d\delta U_{i(l+1)}-\delta
U_{il}}{\Delta t^{(l)}}.
\]
Ligning \ref{Pert:Lin} kan derfor tiln{\ae}rmes med 

\begin{equation}
\frac{d\delta \matrix{U}}{dt} = \matrix{J} \delta\matrix{U},
\end{equation}

hvor kun matricen $\matrix{J}$ er ukendt. Vi kan ikke finde
Jacobimatricen eksakt, men kan minimere fejlen i
l{\o}sningen ved at l{\o}se problemet

\begin{equation}
\label{Pert:Min}
  \min_{\matrix{J}} \|\matrix{J}\delta\matrix{U}-\frac{\delta\matrix{U}}{dt}\|,
\end{equation}

hvor $\|\cdot\|$ er en passende norm. L{\o}sningen til
ligning \ref{Pert:Min} kan findes vha.\ ``mindste-kvadrat''
metoden, som er relativ let at implementere. Vi er
n{\o}dsaget til at benytte os af ``mindste-kvadrat''
metoden, fordi vi har flere ligninger end ubekendte. Men
lad os se p{\aa}, hvordan vi kan implementere en l\o
sningsmetode. Vi begynder med at transponere vort problem.
Derved f{\aa}r vi

\begin{equation}
  \min_{\matrix{J}^T} \|\delta\matrix{U}^T\matrix{J}^T-\frac{\delta\matrix{U}^T}{dt}\|,
\end{equation}
idet vi har brugt regnereglen $(\matrix{A}\matrix{B})^T =
\matrix{B}^T\matrix{A}^T$. Ser vi nu p{\aa} den $j$'te s{\o}jle, f{\aa}s
$n$ ligninger p{\aa} formen

\begin{equation}
  \min_{(\matrix{J}^T)_j} \left\|\delta\matrix{U}^T(\matrix{J}^T)_j -
  \left(\left(\frac{d\delta\matrix{U}}{dt}\right)^T\right)_j \right\|.
\end{equation}

Vi har omskrevet problemet til formen
$\min_{\vec{x}}\|\matrix{A}\vec{x}-\vec{b}\|$. Dette
minimeringsproblem kan l{\o}ses vha.\ {\em singular-value
decomposition} [\citen{NumRec} s.\ 59--70]. Ved at benytte
rutinerne fra [\citen{NumRec}] kan vi reducere problemet
til f{\o}lgende ``program'':

\begin{tabbing}
{\tt svdcmp}($\delta\matrix{U}^t$, $N$, $n$, $w$, $v$) \\
for \= $j = 1,\ldots, n$:\\
\> {\tt svbksb}($\delta\matrix{U}^t$, $w$, $v$, $N$, $n$,
$\left(\left(\frac{d\delta\matrix{U}}{dt}\right)^t\right)_j$,
$(\matrix{J}^t)_j$)
\end{tabbing}

hvor {\tt svdcmp} og {\tt svbksb} er to rutiner fra
[\citen{NumRec}]. Det skal bem{\ae}rkes, at
$\delta\matrix{U}^t$ b{\aa}de er inddata og uddata fra {\tt
svdcmp}. Den f{\o}rste rutine udf{\o}rer selve {\em
singular value-decomposition}, men den anden rutine
l{\o}ser et problem p{\aa} formen
$\min_{\vec{x}}\|\matrix{A}\vec{x}-\vec{b}\|$. De to
rutiner arbejder p{\aa} en m{\aa}de som en almindelig
LU-faktorisering med efterf{\o}lgende forl{\ae}ns- og
bagl{\ae}nssubstitution.

\vspace{4.0mm}
Der findes ogs{\aa} en anden m{\aa}de at implementerer
problemet p{\aa}. Vi kan skrive afvigelsen fra det
station{\ae}re punkt som

\begin{equation}
  \label{pert:ikkelin}
  \delta c_i(t) = \sum_{j=1}^n a_jv_i^je^{\lambda_j t},
\end{equation}

for $i=1,\ldots,n$. Parametrene $a_j$ er konstanter, mens
$v_i^j$ er den $j$'te egenv{\ae}rdi til Jacobimatricen og
$\lambda_j$ er den tilh{\o}rende egenv{\ae}rdi. Hvis vi
s{\ae}tter $w_{ij} = a_jv_i^j$, kan Jacobimatricen
udtrykkes som

\begin{equation}
  \matrix{J} = \matrix{W}\matrix{\Lambda}\matrix{W}^{-1},
\end{equation}

hvor $\matrix{\Lambda}$ er matricen

\[
  \matrix{\Lambda} = \left(
  \begin{array}{ccc}
    \lambda_1 & & \\
     & \ddots &  \\
     & & \lambda_n 
  \end{array}
  \right).
\]

Ideen er, at matricerne $\matrix{W}$ og $\matrix{\Lambda}$
findes vha.\ den eksperimentelle tidsr{\ae}kke. Vi er
n{\o}dsaget til at lave en ikke-line{\ae}r tiln{\ae}rmelse
til den {\o}nskede funktion, alts{\aa} ligning
\ref{pert:ikkelin}. Dette kan g{\o}res vha.\ Levenberg-Marquardts
metode [\citen{NumRec} s.\ 683--688]. En brugbar algoritme er

\begin{tabbing}
for \= $i=1,\ldots,n$: \\
\> gen\= tag  \\
\>\> {\tt mrqmin}($t$, $\delta c_i(t)$, $\vec{w}^i$,
$\lambda_1,\ldots,\lambda_n$) \\
\> indtil konvergens \\
$\matrix{J}=\matrix{W}\matrix{\Lambda}\matrix{W}^{-1}$
\end{tabbing}

I stedet for at udregne Jacobimatricen direkte som den
sidste linie antyder, er det bedre at l{\o}se ligningssystemet
$\matrix{J}\matrix{W}=\matrix{\Lambda}\matrix{W}$, idet det aldrig 
er en god ide at invertere matricer numerisk.

\vspace{4.0mm}
Generelt kan man sige, at begge metoder er sv{\ae}re at
f{\aa} til at give gode v{\ae}rdier. Men den f{\o}rste
({\em singular value-decomposition}) er ofte d{\aa}rligere
end den sidste. Et andet problem er naturligvis, at
perturbationen skal v{\ae}re lille, men man kan ikke p{\aa}
forh{\aa}nd vide, hvor stor en perturbation man kan tillade
sig.

\vspace{4.0mm}
Chevalier {\em et al.} viser i [\citen{pertstat1}] deres
arbejde med den sidste metode. De har set p{\aa}
DOP-modellen, som er en model af oxidation af \chem{NADH}.
Den er

\begin{eqnarray*}
A + B + X &\rightarrow& 2X \\
2X &\rightarrow& 2Y \\
A + B + Y &\rightarrow& 2X \\
Z &\rightarrow& P \\
Y &\rightarrow& Q \\
X_0 &\rightarrow& X \\
A_0 &\rightarrow& A \\
B_0 &\rightarrow& B \\
\end{eqnarray*}

hvor $A = \chem{O_2}$, $B=\chem{NADH}$, $X$ og $Y$ er
radikaler. Kun $A$, $B$, $X$ og $Y$ er dynamiske variable.
Et af de station{\ae}re punkter er 

\begin{equation}
(A,B,X,Y) = (0.85106, 57.146, 0.01539, 0.17996). 
\end{equation}

I det vi kun ser p{\aa} fortegnene af
elementerne i Jacobimatricen, er den

\[
  \matrix{J} = \left( 
  \begin{array}{cccc}
    - & - & - & - \\
    - & - & - & - \\
    + & + & - & + \\
    - & - & + & - \\
  \end{array}
  \right),
\]

mens ved en perturbation f{\aa}s

\[
  \matrix{J} = \left(
  \begin{array}{cccc}
    - & - & - & - \\
    - & - & + & - \\
    + & + & - & + \\
    - & - & + & - \\
  \end{array}
  \right).
\]

Som det ses, s{\aa} er der \'{e}t af elementerne, som har
det gale fortegn. Metoden er p{\aa} m{\aa}de ingen helt sikker, men
den er i stand til at give mere end et fingerpeg om
fortegnene.

\vspace{4.0mm}
Vi har ogs{\aa} fors{\o}gt os med at beregne Jacobimatricen
ud fra perturbationseksperimenter. Vi har dog holdt os til
en model. Som numerisk metode har vi valgt at benytte metoden 
som er baseret p{\aa} {\em Singular-value Decomposition\/}.
Figur~\ref{pert:Oregon} viser en perturbation af Oregonatoren.
Det staton{\ae}re punkt er $\vec{c}^S = ([Y], [X], [Z]) =
(2.17\cdot 10{-7}, 3.199\cdot 10^{-8}, 1.74\cdot 10^{-7})$.
Vores perturbation er i $X$-retningen. Jacobimatricen er
for modellen (p{\aa} symbolsk form)

\[
  \matrix{J} = \left(
  \begin{array}{ccc}
    - & - & + \\
    - & - & 0 \\
    0 & + & - 
  \end{array}
  \right),
\]

mens Jacobimatricen, beregnet p{\aa} baggrund af
perturbationseksperimentet, er

\[
  \matrix{J} = \left(
  \begin{array}{ccc}
    - & - & + \\
    - & - & - \\
    - & + & -
  \end{array}
  \right).
\]

Elementet $J_{23}$ er omkring en faktor 10 mindre end de
andre elementer, s{\aa} det er rimeligt at sige, at det er
nul. Som vi ser, stemmer Jacobimatricen fra
perturbationseksperientet n{\ae}sten overens med
Jacobimatricen fra modellen.

\boxfigure{tbp}{\textwidth}{
 \begin{center}
  \begin{pspicture}(0,5)(14,11.4)
%   \psgrid[](0,5)(0,5)(14,11.4)
   \rput[cc]{*0}(7.0,8.3){% GNUPLOT: LaTeX using TEXDRAW macros
\begin{texdraw}
\normalsize
\ifx\pathDEFINED\relax\else\let\pathDEFINED\relax
 \def\QtGfr{\ifx (\TGre \let\YhetT\cpath\else\let\YhetT\relax\fi\YhetT}
 \def\path (#1 #2){\move (#1 #2)\futurelet\TGre\QtGfr}
 \def\cpath (#1 #2){\lvec (#1 #2)\futurelet\TGre\QtGfr}
\fi
\drawdim pt
\setunitscale 0.24
\linewd 3
\textref h:L v:C
\path (578 424)(578 424)
\path (578 424)(578 423)
\path (578 423)(578 423)
\path (578 423)(578 423)
\path (578 423)(578 422)
\path (578 422)(578 422)
\path (578 422)(578 421)
\path (578 421)(578 421)
\path (578 421)(578 421)
\path (578 421)(578 421)
\path (578 421)(577 420)
\path (577 420)(577 420)
\path (577 420)(577 420)
\path (577 420)(577 420)
\path (577 420)(576 420)
\path (576 420)(576 419)
\path (576 419)(576 419)
\path (576 419)(576 419)
\path (576 419)(575 419)
\path (575 419)(575 419)
\path (575 419)(574 419)
\path (574 419)(574 419)
\path (574 419)(574 420)
\path (574 420)(574 420)
\path (574 420)(573 420)
\path (573 420)(573 420)
\path (573 420)(572 420)
\path (572 420)(572 420)
\path (572 420)(571 421)
\path (571 421)(571 421)
\path (571 421)(571 421)
\path (571 421)(570 421)
\path (570 421)(570 422)
\path (570 422)(570 422)
\path (570 422)(570 422)
\path (570 422)(569 423)
\path (569 423)(569 423)
\path (569 423)(569 424)
\path (569 424)(569 424)
\path (569 424)(569 424)
\path (569 424)(569 425)
\path (569 425)(569 425)
\path (569 425)(569 425)
\path (569 425)(569 425)
\path (569 425)(569 425)
\path (569 425)(569 426)
\path (569 426)(570 426)
\path (570 426)(570 426)
\path (570 426)(570 426)
\path (570 426)(570 426)
\path (570 426)(571 426)
\path (571 426)(571 426)
\path (571 426)(571 427)
\path (571 427)(571 427)
\path (571 427)(572 427)
\path (572 427)(572 427)
\path (572 427)(572 426)
\path (572 426)(573 426)
\path (573 426)(573 426)
\path (573 426)(573 426)
\path (573 426)(573 426)
\path (573 426)(574 426)
\path (574 426)(574 426)
\path (574 426)(574 426)
\path (574 426)(575 425)
\path (575 425)(575 425)
\path (575 425)(575 425)
\path (575 425)(575 425)
\path (575 425)(576 424)
\path (576 424)(576 424)
\path (576 424)(576 424)
\path (576 424)(576 423)
\path (576 423)(576 423)
\path (576 423)(576 423)
\path (576 423)(576 423)
\path (576 423)(576 422)
\path (576 422)(576 422)
\path (576 422)(576 422)
\path (576 422)(576 422)
\path (576 422)(576 422)
\path (576 422)(576 421)
\path (576 421)(576 421)
\path (576 421)(576 421)
\path (576 421)(575 421)
\path (575 421)(575 421)
\path (575 421)(575 421)
\path (575 421)(575 421)
\path (575 421)(575 421)
\path (575 421)(574 421)
\path (574 421)(574 421)
\path (574 421)(574 421)
\path (574 421)(574 421)
\path (574 421)(573 421)
\path (573 421)(573 421)
\path (573 421)(573 421)
\path (573 421)(573 421)
\path (573 421)(572 421)
\path (572 421)(572 421)
\path (572 421)(572 421)
\path (572 421)(572 422)
\path (572 422)(571 422)
\path (571 422)(571 422)
\path (571 422)(571 422)
\path (571 422)(571 423)
\path (571 423)(571 423)
\path (571 423)(571 423)
\path (571 423)(570 423)
\path (570 423)(570 424)
\path (570 424)(825 531)
\path (825 531)(685 579)
\path (685 579)(579 620)
\path (579 620)(516 651)
\path (516 651)(484 673)
\path (484 673)(473 687)
\path (473 687)(472 688)
\path (472 688)(474 698)
\path (474 698)(485 704)
\path (485 704)(504 705)
\path (504 705)(526 702)
\path (526 702)(552 696)
\path (552 696)(571 690)
\path (571 690)(579 688)
\path (579 688)(607 676)
\path (607 676)(636 662)
\path (636 662)(664 645)
\path (664 645)(692 626)
\path (692 626)(718 606)
\path (718 606)(743 583)
\path (743 583)(766 560)
\path (766 560)(787 535)
\path (787 535)(806 509)
\path (806 509)(823 483)
\path (823 483)(838 457)
\path (838 457)(850 432)
\path (850 432)(859 406)
\path (859 406)(866 381)
\path (866 381)(869 356)
\path (869 356)(870 333)
\path (870 333)(868 310)
\path (868 310)(863 290)
\path (863 290)(857 274)
\path (857 274)(855 270)
\path (855 270)(845 253)
\path (845 253)(832 238)
\path (832 238)(817 224)
\path (817 224)(800 213)
\path (800 213)(781 204)
\path (781 204)(760 197)
\path (760 197)(751 195)
\path (751 195)(738 193)
\path (738 193)(714 190)
\path (714 190)(690 190)
\path (690 190)(664 192)
\path (664 192)(638 197)
\path (638 197)(612 203)
\path (612 203)(610 204)
\path (610 204)(586 211)
\path (586 211)(560 221)
\path (560 221)(535 233)
\path (535 233)(510 247)
\path (510 247)(487 262)
\path (487 262)(464 278)
\path (464 278)(443 295)
\path (443 295)(424 313)
\path (424 313)(407 333)
\path (407 333)(391 352)
\path (391 352)(378 373)
\path (378 373)(366 393)
\path (366 393)(357 414)
\path (357 414)(350 434)
\path (350 434)(345 453)
\path (345 453)(343 473)
\path (343 473)(343 492)
\path (343 492)(345 510)
\path (345 510)(349 527)
\path (349 527)(356 543)
\path (356 543)(364 558)
\path (364 558)(375 571)
\path (375 571)(387 583)
\path (387 583)(401 593)
\path (401 593)(416 602)
\path (416 602)(433 608)
\path (433 608)(448 612)
\path (448 612)(451 613)
\path (451 613)(470 615)
\path (470 615)(489 616)
\path (489 616)(510 615)
\path (510 615)(530 612)
\path (530 612)(551 607)
\path (551 607)(558 605)
\path (558 605)(572 601)
\path (572 601)(593 592)
\path (593 592)(613 582)
\path (613 582)(633 571)
\path (633 571)(652 558)
\path (652 558)(670 544)
\path (670 544)(687 529)
\path (687 529)(703 513)
\path (703 513)(717 497)
\path (717 497)(730 480)
\path (730 480)(741 462)
\path (741 462)(750 446)
\path (750 446)(758 428)
\path (758 428)(764 411)
\path (764 411)(768 394)
\path (768 394)(770 378)
\path (770 378)(770 362)
\path (770 362)(769 347)
\path (769 347)(765 333)
\path (765 333)(761 323)
\path (761 323)(760 320)
\path (760 320)(753 308)
\path (753 308)(745 298)
\path (745 298)(735 289)
\path (735 289)(723 281)
\path (723 281)(711 275)
\path (711 275)(696 271)
\path (696 271)(690 269)
\path (690 269)(681 268)
\path (681 268)(665 266)
\path (665 266)(649 266)
\path (649 266)(632 267)
\path (632 267)(614 270)
\path (614 270)(597 275)
\path (597 275)(596 275)
\path (596 275)(579 280)
\path (579 280)(562 287)
\path (562 287)(545 295)
\path (545 295)(528 304)
\path (528 304)(513 315)
\path (513 315)(497 326)
\path (497 326)(484 337)
\path (484 337)(471 350)
\path (471 350)(459 363)
\path (459 363)(449 377)
\path (449 377)(440 390)
\path (440 390)(432 404)
\path (432 404)(426 418)
\path (426 418)(421 432)
\path (421 432)(418 446)
\path (418 446)(416 458)
\path (416 458)(417 471)
\path (417 471)(418 483)
\path (418 483)(421 495)
\path (421 495)(425 504)
\path (425 504)(426 505)
\path (426 505)(431 515)
\path (431 515)(439 524)
\path (439 524)(447 532)
\path (447 532)(456 538)
\path (456 538)(467 544)
\path (467 544)(478 548)
\path (478 548)(486 550)
\path (486 550)(490 551)
\path (490 551)(503 552)
\path (503 552)(517 553)
\path (517 553)(530 552)
\path (530 552)(544 550)
\path (544 550)(559 546)
\path (559 546)(561 545)
\path (561 545)(573 542)
\path (573 542)(587 536)
\path (587 536)(600 529)
\path (600 529)(614 521)
\path (614 521)(627 513)
\path (627 513)(639 503)
\path (639 503)(650 493)
\path (650 493)(661 483)
\path (661 483)(670 472)
\path (670 472)(679 460)
\path (679 460)(686 449)
\path (686 449)(693 438)
\path (693 438)(698 426)
\path (698 426)(702 414)
\path (702 414)(705 403)
\path (705 403)(706 392)
\path (706 392)(706 381)
\path (706 381)(705 371)
\path (705 371)(702 362)
\path (702 362)(700 356)
\path (700 356)(698 353)
\path (698 353)(694 345)
\path (694 345)(688 338)
\path (688 338)(681 332)
\path (681 332)(673 327)
\path (673 327)(664 323)
\path (664 323)(655 320)
\path (655 320)(651 319)
\path (651 319)(644 318)
\path (644 318)(634 317)
\path (634 317)(622 317)
\path (622 317)(611 318)
\path (611 318)(599 320)
\path (599 320)(587 323)
\path (587 323)(575 327)
\path (575 327)(564 332)
\path (564 332)(552 337)
\path (552 337)(541 343)
\path (541 343)(531 350)
\path (531 350)(521 358)
\path (521 358)(511 366)
\path (511 366)(503 375)
\path (503 375)(495 383)
\path (495 383)(488 393)
\path (488 393)(482 402)
\path (482 402)(477 411)
\path (477 411)(473 421)
\path (473 421)(470 430)
\path (470 430)(468 439)
\path (468 439)(467 448)
\path (467 448)(467 456)
\path (467 456)(468 464)
\path (468 464)(470 472)
\path (470 472)(472 477)
\path (472 477)(473 479)
\path (473 479)(477 486)
\path (477 486)(482 491)
\path (482 491)(488 497)
\path (488 497)(494 501)
\path (494 501)(501 505)
\path (501 505)(509 507)
\path (509 507)(513 508)
\path (513 508)(517 509)
\path (517 509)(526 510)
\path (526 510)(535 510)
\path (535 510)(545 509)
\path (545 509)(554 508)
\path (554 508)(564 506)
\path (564 506)(564 505)
\path (564 505)(573 502)
\path (573 502)(583 498)
\path (583 498)(592 494)
\path (592 494)(601 489)
\path (601 489)(609 483)
\path (609 483)(618 476)
\path (618 476)(625 470)
\path (625 470)(632 462)
\path (632 462)(639 455)
\path (639 455)(645 448)
\path (645 448)(650 440)
\path (650 440)(654 432)
\path (654 432)(657 424)
\path (657 424)(660 417)
\path (660 417)(661 409)
\path (661 409)(662 402)
\path (662 402)(662 395)
\path (662 395)(661 388)
\path (661 388)(659 382)
\path (659 382)(658 378)
\path (658 378)(657 376)
\path (657 376)(654 370)
\path (654 370)(650 366)
\path (650 366)(645 362)
\path (645 362)(639 358)
\path (639 358)(633 356)
\path (633 356)(627 354)
\path (627 354)(625 353)
\path (625 353)(620 352)
\path (620 352)(613 351)
\path (613 351)(605 352)
\path (605 352)(597 352)
\path (597 352)(590 354)
\path (590 354)(582 356)
\path (582 356)(574 358)
\path (574 358)(566 362)
\path (566 362)(558 365)
\path (558 365)(551 370)
\path (551 370)(544 374)
\path (544 374)(537 380)
\path (537 380)(531 385)
\path (531 385)(525 391)
\path (525 391)(520 397)
\path (520 397)(515 403)
\path (515 403)(511 409)
\path (511 409)(508 416)
\path (508 416)(505 422)
\path (505 422)(503 429)
\path (503 429)(502 435)
\path (502 435)(501 441)
\path (501 441)(501 447)
\path (501 447)(502 451)
\path (502 451)(503 456)
\path (503 456)(505 459)
\path (505 459)(505 461)
\path (505 461)(508 466)
\path (508 466)(512 469)
\path (512 469)(516 473)
\path (516 473)(520 476)
\path (520 476)(525 478)
\path (525 478)(530 480)
\path (530 480)(532 480)
\path (532 480)(536 481)
\path (536 481)(542 482)
\path (542 482)(548 482)
\path (548 482)(554 481)
\path (554 481)(561 480)
\path (561 480)(567 478)
\path (567 478)(573 476)
\path (573 476)(580 473)
\path (580 473)(586 470)
\path (586 470)(592 467)
\path (592 467)(598 463)
\path (598 463)(604 458)
\path (604 458)(609 454)
\path (609 454)(613 450)
\path (613 450)(618 445)
\path (618 445)(622 440)
\path (622 440)(625 434)
\path (625 434)(628 429)
\path (628 429)(630 424)
\path (630 424)(632 418)
\path (632 418)(633 413)
\path (633 413)(633 408)
\path (633 408)(633 404)
\path (633 404)(632 399)
\path (632 399)(631 395)
\path (631 395)(630 393)
\path (630 393)(629 391)
\path (629 391)(627 387)
\path (627 387)(624 384)
\path (624 384)(621 382)
\path (621 382)(617 379)
\path (617 379)(613 377)
\path (613 377)(609 376)
\path (609 376)(608 376)
\path (608 376)(604 375)
\path (604 375)(599 375)
\path (599 375)(594 375)
\path (594 375)(589 375)
\path (589 375)(584 376)
\path (584 376)(579 378)
\path (579 378)(578 378)
\path (578 378)(573 380)
\path (573 380)(568 382)
\path (568 382)(562 384)
\path (562 384)(558 387)
\path (558 387)(553 391)
\path (553 391)(548 394)
\path (548 394)(544 398)
\path (544 398)(540 402)
\path (540 402)(537 406)
\path (537 406)(534 410)
\path (534 410)(531 414)
\path (531 414)(529 419)
\path (529 419)(527 423)
\path (527 423)(525 427)
\path (525 427)(525 431)
\path (525 431)(524 436)
\path (524 436)(524 439)
\path (524 439)(525 443)
\path (525 443)(526 447)
\path (526 447)(527 449)
\path (527 449)(528 450)
\path (528 450)(529 452)
\path (529 452)(532 454)
\path (532 454)(534 457)
\path (534 457)(537 459)
\path (537 459)(541 460)
\path (541 460)(544 461)
\path (544 461)(545 461)
\path (545 461)(548 462)
\path (548 462)(552 462)
\path (552 462)(556 462)
\path (556 462)(561 462)
\path (561 462)(565 461)
\path (565 461)(569 460)
\path (569 460)(569 460)
\path (569 460)(574 458)
\path (574 458)(578 457)
\path (578 457)(582 454)
\path (582 454)(586 452)
\path (586 452)(590 450)
\path (590 450)(594 447)
\path (594 447)(597 444)
\path (597 444)(600 441)
\path (600 441)(603 438)
\path (603 438)(606 434)
\path (606 434)(608 430)
\path (608 430)(610 427)
\path (610 427)(611 423)
\path (611 423)(612 420)
\path (612 420)(613 416)
\path (613 416)(613 413)
\path (613 413)(613 410)
\path (613 410)(613 407)
\path (613 407)(612 404)
\path (612 404)(611 403)
\path (611 403)(611 401)
\path (611 401)(609 399)
\path (609 399)(607 397)
\path (607 397)(605 395)
\path (605 395)(603 393)
\path (603 393)(600 392)
\path (600 392)(597 391)
\path (597 391)(596 391)
\path (596 391)(594 391)
\path (594 391)(590 391)
\path (590 391)(587 391)
\path (587 391)(583 391)
\path (583 391)(580 392)
\path (580 392)(577 392)
\path (577 392)(576 393)
\path (576 393)(573 394)
\path (573 394)(569 395)
\path (569 395)(565 397)
\path (565 397)(562 399)
\path (562 399)(559 401)
\path (559 401)(556 404)
\path (556 404)(553 406)
\path (553 406)(551 409)
\path (551 409)(548 412)
\path (548 412)(546 415)
\path (546 415)(544 418)
\path (544 418)(543 420)
\path (543 420)(542 423)
\path (542 423)(541 426)
\path (541 426)(540 429)
\path (540 429)(540 432)
\path (540 432)(540 434)
\path (540 434)(541 437)
\path (541 437)(541 439)
\path (541 439)(542 440)
\path (542 440)(542 441)
\path (542 441)(544 443)
\path (544 443)(545 445)
\path (545 445)(547 447)
\path (547 447)(549 448)
\path (549 448)(551 449)
\path (551 449)(554 450)
\path (554 450)(554 450)
\path (554 450)(556 450)
\path (556 450)(559 450)
\path (559 450)(562 450)
\path (562 450)(565 450)
\path (565 450)(568 449)
\path (568 449)(570 449)
\path (570 449)(571 449)
\path (571 449)(574 448)
\path (574 448)(577 446)
\path (577 446)(579 445)
\path (579 445)(582 443)
\path (582 443)(585 441)
\path (585 441)(587 439)
\path (587 439)(590 437)
\path (590 437)(592 435)
\path (592 435)(594 433)
\path (594 433)(595 430)
\path (595 430)(597 428)
\path (597 428)(598 426)
\path (598 426)(599 423)
\path (599 423)(600 421)
\path (600 421)(600 418)
\path (600 418)(600 416)
\path (600 416)(600 414)
\path (600 414)(600 412)
\path (600 412)(599 410)
\path (599 410)(599 409)
\path (599 409)(598 408)
\path (598 408)(597 407)
\path (597 407)(596 405)
\path (596 405)(594 404)
\path (594 404)(593 403)
\path (593 403)(591 402)
\path (591 402)(589 402)
\path (589 402)(589 402)
\path (589 402)(587 401)
\path (587 401)(584 401)
\path (584 401)(582 401)
\path (582 401)(580 402)
\path (580 402)(577 402)
\path (577 402)(576 402)
\path (576 402)(575 403)
\path (575 403)(572 404)
\path (572 404)(570 405)
\path (570 405)(568 406)
\path (568 406)(565 407)
\path (565 407)(563 409)
\path (563 409)(561 410)
\path (561 410)(559 412)
\path (559 412)(558 414)
\path (558 414)(556 416)
\path (556 416)(555 418)
\path (555 418)(554 420)
\path (554 420)(553 422)
\path (553 422)(552 424)
\path (552 424)(551 425)
\path (551 425)(551 427)
\path (551 427)(551 429)
\path (551 429)(551 431)
\path (551 431)(551 433)
\path (551 433)(552 434)
\path (552 434)(552 435)
\path (552 435)(552 436)
\path (552 436)(553 437)
\path (553 437)(554 438)
\path (554 438)(556 439)
\path (556 439)(557 440)
\path (557 440)(558 441)
\path (558 441)(560 441)
\path (560 441)(560 441)
\path (560 441)(562 441)
\path (562 441)(564 442)
\path (564 442)(566 441)
\path (566 441)(568 441)
\path (568 441)(570 441)
\path (570 441)(571 440)
\path (571 440)(572 440)
\path (572 440)(574 440)
\path (574 440)(576 439)
\path (576 439)(577 438)
\path (577 438)(579 437)
\path (579 437)(581 435)
\path (581 435)(583 434)
\path (583 434)(584 433)
\path (584 433)(586 431)
\path (586 431)(587 430)
\path (587 430)(588 428)
\path (588 428)(589 426)
\path (589 426)(590 425)
\path (590 425)(591 423)
\path (591 423)(591 421)
\path (591 421)(591 420)
\path (591 420)(591 418)
\path (591 418)(591 417)
\path (591 417)(591 416)
\path (591 416)(591 414)
\path (591 414)(590 414)
\path (590 414)(590 413)
\path (590 413)(589 412)
\path (589 412)(588 411)
\path (588 411)(587 410)
\path (587 410)(586 410)
\path (586 410)(585 409)
\path (585 409)(584 409)
\path (584 409)(582 408)
\path (582 408)(581 408)
\path (581 408)(579 408)
\path (579 408)(577 409)
\path (577 409)(576 409)
\path (576 409)(575 409)
\path (575 409)(574 409)
\path (574 409)(573 410)
\path (573 410)(571 411)
\path (571 411)(569 412)
\path (569 412)(568 412)
\path (568 412)(566 413)
\path (566 413)(565 415)
\path (565 415)(564 416)
\path (564 416)(563 417)
\path (563 417)(562 418)
\path (562 418)(561 420)
\path (561 420)(560 421)
\path (560 421)(559 422)
\path (559 422)(559 424)
\path (559 424)(558 425)
\path (558 425)(558 426)
\path (558 426)(558 427)
\path (558 427)(558 429)
\path (558 429)(558 430)
\path (558 430)(559 431)
\path (559 431)(559 431)
\path (559 431)(559 432)
\path (559 432)(560 433)
\path (560 433)(560 433)
\path (560 433)(561 434)
\path (561 434)(562 435)
\path (562 435)(563 435)
\path (563 435)(564 435)
\path (564 435)(566 436)
\path (566 436)(567 436)
\path (567 436)(568 436)
\path (568 436)(569 435)
\path (569 435)(571 435)
\path (571 435)(572 435)
\path (572 435)(572 435)
\path (572 435)(573 434)
\path (573 434)(575 434)
\path (575 434)(576 433)
\path (576 433)(577 432)
\path (577 432)(579 431)
\path (579 431)(580 430)
\path (580 430)(581 429)
\path (581 429)(582 428)
\path (582 428)(582 427)
\path (582 427)(583 426)
\path (583 426)(584 425)
\path (584 425)(584 424)
\path (584 424)(585 423)
\path (585 423)(585 422)
\path (585 422)(585 421)
\path (585 421)(585 420)
\path (585 420)(585 419)
\path (585 419)(585 418)
\path (585 418)(585 417)
\path (585 417)(585 417)
\path (585 417)(584 416)
\path (584 416)(584 416)
\path (584 416)(583 415)
\path (583 415)(583 414)
\path (583 414)(582 414)
\path (582 414)(581 414)
\path (581 414)(580 413)
\path (580 413)(579 413)
\path (579 413)(578 413)
\path (578 413)(577 413)
\path (577 413)(576 413)
\path (576 413)(575 414)
\path (575 414)(574 414)
\path (574 414)(574 414)
\path (574 414)(573 414)
\path (573 414)(571 415)
\path (571 415)(570 415)
\path (570 415)(569 416)
\path (569 416)(568 417)
\path (568 417)(567 418)
\path (567 418)(567 418)
\path (567 418)(566 419)
\path (566 419)(565 420)
\path (565 420)(565 421)
\path (565 421)(564 422)
\path (564 422)(564 423)
\path (564 423)(563 424)
\path (563 424)(563 424)
\path (563 424)(563 425)
\path (563 425)(563 426)
\path (563 426)(563 427)
\path (563 427)(563 428)
\path (563 428)(563 428)
\path (563 428)(563 429)
\path (563 429)(564 429)
\path (564 429)(564 430)
\path (564 430)(565 430)
\path (565 430)(565 431)
\path (565 431)(566 431)
\path (566 431)(566 431)
\path (566 431)(567 431)
\path (567 431)(568 432)
\path (568 432)(569 432)
\path (569 432)(570 432)
\path (570 432)(571 431)
\path (571 431)(572 431)
\path (572 431)(572 431)
\path (572 431)(572 431)
\path (572 431)(573 431)
\path (573 431)(574 430)
\path (574 430)(575 430)
\path (575 430)(576 429)
\path (576 429)(577 429)
\path (577 429)(578 428)
\path (578 428)(578 427)
\path (578 427)(579 427)
\path (579 427)(579 426)
\path (579 426)(580 425)
\path (580 425)(580 425)
\path (580 425)(581 424)
\path (581 424)(581 423)
\path (581 423)(581 422)
\path (581 422)(581 422)
\path (581 422)(581 421)
\path (581 421)(581 420)
\path (581 420)(581 420)
\path (581 420)(581 419)
\path (581 419)(581 419)
\path (581 419)(581 419)
\path (581 419)(580 418)
\path (580 418)(580 418)
\path (580 418)(579 417)
\path (579 417)(579 417)
\path (579 417)(578 417)
\path (578 417)(578 417)
\path (578 417)(578 417)
\path (578 417)(577 417)
\path (577 417)(576 416)
\path (576 416)(576 417)
\path (576 417)(575 417)
\path (575 417)(574 417)
\path (574 417)(574 417)
\path (574 417)(573 417)
\path (573 417)(573 417)
\path (573 417)(572 418)
\path (572 418)(571 418)
\path (571 418)(570 418)
\path (570 418)(570 419)
\path (570 419)(569 419)
\path (569 419)(569 420)
\path (569 420)(568 421)
\path (568 421)(568 421)
\path (568 421)(567 422)
\path (567 422)(567 422)
\path (567 422)(567 423)
\path (567 423)(566 424)
\path (566 424)(566 424)
\path (566 424)(566 425)
\path (566 425)(566 425)
\path (566 425)(566 426)
\path (566 426)(566 426)
\path (566 426)(566 427)
\path (566 427)(566 427)
\path (566 427)(567 427)
\path (567 427)(567 428)
\path (567 428)(567 428)
\path (567 428)(568 428)
\path (568 428)(568 428)
\path (568 428)(569 429)
\path (569 429)(569 429)
\path (569 429)(569 429)
\path (569 429)(570 429)
\path (570 429)(570 429)
\path (570 429)(571 429)
\path (571 429)(572 429)
\path (572 429)(572 429)
\path (572 429)(572 429)
\path (572 429)(573 428)
\path (573 428)(573 428)
\path (573 428)(574 428)
\path (574 428)(575 428)
\path (575 428)(575 427)
\path (575 427)(576 427)
\path (576 427)(576 426)
\path (576 426)(577 426)
\path (577 426)(577 426)
\path (577 426)(577 425)
\path (577 425)(578 425)
\path (578 425)(578 424)
\path (578 424)(578 424)
\path (578 424)(578 423)
\path (578 423)(579 423)
\path (579 423)(579 422)
\path (579 422)(579 422)
\path (579 422)(579 421)
\path (579 421)(578 421)
\path (578 421)(578 420)
\path (578 420)(578 420)
\path (578 420)(578 420)
\path (578 420)(578 419)
\path (578 419)(577 419)
\path (577 419)(577 419)
\path (577 419)(577 419)
\path (577 419)(576 419)
\path (576 419)(576 419)
\path (576 419)(576 419)
\path (576 419)(575 419)
\path (575 419)(575 419)
\path (575 419)(574 419)
\path (574 419)(574 419)
\path (574 419)(574 419)
\path (574 419)(573 419)
\path (573 419)(573 419)
\path (573 419)(572 419)
\path (572 419)(572 420)
\path (572 420)(571 420)
\path (571 420)(571 420)
\path (571 420)(570 421)
\path (570 421)(570 421)
\path (570 421)(570 421)
\path (570 421)(569 422)
\path (569 422)(569 422)
\path (569 422)(569 423)
\path (569 423)(569 423)
\path (569 423)(569 424)
\path (569 424)(568 424)
\path (568 424)(568 424)
\path (568 424)(568 425)
\path (568 425)(568 425)
\path (568 425)(569 425)
\path (569 425)(569 426)
\path (569 426)(569 426)
\path (569 426)(569 426)
\path (569 426)(569 426)
\path (569 426)(569 427)
\path (569 427)(570 427)
\path (570 427)(570 427)
\path (570 427)(570 427)
\path (570 427)(570 427)
\path (570 427)(571 427)
\path (571 427)(571 427)
\path (571 427)(572 427)
\path (572 427)(572 427)
\path (572 427)(572 427)
\path (572 427)(573 427)
\path (573 427)(573 427)
\path (573 427)(573 427)
\path (573 427)(574 426)
\path (574 426)(574 426)
\path (574 426)(574 426)
\path (574 426)(575 426)
\path (575 426)(575 425)
\path (575 425)(575 425)
\path (575 425)(576 425)
\path (576 425)(576 424)
\path (576 424)(576 424)
\path (576 424)(576 424)
\path (576 424)(577 423)
\path (577 423)(577 423)
\path (577 423)(577 423)
\path (577 423)(577 422)
\path (577 422)(577 422)
\path (577 422)(577 422)
\path (577 422)(577 422)
\path (577 422)(577 421)
\path (577 421)(576 421)
\path (576 421)(576 421)
\path (576 421)(576 421)
\path (576 421)(576 421)
\path (576 421)(576 420)
\path (576 420)(575 420)
\path (575 420)(575 420)
\path (575 420)(575 420)
\path (575 420)(575 420)
\path (575 420)(574 420)
\path (574 420)(574 420)
\path (574 420)(574 420)
\path (574 420)(573 420)
\path (573 420)(573 420)
\path (573 420)(573 420)
\path (573 420)(573 421)
\path (573 421)(572 421)
\path (572 421)(572 421)
\path (572 421)(572 421)
\path (572 421)(572 421)
\path (572 421)(571 422)
\path (571 422)(571 422)
\path (571 422)(571 422)
\path (571 422)(571 422)
\path (571 422)(570 423)
\path (570 423)(570 423)
\path (570 423)(570 423)
\path (570 423)(570 423)
\path (570 423)(570 424)
\path (570 424)(570 424)
\path (570 424)(570 424)
\path (570 424)(570 424)
\path (570 424)(570 425)
\path (570 425)(570 425)
\path (570 425)(570 425)
\path (570 425)(570 425)
\path (570 425)(570 425)
\path (570 425)(571 426)
\path (571 426)(571 426)
\path (571 426)(571 426)
\path (571 426)(571 426)
\path (571 426)(571 426)
\path (571 426)(572 426)
\path (572 426)(572 426)
\path (572 426)(572 426)
\path (572 426)(572 426)
\path (572 426)(573 426)
\path (573 426)(573 426)
\path (573 426)(573 426)
\path (573 426)(573 426)
\path (573 426)(574 425)
\path (574 425)(574 425)
\path (574 425)(574 425)
\path (574 425)(574 425)
\path (574 425)(574 425)
\path (574 425)(575 425)
\path (575 425)(575 424)
\path (575 424)(575 424)
\path (575 424)(575 424)
\path (575 424)(575 424)
\path (575 424)(575 423)
\path (575 423)(576 423)
\path (576 423)(576 423)
\path (576 423)(576 423)
\path (576 423)(576 423)
\path (576 423)(576 422)
\path (576 422)(576 422)
\path (576 422)(575 422)
\path (575 422)(575 422)
\path (575 422)(575 422)
\path (575 422)(575 422)
\path (575 422)(575 422)
\path (575 422)(575 421)
\path (575 421)(575 421)
\path (575 421)(575 421)
\path (575 421)(574 421)
\path (574 421)(574 421)
\path (574 421)(574 421)
\path (574 421)(574 421)
\path (574 421)(574 421)
\path (574 421)(574 421)
\path (574 421)(573 421)
\path (573 421)(573 421)
\path (573 421)(573 421)
\path (573 421)(573 421)
\path (573 421)(573 422)
\path (573 422)(572 422)
\path (572 422)(572 422)
\path (572 422)(572 422)
\path (572 422)(572 422)
\path (572 422)(572 422)
\path (572 422)(571 423)
\path (571 423)(571 423)
\path (571 423)(571 423)
\path (571 423)(571 423)
\path (571 423)(571 423)
\path (571 423)(571 423)
\path (571 423)(571 424)
\path (571 424)(571 424)
\path (571 424)(571 424)
\path (571 424)(571 424)
\path (571 424)(571 424)
\path (571 424)(571 424)
\path (571 424)(571 424)
\path (571 424)(571 425)
\path (571 425)(571 425)
\path (571 425)(571 425)
\path (571 425)(571 425)
\path (571 425)(572 425)
\path (572 425)(572 425)
\path (572 425)(572 425)
\path (572 425)(572 425)
\path (572 425)(572 425)
\path (572 425)(572 425)
\path (572 425)(572 425)
\path (572 425)(573 425)
\path (573 425)(573 425)
\path (573 425)(573 425)
\path (573 425)(573 425)
\path (573 425)(573 425)
\path (573 425)(573 425)
\path (573 425)(574 425)
\path (574 425)(574 424)
\path (574 424)(574 424)
\path (574 424)(574 424)
\path (574 424)(574 424)
\path (574 424)(574 424)
\path (574 424)(574 424)
\path (574 424)(575 424)
\path (575 424)(575 424)
\path (575 424)(575 423)
\path (575 423)(575 423)
\path (575 423)(575 423)
\path (575 423)(575 423)
\path (575 423)(575 423)
\path (575 423)(575 423)
\path (575 423)(575 423)
\path (575 423)(575 422)
\path (575 422)(575 422)
\path (575 422)(575 422)
\path (575 422)(574 422)
\path (574 422)(574 422)
\path (574 422)(574 422)
\path (574 422)(574 422)
\path (574 422)(574 422)
\path (574 422)(574 422)
\path (574 422)(574 422)
\path (574 422)(574 422)
\path (574 422)(574 422)
\path (574 422)(574 422)
\path (574 422)(573 422)
\path (573 422)(573 422)
\path (573 422)(573 422)
\path (573 422)(573 422)
\path (573 422)(573 422)
\path (573 422)(573 422)
\path (573 422)(573 422)
\path (573 422)(572 422)
\path (572 422)(572 422)
\path (572 422)(572 423)
\path (572 423)(572 423)
\path (572 423)(572 423)
\path (572 423)(572 423)
\path (572 423)(572 423)
\path (572 423)(572 423)
\path (572 423)(572 423)
\path (572 423)(572 423)
\path (572 423)(572 424)
\path (572 424)(572 424)
\path (572 424)(572 424)
\path (572 424)(572 424)
\path (572 424)(572 424)
\path (572 424)(572 424)
\path (572 424)(572 424)
\path (572 424)(572 424)
\path (572 424)(572 424)
\path (572 424)(572 424)
\path (572 424)(572 424)
\path (572 424)(572 424)
\path (572 424)(572 424)
\path (572 424)(572 424)
\path (572 424)(572 424)
\linewd 4
\path (215 249)(893 152)
\path (893 152)(1285 319)
\path (1285 319)(607 415)
\path (607 415)(215 249)
\end{texdraw}
}
   \psline[arrowinset=0]{->}(2.43,6.78)(2.43,10.7)
   \rput[ct]{*0}(5.1,6.1){\footnotesize [Br$^-$]}
   \rput[cl]{*0}( 10,6.4){\footnotesize [HBrO$_2$]}
   \rput[cr]{*0}(2.3,8.6){\footnotesize [Ce$^{4+}$]}
   \rput[cl]{*0}(9,9.4){
    \tiny [Br$^-$] $\in [2.16\cdot 10^{-7};2.21\cdot 10^{-7}]$}
   \rput[cl]{*0}(9,9.0){
    \tiny [HBrO$_2$] $\in [3.0\cdot 10^{-8};3.5\cdot 10^{-8}]$}
   \rput[cl]{*0}(9,8.6){
    \tiny [Ce$^{4+}$] $\in [0;1.9\cdot 10^{-7}]$}
  \end{pspicture}
 \end{center}
}
{
\caption{\protect\capsize
En perturbation af Oregonatoren relakserer tilbage til det
station{\ae}re punkt. Intervallerne angiver max.\ og min.\
v{\ae}rdierne for de respektive akser.}
\label{pert:Oregon}
}

I {\aa}bne systemer sker der en tilf{\o}relse af stof
udenfra. Den type eksperimenter udf{\o}res som
reglen i en CSTR. Derfor m{\aa} ligning \ref{Pert:KinEq}
{\ae}ndres til

\begin{equation}
\label{Pert:CSTR}
\frac{d\vec{c}}{dt} = \vec{f}(\vec{c}) + j_0(\vec{c}^0 - \vec{c}),
\end{equation}

hvor $j_0$ er den reciprokke residenstid, og $\vec{c}^0$ er
indf{\o}relseskoncentrationerne. Ved en station{\ae}r
tilstand {\ae}ndres indf{\o}relse af stof $k$ med $\delta
c_k^0$. Efter en transient indtr{\ae}der en ny
station{\ae}r tilstand. Den relative {\ae}ndring i
koncentrationen er $\frac{\delta c_i^s}{\delta c_k^0}$, der
kan tiln{\ae}rmes med $\frac{d c_i^s}{d c_k^0}$. Vi kan
udregne Jacobimatricens elementer i $\vec{c}^0$ ved

\begin{equation}
\frac{d f_i}{d c_k^0} = \sum_{m=1}^n \underbrace{\frac{\partial f_i}{\partial
c_m^s}}_{J_{im}} \underbrace{\frac{d c_m^s}{d c_k^0}}_{\Delta c_{mk}} +
\underbrace{\frac{\partial f_i}{\partial c_k^0}}_{j_0} = 0,
\end{equation}

eller p{\aa} matrixform

\begin{equation}
\Delta\matrix{c} = -j_0\matrix{J}^{-1}.
\end{equation}

Med andre ord: Jacobimatricen kan findes ved {\ae}ndring af
ind\-f{\o}d\-nings\-kon\-cen\-tra\-tionen af stoffer og
kendskab til residenstiden.
 
\section{Mekanismen}
Det er et ganske ambi{\o}st projekt at finde en mekanisme
til et reaktionssystem, som udviser den type opf{\o}rsler,
som denne afhandling omhandler. Men det er dog muligt at
n{\ae}rme sig et resultat, som vi skal se i dette afsnit.
Vi vil antage, at hastighedslovene for mekanismen er
potenslove. Som vi s{\aa} i afsnit \ref{Pert:jac} er det
muligt at finde Jacobimatricen i et station{\ae}rt punkt
ved eksperimenter. Ud fra Jacobimatricen f{\aa}r vi en del
information. Element $(i,j)$ i Jacobimatricen beskriver sig
om stofferne $i$ og $j$. Denne information er summeret i
tabellen nedenfor

\vspace{0.15cm}
\begin{tabular}[h]{cl}
Jacobimatricen & Egenskab \\ \hline
$J_{ii} > 0$    & direkte autokatalyse \\
$J_{ii} = 0$    & produktionen opvejer forbruget af stof $i$ \\
$J_{ii} < 0$    & en af f{\o}lgende muligheder \\
                & \begin{minipage}[t]{5cm}
                  \begin{enumerate}
                     \item ingen direk\-te autokatalyse
                     \item forbruget af det $i$'te stof do\-mi\-ne\-rer
                  \end{enumerate} 
                  \end{minipage} \\
$J_{ij} > 0$    & stof $i$ produceres udfra stof $j$ \\
$J_{ij} = 0$    & ingen reaktion, hvor stof $i$ producerer stof $j$ \\
$J_{ij} < 0$    & stof $i$ forbruges og $j$ ``deltager'' \\ \hline
\end{tabular}
\vspace{0.15cm}

Ved at bruge disse ``regler'', kan vi udlede en mulig
mekanisme for systemet.

\vspace{4.0mm}
Vi vil vise metodens styrke ved at bruge den p{\aa}
Brusselatoren. Brusselatoren er givet ved fire reaktioner
[\citen{Marek2}]

\begin{subequations}
\begin{eqnarray}
A &\rightarrow& X \label{brusA} \\
B+X &\rightarrow& Y + D \label{brusB} \\
2X + Y &\rightarrow& 3X \label{brusC} \\
X &\rightarrow& E \label{brusD}
\end{eqnarray}
\end{subequations}
som svarer til to koblede differentialligninger

\begin{eqnarray}
\frac{dX}{dt} &=& A - (B+1)X + X^2Y \\
\frac{dY}{dt} &=& BX - X^2Y.
\end{eqnarray}

Det station{\ae}re punkt er $\vec{c}^S = (A, \frac{B}{A}$).
Vi er ogs{\aa} i stand til at udregne Jacobimatricen eksakt
i det station{\ae}re punkt

\[
  \matrix{J} = \left( 
  \begin{array}{cc}
  B-1 & A^2 \\
  -B  & -A^2
  \end{array}
  \right).
\]

Vi vil antage, at $B>1$, samt at Jacobimatricen symbolsk
kan skrives som

\[
\matrix{J} = \left(
\begin{array}{cc}
+ & + \\
- & -
\end{array} 
\right),
\]

hvor ``$+$'' betyder en positiv v{\ae}rdi og ``$-$''
betyder en negativ v{\ae}rdi. Elementet $J_{11}$ viser, at
der i Brusselatoren forekommer en reaktion med direkte
autokatalyse af $X$. Dette er reaktion \ref{brusC}. At $X$
produceres ud fra $Y$ svarer til den informationen, der er
indeholdt i elementet $J_{12}$, mens $J_{21}$
fort{\ae}ller, at $X$ forbruges under deltagelse af $Y$
svarende til reaktion \ref{brusB}.

\vspace{4.0mm}
I [\citen{pertstat1}] vises, at en st{\o}rre model ogs{\aa}
kan udledes fra Jacobimatricen. Her er der igen tale om
DOP-modellen. Chevalier {\em et al.} er i stand til
redeg{\o}re for reaktionerne ud fra elementerne i
Jacobimatricen.
