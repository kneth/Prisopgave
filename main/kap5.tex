\chapter{Oscillationer i kemiske reaktioner}
En lang r{\ae}kke matematiske metoder har v{\ae}ret anvendt
til at beskrive ikke-line{\ae}re f{\ae}nomener i kemisk
kinetik. Med hensyn til beskrivelse og
stabili\-tets\-ana\-lyse af oscillationer og mere
kompliceret dynamisk opf{\o}rsel har specielt en r{\ae}kke
bidrag fra Poincar\'{e} og Lyapunov \cite{PoinOrig} vist
sig at v{\ae}re af stor betydning. Endvidere har en
ana\-lyse af line{\ae}re differentiallig\-ninger med
periodiske koefficienter af Floquet i forrige
{\aa}rhundrede vist sig at kunne danne grundlag for at
bestemme periodiske banekurvers stabilitet. Vi vil i dette
afsnit fors{\o}ge at redeg{\o}re for indholdet i disse
teorier, samtidig med at disse uddybes med en r{\ae}kke
eksempler.

\section{Floquetteori og periodiske l{\o}sninger}
Lad os antage, at den kinetiske lig\-ning

\begin{equation}
 \dot{\bf c} = {\bf f}({\bf c},\mu)  
 \label{eq:OdeBasic}
\end{equation}

har en $T$-periodisk l{\o}sning $\varphi_t({\bf c}_p)$
s{\aa}ledes, at denne tilfredsstiller

\begin{equation}
 \varphi_t({\bf c}_p) = \varphi_{t+T}({\bf c}_p)
\end{equation}

Her angiver $T$ alts{\aa} den tid, det tager sy\-stemet at
returnere til punktet ${\bf c}_p$ i faserummet. Lad
ydermere $\gamma$ angive den banekurve som $\varphi_t({\bf
c}_p)$ beskriver i faserummet. Vi {\o}nsker nu at foretage
en generel line{\ae}r stabilitets\-ana\-lyse af denne
l{\o}sning for herved at opstille den periodiske
l{\o}snings eksakte stabilitets\-kriterier. Dette g{\o}res
ved at betragte en infinitesimal perturbation $\epsilon$ af
den periodiske l{\o}sning v{\ae}k fra banekurven $\gamma$
og ana\-lysere dennes tidsudvikling i gr{\ae}nsen $t
\rightarrow \infty$. Formelt kan perturbationens variation
som funktion af tiden udtrykkes som

\begin{equation}
 \epsilon(t) = \varphi_t({\bf c}_p+\epsilon) - \varphi_{t}({\bf c}_p)
\end{equation}

Ved Taylorudvikling af $\epsilon$ omkring ${\bf c}_p$ til
f{\o}rste orden finder man

\begin{eqnarray}
 \epsilon(t) & = &
 \varphi_t({\bf c}_p) + 
 \left.
 \frac{\partial \varphi_t({\bf c}_0)}{\partial {\bf c}_0}
 \right|_{{\bf c}_p}\negsp\epsilon - 
 \varphi_t({\bf c}_p) \nonumber\\
 & = &
 \left.
 \frac{\partial \varphi_t({\bf c}_0)}{\partial {\bf c}_0}
 \right|_{{\bf c}_p}\negsp\epsilon
 \label{eq:basicmono}
\end{eqnarray}

Tidsudviklingen for perturbationen $\epsilon$ er alts{\aa} fuldst{\ae}ndigt
bestemt udfra matricen 

$$
 \left.\frac{\partial \varphi_t({\bf c}_0)}
 {\partial {\bf c}_0}\right|_{{\bf c}_p}
$$ 

der ofte angives ved symbolet ${\bf M}(t)$. Dennes
v{\ae}rdi ${\bf M}(T)$ for $t=T$ kaldes for
monodromimatricen og skal senere vise sig at udg{\o}re
hovedredskabet i de stabilitetskriterier, vi er p{\aa} vej
til at udlede. Den form, hvormed ${\bf M}(t)$ er angivet i
lig\-ning~\ref{eq:basicmono} er dog ikke s{\ae}rlig
hensigtsm{\ae}ssig at arbejde videre med. Vi skal derfor
bestemme en differentiallig\-ning, der angivet med de rette
begyndelsesbetingelser netop har ${\bf M}(t)$ som
l{\o}sning. Hertil betragtes den generelle l{\o}sning
$\varphi_t({\bf c}_0)$ til lig\-ning~\ref{eq:basicmono}.
Differentieres den generelle l{\o}sning b{\aa}de med hensyn
til tiden $t$ og begyndelsesbetingelserne ${\bf c}_0$,
finder man

\begin{eqnarray}
 \frac{\partial}{\partial {\bf c}_0}
 \frac{d \varphi_t({\bf c}_0)}{dt} & = &
 \frac{\partial {\bf f}(\varphi_t({\bf c}_0),\mu)}
      {\partial {\bf c}_0} \Rightarrow  \nonumber\\
 & &                                    \nonumber\\
 \frac{d}{dt}
 \frac{\partial\varphi_t({\bf c}_0)}{\partial{\bf c}_0} & = &
 \frac{\partial {\bf f}(\varphi_t({\bf c}_0),\mu)}
      {\partial \varphi_t({\bf c}_0)}
 \frac{\partial \varphi_t({\bf c}_0)} {\partial {\bf c}_0}
 \label{eq:monotempdiff}
\end{eqnarray}

S{\ae}ttes ${\bf c}_0$ nu lig punktet ${\bf c}_p$ p{\aa}
den periodiske banekurve $\gamma$, ses at
lig\-ning~\ref{eq:monotempdiff} er {\ae}kvivalent med
differentiallig\-ningen

\begin{equation}
 \dot{{\bf M}} = \left.{\bf J}\right|_{\varphi_t({\bf c}_p)} {\bf M}, 
 \mbox{\ \ hvor \ \ } {\bf M}(0) = {\bf I}
 \label{eq:monodiff}
\end{equation}

hvor ${\bf J}$ angiver Jacobimatricens v{\ae}rdi taget
langs med gr{\ae}nsecyklusen.
Lig\-ning~\ref{eq:monotempdiff} er alts{\aa} en line{\ae}r
differentiallig\-ning med periodiske koefficienter, da
v{\ae}rdien af ${\bf J}$ netop er periodisk langs med
l{\o}sningen $\varphi_t({\bf c}_0)$. Egenskaberne ved denne
familie af differentiallig\-ninger, der for f{\o}rste gang
blev unders{\o}gt af Floquet, viser sig at v{\ae}re af stor
betydning for den kommende stabilitetsana\-lyse, hvorfor vi
vil ofre et par linier p{\aa} at opn{\aa} en st{\o}rre
indsigt i s{\aa}danne lig\-ninger. Floquet teoremet
udsiger, at for differentiallig\-ningen

\begin{equation}
 \dot{\bf v} = A(t){\bf v}, \mbox{\ \ hvor \ \ }
 A(t) = A(t+T)
 \label{eq:floquetdiff}
\end{equation}

vil enhver l{\o}sning ${\bf v}(t)$ kunne skrives som
produktet af en periodisk matrix ${\bf P}(t)$ og
eksponentialfunktionen til en konstant matrix ${\bf B}$:

\begin{equation}
 {\bf v}(t) = {\bf P}(t) e^{{\bf B}t}.
\end{equation}

Lad nu ${\bf v}_1,\ldots,{\bf v}_n$ v{\ae}re $n$
line{\ae}rt uafh{\ae}ngige l{\o}sninger til
lig\-ning~\ref{eq:floquetdiff}. Herved defineres den
fundamentale l{\o}snings matrix ${\bf V}(t)$ som

\begin{equation}
{\bf V}(t) = \left[{\bf v}_1,\ldots,{\bf v}_n\right]
\end{equation}

hvor ${\bf v}_i$ alts{\aa} udg{\o}r den $i$'te s{\o}jle i
matricen ${\bf V}(t)$. Betragtes nu ${\bf V}(t+T)$ ser vi
via variabelskiftet $\tau=t+T$, at

\begin{equation}
\dot{\bf V}(\tau) = {\bf A}(t) {\bf V}(\tau)
\end{equation}

hvorfor ${\bf V}(t+T)$ ogs{\aa} er en l{\o}sning til
lig\-ning~\ref{eq:floquetdiff}. Med andre ord m{\aa} der
alts{\aa} findes et basisskift ${\bf C}$, s{\aa}ledes at

\begin{equation}
{\bf V}(t+T) = {\bf V}(t){\bf C}
\end{equation}

Lad nu ${\bf M}(t)$ v{\ae}re en fundamental
l{\o}sningsmatrix med egenskaben ${\bf M}(0) = {\bf I}$.
Generelt m{\aa} der for ${\bf M}(t)$ if{\o}lge de
ovenst{\aa}ende betragtninger g{\ae}lde

$$
 {\bf M}(t+T) = {\bf M}(t) {\bf C} 
$$ 

der ved inds{\ae}ttelse af $t=0$ giver

\begin{equation}
 {\bf M}(T) = {\bf C} 
\end{equation}

Vi har alts{\aa} ${\bf M}(t+T)={\bf M}(t){\bf M}(T)$,
hvorfor vi med et simpelt induktionsargument slutter

\begin{equation}
 {\bf M}(nT) = {\bf M}^n(T)
\end{equation}

Lad nu $\Psi_1(T),\ldots,\Psi_n(T)$ og
$\lambda_1(T),\ldots,\lambda_n(T)$ v{\ae}re henholdsvis
egenvektorer og egenv{\ae}rdier til matricen ${\bf M}(T)$.
Per definition har vi alts{\aa}

\begin{eqnarray}
 {\bf M}(T) \Psi_j(T)  & = & \lambda_j(T) \Psi_j(T)   \Rightarrow \nonumber\\ 
{\bf M}(nT) \Psi_j(T) = 
{\bf M}^n(T)\Psi_j(T)  & = & \lambda^n_j(T) \Psi_j(T) \Rightarrow \nonumber\\ 
	 \lambda_j(nT) & = & \lambda^n_j(T)  
 \label{eq:expfunk}
\end{eqnarray}

Da lig\-ning~\ref{eq:expfunk} netop er
funktionallig\-ningen for eksponentialfunktionen, m{\aa}
$\lambda_j(T)$ alts{\aa} tilfredsstille

\begin{equation}
 \lambda_j(T) = e^{\sigma_j T},\mbox{\ \ }
 \sigma_j = \alpha_j + i\beta_j + p\frac{2\pi}{T},
 \label{eq:floq-mult-exp}
\end{equation}

hvor $p \in \Z$. $\lambda_j(T)$ og $\sigma_j$ kaldes
henholdsvis Floquetmultiplikatorer og Floqueteksponenter.
Man b{\o}r bem{\ae}rke, at Floqueteksponenterne ikke er
entydigt fastlagt svarende til et multiplum
$\frac{2\pi}{T}$. Lad os nu betragte vektorfunktionerne
${\bf v}_1(t),\ldots,{\bf v}_n(t)$ defineret ved

\begin{equation}
 {\bf v}_j(t) = {\bf M}(t) \Psi_j
\end{equation}

hvor vi specielt bem{\ae}rker, at systemet af
l{\o}sningsvektorer $\left[{\bf v}_1(t),\ldots,{\bf
v}_n(t)\right]$ udg{\o}r en fundamental l{\o}sningsmatrix
til lig\-ning~\ref{eq:floquetdiff}. Udfra betragtningen

\begin{eqnarray} {\bf v}_j(t+T) & = & {\bf M}(t+T)            \Psi_j
\nonumber\\ & = & {\bf M}(t){\bf M}(T)    \Psi_j
\nonumber\\ & = & e^{\sigma_j T}{\bf M}(t)\Psi_j
\nonumber\\ {\bf v}_j(t+T) & = & e^{\sigma_j T}{\bf v}_j(t)
\end{eqnarray}

ser vi, at ${\bf v}_j(t)$ er $T$-periodisk bortset fra en
eksponential faktor. Definerer vi derfor vektorfunktionen
${\bf p}_j(t)$ som

\begin{equation}
 {\bf p}_j(t) = e^{-\sigma_j t}{\bf v}_j(t)
\end{equation}

ser vi udfra overvejelserne

\begin{eqnarray}
 {\bf p}_j(t+T) & = & e^{-\sigma_j (t+T)}{\bf v}_j(t+T)    \nonumber\\
		& = & e^{-\sigma_j t} e^{-\sigma_j T} 
		      e^{ \sigma_j T}{\bf v}_j(t)          \nonumber\\
		& = & e^{-\sigma_j t}{\bf v}_j(t)          \nonumber\\
 {\bf p}_j(t+T) & = & {\bf p}_j(t)
\end{eqnarray}

at ${\bf p}_j(t)$ er $T$-periodisk, hvorfor
vektorfunktionen ${\bf v}_j(t)$ netop kan skrives som et
produkt mellem en periodisk funktion og en eksponential
faktor. Med denne viden kan vi nu drage de {\o}nskede
konklusioner vedr{\o}rende stabiliteten af den periodiske
l{\o}sning $\varphi_t({\bf c}_p)$. Antager vi nu, at
egenvektorene $\Psi_1(T),\ldots,\Psi_n(T)$ udg{\o}r en
basis i \R$^n$, f{\o}lger det, at tidsudviklingen af
perturbationen $\epsilon$ kan skrives som

\begin{eqnarray}
 \epsilon (t)& = & {\bf M}(t) \epsilon                \nonumber\\
	     & = & {\bf M}(t) \sum_{j=1}^n a_j \Psi_j \nonumber\\
 \epsilon (t)& = & \sum_{j=1}^n a_j {\bf p}_j(t) e^{\sigma_j t}
 \label{eq:PertExpan}
\end{eqnarray}

hvorfor l{\o}sningen $\varphi_t({\bf c}_p)$ netop er
stabil, hvis der for alle $j\in\{1,\ldots,n\}$ g{\ae}lder

\begin{equation}
 \left| \lambda_j(T) \right| < 1 \mbox{\ \ eller \ \ }
 \sigma_j < 0
\end{equation}

Omvendt er $\varphi_t({\bf c}_p)$ ustabil, hvis mindst et
$j\in\{1,\ldots,n\}$ tilfredsstiller

\begin{equation}
 \left| \lambda_j(T) \right| \geq 1 \mbox{\ \ eller \ \ }
 \sigma_j \geq 0
\end{equation}

\boxfigure{t}{\textwidth}
{
 \begin{minipage}[t]{6cm}
\vspace{7.5mm}
\begin{picture}(159,100)(-110,-50)
 
%circle 1
\put (-50,0){\vector(1,0){100}}
\put (0,-50){\vector(0,1){100}}
\put (0,0){\circle{100}}
 
\put (20,0){\circle*{3}}
\put (-15,0){\circle*{3}}
\put (-10,10){\circle*{3}}
\put (-10,-10){\circle*{3}}
\put (15,5){\circle*{3}}
\put (15,-5){\circle*{3}}
\put (-5,0){\circle*{3}}
%circle1
\put (52,-2){$Re\,  \lambda_j$}
\put (-12,52){$Im\,  \lambda_j$}
\put (-45,40){a)}
\end{picture}
\end{minipage}
\ \hfill \
\begin{minipage}[t]{6cm}
\vspace{7.5mm}
\begin{picture}(159,100)(70,-50)
%circle2
\put (90,0){\vector(1,0){100}}
\put (140,-50){\vector(0,1){100}}
 
\put (140,0){\circle*{3}}
\put (130,15){\circle*{3}}
\put (130,-15){\circle*{3}}
\put (125,5){\circle*{3}}
\put (125,-5){\circle*{3}}
\put (115,0){\circle*{3}}
\put (105,0){\circle*{3}}
%circle2
\put (192,-2){$Re\,  \sigma_j$}
\put (128,52){$Im\,  \sigma_j$}
\put (95,40){b)}
\end{picture}
\end{minipage}
\vspace{5mm}


}
{
\caption{\protect\capsize
	 Stabilitetskriterier for en periodisk l{\o}sning til
	 en s{\ae}dvanlig differentiallig\-ning:
	 a) \ $\mid \lambda_j(T) \mid < 1$ og
	 b) \ $Re\, \sigma_j < 0$.}
\label{fig:stab}
}

Disse stabilitetsresultater kan bekvemt illustreres
grafisk som vist i figur~\ref{fig:stab}. 

\vspace{4.0mm}
Om disse Floquetmuliplikatorer og -eksponenter g{\ae}lder
der ydermere, at en enkelt af disse altid tilfredsstiller
$\lambda=1$ og $\sigma=0$. Umiddelbart virker denne
egenskab intuitivt naturlig, idet vi jo altid kan foretage
perturbationen i gr{\ae}nsecyklusens retning, svarende til
at vi forbliver p{\aa} gr{\ae}nsecyklusen. Beviset for
dette er meget simpelt, men er sj{\ae}ldent forekommende i
litteraturen, hvorfor gangen i dette vil blive illustreret
her.

\vspace{4.0mm}
Lad $\epsilon_0$ v{\ae}re en perturbation af
gr{\ae}nsecyklusen $\gamma$ s{\aa}ledes, at denne blot
``rammer'' et andet punkt p{\aa} $\gamma$ (se
figur~\ref{fig:lambda=1}). Formelt kan tidsudviklingen
$\epsilon (t)$ af perturbationen udtrykkes som

\begin{equation}
\epsilon (t) = \varphi_t({\bf c}_p+\epsilon_0) - \varphi_t({\bf c}_p)
\end{equation}

Men da perturbationen udelukkende svarer til et skift fra
\'{e}t punkt p{\aa} $\gamma$ til et andet, betyder dette,
at der findes et tidsrum $dt$, s{\aa}ledes at
$\varphi_{t+dt}({\bf c}_p) = \varphi_t({\bf
c}_p+\epsilon_0)$. Med andre ord har vi alts{\aa}

\begin{equation}
 \epsilon (t) = \varphi_{t+dt}({\bf c}_p) - \varphi_t({\bf c}_p)
\end{equation}

Men da $\varphi_{t}({\bf c}_p)$ jo var $T$-periodisk,
m{\aa} der alts{\aa} g{\ae}lde

\begin{equation}
 \epsilon (t) = \epsilon (t+T)
\end{equation}

Tidsudviklingen for $\epsilon (t)$ kan, som vi tidligere
har vist, skrives som $\epsilon (t) = {\bf
M}(t)\epsilon_0$, hvorfor der specielt til tiden $t+T$
g{\ae}lder

\begin{equation}
 \epsilon (t+T) = {\bf M}(t){\bf M}(T)\epsilon_0
\end{equation}

Men da $\epsilon (t) = \epsilon (t+T)$, f{\aa}r vi, at der
m{\aa} g{\ae}lde

\begin{equation}
  \epsilon (t+T) = {\bf M}(t){\bf M}(T)\epsilon_0 = 
  \epsilon (t) = {\bf M}(t)\epsilon_0
\end{equation}

Denne relation udtrykker netop, at ${\bf M}(T)\epsilon_0 =
\epsilon_0$, hvorfor perturbationen $\epsilon_0$ er en
Floquetegenvektor til monodromimatricen med tilh{\o}rende
Floquet\-multi\-plikatorer og -eksponenter $\lambda=1$ og
$\sigma=0$. Dette faktum er af stor betydning for numeriske
bestemmelser af s{\aa}vel monodromimatricen og de
tilh{\o}rende Floquetmultiplikatorer og eksponenter, idet
dette tillader en vurdering af den numeriske
n{\o}jagtighed, med hvilken disse er bestemt.

\boxfigure{t}{\textwidth}{
 \vspace{-2cm}
 % GNUPLOT: LaTeX picture
\setlength{\unitlength}{0.240900pt}
\ifx\plotpoint\undefined\newsavebox{\plotpoint}\fi
\sbox{\plotpoint}{\rule[-0.175pt]{0.350pt}{0.350pt}}%
\begin{picture}(1500,900)(0,0)
\tenrm
\sbox{\plotpoint}{\rule[-0.175pt]{0.350pt}{0.350pt}}%
\put(616,204){\makebox(0,0)[l]{{\small $\varphi_t({\bf x}_p)$}}}
\put(1028,466){\makebox(0,0)[l]{{\small $\varphi_{t+dt}({\bf x}_p)$}}}
\put(755,362){\makebox(0,0)[l]{{\small $\epsilon_0$}}}
\put(611,483){\makebox(0,0)[l]{{\small $\gamma$}}}
\put(616,229){\rule[-0.175pt]{0.390pt}{0.350pt}}
\put(617,230){\rule[-0.175pt]{0.390pt}{0.350pt}}
\put(619,231){\rule[-0.175pt]{0.390pt}{0.350pt}}
\put(620,232){\rule[-0.175pt]{0.390pt}{0.350pt}}
\put(622,233){\rule[-0.175pt]{0.390pt}{0.350pt}}
\put(624,234){\rule[-0.175pt]{0.390pt}{0.350pt}}
\put(625,235){\rule[-0.175pt]{0.390pt}{0.350pt}}
\put(627,236){\rule[-0.175pt]{0.390pt}{0.350pt}}
\put(628,237){\rule[-0.175pt]{0.390pt}{0.350pt}}
\put(630,238){\rule[-0.175pt]{0.390pt}{0.350pt}}
\put(632,239){\rule[-0.175pt]{0.390pt}{0.350pt}}
\put(633,240){\rule[-0.175pt]{0.390pt}{0.350pt}}
\put(635,241){\rule[-0.175pt]{0.390pt}{0.350pt}}
\put(637,242){\rule[-0.175pt]{0.390pt}{0.350pt}}
\put(638,243){\rule[-0.175pt]{0.390pt}{0.350pt}}
\put(640,244){\rule[-0.175pt]{0.390pt}{0.350pt}}
\put(641,245){\rule[-0.175pt]{0.390pt}{0.350pt}}
\put(643,246){\rule[-0.175pt]{0.390pt}{0.350pt}}
\put(645,247){\rule[-0.175pt]{0.390pt}{0.350pt}}
\put(646,248){\rule[-0.175pt]{0.390pt}{0.350pt}}
\put(648,249){\rule[-0.175pt]{0.390pt}{0.350pt}}
\put(650,250){\rule[-0.175pt]{0.390pt}{0.350pt}}
\put(651,251){\rule[-0.175pt]{0.390pt}{0.350pt}}
\put(653,252){\rule[-0.175pt]{0.390pt}{0.350pt}}
\put(654,253){\rule[-0.175pt]{0.390pt}{0.350pt}}
\put(656,254){\rule[-0.175pt]{0.390pt}{0.350pt}}
\put(658,255){\rule[-0.175pt]{0.390pt}{0.350pt}}
\put(659,256){\rule[-0.175pt]{0.390pt}{0.350pt}}
\put(661,257){\rule[-0.175pt]{0.390pt}{0.350pt}}
\put(662,258){\rule[-0.175pt]{0.390pt}{0.350pt}}
\put(664,259){\rule[-0.175pt]{0.390pt}{0.350pt}}
\put(666,260){\rule[-0.175pt]{0.390pt}{0.350pt}}
\put(667,261){\rule[-0.175pt]{0.390pt}{0.350pt}}
\put(669,262){\rule[-0.175pt]{0.390pt}{0.350pt}}
\put(671,263){\rule[-0.175pt]{0.390pt}{0.350pt}}
\put(672,264){\rule[-0.175pt]{0.390pt}{0.350pt}}
\put(674,265){\rule[-0.175pt]{0.390pt}{0.350pt}}
\put(675,266){\rule[-0.175pt]{0.390pt}{0.350pt}}
\put(677,267){\rule[-0.175pt]{0.390pt}{0.350pt}}
\put(679,268){\rule[-0.175pt]{0.390pt}{0.350pt}}
\put(680,269){\rule[-0.175pt]{0.390pt}{0.350pt}}
\put(682,270){\rule[-0.175pt]{0.390pt}{0.350pt}}
\put(684,271){\rule[-0.175pt]{0.390pt}{0.350pt}}
\put(685,272){\rule[-0.175pt]{0.390pt}{0.350pt}}
\put(687,273){\rule[-0.175pt]{0.390pt}{0.350pt}}
\put(688,274){\rule[-0.175pt]{0.390pt}{0.350pt}}
\put(690,275){\rule[-0.175pt]{0.390pt}{0.350pt}}
\put(692,276){\rule[-0.175pt]{0.390pt}{0.350pt}}
\put(693,277){\rule[-0.175pt]{0.390pt}{0.350pt}}
\put(695,278){\rule[-0.175pt]{0.390pt}{0.350pt}}
\put(696,279){\rule[-0.175pt]{0.390pt}{0.350pt}}
\put(698,280){\rule[-0.175pt]{0.390pt}{0.350pt}}
\put(700,281){\rule[-0.175pt]{0.390pt}{0.350pt}}
\put(701,282){\rule[-0.175pt]{0.390pt}{0.350pt}}
\put(703,283){\rule[-0.175pt]{0.390pt}{0.350pt}}
\put(705,284){\rule[-0.175pt]{0.390pt}{0.350pt}}
\put(706,285){\rule[-0.175pt]{0.390pt}{0.350pt}}
\put(708,286){\rule[-0.175pt]{0.390pt}{0.350pt}}
\put(709,287){\rule[-0.175pt]{0.390pt}{0.350pt}}
\put(711,288){\rule[-0.175pt]{0.390pt}{0.350pt}}
\put(713,289){\rule[-0.175pt]{0.390pt}{0.350pt}}
\put(714,290){\rule[-0.175pt]{0.390pt}{0.350pt}}
\put(716,291){\rule[-0.175pt]{0.390pt}{0.350pt}}
\put(718,292){\rule[-0.175pt]{0.390pt}{0.350pt}}
\put(719,293){\rule[-0.175pt]{0.390pt}{0.350pt}}
\put(721,294){\rule[-0.175pt]{0.390pt}{0.350pt}}
\put(722,295){\rule[-0.175pt]{0.390pt}{0.350pt}}
\put(724,296){\rule[-0.175pt]{0.390pt}{0.350pt}}
\put(726,297){\rule[-0.175pt]{0.390pt}{0.350pt}}
\put(727,298){\rule[-0.175pt]{0.390pt}{0.350pt}}
\put(729,299){\rule[-0.175pt]{0.390pt}{0.350pt}}
\put(730,300){\rule[-0.175pt]{0.390pt}{0.350pt}}
\put(732,301){\rule[-0.175pt]{0.390pt}{0.350pt}}
\put(734,302){\rule[-0.175pt]{0.390pt}{0.350pt}}
\put(735,303){\rule[-0.175pt]{0.390pt}{0.350pt}}
\put(737,304){\rule[-0.175pt]{0.390pt}{0.350pt}}
\put(739,305){\rule[-0.175pt]{0.390pt}{0.350pt}}
\put(740,306){\rule[-0.175pt]{0.390pt}{0.350pt}}
\put(742,307){\rule[-0.175pt]{0.390pt}{0.350pt}}
\put(743,308){\rule[-0.175pt]{0.390pt}{0.350pt}}
\put(745,309){\rule[-0.175pt]{0.390pt}{0.350pt}}
\put(747,310){\rule[-0.175pt]{0.390pt}{0.350pt}}
\put(748,311){\rule[-0.175pt]{0.390pt}{0.350pt}}
\put(750,312){\rule[-0.175pt]{0.390pt}{0.350pt}}
\put(752,313){\rule[-0.175pt]{0.390pt}{0.350pt}}
\put(753,314){\rule[-0.175pt]{0.390pt}{0.350pt}}
\put(755,315){\rule[-0.175pt]{0.390pt}{0.350pt}}
\put(756,316){\rule[-0.175pt]{0.390pt}{0.350pt}}
\put(758,317){\rule[-0.175pt]{0.390pt}{0.350pt}}
\put(760,318){\rule[-0.175pt]{0.390pt}{0.350pt}}
\put(761,319){\rule[-0.175pt]{0.390pt}{0.350pt}}
\put(763,320){\rule[-0.175pt]{0.390pt}{0.350pt}}
\put(764,321){\rule[-0.175pt]{0.390pt}{0.350pt}}
\put(766,322){\rule[-0.175pt]{0.390pt}{0.350pt}}
\put(768,323){\rule[-0.175pt]{0.390pt}{0.350pt}}
\put(769,324){\rule[-0.175pt]{0.390pt}{0.350pt}}
\put(771,325){\rule[-0.175pt]{0.390pt}{0.350pt}}
\put(773,326){\rule[-0.175pt]{0.390pt}{0.350pt}}
\put(774,327){\rule[-0.175pt]{0.390pt}{0.350pt}}
\put(776,328){\rule[-0.175pt]{0.390pt}{0.350pt}}
\put(777,329){\rule[-0.175pt]{0.390pt}{0.350pt}}
\put(779,330){\rule[-0.175pt]{0.390pt}{0.350pt}}
\put(781,331){\rule[-0.175pt]{0.390pt}{0.350pt}}
\put(782,332){\rule[-0.175pt]{0.390pt}{0.350pt}}
\put(784,333){\rule[-0.175pt]{0.390pt}{0.350pt}}
\put(786,334){\rule[-0.175pt]{0.390pt}{0.350pt}}
\put(787,335){\rule[-0.175pt]{0.390pt}{0.350pt}}
\put(789,336){\rule[-0.175pt]{0.390pt}{0.350pt}}
\put(790,337){\rule[-0.175pt]{0.390pt}{0.350pt}}
\put(792,338){\rule[-0.175pt]{0.390pt}{0.350pt}}
\put(794,339){\rule[-0.175pt]{0.390pt}{0.350pt}}
\put(795,340){\rule[-0.175pt]{0.390pt}{0.350pt}}
\put(797,341){\rule[-0.175pt]{0.390pt}{0.350pt}}
\put(798,342){\rule[-0.175pt]{0.390pt}{0.350pt}}
\put(800,343){\rule[-0.175pt]{0.390pt}{0.350pt}}
\put(802,344){\rule[-0.175pt]{0.390pt}{0.350pt}}
\put(803,345){\rule[-0.175pt]{0.390pt}{0.350pt}}
\put(805,346){\rule[-0.175pt]{0.390pt}{0.350pt}}
\put(807,347){\rule[-0.175pt]{0.390pt}{0.350pt}}
\put(808,348){\rule[-0.175pt]{0.390pt}{0.350pt}}
\put(810,349){\rule[-0.175pt]{0.390pt}{0.350pt}}
\put(811,350){\rule[-0.175pt]{0.390pt}{0.350pt}}
\put(813,351){\rule[-0.175pt]{0.390pt}{0.350pt}}
\put(815,352){\rule[-0.175pt]{0.390pt}{0.350pt}}
\put(816,353){\rule[-0.175pt]{0.390pt}{0.350pt}}
\put(818,354){\rule[-0.175pt]{0.390pt}{0.350pt}}
\put(820,355){\rule[-0.175pt]{0.390pt}{0.350pt}}
\put(821,356){\rule[-0.175pt]{0.390pt}{0.350pt}}
\put(823,357){\rule[-0.175pt]{0.390pt}{0.350pt}}
\put(824,358){\rule[-0.175pt]{0.390pt}{0.350pt}}
\put(826,359){\rule[-0.175pt]{0.390pt}{0.350pt}}
\put(828,360){\rule[-0.175pt]{0.390pt}{0.350pt}}
\put(829,361){\rule[-0.175pt]{0.390pt}{0.350pt}}
\put(831,362){\rule[-0.175pt]{0.390pt}{0.350pt}}
\put(833,363){\rule[-0.175pt]{0.390pt}{0.350pt}}
\put(834,364){\rule[-0.175pt]{0.390pt}{0.350pt}}
\put(836,365){\rule[-0.175pt]{0.390pt}{0.350pt}}
\put(837,366){\rule[-0.175pt]{0.390pt}{0.350pt}}
\put(839,367){\rule[-0.175pt]{0.390pt}{0.350pt}}
\put(841,368){\rule[-0.175pt]{0.390pt}{0.350pt}}
\put(842,369){\rule[-0.175pt]{0.390pt}{0.350pt}}
\put(844,370){\rule[-0.175pt]{0.390pt}{0.350pt}}
\put(845,371){\rule[-0.175pt]{0.390pt}{0.350pt}}
\put(847,372){\rule[-0.175pt]{0.390pt}{0.350pt}}
\put(849,373){\rule[-0.175pt]{0.390pt}{0.350pt}}
\put(850,374){\rule[-0.175pt]{0.390pt}{0.350pt}}
\put(852,375){\rule[-0.175pt]{0.390pt}{0.350pt}}
\put(854,376){\rule[-0.175pt]{0.390pt}{0.350pt}}
\put(855,377){\rule[-0.175pt]{0.390pt}{0.350pt}}
\put(857,378){\rule[-0.175pt]{0.390pt}{0.350pt}}
\put(858,379){\rule[-0.175pt]{0.390pt}{0.350pt}}
\put(860,380){\rule[-0.175pt]{0.390pt}{0.350pt}}
\put(862,381){\rule[-0.175pt]{0.390pt}{0.350pt}}
\put(863,382){\rule[-0.175pt]{0.390pt}{0.350pt}}
\put(865,383){\rule[-0.175pt]{0.390pt}{0.350pt}}
\put(867,384){\rule[-0.175pt]{0.390pt}{0.350pt}}
\put(868,385){\rule[-0.175pt]{0.390pt}{0.350pt}}
\put(870,386){\rule[-0.175pt]{0.390pt}{0.350pt}}
\put(871,387){\rule[-0.175pt]{0.390pt}{0.350pt}}
\put(873,388){\rule[-0.175pt]{0.390pt}{0.350pt}}
\put(875,389){\rule[-0.175pt]{0.390pt}{0.350pt}}
\put(876,390){\rule[-0.175pt]{0.390pt}{0.350pt}}
\put(878,391){\rule[-0.175pt]{0.390pt}{0.350pt}}
\put(879,392){\rule[-0.175pt]{0.390pt}{0.350pt}}
\put(881,393){\rule[-0.175pt]{0.390pt}{0.350pt}}
\put(883,394){\rule[-0.175pt]{0.390pt}{0.350pt}}
\put(884,395){\rule[-0.175pt]{0.390pt}{0.350pt}}
\put(886,396){\rule[-0.175pt]{0.390pt}{0.350pt}}
\put(888,397){\rule[-0.175pt]{0.390pt}{0.350pt}}
\put(889,398){\rule[-0.175pt]{0.390pt}{0.350pt}}
\put(891,399){\rule[-0.175pt]{0.390pt}{0.350pt}}
\put(892,400){\rule[-0.175pt]{0.390pt}{0.350pt}}
\put(894,401){\rule[-0.175pt]{0.390pt}{0.350pt}}
\put(896,402){\rule[-0.175pt]{0.390pt}{0.350pt}}
\put(897,403){\rule[-0.175pt]{0.390pt}{0.350pt}}
\put(899,404){\rule[-0.175pt]{0.390pt}{0.350pt}}
\put(901,405){\rule[-0.175pt]{0.390pt}{0.350pt}}
\put(902,406){\rule[-0.175pt]{0.390pt}{0.350pt}}
\put(904,407){\rule[-0.175pt]{0.390pt}{0.350pt}}
\put(905,408){\rule[-0.175pt]{0.390pt}{0.350pt}}
\put(907,409){\rule[-0.175pt]{0.390pt}{0.350pt}}
\put(909,410){\rule[-0.175pt]{0.390pt}{0.350pt}}
\put(910,411){\rule[-0.175pt]{0.390pt}{0.350pt}}
\put(912,412){\rule[-0.175pt]{0.390pt}{0.350pt}}
\put(913,413){\rule[-0.175pt]{0.390pt}{0.350pt}}
\put(915,414){\rule[-0.175pt]{0.390pt}{0.350pt}}
\put(917,415){\rule[-0.175pt]{0.390pt}{0.350pt}}
\put(918,416){\rule[-0.175pt]{0.390pt}{0.350pt}}
\put(920,417){\rule[-0.175pt]{0.390pt}{0.350pt}}
\put(922,418){\rule[-0.175pt]{0.390pt}{0.350pt}}
\put(923,419){\rule[-0.175pt]{0.390pt}{0.350pt}}
\put(925,420){\rule[-0.175pt]{0.390pt}{0.350pt}}
\put(926,421){\rule[-0.175pt]{0.390pt}{0.350pt}}
\put(928,422){\rule[-0.175pt]{0.390pt}{0.350pt}}
\put(930,423){\rule[-0.175pt]{0.390pt}{0.350pt}}
\put(931,424){\rule[-0.175pt]{0.390pt}{0.350pt}}
\put(933,425){\rule[-0.175pt]{0.390pt}{0.350pt}}
\put(935,426){\rule[-0.175pt]{0.390pt}{0.350pt}}
\put(936,427){\rule[-0.175pt]{0.390pt}{0.350pt}}
\put(938,428){\rule[-0.175pt]{0.390pt}{0.350pt}}
\put(939,429){\rule[-0.175pt]{0.390pt}{0.350pt}}
\put(941,430){\rule[-0.175pt]{0.390pt}{0.350pt}}
\put(943,431){\rule[-0.175pt]{0.390pt}{0.350pt}}
\put(944,432){\rule[-0.175pt]{0.390pt}{0.350pt}}
\put(946,433){\rule[-0.175pt]{0.390pt}{0.350pt}}
\put(947,434){\rule[-0.175pt]{0.390pt}{0.350pt}}
\put(949,435){\rule[-0.175pt]{0.390pt}{0.350pt}}
\put(951,436){\rule[-0.175pt]{0.390pt}{0.350pt}}
\put(952,437){\rule[-0.175pt]{0.390pt}{0.350pt}}
\put(954,438){\rule[-0.175pt]{0.390pt}{0.350pt}}
\put(956,439){\rule[-0.175pt]{0.390pt}{0.350pt}}
\put(957,440){\rule[-0.175pt]{0.390pt}{0.350pt}}
\put(959,441){\rule[-0.175pt]{0.390pt}{0.350pt}}
\put(960,442){\rule[-0.175pt]{0.390pt}{0.350pt}}
\put(962,443){\rule[-0.175pt]{0.390pt}{0.350pt}}
\put(964,444){\rule[-0.175pt]{0.390pt}{0.350pt}}
\put(965,445){\rule[-0.175pt]{0.390pt}{0.350pt}}
\put(967,446){\rule[-0.175pt]{0.390pt}{0.350pt}}
\put(969,447){\rule[-0.175pt]{0.390pt}{0.350pt}}
\put(970,448){\rule[-0.175pt]{0.390pt}{0.350pt}}
\put(972,449){\rule[-0.175pt]{0.390pt}{0.350pt}}
\put(973,450){\rule[-0.175pt]{0.390pt}{0.350pt}}
\put(975,451){\rule[-0.175pt]{0.390pt}{0.350pt}}
\put(977,452){\rule[-0.175pt]{0.390pt}{0.350pt}}
\put(978,453){\rule[-0.175pt]{0.390pt}{0.350pt}}
\put(980,454){\rule[-0.175pt]{0.390pt}{0.350pt}}
\put(981,455){\rule[-0.175pt]{0.390pt}{0.350pt}}
\put(983,456){\rule[-0.175pt]{0.390pt}{0.350pt}}
\put(985,457){\rule[-0.175pt]{0.390pt}{0.350pt}}
\put(986,458){\rule[-0.175pt]{0.390pt}{0.350pt}}
\put(988,459){\rule[-0.175pt]{0.390pt}{0.350pt}}
\put(990,460){\rule[-0.175pt]{0.390pt}{0.350pt}}
\put(991,461){\rule[-0.175pt]{0.390pt}{0.350pt}}
\put(993,462){\rule[-0.175pt]{0.390pt}{0.350pt}}
\put(994,463){\rule[-0.175pt]{0.390pt}{0.350pt}}
\put(996,464){\rule[-0.175pt]{0.390pt}{0.350pt}}
\put(998,465){\rule[-0.175pt]{0.390pt}{0.350pt}}
\put(999,466){\rule[-0.175pt]{0.390pt}{0.350pt}}
\put(1001,467){\rule[-0.175pt]{0.390pt}{0.350pt}}
\put(1003,468){\rule[-0.175pt]{0.390pt}{0.350pt}}
\put(1004,469){\rule[-0.175pt]{0.390pt}{0.350pt}}
\put(1006,470){\rule[-0.175pt]{0.390pt}{0.350pt}}
\put(1007,471){\rule[-0.175pt]{0.390pt}{0.350pt}}
\put(1009,472){\rule[-0.175pt]{0.390pt}{0.350pt}}
\put(1011,473){\rule[-0.175pt]{0.390pt}{0.350pt}}
\put(1012,474){\rule[-0.175pt]{0.390pt}{0.350pt}}
\put(1014,475){\rule[-0.175pt]{0.389pt}{0.350pt}}
\sbox{\plotpoint}{\rule[-0.350pt]{0.700pt}{0.700pt}}%
\put(1041,508){\usebox{\plotpoint}}
\put(1042,509){\usebox{\plotpoint}}
\put(1043,510){\usebox{\plotpoint}}
\put(1044,512){\usebox{\plotpoint}}
\put(1045,513){\usebox{\plotpoint}}
\put(1046,515){\usebox{\plotpoint}}
\put(1047,516){\usebox{\plotpoint}}
\put(1048,518){\usebox{\plotpoint}}
\put(1049,519){\usebox{\plotpoint}}
\put(1050,521){\usebox{\plotpoint}}
\put(1051,522){\usebox{\plotpoint}}
\put(1052,523){\usebox{\plotpoint}}
\put(1053,525){\usebox{\plotpoint}}
\put(1054,526){\usebox{\plotpoint}}
\put(1055,528){\usebox{\plotpoint}}
\put(1056,529){\usebox{\plotpoint}}
\put(1057,531){\usebox{\plotpoint}}
\put(1058,532){\usebox{\plotpoint}}
\put(1059,534){\usebox{\plotpoint}}
\put(1060,535){\usebox{\plotpoint}}
\put(1061,537){\usebox{\plotpoint}}
\put(1062,538){\usebox{\plotpoint}}
\put(1063,540){\usebox{\plotpoint}}
\put(1064,542){\usebox{\plotpoint}}
\put(1065,543){\usebox{\plotpoint}}
\put(1066,545){\usebox{\plotpoint}}
\put(1067,547){\usebox{\plotpoint}}
\put(1068,549){\usebox{\plotpoint}}
\put(1069,550){\usebox{\plotpoint}}
\put(1070,552){\usebox{\plotpoint}}
\put(1071,554){\usebox{\plotpoint}}
\put(1072,555){\usebox{\plotpoint}}
\put(1073,557){\usebox{\plotpoint}}
\put(1074,559){\usebox{\plotpoint}}
\put(1075,561){\usebox{\plotpoint}}
\put(1076,563){\usebox{\plotpoint}}
\put(1077,565){\usebox{\plotpoint}}
\put(1078,567){\usebox{\plotpoint}}
\put(1079,569){\usebox{\plotpoint}}
\put(1080,572){\usebox{\plotpoint}}
\put(1081,574){\usebox{\plotpoint}}
\put(1082,576){\usebox{\plotpoint}}
\put(1083,578){\usebox{\plotpoint}}
\put(1084,581){\rule[-0.350pt]{0.700pt}{1.124pt}}
\put(1085,585){\rule[-0.350pt]{0.700pt}{1.124pt}}
\put(1086,590){\rule[-0.350pt]{0.700pt}{1.124pt}}
\put(1087,595){\rule[-0.350pt]{0.700pt}{0.964pt}}
\put(1086,599){\rule[-0.350pt]{0.700pt}{0.964pt}}
\put(1082,603){\usebox{\plotpoint}}
\put(1079,604){\usebox{\plotpoint}}
\put(1077,605){\usebox{\plotpoint}}
\put(1077,606){\usebox{\plotpoint}}
\put(1070,606){\rule[-0.350pt]{1.686pt}{0.700pt}}
\put(1063,605){\rule[-0.350pt]{1.686pt}{0.700pt}}
\put(1059,604){\rule[-0.350pt]{0.763pt}{0.700pt}}
\put(1056,603){\rule[-0.350pt]{0.763pt}{0.700pt}}
\put(1053,602){\rule[-0.350pt]{0.763pt}{0.700pt}}
\put(1050,601){\rule[-0.350pt]{0.763pt}{0.700pt}}
\put(1047,600){\rule[-0.350pt]{0.763pt}{0.700pt}}
\put(1044,599){\rule[-0.350pt]{0.763pt}{0.700pt}}
\put(1044,598){\usebox{\plotpoint}}
\put(1041,598){\usebox{\plotpoint}}
\put(1039,597){\usebox{\plotpoint}}
\put(1036,596){\usebox{\plotpoint}}
\put(1034,595){\usebox{\plotpoint}}
\put(1031,594){\usebox{\plotpoint}}
\put(1029,593){\usebox{\plotpoint}}
\put(1027,592){\usebox{\plotpoint}}
\put(1024,591){\usebox{\plotpoint}}
\put(1022,590){\usebox{\plotpoint}}
\put(1020,589){\usebox{\plotpoint}}
\put(1017,588){\usebox{\plotpoint}}
\put(1015,587){\usebox{\plotpoint}}
\put(1013,586){\usebox{\plotpoint}}
\put(1011,585){\usebox{\plotpoint}}
\put(1009,584){\usebox{\plotpoint}}
\put(1007,583){\usebox{\plotpoint}}
\put(1004,582){\usebox{\plotpoint}}
\put(1002,581){\usebox{\plotpoint}}
\put(1000,580){\usebox{\plotpoint}}
\put(998,579){\usebox{\plotpoint}}
\put(996,578){\usebox{\plotpoint}}
\put(994,577){\usebox{\plotpoint}}
\put(992,576){\usebox{\plotpoint}}
\put(989,575){\usebox{\plotpoint}}
\put(987,574){\usebox{\plotpoint}}
\put(984,573){\usebox{\plotpoint}}
\put(982,572){\usebox{\plotpoint}}
\put(980,571){\usebox{\plotpoint}}
\put(977,570){\usebox{\plotpoint}}
\put(975,569){\usebox{\plotpoint}}
\put(973,568){\usebox{\plotpoint}}
\put(970,567){\usebox{\plotpoint}}
\put(968,566){\usebox{\plotpoint}}
\put(966,565){\usebox{\plotpoint}}
\put(963,564){\usebox{\plotpoint}}
\put(961,563){\usebox{\plotpoint}}
\put(959,562){\usebox{\plotpoint}}
\put(959,561){\usebox{\plotpoint}}
\put(956,561){\usebox{\plotpoint}}
\put(954,560){\usebox{\plotpoint}}
\put(952,559){\usebox{\plotpoint}}
\put(950,558){\usebox{\plotpoint}}
\put(947,557){\usebox{\plotpoint}}
\put(945,556){\usebox{\plotpoint}}
\put(943,555){\usebox{\plotpoint}}
\put(941,554){\usebox{\plotpoint}}
\put(938,553){\usebox{\plotpoint}}
\put(936,552){\usebox{\plotpoint}}
\put(934,551){\usebox{\plotpoint}}
\put(932,550){\usebox{\plotpoint}}
\put(929,549){\usebox{\plotpoint}}
\put(927,548){\usebox{\plotpoint}}
\put(925,547){\usebox{\plotpoint}}
\put(923,546){\usebox{\plotpoint}}
\put(920,545){\usebox{\plotpoint}}
\put(918,544){\usebox{\plotpoint}}
\put(915,543){\usebox{\plotpoint}}
\put(913,542){\usebox{\plotpoint}}
\put(910,541){\usebox{\plotpoint}}
\put(908,540){\usebox{\plotpoint}}
\put(905,539){\usebox{\plotpoint}}
\put(903,538){\usebox{\plotpoint}}
\put(901,537){\usebox{\plotpoint}}
\put(898,536){\usebox{\plotpoint}}
\put(896,535){\usebox{\plotpoint}}
\put(893,534){\usebox{\plotpoint}}
\put(891,533){\usebox{\plotpoint}}
\put(888,532){\usebox{\plotpoint}}
\put(886,531){\usebox{\plotpoint}}
\put(884,530){\usebox{\plotpoint}}
\put(881,529){\usebox{\plotpoint}}
\put(878,528){\usebox{\plotpoint}}
\put(875,527){\usebox{\plotpoint}}
\put(872,526){\usebox{\plotpoint}}
\put(870,525){\usebox{\plotpoint}}
\put(867,524){\usebox{\plotpoint}}
\put(864,523){\usebox{\plotpoint}}
\put(861,522){\usebox{\plotpoint}}
\put(858,521){\usebox{\plotpoint}}
\put(856,520){\usebox{\plotpoint}}
\put(853,519){\usebox{\plotpoint}}
\put(850,518){\usebox{\plotpoint}}
\put(847,517){\usebox{\plotpoint}}
\put(844,516){\usebox{\plotpoint}}
\put(842,515){\usebox{\plotpoint}}
\put(842,514){\usebox{\plotpoint}}
\put(838,514){\rule[-0.350pt]{0.815pt}{0.700pt}}
\put(835,513){\rule[-0.350pt]{0.815pt}{0.700pt}}
\put(831,512){\rule[-0.350pt]{0.815pt}{0.700pt}}
\put(828,511){\rule[-0.350pt]{0.815pt}{0.700pt}}
\put(825,510){\rule[-0.350pt]{0.815pt}{0.700pt}}
\put(821,509){\rule[-0.350pt]{0.815pt}{0.700pt}}
\put(818,508){\rule[-0.350pt]{0.815pt}{0.700pt}}
\put(814,507){\rule[-0.350pt]{0.815pt}{0.700pt}}
\put(811,506){\rule[-0.350pt]{0.815pt}{0.700pt}}
\put(808,505){\rule[-0.350pt]{0.815pt}{0.700pt}}
\put(804,504){\rule[-0.350pt]{0.815pt}{0.700pt}}
\put(801,503){\rule[-0.350pt]{0.815pt}{0.700pt}}
\put(798,502){\rule[-0.350pt]{0.815pt}{0.700pt}}
\put(794,501){\rule[-0.350pt]{0.883pt}{0.700pt}}
\put(790,500){\rule[-0.350pt]{0.883pt}{0.700pt}}
\put(786,499){\rule[-0.350pt]{0.883pt}{0.700pt}}
\put(783,498){\rule[-0.350pt]{0.883pt}{0.700pt}}
\put(779,497){\rule[-0.350pt]{0.883pt}{0.700pt}}
\put(775,496){\rule[-0.350pt]{0.883pt}{0.700pt}}
\put(772,495){\rule[-0.350pt]{0.883pt}{0.700pt}}
\put(768,494){\rule[-0.350pt]{0.883pt}{0.700pt}}
\put(764,493){\rule[-0.350pt]{0.883pt}{0.700pt}}
\put(761,492){\rule[-0.350pt]{0.883pt}{0.700pt}}
\put(757,491){\rule[-0.350pt]{0.883pt}{0.700pt}}
\put(754,490){\rule[-0.350pt]{0.883pt}{0.700pt}}
\put(750,489){\rule[-0.350pt]{0.964pt}{0.700pt}}
\put(746,488){\rule[-0.350pt]{0.964pt}{0.700pt}}
\put(742,487){\rule[-0.350pt]{0.964pt}{0.700pt}}
\put(738,486){\rule[-0.350pt]{0.964pt}{0.700pt}}
\put(734,485){\rule[-0.350pt]{0.964pt}{0.700pt}}
\put(730,484){\rule[-0.350pt]{0.964pt}{0.700pt}}
\put(726,483){\rule[-0.350pt]{0.964pt}{0.700pt}}
\put(722,482){\rule[-0.350pt]{0.964pt}{0.700pt}}
\put(718,481){\rule[-0.350pt]{0.964pt}{0.700pt}}
\put(714,480){\rule[-0.350pt]{0.964pt}{0.700pt}}
\put(710,479){\rule[-0.350pt]{0.964pt}{0.700pt}}
\put(704,478){\rule[-0.350pt]{1.325pt}{0.700pt}}
\put(699,477){\rule[-0.350pt]{1.325pt}{0.700pt}}
\put(693,476){\rule[-0.350pt]{1.325pt}{0.700pt}}
\put(688,475){\rule[-0.350pt]{1.325pt}{0.700pt}}
\put(682,474){\rule[-0.350pt]{1.325pt}{0.700pt}}
\put(677,473){\rule[-0.350pt]{1.325pt}{0.700pt}}
\put(671,472){\rule[-0.350pt]{1.325pt}{0.700pt}}
\put(666,471){\rule[-0.350pt]{1.325pt}{0.700pt}}
\put(659,470){\rule[-0.350pt]{1.480pt}{0.700pt}}
\put(653,469){\rule[-0.350pt]{1.480pt}{0.700pt}}
\put(647,468){\rule[-0.350pt]{1.480pt}{0.700pt}}
\put(641,467){\rule[-0.350pt]{1.480pt}{0.700pt}}
\put(635,466){\rule[-0.350pt]{1.480pt}{0.700pt}}
\put(629,465){\rule[-0.350pt]{1.480pt}{0.700pt}}
\put(623,464){\rule[-0.350pt]{1.480pt}{0.700pt}}
\put(616,463){\rule[-0.350pt]{1.646pt}{0.700pt}}
\put(609,462){\rule[-0.350pt]{1.646pt}{0.700pt}}
\put(602,461){\rule[-0.350pt]{1.646pt}{0.700pt}}
\put(595,460){\rule[-0.350pt]{1.646pt}{0.700pt}}
\put(588,459){\rule[-0.350pt]{1.646pt}{0.700pt}}
\put(582,458){\rule[-0.350pt]{1.646pt}{0.700pt}}
\put(582,457){\usebox{\plotpoint}}
\put(574,457){\rule[-0.350pt]{1.831pt}{0.700pt}}
\put(566,456){\rule[-0.350pt]{1.831pt}{0.700pt}}
\put(559,455){\rule[-0.350pt]{1.831pt}{0.700pt}}
\put(551,454){\rule[-0.350pt]{1.831pt}{0.700pt}}
\put(544,453){\rule[-0.350pt]{1.831pt}{0.700pt}}
\put(544,452){\usebox{\plotpoint}}
\put(537,452){\rule[-0.350pt]{1.686pt}{0.700pt}}
\put(530,451){\rule[-0.350pt]{1.686pt}{0.700pt}}
\put(523,450){\rule[-0.350pt]{1.686pt}{0.700pt}}
\put(516,449){\rule[-0.350pt]{1.686pt}{0.700pt}}
\put(509,448){\rule[-0.350pt]{1.686pt}{0.700pt}}
\put(503,447){\rule[-0.350pt]{1.285pt}{0.700pt}}
\put(498,446){\rule[-0.350pt]{1.285pt}{0.700pt}}
\put(492,445){\rule[-0.350pt]{1.285pt}{0.700pt}}
\put(487,444){\rule[-0.350pt]{1.285pt}{0.700pt}}
\put(482,443){\rule[-0.350pt]{1.285pt}{0.700pt}}
\put(477,442){\rule[-0.350pt]{1.285pt}{0.700pt}}
\put(473,441){\rule[-0.350pt]{0.813pt}{0.700pt}}
\put(470,440){\rule[-0.350pt]{0.813pt}{0.700pt}}
\put(466,439){\rule[-0.350pt]{0.813pt}{0.700pt}}
\put(463,438){\rule[-0.350pt]{0.813pt}{0.700pt}}
\put(460,437){\rule[-0.350pt]{0.813pt}{0.700pt}}
\put(456,436){\rule[-0.350pt]{0.813pt}{0.700pt}}
\put(453,435){\rule[-0.350pt]{0.813pt}{0.700pt}}
\put(450,434){\rule[-0.350pt]{0.813pt}{0.700pt}}
\put(447,433){\usebox{\plotpoint}}
\put(445,432){\usebox{\plotpoint}}
\put(442,431){\usebox{\plotpoint}}
\put(440,430){\usebox{\plotpoint}}
\put(437,429){\usebox{\plotpoint}}
\put(435,428){\usebox{\plotpoint}}
\put(432,427){\usebox{\plotpoint}}
\put(430,426){\usebox{\plotpoint}}
\put(428,425){\usebox{\plotpoint}}
\put(426,424){\usebox{\plotpoint}}
\put(425,423){\usebox{\plotpoint}}
\put(423,422){\usebox{\plotpoint}}
\put(422,421){\usebox{\plotpoint}}
\put(420,420){\usebox{\plotpoint}}
\put(419,419){\usebox{\plotpoint}}
\put(418,418){\usebox{\plotpoint}}
\put(416,417){\usebox{\plotpoint}}
\put(415,416){\usebox{\plotpoint}}
\put(413,415){\usebox{\plotpoint}}
\put(412,414){\usebox{\plotpoint}}
\put(411,413){\usebox{\plotpoint}}
\put(411,412){\usebox{\plotpoint}}
\put(411,410){\usebox{\plotpoint}}
\put(410,409){\usebox{\plotpoint}}
\put(409,408){\usebox{\plotpoint}}
\put(408,407){\usebox{\plotpoint}}
\put(407,406){\usebox{\plotpoint}}
\put(406,405){\usebox{\plotpoint}}
\put(405,403){\usebox{\plotpoint}}
\put(404,402){\usebox{\plotpoint}}
\put(403,401){\usebox{\plotpoint}}
\put(402,400){\usebox{\plotpoint}}
\put(401,399){\usebox{\plotpoint}}
\put(400,398){\usebox{\plotpoint}}
\put(399,398){\usebox{\plotpoint}}
\put(399,395){\usebox{\plotpoint}}
\put(398,392){\usebox{\plotpoint}}
\put(397,390){\usebox{\plotpoint}}
\put(396,387){\usebox{\plotpoint}}
\put(395,384){\usebox{\plotpoint}}
\put(394,382){\usebox{\plotpoint}}
\put(393,382){\usebox{\plotpoint}}
\put(393,364){\rule[-0.350pt]{0.700pt}{4.336pt}}
\put(392,360){\rule[-0.350pt]{0.700pt}{0.803pt}}
\put(393,357){\rule[-0.350pt]{0.700pt}{0.803pt}}
\put(394,353){\rule[-0.350pt]{0.700pt}{0.803pt}}
\put(395,350){\rule[-0.350pt]{0.700pt}{0.803pt}}
\put(396,347){\rule[-0.350pt]{0.700pt}{0.803pt}}
\put(397,344){\rule[-0.350pt]{0.700pt}{0.803pt}}
\put(398,341){\usebox{\plotpoint}}
\put(399,339){\usebox{\plotpoint}}
\put(400,337){\usebox{\plotpoint}}
\put(401,335){\usebox{\plotpoint}}
\put(402,333){\usebox{\plotpoint}}
\put(403,331){\usebox{\plotpoint}}
\put(404,329){\usebox{\plotpoint}}
\put(405,327){\usebox{\plotpoint}}
\put(406,325){\usebox{\plotpoint}}
\put(407,323){\usebox{\plotpoint}}
\put(408,321){\usebox{\plotpoint}}
\put(409,320){\usebox{\plotpoint}}
\put(410,319){\usebox{\plotpoint}}
\put(411,318){\usebox{\plotpoint}}
\put(412,316){\usebox{\plotpoint}}
\put(413,315){\usebox{\plotpoint}}
\put(414,314){\usebox{\plotpoint}}
\put(415,313){\usebox{\plotpoint}}
\put(416,311){\usebox{\plotpoint}}
\put(417,310){\usebox{\plotpoint}}
\put(418,309){\usebox{\plotpoint}}
\put(419,308){\usebox{\plotpoint}}
\put(420,306){\usebox{\plotpoint}}
\put(421,305){\usebox{\plotpoint}}
\put(422,304){\usebox{\plotpoint}}
\put(423,303){\usebox{\plotpoint}}
\put(424,302){\usebox{\plotpoint}}
\put(425,302){\usebox{\plotpoint}}
\put(425,302){\usebox{\plotpoint}}
\put(426,301){\usebox{\plotpoint}}
\put(427,300){\usebox{\plotpoint}}
\put(428,299){\usebox{\plotpoint}}
\put(429,298){\usebox{\plotpoint}}
\put(430,297){\usebox{\plotpoint}}
\put(431,296){\usebox{\plotpoint}}
\put(432,295){\usebox{\plotpoint}}
\put(433,294){\usebox{\plotpoint}}
\put(434,293){\usebox{\plotpoint}}
\put(435,292){\usebox{\plotpoint}}
\put(436,291){\usebox{\plotpoint}}
\put(437,290){\usebox{\plotpoint}}
\put(438,289){\usebox{\plotpoint}}
\put(439,288){\usebox{\plotpoint}}
\put(440,287){\usebox{\plotpoint}}
\put(441,286){\usebox{\plotpoint}}
\put(442,285){\usebox{\plotpoint}}
\put(443,284){\usebox{\plotpoint}}
\put(444,283){\usebox{\plotpoint}}
\put(445,282){\usebox{\plotpoint}}
\put(446,282){\usebox{\plotpoint}}
\put(447,281){\usebox{\plotpoint}}
\put(449,280){\usebox{\plotpoint}}
\put(450,279){\usebox{\plotpoint}}
\put(452,278){\usebox{\plotpoint}}
\put(453,277){\usebox{\plotpoint}}
\put(455,276){\usebox{\plotpoint}}
\put(456,275){\usebox{\plotpoint}}
\put(458,274){\usebox{\plotpoint}}
\put(459,273){\usebox{\plotpoint}}
\put(461,272){\usebox{\plotpoint}}
\put(462,271){\usebox{\plotpoint}}
\put(464,270){\usebox{\plotpoint}}
\put(465,269){\usebox{\plotpoint}}
\put(467,268){\usebox{\plotpoint}}
\put(468,267){\usebox{\plotpoint}}
\put(470,266){\usebox{\plotpoint}}
\put(471,265){\usebox{\plotpoint}}
\put(473,264){\usebox{\plotpoint}}
\put(474,263){\usebox{\plotpoint}}
\put(476,262){\usebox{\plotpoint}}
\put(478,261){\usebox{\plotpoint}}
\put(480,260){\usebox{\plotpoint}}
\put(482,259){\usebox{\plotpoint}}
\put(484,258){\usebox{\plotpoint}}
\put(486,257){\usebox{\plotpoint}}
\put(488,256){\usebox{\plotpoint}}
\put(489,255){\usebox{\plotpoint}}
\put(491,254){\usebox{\plotpoint}}
\put(493,253){\usebox{\plotpoint}}
\put(495,252){\usebox{\plotpoint}}
\put(497,251){\usebox{\plotpoint}}
\put(499,250){\usebox{\plotpoint}}
\put(501,249){\usebox{\plotpoint}}
\put(503,248){\rule[-0.350pt]{0.767pt}{0.700pt}}
\put(506,247){\rule[-0.350pt]{0.767pt}{0.700pt}}
\put(509,246){\rule[-0.350pt]{0.767pt}{0.700pt}}
\put(512,245){\rule[-0.350pt]{0.767pt}{0.700pt}}
\put(515,244){\rule[-0.350pt]{0.767pt}{0.700pt}}
\put(518,243){\rule[-0.350pt]{0.767pt}{0.700pt}}
\put(522,242){\rule[-0.350pt]{0.767pt}{0.700pt}}
\put(525,241){\rule[-0.350pt]{0.767pt}{0.700pt}}
\put(528,240){\rule[-0.350pt]{0.767pt}{0.700pt}}
\put(531,239){\rule[-0.350pt]{0.767pt}{0.700pt}}
\put(534,238){\rule[-0.350pt]{0.766pt}{0.700pt}}
\put(538,237){\rule[-0.350pt]{1.308pt}{0.700pt}}
\put(543,236){\rule[-0.350pt]{1.308pt}{0.700pt}}
\put(548,235){\rule[-0.350pt]{1.308pt}{0.700pt}}
\put(554,234){\rule[-0.350pt]{1.308pt}{0.700pt}}
\put(559,233){\rule[-0.350pt]{1.308pt}{0.700pt}}
\put(565,232){\rule[-0.350pt]{1.308pt}{0.700pt}}
\put(570,231){\rule[-0.350pt]{1.308pt}{0.700pt}}
\put(576,230){\rule[-0.350pt]{9.636pt}{0.700pt}}
\put(616,229){\rule[-0.350pt]{2.072pt}{0.700pt}}
\put(624,230){\rule[-0.350pt]{2.072pt}{0.700pt}}
\put(633,231){\rule[-0.350pt]{2.072pt}{0.700pt}}
\put(641,232){\rule[-0.350pt]{2.072pt}{0.700pt}}
\put(650,233){\rule[-0.350pt]{2.072pt}{0.700pt}}
\put(658,234){\usebox{\plotpoint}}
\put(659,234){\rule[-0.350pt]{1.060pt}{0.700pt}}
\put(663,235){\rule[-0.350pt]{1.060pt}{0.700pt}}
\put(667,236){\rule[-0.350pt]{1.060pt}{0.700pt}}
\put(672,237){\rule[-0.350pt]{1.060pt}{0.700pt}}
\put(676,238){\rule[-0.350pt]{1.060pt}{0.700pt}}
\put(681,239){\rule[-0.350pt]{1.060pt}{0.700pt}}
\put(685,240){\rule[-0.350pt]{1.060pt}{0.700pt}}
\put(689,241){\rule[-0.350pt]{1.060pt}{0.700pt}}
\put(694,242){\rule[-0.350pt]{1.060pt}{0.700pt}}
\put(698,243){\rule[-0.350pt]{1.060pt}{0.700pt}}
\put(703,244){\usebox{\plotpoint}}
\put(705,245){\usebox{\plotpoint}}
\put(708,246){\usebox{\plotpoint}}
\put(711,247){\usebox{\plotpoint}}
\put(714,248){\usebox{\plotpoint}}
\put(716,249){\usebox{\plotpoint}}
\put(719,250){\usebox{\plotpoint}}
\put(722,251){\usebox{\plotpoint}}
\put(725,252){\usebox{\plotpoint}}
\put(727,253){\usebox{\plotpoint}}
\put(730,254){\usebox{\plotpoint}}
\put(733,255){\usebox{\plotpoint}}
\put(736,256){\usebox{\plotpoint}}
\put(738,257){\usebox{\plotpoint}}
\put(741,258){\usebox{\plotpoint}}
\put(744,259){\usebox{\plotpoint}}
\put(747,260){\usebox{\plotpoint}}
\put(749,261){\usebox{\plotpoint}}
\put(751,262){\usebox{\plotpoint}}
\put(753,263){\usebox{\plotpoint}}
\put(755,264){\usebox{\plotpoint}}
\put(757,265){\usebox{\plotpoint}}
\put(759,266){\usebox{\plotpoint}}
\put(761,267){\usebox{\plotpoint}}
\put(763,268){\usebox{\plotpoint}}
\put(765,269){\usebox{\plotpoint}}
\put(767,270){\usebox{\plotpoint}}
\put(769,271){\usebox{\plotpoint}}
\put(771,272){\usebox{\plotpoint}}
\put(773,273){\usebox{\plotpoint}}
\put(775,274){\usebox{\plotpoint}}
\put(777,275){\usebox{\plotpoint}}
\put(779,276){\usebox{\plotpoint}}
\put(781,277){\usebox{\plotpoint}}
\put(783,278){\usebox{\plotpoint}}
\put(785,279){\usebox{\plotpoint}}
\put(787,280){\usebox{\plotpoint}}
\put(789,281){\usebox{\plotpoint}}
\put(791,282){\usebox{\plotpoint}}
\put(792,283){\usebox{\plotpoint}}
\put(794,284){\usebox{\plotpoint}}
\put(795,285){\usebox{\plotpoint}}
\put(797,286){\usebox{\plotpoint}}
\put(799,287){\usebox{\plotpoint}}
\put(800,288){\usebox{\plotpoint}}
\put(802,289){\usebox{\plotpoint}}
\put(804,290){\usebox{\plotpoint}}
\put(805,291){\usebox{\plotpoint}}
\put(807,292){\usebox{\plotpoint}}
\put(808,293){\usebox{\plotpoint}}
\put(810,294){\usebox{\plotpoint}}
\put(812,295){\usebox{\plotpoint}}
\put(813,296){\usebox{\plotpoint}}
\put(815,297){\usebox{\plotpoint}}
\put(817,298){\usebox{\plotpoint}}
\put(818,299){\usebox{\plotpoint}}
\put(820,300){\usebox{\plotpoint}}
\put(821,301){\usebox{\plotpoint}}
\put(823,302){\usebox{\plotpoint}}
\put(825,303){\usebox{\plotpoint}}
\put(826,304){\usebox{\plotpoint}}
\put(828,305){\usebox{\plotpoint}}
\put(830,306){\usebox{\plotpoint}}
\put(831,307){\usebox{\plotpoint}}
\put(833,308){\usebox{\plotpoint}}
\put(835,309){\usebox{\plotpoint}}
\put(836,310){\usebox{\plotpoint}}
\put(837,311){\usebox{\plotpoint}}
\put(839,312){\usebox{\plotpoint}}
\put(840,313){\usebox{\plotpoint}}
\put(842,314){\usebox{\plotpoint}}
\put(843,315){\usebox{\plotpoint}}
\put(844,316){\usebox{\plotpoint}}
\put(846,317){\usebox{\plotpoint}}
\put(847,318){\usebox{\plotpoint}}
\put(849,319){\usebox{\plotpoint}}
\put(850,320){\usebox{\plotpoint}}
\put(851,321){\usebox{\plotpoint}}
\put(853,322){\usebox{\plotpoint}}
\put(854,323){\usebox{\plotpoint}}
\put(856,324){\usebox{\plotpoint}}
\put(857,325){\usebox{\plotpoint}}
\put(858,326){\usebox{\plotpoint}}
\put(860,327){\usebox{\plotpoint}}
\put(861,328){\usebox{\plotpoint}}
\put(863,329){\usebox{\plotpoint}}
\put(864,330){\usebox{\plotpoint}}
\put(865,331){\usebox{\plotpoint}}
\put(867,332){\usebox{\plotpoint}}
\put(868,333){\usebox{\plotpoint}}
\put(870,334){\usebox{\plotpoint}}
\put(871,335){\usebox{\plotpoint}}
\put(872,336){\usebox{\plotpoint}}
\put(874,337){\usebox{\plotpoint}}
\put(875,338){\usebox{\plotpoint}}
\put(877,339){\usebox{\plotpoint}}
\put(878,340){\usebox{\plotpoint}}
\put(879,341){\usebox{\plotpoint}}
\put(880,342){\usebox{\plotpoint}}
\put(881,343){\usebox{\plotpoint}}
\put(882,344){\usebox{\plotpoint}}
\put(884,345){\usebox{\plotpoint}}
\put(885,346){\usebox{\plotpoint}}
\put(886,347){\usebox{\plotpoint}}
\put(887,348){\usebox{\plotpoint}}
\put(888,349){\usebox{\plotpoint}}
\put(889,350){\usebox{\plotpoint}}
\put(891,351){\usebox{\plotpoint}}
\put(892,352){\usebox{\plotpoint}}
\put(893,353){\usebox{\plotpoint}}
\put(894,354){\usebox{\plotpoint}}
\put(895,355){\usebox{\plotpoint}}
\put(896,356){\usebox{\plotpoint}}
\put(898,357){\usebox{\plotpoint}}
\put(899,358){\usebox{\plotpoint}}
\put(900,359){\usebox{\plotpoint}}
\put(901,360){\usebox{\plotpoint}}
\put(902,361){\usebox{\plotpoint}}
\put(904,362){\usebox{\plotpoint}}
\put(905,363){\usebox{\plotpoint}}
\put(906,364){\usebox{\plotpoint}}
\put(907,365){\usebox{\plotpoint}}
\put(908,366){\usebox{\plotpoint}}
\put(909,367){\usebox{\plotpoint}}
\put(911,368){\usebox{\plotpoint}}
\put(912,369){\usebox{\plotpoint}}
\put(913,370){\usebox{\plotpoint}}
\put(914,371){\usebox{\plotpoint}}
\put(915,372){\usebox{\plotpoint}}
\put(916,373){\usebox{\plotpoint}}
\put(917,373){\usebox{\plotpoint}}
\put(918,374){\usebox{\plotpoint}}
\put(919,375){\usebox{\plotpoint}}
\put(920,376){\usebox{\plotpoint}}
\put(921,377){\usebox{\plotpoint}}
\put(922,378){\usebox{\plotpoint}}
\put(923,379){\usebox{\plotpoint}}
\put(924,380){\usebox{\plotpoint}}
\put(925,381){\usebox{\plotpoint}}
\put(926,382){\usebox{\plotpoint}}
\put(927,383){\usebox{\plotpoint}}
\put(928,384){\usebox{\plotpoint}}
\put(929,385){\usebox{\plotpoint}}
\put(930,386){\usebox{\plotpoint}}
\put(931,387){\usebox{\plotpoint}}
\put(932,388){\usebox{\plotpoint}}
\put(933,389){\usebox{\plotpoint}}
\put(934,390){\usebox{\plotpoint}}
\put(935,391){\usebox{\plotpoint}}
\put(936,392){\usebox{\plotpoint}}
\put(937,393){\usebox{\plotpoint}}
\put(938,394){\usebox{\plotpoint}}
\put(939,395){\usebox{\plotpoint}}
\put(940,396){\usebox{\plotpoint}}
\put(941,397){\usebox{\plotpoint}}
\put(942,398){\usebox{\plotpoint}}
\put(943,399){\usebox{\plotpoint}}
\put(944,400){\usebox{\plotpoint}}
\put(945,401){\usebox{\plotpoint}}
\put(946,402){\usebox{\plotpoint}}
\put(947,403){\usebox{\plotpoint}}
\put(948,404){\usebox{\plotpoint}}
\put(949,405){\usebox{\plotpoint}}
\put(950,406){\usebox{\plotpoint}}
\put(951,407){\usebox{\plotpoint}}
\put(952,408){\usebox{\plotpoint}}
\put(953,408){\usebox{\plotpoint}}
\put(954,409){\usebox{\plotpoint}}
\put(955,410){\usebox{\plotpoint}}
\put(956,411){\usebox{\plotpoint}}
\put(957,412){\usebox{\plotpoint}}
\put(958,413){\usebox{\plotpoint}}
\put(959,414){\usebox{\plotpoint}}
\put(960,415){\usebox{\plotpoint}}
\put(961,416){\usebox{\plotpoint}}
\put(962,417){\usebox{\plotpoint}}
\put(963,418){\usebox{\plotpoint}}
\put(964,419){\usebox{\plotpoint}}
\put(965,420){\usebox{\plotpoint}}
\put(966,421){\usebox{\plotpoint}}
\put(967,422){\usebox{\plotpoint}}
\put(968,423){\usebox{\plotpoint}}
\put(969,424){\usebox{\plotpoint}}
\put(970,425){\usebox{\plotpoint}}
\put(971,426){\usebox{\plotpoint}}
\put(972,427){\usebox{\plotpoint}}
\put(973,428){\usebox{\plotpoint}}
\put(974,429){\usebox{\plotpoint}}
\put(975,430){\usebox{\plotpoint}}
\put(976,431){\usebox{\plotpoint}}
\put(977,432){\usebox{\plotpoint}}
\put(978,433){\usebox{\plotpoint}}
\put(979,434){\usebox{\plotpoint}}
\put(980,435){\usebox{\plotpoint}}
\put(981,436){\usebox{\plotpoint}}
\put(982,437){\usebox{\plotpoint}}
\put(983,438){\usebox{\plotpoint}}
\put(984,439){\usebox{\plotpoint}}
\put(985,440){\usebox{\plotpoint}}
\put(986,441){\usebox{\plotpoint}}
\put(987,443){\usebox{\plotpoint}}
\put(988,444){\usebox{\plotpoint}}
\put(989,445){\usebox{\plotpoint}}
\put(990,446){\usebox{\plotpoint}}
\put(991,447){\usebox{\plotpoint}}
\put(992,448){\usebox{\plotpoint}}
\put(993,449){\usebox{\plotpoint}}
\put(994,450){\usebox{\plotpoint}}
\put(995,452){\usebox{\plotpoint}}
\put(996,453){\usebox{\plotpoint}}
\put(997,454){\usebox{\plotpoint}}
\put(998,455){\usebox{\plotpoint}}
\put(999,456){\usebox{\plotpoint}}
\put(1000,457){\usebox{\plotpoint}}
\put(1001,458){\usebox{\plotpoint}}
\put(1002,460){\usebox{\plotpoint}}
\put(1003,461){\usebox{\plotpoint}}
\put(1004,462){\usebox{\plotpoint}}
\put(1005,463){\usebox{\plotpoint}}
\put(1006,464){\usebox{\plotpoint}}
\put(1007,465){\usebox{\plotpoint}}
\put(1008,466){\usebox{\plotpoint}}
\put(1009,468){\usebox{\plotpoint}}
\put(1010,469){\usebox{\plotpoint}}
\put(1011,470){\usebox{\plotpoint}}
\put(1012,471){\usebox{\plotpoint}}
\put(1013,472){\usebox{\plotpoint}}
\put(1014,473){\usebox{\plotpoint}}
\put(1015,474){\usebox{\plotpoint}}
\put(1016,476){\usebox{\plotpoint}}
\put(1017,477){\usebox{\plotpoint}}
\put(1018,478){\usebox{\plotpoint}}
\put(1019,479){\usebox{\plotpoint}}
\put(1020,481){\usebox{\plotpoint}}
\put(1021,482){\usebox{\plotpoint}}
\put(1022,483){\usebox{\plotpoint}}
\put(1023,484){\usebox{\plotpoint}}
\put(1024,486){\usebox{\plotpoint}}
\put(1025,487){\usebox{\plotpoint}}
\put(1026,488){\usebox{\plotpoint}}
\put(1027,490){\usebox{\plotpoint}}
\put(1028,491){\usebox{\plotpoint}}
\put(1029,492){\usebox{\plotpoint}}
\put(1030,493){\usebox{\plotpoint}}
\put(1031,495){\usebox{\plotpoint}}
\put(1032,496){\usebox{\plotpoint}}
\put(1033,497){\usebox{\plotpoint}}
\put(1034,499){\usebox{\plotpoint}}
\put(1035,500){\usebox{\plotpoint}}
\put(1036,501){\usebox{\plotpoint}}
\put(1037,502){\usebox{\plotpoint}}
\put(1038,504){\usebox{\plotpoint}}
\put(1039,505){\usebox{\plotpoint}}
\put(1040,506){\usebox{\plotpoint}}
\put(1041,507){\usebox{\plotpoint}}
\sbox{\plotpoint}{\rule[-0.175pt]{0.350pt}{0.350pt}}%
\sbox{\plotpoint}{\rule[-0.250pt]{0.500pt}{0.500pt}}%
\put(1020,476){\raisebox{-1.2pt}{\makebox(0,0){\circle*{12}}}}
\put(623,233){\raisebox{-1.2pt}{\makebox(0,0){\circle*{12}}}}
\end{picture}

 \vspace{-1cm}
}
{
\caption{\protect\capsize
Skematisk bevis for at en af Floquetmultiplikatorerne
opfylder $\lambda=1$. Et punkt p{\aa} banekurven
$\varphi_t({\bf c}_p)$ perturberes i gr{\ae}nsecyklusens
retning med $\epsilon_0$, hvorfor det resulterende punkt
kan beskrives ved $\varphi_{t+dt}({\bf c}_p)$.}
\label{fig:lambda=1}
}

\section{Bifurkationer af periodiske l{\o}sninger}
Lad os stadig betragte den periodiske l{\o}sning
$\varphi_t({\bf c}_p)$ til differentiallig\-ningen
$\dot{\bf c} = {\bf f}({\bf c},\mu)$, idet vi nu lader
vores vektorfelt, der beskriver kinetiken af det kemiske
sy\-stem, v{\ae}re afh{\ae}ngig af en kontrol- eller
bifurkationsparameter $\mu$. I dette tilf{\ae}lde vil
gr{\ae}nsecyklusen $\gamma$ og de tilh{\o}rende
Floquetmultiplikatorer $\lambda_1,\ldots,\lambda_n$ og
eksponenter if{\o}lge det implicitte funktionsteorem
v{\ae}re ana\-lytiske funktioner af $\mu$, s{\aa}l{\ae}nge
monodromimatricen ikke er singul{\ae}r. En s{\aa}dan
singularitet kan forekomme ved, at en eller flere af
multiplikatorerne krydser enhedscirklen i det komplekse
plan. Denne situation kan kvalitativt forekomme p{\aa} tre
forskellige m{\aa}der (fig~\ref{fig:perbif}) svarende til
tre forskellige bifurkationstyper.

\vspace{4.0mm}
Vi skal ikke her komme n{\ae}rmere ind p{\aa} disses
pr{\ae}cise matematiske karakteri\-sering, men henviser i
stedet til \cite{Marek}. Kvalitativt kan de tre
bifurkationstyper beskrives som f{\o}lger

\boxfigure{t}{\textwidth}
{
 \setlength{\unitlength}{0.8pt}
\vspace{7.5mm}
\begin{center}
\begin{picture}(330,100)(-14.35,-50)
 
%circle 1
\put (-70,0){\vector(1,0){100}}
\put (-20,-50){\vector(0,1){100}}
\put (-20,0){\circle{60}}
 
\put (-30,10){\circle*{3.75}}
\put (-30,-10){\circle*{3.75}}
\put (-10,5){\circle*{3.75}}
\put (-10,-5){\circle*{3.75}}
\put (-25,0){\circle*{3.75}}
\put (25,0){\circle*{3.75}}
 
\thicklines
\put (-30,10){\vector(-1,1){15}}
\put (-30,-10){\vector(-1,-1){15}}
\thinlines
%circle2
\put (90,0){\vector(1,0){100}}
\put (140,-50){\vector(0,1){100}}
\put (140,0){\circle{60}}
 
\put (125,0){\circle*{3.75}}
\put (130,10){\circle*{3.75}}
\put (130,-10){\circle*{3.75}}
\put (150,5){\circle*{3.75}}
\put (150,-5){\circle*{3.75}}
\put (165,0){\circle*{3.75}}
 
\thicklines
\put (125,0){\vector(-1,0){20}}
\thinlines
%circle2
%circle3
\put (250,0){\vector(1,0){100}}
\put (300,-50){\vector(0,1){100}}
\put (300,0){\circle{60}}
 
\put (290,10){\circle*{3.75}}
\put (290,-10){\circle*{3.75}}
\put (310,5){\circle*{3.75}}
\put (310,-5){\circle*{3.75}}
\put (317,0){\circle*{3.75}}
\put (295,0){\circle*{3.75}}
 
\thicklines
\put (317,0){\vector(1,0){20}}
\thinlines
%circle3
 
\small
\put (32,-2){\footnotesize $Re\,  \lambda_j$}
\put (-32,52){\footnotesize $Im\,  \lambda_j$}
\put (-65,40){a)}
 
\put (192,-2){\footnotesize $Re\,  \lambda_j$}
\put (128,52){\footnotesize $Im\,  \lambda_j$}
\put (95,40){b)}
 
\put (352,-2){\footnotesize $Re\,  \lambda_j$}
\put (288,52){\footnotesize $Im\,  \lambda_j$}
\put (255,40){c)}
\end{picture}
\end{center}
\vspace{5mm}



}
{
\caption{\protect\capsize
	 De mulige sk{\ae}ringer mellem Floquetmultiplikatoren 
         $\lambda_j$ og enhedscirklen i det komplekse plan:
	 a) \ \ $\lambda_j = \alpha_j \pm i\beta_j$, \
	 b) \ \ $\lambda_j = -1$ and
	 c) \ \ $\lambda_j = +1$.
}

\label{fig:perbif}
}

\begin{itemize}
 \item {\em Torusbifurkationen}\\
 To komplekskonjugerede multiplikatorer krydser 
 enhedscirklen i det komplekse plan. Den tidligere 
 stabile gr{\ae}nsecyklus bliver ustabil og bifurkerer 
 ud p{\aa} en torus.
 
 \item {\em Periodefordoblingsbifurkationen}\\
 En multiplikator krydser $-1$ langs den reelle akse. 
 Gr{\ae}nsecyklusen bliver ustabil og bifurkerer ud p{\aa} et
 M\"{o}biusb{\aa}nd svarende til en periodefordobling.
 
\item {\em Generaliseret pitchfork (gaffelgrens)
 bifurkation}\\
 En multiplikator krydser $+1$ langs den reelle akse.
 Gr{\ae}nsecyklusen er nu ustabil og to nye uafh{\ae}ngige og
 stabile gr{\ae}nsecykluser opst{\aa}r. 
\end{itemize}

Den geometriske tolkning af disse tre bifurkationer er
illustreret i figur~\ref{fig:perbifgeo}. Af de n{\ae}vnte
tre bifurkationstyper viser det sig, at torusbifurkationen
og periodefor\-doblingsbifurkationen begge indtager en
v{\ae}sentlig rolle i de m{\aa}der, hvorp{\aa} kaos kan
introduceres i dynamiske sy\-stemer.

\vspace{4.0mm}
Ydermere g{\ae}lder det, at den m{\aa}de, hvorp{\aa} kaos
genereres i det dynamiske sy\-stem, har en universal
karakter, der genfindes i s{\aa}vel s{\ae}dvanlige
differentiallig\-ninger ($\dot{\bf x} = {\bf f}({\bf x}),
\mbox{\ }{\bf x}\in\R^n, \mbox{\ }{\bf f}:
\R^n\mapsto\R^n$) som diskrete afbildninger (${\bf x}_{i+1}
= {\bf f}({\bf x}_i),\mbox{\ } {\bf x}_n\in\R^n, \mbox{\
}{\bf f}: \R^n\mapsto\R^n$). Vi skal se, at ligesom det
irrationelle tal $\pi$ er indeholdt i stort set enhver
manipulation indenfor den klassiske geometri, s{\aa} er
dynamiske sy\-stemers natur p{\aa} samme m{\aa}de
karakteriseret af en r{\ae}kke ana\-loge konstanter.

\vspace*{\fill}

\boxfigure{htbp}{\textwidth}{
 \vspace{16cm}
}
{
\caption{\protect\capsize
Illustration af a) torusbifurkationen, b)
periodefordoblingsbifurkationen og c) generaliseret
pitchfork bifurkation.}
\label{fig:perbifgeo}
}

\vspace*{\fill}

\newpage
\subsection{Periodefordoblings-bifurkationen}
For at give et klart og forholdsvist simpelt eksempel
p{\aa} et dynamisk sy\-stem, der indeholder
periodefordoblinger, vil vi betragte en reel diskret
afbildning $f_a(x)$ af enhedsintervallet p{\aa}
enhedsintervallet

\begin{equation}
 f_a(x) = ax(1-x), \mbox{\ } a \in [0;4]
\end{equation}

kaldet {\em den logistiske afbildning}. Denne ``simple''
model, der blandt andet er blevet anvendt til at beskrive
variationer i biologiske populationer, blev gennem en
r{\ae}kke pionerarbejder af Mitchell Feigenbaum i 1978
\cite{Feig1,Feig2} vist at besidde en r{\ae}kke egenskaber,
der senere er blevet bevist at v{\ae}re
universelle/generelle for de n{\ae}vnte typer af dynamiske
sy\-stemer. Lad os derfor foretage en ana\-lyse af den
logistiske afbildning med henblik p{\aa} at beskrive disse
universelle egenskaber.

\vspace{4.0mm}
Umiddelbart ser vi, at den logistiske afbildning har
f{\o}lgende ikke-trivielle fikspunkt\footnote{Dvs.\ $f(x_f)=x_f$
og $x_f \neq 0$} $x_f$.

\begin{equation}
 x_f = \frac{a-1}{a}
\end{equation}

Vi {\o}nsker nu at diskutere stabiliteten af $x_f$, dvs.,
for hvilke v{\ae}rdier af para\-meteren $a$ g{\ae}lder

\begin{equation}
 \lim_{n \rightarrow \infty} f^n_a(x_f + \delta x) = x_f
\end{equation}

Taylorudvikler vi nu $f^n_a(x_f + \delta x)$ omkring $x_f$
til f{\o}rste orden f{\aa}s ved anvendelse af
k{\ae}dereglen

\begin{eqnarray}
 f^n_a(x_f + \delta x) & \simeq & f^n_a(x_f) + 
 \left.\frac{df_a^n}{dx}\right|_{x_f} \negsp\delta x                \nonumber\\
 & = & x_f + \left.\frac{df_a^n}{dx}\right|_{x_f}  \negsp\delta x\nonumber\\
 & = & x_f + \left(\left.\frac{df_a}{dx}\right|_{x_f}\right)^{\!\! n}\delta x
\end{eqnarray}

Af disse udregninger konkluderes, at fikspunktet $x_f$ er
stabilt, hvis kravet $|f_a'(x_f)| < 1$ er opfyldt. Da
$f_a'(x_f)=2-a$ ses, at $x_f$ er stabilt for $a \in [1;3]$.
Sp{\o}rgsm{\aa}let er nu, hvad der sker med dynamikken for
$f_a(x)$ for $a>3$? Dette lader sig bedst besvare ved
hj{\ae}lp af en grafisk argumentation.

\vspace{4.0mm}
Lad os f{\o}rst indf{\o}re notationen $a_0$ for $a=3$. Idet
der oplagt g{\ae}lder $f(x_f)=x_f \Rightarrow f^n(x_f)=x_f$
for $n=0,1,2,\ldots$, ser vi, at fikspunktet
$\frac{a-1}{a}$ ogs{\aa} vil v{\ae}re stabilt for den
dobbeltiterede afbildning $f^2_a(x)$ for $a<a_0$
(figur~\ref{fig:LogGraf}a). For $a=a_0$ vil kurven $y=x$
v{\ae}re tangent til $f^2_a(x)$ i $x_f$, da
$(f^2_{a=a_0}(x_f))' = f_{a=a_0}'(x_f) f_{a=a_0}'(x_f) =
(-1)(-1) = 1$ (figur~\ref{fig:LogGraf}b). For $a>a_0$
``l{\o}ftes'' den {\o}verste del af $f_a^2(x)$ over $y=x$,
hvorimod den nederste del s{\ae}nkes ned under $y=x$
(figur~\ref{fig:LogGraf}c). Herved er genereret to nye
stabile fikspunkter $x_1$ og $x_2$ for $f^2_a(x)$, samtidig
med fikspunktet $x_f=\frac{a-1}{a}$ nu er ustabilt. $x_1$
og $x_2$ vil udg{\o}re en m{\ae}ngde, hvorp{\aa} $f_a(x)$
er periodisk, da der g{\ae}lder

\begin{equation}
 \begin{array}{lcl}
  f_a(x_1) & = & x_2 \\
  f_a(x_2) & = & x_1 
 \end{array}
 \label{eq:Pitch}
\end{equation}

%%%%%%%%%%%%%%%%%%%%%%%%%%%%%%%%%%%%%%%%%%
%                                        %
% Figure for logistic map, three plots,  %
% fig11a, fig11b, fig11c                 %
%                                        %
% gnuplot file: fig11.plt                %
% LaTeX   file: fig11a.tex               %
%               fig11b.tex               %
%               fig11c.tex               %
%%%%%%%%%%%%%%%%%%%%%%%%%%%%%%%%%%%%%%%%%%
\boxfigure{t}{\textwidth}
{
\vspace{0.15cm}
 \hspace*{-1.4cm}
 \begin{minipage}{4cm}
  % GNUPLOT: LaTeX picture
\setlength{\unitlength}{0.240900pt}
\ifx\plotpoint\undefined\newsavebox{\plotpoint}\fi
\sbox{\plotpoint}{\rule[-0.175pt]{0.350pt}{0.350pt}}%
\begin{picture}(469,612)(0,0)
\tenrm
\sbox{\plotpoint}{\rule[-0.175pt]{0.350pt}{0.350pt}}%
\put(264,158){\rule[-0.175pt]{87.206pt}{0.350pt}}
\put(264,158){\rule[-0.175pt]{0.350pt}{82.147pt}}
\put(264,158){\rule[-0.175pt]{87.206pt}{0.350pt}}
\put(626,158){\rule[-0.175pt]{0.350pt}{82.147pt}}
\put(264,499){\rule[-0.175pt]{87.206pt}{0.350pt}}
\put(300,465){\makebox(0,0)[l]{{\footnotesize $a<3$}}}
\put(497,127){\makebox(0,0){{\footnotesize $x_f$}}}
\put(264,158){\rule[-0.175pt]{0.350pt}{82.147pt}}
\put(264,158){\usebox{\plotpoint}}
\put(264,158){\rule[-0.175pt]{0.350pt}{1.660pt}}
\put(265,164){\rule[-0.175pt]{0.350pt}{1.660pt}}
\put(266,171){\rule[-0.175pt]{0.350pt}{1.660pt}}
\put(267,178){\rule[-0.175pt]{0.350pt}{1.660pt}}
\put(268,185){\rule[-0.175pt]{0.350pt}{1.660pt}}
\put(269,192){\rule[-0.175pt]{0.350pt}{1.660pt}}
\put(270,199){\rule[-0.175pt]{0.350pt}{1.660pt}}
\put(271,206){\rule[-0.175pt]{0.350pt}{1.660pt}}
\put(272,213){\rule[-0.175pt]{0.350pt}{1.660pt}}
\put(273,219){\rule[-0.175pt]{0.350pt}{1.205pt}}
\put(274,225){\rule[-0.175pt]{0.350pt}{1.204pt}}
\put(275,230){\rule[-0.175pt]{0.350pt}{1.204pt}}
\put(276,235){\rule[-0.175pt]{0.350pt}{1.204pt}}
\put(277,240){\rule[-0.175pt]{0.350pt}{1.204pt}}
\put(278,245){\rule[-0.175pt]{0.350pt}{1.204pt}}
\put(279,250){\rule[-0.175pt]{0.350pt}{1.204pt}}
\put(280,255){\rule[-0.175pt]{0.350pt}{1.204pt}}
\put(281,260){\rule[-0.175pt]{0.350pt}{1.204pt}}
\put(282,265){\rule[-0.175pt]{0.350pt}{1.204pt}}
\put(283,270){\rule[-0.175pt]{0.350pt}{1.071pt}}
\put(284,274){\rule[-0.175pt]{0.350pt}{1.071pt}}
\put(285,278){\rule[-0.175pt]{0.350pt}{1.071pt}}
\put(286,283){\rule[-0.175pt]{0.350pt}{1.071pt}}
\put(287,287){\rule[-0.175pt]{0.350pt}{1.071pt}}
\put(288,292){\rule[-0.175pt]{0.350pt}{1.071pt}}
\put(289,296){\rule[-0.175pt]{0.350pt}{1.071pt}}
\put(290,301){\rule[-0.175pt]{0.350pt}{1.071pt}}
\put(291,305){\rule[-0.175pt]{0.350pt}{1.071pt}}
\put(292,310){\rule[-0.175pt]{0.350pt}{0.830pt}}
\put(293,313){\rule[-0.175pt]{0.350pt}{0.830pt}}
\put(294,316){\rule[-0.175pt]{0.350pt}{0.830pt}}
\put(295,320){\rule[-0.175pt]{0.350pt}{0.830pt}}
\put(296,323){\rule[-0.175pt]{0.350pt}{0.830pt}}
\put(297,327){\rule[-0.175pt]{0.350pt}{0.830pt}}
\put(298,330){\rule[-0.175pt]{0.350pt}{0.830pt}}
\put(299,334){\rule[-0.175pt]{0.350pt}{0.830pt}}
\put(300,337){\rule[-0.175pt]{0.350pt}{0.830pt}}
\put(301,341){\rule[-0.175pt]{0.350pt}{0.589pt}}
\put(302,343){\rule[-0.175pt]{0.350pt}{0.589pt}}
\put(303,345){\rule[-0.175pt]{0.350pt}{0.589pt}}
\put(304,348){\rule[-0.175pt]{0.350pt}{0.589pt}}
\put(305,350){\rule[-0.175pt]{0.350pt}{0.589pt}}
\put(306,353){\rule[-0.175pt]{0.350pt}{0.589pt}}
\put(307,355){\rule[-0.175pt]{0.350pt}{0.589pt}}
\put(308,358){\rule[-0.175pt]{0.350pt}{0.589pt}}
\put(309,360){\rule[-0.175pt]{0.350pt}{0.589pt}}
\put(310,363){\rule[-0.175pt]{0.350pt}{0.385pt}}
\put(311,364){\rule[-0.175pt]{0.350pt}{0.385pt}}
\put(312,366){\rule[-0.175pt]{0.350pt}{0.385pt}}
\put(313,367){\rule[-0.175pt]{0.350pt}{0.385pt}}
\put(314,369){\rule[-0.175pt]{0.350pt}{0.385pt}}
\put(315,371){\rule[-0.175pt]{0.350pt}{0.385pt}}
\put(316,372){\rule[-0.175pt]{0.350pt}{0.385pt}}
\put(317,374){\rule[-0.175pt]{0.350pt}{0.385pt}}
\put(318,375){\rule[-0.175pt]{0.350pt}{0.385pt}}
\put(319,377){\rule[-0.175pt]{0.350pt}{0.385pt}}
\put(320,379){\usebox{\plotpoint}}
\put(321,380){\usebox{\plotpoint}}
\put(322,381){\usebox{\plotpoint}}
\put(323,382){\usebox{\plotpoint}}
\put(324,383){\usebox{\plotpoint}}
\put(325,384){\usebox{\plotpoint}}
\put(326,385){\usebox{\plotpoint}}
\put(327,386){\usebox{\plotpoint}}
\put(328,387){\usebox{\plotpoint}}
\put(329,389){\rule[-0.175pt]{0.361pt}{0.350pt}}
\put(330,390){\rule[-0.175pt]{0.361pt}{0.350pt}}
\put(332,391){\rule[-0.175pt]{0.361pt}{0.350pt}}
\put(333,392){\rule[-0.175pt]{0.361pt}{0.350pt}}
\put(335,393){\rule[-0.175pt]{0.361pt}{0.350pt}}
\put(336,394){\rule[-0.175pt]{0.361pt}{0.350pt}}
\put(338,395){\rule[-0.175pt]{1.204pt}{0.350pt}}
\put(343,396){\rule[-0.175pt]{1.204pt}{0.350pt}}
\put(348,397){\rule[-0.175pt]{2.168pt}{0.350pt}}
\put(357,396){\rule[-0.175pt]{0.542pt}{0.350pt}}
\put(359,395){\rule[-0.175pt]{0.542pt}{0.350pt}}
\put(361,394){\rule[-0.175pt]{0.542pt}{0.350pt}}
\put(363,393){\rule[-0.175pt]{0.542pt}{0.350pt}}
\put(366,392){\rule[-0.175pt]{0.542pt}{0.350pt}}
\put(368,391){\rule[-0.175pt]{0.542pt}{0.350pt}}
\put(370,390){\rule[-0.175pt]{0.542pt}{0.350pt}}
\put(372,389){\rule[-0.175pt]{0.542pt}{0.350pt}}
\put(375,388){\rule[-0.175pt]{0.401pt}{0.350pt}}
\put(376,387){\rule[-0.175pt]{0.401pt}{0.350pt}}
\put(378,386){\rule[-0.175pt]{0.401pt}{0.350pt}}
\put(379,385){\rule[-0.175pt]{0.401pt}{0.350pt}}
\put(381,384){\rule[-0.175pt]{0.401pt}{0.350pt}}
\put(383,383){\rule[-0.175pt]{0.401pt}{0.350pt}}
\put(384,382){\rule[-0.175pt]{0.434pt}{0.350pt}}
\put(386,381){\rule[-0.175pt]{0.434pt}{0.350pt}}
\put(388,380){\rule[-0.175pt]{0.434pt}{0.350pt}}
\put(390,379){\rule[-0.175pt]{0.434pt}{0.350pt}}
\put(392,378){\rule[-0.175pt]{0.434pt}{0.350pt}}
\put(393,377){\rule[-0.175pt]{0.361pt}{0.350pt}}
\put(395,376){\rule[-0.175pt]{0.361pt}{0.350pt}}
\put(397,375){\rule[-0.175pt]{0.361pt}{0.350pt}}
\put(398,374){\rule[-0.175pt]{0.361pt}{0.350pt}}
\put(400,373){\rule[-0.175pt]{0.361pt}{0.350pt}}
\put(401,372){\rule[-0.175pt]{0.361pt}{0.350pt}}
\put(403,371){\rule[-0.175pt]{0.602pt}{0.350pt}}
\put(405,370){\rule[-0.175pt]{0.602pt}{0.350pt}}
\put(408,369){\rule[-0.175pt]{0.602pt}{0.350pt}}
\put(410,368){\rule[-0.175pt]{0.602pt}{0.350pt}}
\put(413,367){\rule[-0.175pt]{0.542pt}{0.350pt}}
\put(415,366){\rule[-0.175pt]{0.542pt}{0.350pt}}
\put(417,365){\rule[-0.175pt]{0.542pt}{0.350pt}}
\put(419,364){\rule[-0.175pt]{0.542pt}{0.350pt}}
\put(422,363){\rule[-0.175pt]{0.723pt}{0.350pt}}
\put(425,362){\rule[-0.175pt]{0.723pt}{0.350pt}}
\put(428,361){\rule[-0.175pt]{0.723pt}{0.350pt}}
\put(431,360){\rule[-0.175pt]{2.168pt}{0.350pt}}
\put(440,359){\rule[-0.175pt]{4.577pt}{0.350pt}}
\put(459,360){\rule[-0.175pt]{0.723pt}{0.350pt}}
\put(462,361){\rule[-0.175pt]{0.723pt}{0.350pt}}
\put(465,362){\rule[-0.175pt]{0.723pt}{0.350pt}}
\put(468,363){\rule[-0.175pt]{0.542pt}{0.350pt}}
\put(470,364){\rule[-0.175pt]{0.542pt}{0.350pt}}
\put(472,365){\rule[-0.175pt]{0.542pt}{0.350pt}}
\put(474,366){\rule[-0.175pt]{0.542pt}{0.350pt}}
\put(477,367){\rule[-0.175pt]{0.602pt}{0.350pt}}
\put(479,368){\rule[-0.175pt]{0.602pt}{0.350pt}}
\put(482,369){\rule[-0.175pt]{0.602pt}{0.350pt}}
\put(484,370){\rule[-0.175pt]{0.602pt}{0.350pt}}
\put(487,371){\rule[-0.175pt]{0.361pt}{0.350pt}}
\put(488,372){\rule[-0.175pt]{0.361pt}{0.350pt}}
\put(490,373){\rule[-0.175pt]{0.361pt}{0.350pt}}
\put(491,374){\rule[-0.175pt]{0.361pt}{0.350pt}}
\put(493,375){\rule[-0.175pt]{0.361pt}{0.350pt}}
\put(494,376){\rule[-0.175pt]{0.361pt}{0.350pt}}
\put(496,377){\rule[-0.175pt]{0.434pt}{0.350pt}}
\put(497,378){\rule[-0.175pt]{0.434pt}{0.350pt}}
\put(499,379){\rule[-0.175pt]{0.434pt}{0.350pt}}
\put(501,380){\rule[-0.175pt]{0.434pt}{0.350pt}}
\put(503,381){\rule[-0.175pt]{0.434pt}{0.350pt}}
\put(504,382){\rule[-0.175pt]{0.402pt}{0.350pt}}
\put(506,383){\rule[-0.175pt]{0.401pt}{0.350pt}}
\put(508,384){\rule[-0.175pt]{0.401pt}{0.350pt}}
\put(509,385){\rule[-0.175pt]{0.401pt}{0.350pt}}
\put(511,386){\rule[-0.175pt]{0.402pt}{0.350pt}}
\put(513,387){\rule[-0.175pt]{0.402pt}{0.350pt}}
\put(515,388){\rule[-0.175pt]{0.542pt}{0.350pt}}
\put(517,389){\rule[-0.175pt]{0.542pt}{0.350pt}}
\put(519,390){\rule[-0.175pt]{0.542pt}{0.350pt}}
\put(521,391){\rule[-0.175pt]{0.542pt}{0.350pt}}
\put(524,392){\rule[-0.175pt]{0.542pt}{0.350pt}}
\put(526,393){\rule[-0.175pt]{0.542pt}{0.350pt}}
\put(528,394){\rule[-0.175pt]{0.542pt}{0.350pt}}
\put(530,395){\rule[-0.175pt]{0.542pt}{0.350pt}}
\put(533,396){\rule[-0.175pt]{2.168pt}{0.350pt}}
\put(542,397){\rule[-0.175pt]{1.204pt}{0.350pt}}
\put(547,396){\rule[-0.175pt]{1.204pt}{0.350pt}}
\put(552,395){\rule[-0.175pt]{0.361pt}{0.350pt}}
\put(553,394){\rule[-0.175pt]{0.361pt}{0.350pt}}
\put(555,393){\rule[-0.175pt]{0.361pt}{0.350pt}}
\put(556,392){\rule[-0.175pt]{0.361pt}{0.350pt}}
\put(558,391){\rule[-0.175pt]{0.361pt}{0.350pt}}
\put(559,390){\rule[-0.175pt]{0.361pt}{0.350pt}}
\put(561,387){\usebox{\plotpoint}}
\put(562,386){\usebox{\plotpoint}}
\put(563,385){\usebox{\plotpoint}}
\put(564,384){\usebox{\plotpoint}}
\put(565,383){\usebox{\plotpoint}}
\put(566,382){\usebox{\plotpoint}}
\put(567,381){\usebox{\plotpoint}}
\put(568,380){\usebox{\plotpoint}}
\put(569,379){\usebox{\plotpoint}}
\put(570,377){\rule[-0.175pt]{0.350pt}{0.385pt}}
\put(571,375){\rule[-0.175pt]{0.350pt}{0.385pt}}
\put(572,374){\rule[-0.175pt]{0.350pt}{0.385pt}}
\put(573,372){\rule[-0.175pt]{0.350pt}{0.385pt}}
\put(574,370){\rule[-0.175pt]{0.350pt}{0.385pt}}
\put(575,369){\rule[-0.175pt]{0.350pt}{0.385pt}}
\put(576,367){\rule[-0.175pt]{0.350pt}{0.385pt}}
\put(577,366){\rule[-0.175pt]{0.350pt}{0.385pt}}
\put(578,364){\rule[-0.175pt]{0.350pt}{0.385pt}}
\put(579,363){\rule[-0.175pt]{0.350pt}{0.385pt}}
\put(580,360){\rule[-0.175pt]{0.350pt}{0.589pt}}
\put(581,358){\rule[-0.175pt]{0.350pt}{0.589pt}}
\put(582,355){\rule[-0.175pt]{0.350pt}{0.589pt}}
\put(583,353){\rule[-0.175pt]{0.350pt}{0.589pt}}
\put(584,350){\rule[-0.175pt]{0.350pt}{0.589pt}}
\put(585,348){\rule[-0.175pt]{0.350pt}{0.589pt}}
\put(586,345){\rule[-0.175pt]{0.350pt}{0.589pt}}
\put(587,343){\rule[-0.175pt]{0.350pt}{0.589pt}}
\put(588,341){\rule[-0.175pt]{0.350pt}{0.589pt}}
\put(589,337){\rule[-0.175pt]{0.350pt}{0.830pt}}
\put(590,334){\rule[-0.175pt]{0.350pt}{0.830pt}}
\put(591,330){\rule[-0.175pt]{0.350pt}{0.830pt}}
\put(592,327){\rule[-0.175pt]{0.350pt}{0.830pt}}
\put(593,323){\rule[-0.175pt]{0.350pt}{0.830pt}}
\put(594,320){\rule[-0.175pt]{0.350pt}{0.830pt}}
\put(595,316){\rule[-0.175pt]{0.350pt}{0.830pt}}
\put(596,313){\rule[-0.175pt]{0.350pt}{0.830pt}}
\put(597,310){\rule[-0.175pt]{0.350pt}{0.830pt}}
\put(598,305){\rule[-0.175pt]{0.350pt}{1.071pt}}
\put(599,301){\rule[-0.175pt]{0.350pt}{1.071pt}}
\put(600,296){\rule[-0.175pt]{0.350pt}{1.071pt}}
\put(601,292){\rule[-0.175pt]{0.350pt}{1.071pt}}
\put(602,287){\rule[-0.175pt]{0.350pt}{1.071pt}}
\put(603,283){\rule[-0.175pt]{0.350pt}{1.071pt}}
\put(604,278){\rule[-0.175pt]{0.350pt}{1.071pt}}
\put(605,274){\rule[-0.175pt]{0.350pt}{1.071pt}}
\put(606,270){\rule[-0.175pt]{0.350pt}{1.071pt}}
\put(607,265){\rule[-0.175pt]{0.350pt}{1.204pt}}
\put(608,260){\rule[-0.175pt]{0.350pt}{1.204pt}}
\put(609,255){\rule[-0.175pt]{0.350pt}{1.204pt}}
\put(610,250){\rule[-0.175pt]{0.350pt}{1.204pt}}
\put(611,245){\rule[-0.175pt]{0.350pt}{1.204pt}}
\put(612,240){\rule[-0.175pt]{0.350pt}{1.204pt}}
\put(613,235){\rule[-0.175pt]{0.350pt}{1.204pt}}
\put(614,230){\rule[-0.175pt]{0.350pt}{1.204pt}}
\put(615,225){\rule[-0.175pt]{0.350pt}{1.204pt}}
\put(616,220){\rule[-0.175pt]{0.350pt}{1.204pt}}
\put(617,213){\rule[-0.175pt]{0.350pt}{1.660pt}}
\put(618,206){\rule[-0.175pt]{0.350pt}{1.660pt}}
\put(619,199){\rule[-0.175pt]{0.350pt}{1.660pt}}
\put(620,192){\rule[-0.175pt]{0.350pt}{1.660pt}}
\put(621,185){\rule[-0.175pt]{0.350pt}{1.660pt}}
\put(622,178){\rule[-0.175pt]{0.350pt}{1.660pt}}
\put(623,171){\rule[-0.175pt]{0.350pt}{1.660pt}}
\put(624,164){\rule[-0.175pt]{0.350pt}{1.660pt}}
\put(625,158){\rule[-0.175pt]{0.350pt}{1.660pt}}
\put(626,158){\usebox{\plotpoint}}
\sbox{\plotpoint}{\rule[-0.350pt]{0.700pt}{0.700pt}}%
\put(264,158){\usebox{\plotpoint}}
\put(264,158){\usebox{\plotpoint}}
\put(265,159){\usebox{\plotpoint}}
\put(266,160){\usebox{\plotpoint}}
\put(267,161){\usebox{\plotpoint}}
\put(268,162){\usebox{\plotpoint}}
\put(269,163){\usebox{\plotpoint}}
\put(270,164){\usebox{\plotpoint}}
\put(271,165){\usebox{\plotpoint}}
\put(272,166){\usebox{\plotpoint}}
\put(273,167){\usebox{\plotpoint}}
\put(274,168){\usebox{\plotpoint}}
\put(275,169){\usebox{\plotpoint}}
\put(276,170){\usebox{\plotpoint}}
\put(278,171){\usebox{\plotpoint}}
\put(279,172){\usebox{\plotpoint}}
\put(280,173){\usebox{\plotpoint}}
\put(281,174){\usebox{\plotpoint}}
\put(283,175){\usebox{\plotpoint}}
\put(284,176){\usebox{\plotpoint}}
\put(285,177){\usebox{\plotpoint}}
\put(286,178){\usebox{\plotpoint}}
\put(287,179){\usebox{\plotpoint}}
\put(288,180){\usebox{\plotpoint}}
\put(289,181){\usebox{\plotpoint}}
\put(290,182){\usebox{\plotpoint}}
\put(291,183){\usebox{\plotpoint}}
\put(292,184){\usebox{\plotpoint}}
\put(293,185){\usebox{\plotpoint}}
\put(294,186){\usebox{\plotpoint}}
\put(295,187){\usebox{\plotpoint}}
\put(296,188){\usebox{\plotpoint}}
\put(297,189){\usebox{\plotpoint}}
\put(298,190){\usebox{\plotpoint}}
\put(299,191){\usebox{\plotpoint}}
\put(300,192){\usebox{\plotpoint}}
\put(301,193){\usebox{\plotpoint}}
\put(302,194){\usebox{\plotpoint}}
\put(303,195){\usebox{\plotpoint}}
\put(304,196){\usebox{\plotpoint}}
\put(305,197){\usebox{\plotpoint}}
\put(306,198){\usebox{\plotpoint}}
\put(307,199){\usebox{\plotpoint}}
\put(308,200){\usebox{\plotpoint}}
\put(309,201){\usebox{\plotpoint}}
\put(310,202){\usebox{\plotpoint}}
\put(311,203){\usebox{\plotpoint}}
\put(312,204){\usebox{\plotpoint}}
\put(313,205){\usebox{\plotpoint}}
\put(315,206){\usebox{\plotpoint}}
\put(316,207){\usebox{\plotpoint}}
\put(317,208){\usebox{\plotpoint}}
\put(318,209){\usebox{\plotpoint}}
\put(320,210){\usebox{\plotpoint}}
\put(321,211){\usebox{\plotpoint}}
\put(322,212){\usebox{\plotpoint}}
\put(323,213){\usebox{\plotpoint}}
\put(324,214){\usebox{\plotpoint}}
\put(325,215){\usebox{\plotpoint}}
\put(326,216){\usebox{\plotpoint}}
\put(327,217){\usebox{\plotpoint}}
\put(328,218){\usebox{\plotpoint}}
\put(329,219){\usebox{\plotpoint}}
\put(330,220){\usebox{\plotpoint}}
\put(331,221){\usebox{\plotpoint}}
\put(332,222){\usebox{\plotpoint}}
\put(333,223){\usebox{\plotpoint}}
\put(334,224){\usebox{\plotpoint}}
\put(335,225){\usebox{\plotpoint}}
\put(336,226){\usebox{\plotpoint}}
\put(337,227){\usebox{\plotpoint}}
\put(338,228){\usebox{\plotpoint}}
\put(339,229){\usebox{\plotpoint}}
\put(340,230){\usebox{\plotpoint}}
\put(341,231){\usebox{\plotpoint}}
\put(342,232){\usebox{\plotpoint}}
\put(343,233){\usebox{\plotpoint}}
\put(344,234){\usebox{\plotpoint}}
\put(345,235){\usebox{\plotpoint}}
\put(346,236){\usebox{\plotpoint}}
\put(348,237){\usebox{\plotpoint}}
\put(349,238){\usebox{\plotpoint}}
\put(350,239){\usebox{\plotpoint}}
\put(351,240){\usebox{\plotpoint}}
\put(352,241){\usebox{\plotpoint}}
\put(353,242){\usebox{\plotpoint}}
\put(354,243){\usebox{\plotpoint}}
\put(355,244){\usebox{\plotpoint}}
\put(357,245){\usebox{\plotpoint}}
\put(358,246){\usebox{\plotpoint}}
\put(359,247){\usebox{\plotpoint}}
\put(360,248){\usebox{\plotpoint}}
\put(361,249){\usebox{\plotpoint}}
\put(362,250){\usebox{\plotpoint}}
\put(363,251){\usebox{\plotpoint}}
\put(364,252){\usebox{\plotpoint}}
\put(365,253){\usebox{\plotpoint}}
\put(366,254){\usebox{\plotpoint}}
\put(367,255){\usebox{\plotpoint}}
\put(368,256){\usebox{\plotpoint}}
\put(369,257){\usebox{\plotpoint}}
\put(370,258){\usebox{\plotpoint}}
\put(371,259){\usebox{\plotpoint}}
\put(372,260){\usebox{\plotpoint}}
\put(373,261){\usebox{\plotpoint}}
\put(374,262){\usebox{\plotpoint}}
\put(375,263){\usebox{\plotpoint}}
\put(376,264){\usebox{\plotpoint}}
\put(377,265){\usebox{\plotpoint}}
\put(378,266){\usebox{\plotpoint}}
\put(379,267){\usebox{\plotpoint}}
\put(380,268){\usebox{\plotpoint}}
\put(381,269){\usebox{\plotpoint}}
\put(382,270){\usebox{\plotpoint}}
\put(383,271){\usebox{\plotpoint}}
\put(385,272){\usebox{\plotpoint}}
\put(386,273){\usebox{\plotpoint}}
\put(387,274){\usebox{\plotpoint}}
\put(388,275){\usebox{\plotpoint}}
\put(389,276){\usebox{\plotpoint}}
\put(390,277){\usebox{\plotpoint}}
\put(391,278){\usebox{\plotpoint}}
\put(392,279){\usebox{\plotpoint}}
\put(394,280){\usebox{\plotpoint}}
\put(395,281){\usebox{\plotpoint}}
\put(396,282){\usebox{\plotpoint}}
\put(397,283){\usebox{\plotpoint}}
\put(398,284){\usebox{\plotpoint}}
\put(399,285){\usebox{\plotpoint}}
\put(400,286){\usebox{\plotpoint}}
\put(401,287){\usebox{\plotpoint}}
\put(402,288){\usebox{\plotpoint}}
\put(403,289){\usebox{\plotpoint}}
\put(404,290){\usebox{\plotpoint}}
\put(405,291){\usebox{\plotpoint}}
\put(406,292){\usebox{\plotpoint}}
\put(407,293){\usebox{\plotpoint}}
\put(408,294){\usebox{\plotpoint}}
\put(409,295){\usebox{\plotpoint}}
\put(410,296){\usebox{\plotpoint}}
\put(411,297){\usebox{\plotpoint}}
\put(413,298){\usebox{\plotpoint}}
\put(414,299){\usebox{\plotpoint}}
\put(415,300){\usebox{\plotpoint}}
\put(416,301){\usebox{\plotpoint}}
\put(417,302){\usebox{\plotpoint}}
\put(418,303){\usebox{\plotpoint}}
\put(419,304){\usebox{\plotpoint}}
\put(420,305){\usebox{\plotpoint}}
\put(421,306){\usebox{\plotpoint}}
\put(422,307){\usebox{\plotpoint}}
\put(423,308){\usebox{\plotpoint}}
\put(424,309){\usebox{\plotpoint}}
\put(425,310){\usebox{\plotpoint}}
\put(426,311){\usebox{\plotpoint}}
\put(427,312){\usebox{\plotpoint}}
\put(428,313){\usebox{\plotpoint}}
\put(429,314){\usebox{\plotpoint}}
\put(431,315){\usebox{\plotpoint}}
\put(432,316){\usebox{\plotpoint}}
\put(433,317){\usebox{\plotpoint}}
\put(434,318){\usebox{\plotpoint}}
\put(435,319){\usebox{\plotpoint}}
\put(436,320){\usebox{\plotpoint}}
\put(437,321){\usebox{\plotpoint}}
\put(438,322){\usebox{\plotpoint}}
\put(439,323){\usebox{\plotpoint}}
\put(440,324){\usebox{\plotpoint}}
\put(441,325){\usebox{\plotpoint}}
\put(442,326){\usebox{\plotpoint}}
\put(443,327){\usebox{\plotpoint}}
\put(444,328){\usebox{\plotpoint}}
\put(445,329){\usebox{\plotpoint}}
\put(446,330){\usebox{\plotpoint}}
\put(447,331){\usebox{\plotpoint}}
\put(448,332){\usebox{\plotpoint}}
\put(450,333){\usebox{\plotpoint}}
\put(451,334){\usebox{\plotpoint}}
\put(452,335){\usebox{\plotpoint}}
\put(453,336){\usebox{\plotpoint}}
\put(454,337){\usebox{\plotpoint}}
\put(455,338){\usebox{\plotpoint}}
\put(456,339){\usebox{\plotpoint}}
\put(457,340){\usebox{\plotpoint}}
\put(458,341){\usebox{\plotpoint}}
\put(459,342){\usebox{\plotpoint}}
\put(460,343){\usebox{\plotpoint}}
\put(461,344){\usebox{\plotpoint}}
\put(462,345){\usebox{\plotpoint}}
\put(463,346){\usebox{\plotpoint}}
\put(464,347){\usebox{\plotpoint}}
\put(465,348){\usebox{\plotpoint}}
\put(466,349){\usebox{\plotpoint}}
\put(468,350){\usebox{\plotpoint}}
\put(469,351){\usebox{\plotpoint}}
\put(470,352){\usebox{\plotpoint}}
\put(471,353){\usebox{\plotpoint}}
\put(472,354){\usebox{\plotpoint}}
\put(473,355){\usebox{\plotpoint}}
\put(474,356){\usebox{\plotpoint}}
\put(475,357){\usebox{\plotpoint}}
\put(476,358){\usebox{\plotpoint}}
\put(477,359){\usebox{\plotpoint}}
\put(478,360){\usebox{\plotpoint}}
\put(479,361){\usebox{\plotpoint}}
\put(480,362){\usebox{\plotpoint}}
\put(481,363){\usebox{\plotpoint}}
\put(482,364){\usebox{\plotpoint}}
\put(483,365){\usebox{\plotpoint}}
\put(484,366){\usebox{\plotpoint}}
\put(485,367){\usebox{\plotpoint}}
\put(487,368){\usebox{\plotpoint}}
\put(488,369){\usebox{\plotpoint}}
\put(489,370){\usebox{\plotpoint}}
\put(490,371){\usebox{\plotpoint}}
\put(491,372){\usebox{\plotpoint}}
\put(492,373){\usebox{\plotpoint}}
\put(493,374){\usebox{\plotpoint}}
\put(494,375){\usebox{\plotpoint}}
\put(495,376){\usebox{\plotpoint}}
\put(496,377){\usebox{\plotpoint}}
\put(497,378){\usebox{\plotpoint}}
\put(498,379){\usebox{\plotpoint}}
\put(499,380){\usebox{\plotpoint}}
\put(500,381){\usebox{\plotpoint}}
\put(501,382){\usebox{\plotpoint}}
\put(502,383){\usebox{\plotpoint}}
\put(503,384){\usebox{\plotpoint}}
\put(505,385){\usebox{\plotpoint}}
\put(506,386){\usebox{\plotpoint}}
\put(507,387){\usebox{\plotpoint}}
\put(508,388){\usebox{\plotpoint}}
\put(509,389){\usebox{\plotpoint}}
\put(510,390){\usebox{\plotpoint}}
\put(511,391){\usebox{\plotpoint}}
\put(512,392){\usebox{\plotpoint}}
\put(513,393){\usebox{\plotpoint}}
\put(514,394){\usebox{\plotpoint}}
\put(516,395){\usebox{\plotpoint}}
\put(517,396){\usebox{\plotpoint}}
\put(518,397){\usebox{\plotpoint}}
\put(519,398){\usebox{\plotpoint}}
\put(520,399){\usebox{\plotpoint}}
\put(521,400){\usebox{\plotpoint}}
\put(522,401){\usebox{\plotpoint}}
\put(523,402){\usebox{\plotpoint}}
\put(524,403){\usebox{\plotpoint}}
\put(525,404){\usebox{\plotpoint}}
\put(526,405){\usebox{\plotpoint}}
\put(527,406){\usebox{\plotpoint}}
\put(528,407){\usebox{\plotpoint}}
\put(529,408){\usebox{\plotpoint}}
\put(530,409){\usebox{\plotpoint}}
\put(531,410){\usebox{\plotpoint}}
\put(532,411){\usebox{\plotpoint}}
\put(533,412){\usebox{\plotpoint}}
\put(534,413){\usebox{\plotpoint}}
\put(535,414){\usebox{\plotpoint}}
\put(536,415){\usebox{\plotpoint}}
\put(537,416){\usebox{\plotpoint}}
\put(538,417){\usebox{\plotpoint}}
\put(539,418){\usebox{\plotpoint}}
\put(540,419){\usebox{\plotpoint}}
\put(542,420){\usebox{\plotpoint}}
\put(543,421){\usebox{\plotpoint}}
\put(544,422){\usebox{\plotpoint}}
\put(545,423){\usebox{\plotpoint}}
\put(546,424){\usebox{\plotpoint}}
\put(547,425){\usebox{\plotpoint}}
\put(548,426){\usebox{\plotpoint}}
\put(549,427){\usebox{\plotpoint}}
\put(550,428){\usebox{\plotpoint}}
\put(551,429){\usebox{\plotpoint}}
\put(553,430){\usebox{\plotpoint}}
\put(554,431){\usebox{\plotpoint}}
\put(555,432){\usebox{\plotpoint}}
\put(556,433){\usebox{\plotpoint}}
\put(557,434){\usebox{\plotpoint}}
\put(558,435){\usebox{\plotpoint}}
\put(559,436){\usebox{\plotpoint}}
\put(560,437){\usebox{\plotpoint}}
\put(561,438){\usebox{\plotpoint}}
\put(562,439){\usebox{\plotpoint}}
\put(563,440){\usebox{\plotpoint}}
\put(564,441){\usebox{\plotpoint}}
\put(565,442){\usebox{\plotpoint}}
\put(566,443){\usebox{\plotpoint}}
\put(567,444){\usebox{\plotpoint}}
\put(568,445){\usebox{\plotpoint}}
\put(569,446){\usebox{\plotpoint}}
\put(570,447){\usebox{\plotpoint}}
\put(571,448){\usebox{\plotpoint}}
\put(572,449){\usebox{\plotpoint}}
\put(573,450){\usebox{\plotpoint}}
\put(575,451){\usebox{\plotpoint}}
\put(576,452){\usebox{\plotpoint}}
\put(577,453){\usebox{\plotpoint}}
\put(578,454){\usebox{\plotpoint}}
\put(580,455){\usebox{\plotpoint}}
\put(581,456){\usebox{\plotpoint}}
\put(582,457){\usebox{\plotpoint}}
\put(583,458){\usebox{\plotpoint}}
\put(584,459){\usebox{\plotpoint}}
\put(585,460){\usebox{\plotpoint}}
\put(586,461){\usebox{\plotpoint}}
\put(587,462){\usebox{\plotpoint}}
\put(588,463){\usebox{\plotpoint}}
\put(589,464){\usebox{\plotpoint}}
\put(590,465){\usebox{\plotpoint}}
\put(591,466){\usebox{\plotpoint}}
\put(592,467){\usebox{\plotpoint}}
\put(593,468){\usebox{\plotpoint}}
\put(594,469){\usebox{\plotpoint}}
\put(595,470){\usebox{\plotpoint}}
\put(596,471){\usebox{\plotpoint}}
\put(597,472){\usebox{\plotpoint}}
\put(598,473){\usebox{\plotpoint}}
\put(599,474){\usebox{\plotpoint}}
\put(600,475){\usebox{\plotpoint}}
\put(601,476){\usebox{\plotpoint}}
\put(602,477){\usebox{\plotpoint}}
\put(603,478){\usebox{\plotpoint}}
\put(604,479){\usebox{\plotpoint}}
\put(605,480){\usebox{\plotpoint}}
\put(606,481){\usebox{\plotpoint}}
\put(607,482){\usebox{\plotpoint}}
\put(608,483){\usebox{\plotpoint}}
\put(609,484){\usebox{\plotpoint}}
\put(610,485){\usebox{\plotpoint}}
\put(612,486){\usebox{\plotpoint}}
\put(613,487){\usebox{\plotpoint}}
\put(614,488){\usebox{\plotpoint}}
\put(615,489){\usebox{\plotpoint}}
\put(617,490){\usebox{\plotpoint}}
\put(618,491){\usebox{\plotpoint}}
\put(619,492){\usebox{\plotpoint}}
\put(620,493){\usebox{\plotpoint}}
\put(621,494){\usebox{\plotpoint}}
\put(622,495){\usebox{\plotpoint}}
\put(623,496){\usebox{\plotpoint}}
\put(624,497){\usebox{\plotpoint}}
\put(625,498){\usebox{\plotpoint}}
\sbox{\plotpoint}{\rule[-0.500pt]{1.000pt}{1.000pt}}%
\put(497,158){\rule{.1pt}{.1pt}}
\put(497,164){\rule{.1pt}{.1pt}}
\put(497,169){\rule{.1pt}{.1pt}}
\put(497,175){\rule{.1pt}{.1pt}}
\put(497,180){\rule{.1pt}{.1pt}}
\put(497,186){\rule{.1pt}{.1pt}}
\put(497,192){\rule{.1pt}{.1pt}}
\put(497,197){\rule{.1pt}{.1pt}}
\put(497,203){\rule{.1pt}{.1pt}}
\put(497,209){\rule{.1pt}{.1pt}}
\put(497,214){\rule{.1pt}{.1pt}}
\put(497,220){\rule{.1pt}{.1pt}}
\put(497,225){\rule{.1pt}{.1pt}}
\put(497,231){\rule{.1pt}{.1pt}}
\put(497,237){\rule{.1pt}{.1pt}}
\put(497,242){\rule{.1pt}{.1pt}}
\put(497,248){\rule{.1pt}{.1pt}}
\put(497,254){\rule{.1pt}{.1pt}}
\put(497,259){\rule{.1pt}{.1pt}}
\put(497,265){\rule{.1pt}{.1pt}}
\put(497,270){\rule{.1pt}{.1pt}}
\put(497,276){\rule{.1pt}{.1pt}}
\put(497,282){\rule{.1pt}{.1pt}}
\put(497,287){\rule{.1pt}{.1pt}}
\put(497,293){\rule{.1pt}{.1pt}}
\put(497,299){\rule{.1pt}{.1pt}}
\put(497,304){\rule{.1pt}{.1pt}}
\put(497,310){\rule{.1pt}{.1pt}}
\put(497,315){\rule{.1pt}{.1pt}}
\put(497,321){\rule{.1pt}{.1pt}}
\put(497,327){\rule{.1pt}{.1pt}}
\put(497,332){\rule{.1pt}{.1pt}}
\put(497,338){\rule{.1pt}{.1pt}}
\put(497,343){\rule{.1pt}{.1pt}}
\put(497,349){\rule{.1pt}{.1pt}}
\put(497,355){\rule{.1pt}{.1pt}}
\put(497,360){\rule{.1pt}{.1pt}}
\put(497,366){\rule{.1pt}{.1pt}}
\put(497,372){\rule{.1pt}{.1pt}}
\put(497,377){\rule{.1pt}{.1pt}}
\end{picture}

\hfill

 \end{minipage}
 \begin{minipage}{4cm}
  % GNUPLOT: LaTeX picture
\setlength{\unitlength}{0.240900pt}
\ifx\plotpoint\undefined\newsavebox{\plotpoint}\fi
\sbox{\plotpoint}{\rule[-0.175pt]{0.350pt}{0.350pt}}%
\begin{picture}(469,612)(0,0)
\tenrm
\sbox{\plotpoint}{\rule[-0.175pt]{0.350pt}{0.350pt}}%
\put(264,158){\rule[-0.175pt]{87.206pt}{0.350pt}}
\put(264,158){\rule[-0.175pt]{0.350pt}{82.147pt}}
\put(264,158){\rule[-0.175pt]{87.206pt}{0.350pt}}
\put(626,158){\rule[-0.175pt]{0.350pt}{82.147pt}}
\put(264,499){\rule[-0.175pt]{87.206pt}{0.350pt}}
\put(300,465){\makebox(0,0)[l]{{\footnotesize $a=3$}}}
\put(505,127){\makebox(0,0){{\footnotesize $x_f$}}}
\put(264,158){\rule[-0.175pt]{0.350pt}{82.147pt}}
\put(264,158){\usebox{\plotpoint}}
\put(264,158){\rule[-0.175pt]{0.350pt}{1.900pt}}
\put(265,165){\rule[-0.175pt]{0.350pt}{1.900pt}}
\put(266,173){\rule[-0.175pt]{0.350pt}{1.900pt}}
\put(267,181){\rule[-0.175pt]{0.350pt}{1.900pt}}
\put(268,189){\rule[-0.175pt]{0.350pt}{1.900pt}}
\put(269,197){\rule[-0.175pt]{0.350pt}{1.900pt}}
\put(270,205){\rule[-0.175pt]{0.350pt}{1.900pt}}
\put(271,213){\rule[-0.175pt]{0.350pt}{1.900pt}}
\put(272,221){\rule[-0.175pt]{0.350pt}{1.900pt}}
\put(273,228){\rule[-0.175pt]{0.350pt}{1.373pt}}
\put(274,234){\rule[-0.175pt]{0.350pt}{1.373pt}}
\put(275,240){\rule[-0.175pt]{0.350pt}{1.373pt}}
\put(276,246){\rule[-0.175pt]{0.350pt}{1.373pt}}
\put(277,251){\rule[-0.175pt]{0.350pt}{1.373pt}}
\put(278,257){\rule[-0.175pt]{0.350pt}{1.373pt}}
\put(279,263){\rule[-0.175pt]{0.350pt}{1.373pt}}
\put(280,268){\rule[-0.175pt]{0.350pt}{1.373pt}}
\put(281,274){\rule[-0.175pt]{0.350pt}{1.373pt}}
\put(282,280){\rule[-0.175pt]{0.350pt}{1.373pt}}
\put(283,286){\rule[-0.175pt]{0.350pt}{1.151pt}}
\put(284,290){\rule[-0.175pt]{0.350pt}{1.151pt}}
\put(285,295){\rule[-0.175pt]{0.350pt}{1.151pt}}
\put(286,300){\rule[-0.175pt]{0.350pt}{1.151pt}}
\put(287,305){\rule[-0.175pt]{0.350pt}{1.151pt}}
\put(288,309){\rule[-0.175pt]{0.350pt}{1.151pt}}
\put(289,314){\rule[-0.175pt]{0.350pt}{1.151pt}}
\put(290,319){\rule[-0.175pt]{0.350pt}{1.151pt}}
\put(291,324){\rule[-0.175pt]{0.350pt}{1.151pt}}
\put(292,328){\rule[-0.175pt]{0.350pt}{0.883pt}}
\put(293,332){\rule[-0.175pt]{0.350pt}{0.883pt}}
\put(294,336){\rule[-0.175pt]{0.350pt}{0.883pt}}
\put(295,339){\rule[-0.175pt]{0.350pt}{0.883pt}}
\put(296,343){\rule[-0.175pt]{0.350pt}{0.883pt}}
\put(297,347){\rule[-0.175pt]{0.350pt}{0.883pt}}
\put(298,350){\rule[-0.175pt]{0.350pt}{0.883pt}}
\put(299,354){\rule[-0.175pt]{0.350pt}{0.883pt}}
\put(300,358){\rule[-0.175pt]{0.350pt}{0.883pt}}
\put(301,361){\rule[-0.175pt]{0.350pt}{0.642pt}}
\put(302,364){\rule[-0.175pt]{0.350pt}{0.642pt}}
\put(303,367){\rule[-0.175pt]{0.350pt}{0.642pt}}
\put(304,369){\rule[-0.175pt]{0.350pt}{0.642pt}}
\put(305,372){\rule[-0.175pt]{0.350pt}{0.642pt}}
\put(306,375){\rule[-0.175pt]{0.350pt}{0.642pt}}
\put(307,377){\rule[-0.175pt]{0.350pt}{0.642pt}}
\put(308,380){\rule[-0.175pt]{0.350pt}{0.642pt}}
\put(309,383){\rule[-0.175pt]{0.350pt}{0.642pt}}
\put(310,385){\rule[-0.175pt]{0.350pt}{0.361pt}}
\put(311,387){\rule[-0.175pt]{0.350pt}{0.361pt}}
\put(312,389){\rule[-0.175pt]{0.350pt}{0.361pt}}
\put(313,390){\rule[-0.175pt]{0.350pt}{0.361pt}}
\put(314,392){\rule[-0.175pt]{0.350pt}{0.361pt}}
\put(315,393){\rule[-0.175pt]{0.350pt}{0.361pt}}
\put(316,395){\rule[-0.175pt]{0.350pt}{0.361pt}}
\put(317,396){\rule[-0.175pt]{0.350pt}{0.361pt}}
\put(318,398){\rule[-0.175pt]{0.350pt}{0.361pt}}
\put(319,399){\rule[-0.175pt]{0.350pt}{0.361pt}}
\put(320,401){\usebox{\plotpoint}}
\put(321,402){\usebox{\plotpoint}}
\put(322,403){\usebox{\plotpoint}}
\put(323,404){\usebox{\plotpoint}}
\put(324,405){\usebox{\plotpoint}}
\put(325,406){\usebox{\plotpoint}}
\put(326,407){\usebox{\plotpoint}}
\put(327,408){\usebox{\plotpoint}}
\put(328,409){\usebox{\plotpoint}}
\put(329,410){\rule[-0.175pt]{0.542pt}{0.350pt}}
\put(331,411){\rule[-0.175pt]{0.542pt}{0.350pt}}
\put(333,412){\rule[-0.175pt]{0.542pt}{0.350pt}}
\put(335,413){\rule[-0.175pt]{0.542pt}{0.350pt}}
\put(338,414){\rule[-0.175pt]{2.409pt}{0.350pt}}
\put(348,413){\rule[-0.175pt]{0.434pt}{0.350pt}}
\put(349,412){\rule[-0.175pt]{0.434pt}{0.350pt}}
\put(351,411){\rule[-0.175pt]{0.434pt}{0.350pt}}
\put(353,410){\rule[-0.175pt]{0.434pt}{0.350pt}}
\put(355,409){\rule[-0.175pt]{0.434pt}{0.350pt}}
\put(356,408){\rule[-0.175pt]{0.361pt}{0.350pt}}
\put(358,407){\rule[-0.175pt]{0.361pt}{0.350pt}}
\put(360,406){\rule[-0.175pt]{0.361pt}{0.350pt}}
\put(361,405){\rule[-0.175pt]{0.361pt}{0.350pt}}
\put(363,404){\rule[-0.175pt]{0.361pt}{0.350pt}}
\put(364,403){\rule[-0.175pt]{0.361pt}{0.350pt}}
\put(366,402){\usebox{\plotpoint}}
\put(367,401){\usebox{\plotpoint}}
\put(368,400){\usebox{\plotpoint}}
\put(369,399){\usebox{\plotpoint}}
\put(370,398){\usebox{\plotpoint}}
\put(371,397){\usebox{\plotpoint}}
\put(372,396){\usebox{\plotpoint}}
\put(373,395){\usebox{\plotpoint}}
\put(375,394){\usebox{\plotpoint}}
\put(376,393){\usebox{\plotpoint}}
\put(377,392){\usebox{\plotpoint}}
\put(378,391){\usebox{\plotpoint}}
\put(379,390){\usebox{\plotpoint}}
\put(380,389){\usebox{\plotpoint}}
\put(381,388){\usebox{\plotpoint}}
\put(382,387){\usebox{\plotpoint}}
\put(383,386){\usebox{\plotpoint}}
\put(385,385){\usebox{\plotpoint}}
\put(386,384){\usebox{\plotpoint}}
\put(387,383){\usebox{\plotpoint}}
\put(388,382){\usebox{\plotpoint}}
\put(389,381){\usebox{\plotpoint}}
\put(390,380){\usebox{\plotpoint}}
\put(391,379){\usebox{\plotpoint}}
\put(392,378){\usebox{\plotpoint}}
\put(394,377){\usebox{\plotpoint}}
\put(395,376){\usebox{\plotpoint}}
\put(396,375){\usebox{\plotpoint}}
\put(397,374){\usebox{\plotpoint}}
\put(398,373){\usebox{\plotpoint}}
\put(399,372){\usebox{\plotpoint}}
\put(400,371){\usebox{\plotpoint}}
\put(401,370){\usebox{\plotpoint}}
\put(403,369){\usebox{\plotpoint}}
\put(404,368){\usebox{\plotpoint}}
\put(405,367){\usebox{\plotpoint}}
\put(407,366){\usebox{\plotpoint}}
\put(408,365){\usebox{\plotpoint}}
\put(410,364){\usebox{\plotpoint}}
\put(411,363){\usebox{\plotpoint}}
\put(412,362){\rule[-0.175pt]{0.361pt}{0.350pt}}
\put(414,361){\rule[-0.175pt]{0.361pt}{0.350pt}}
\put(416,360){\rule[-0.175pt]{0.361pt}{0.350pt}}
\put(417,359){\rule[-0.175pt]{0.361pt}{0.350pt}}
\put(419,358){\rule[-0.175pt]{0.361pt}{0.350pt}}
\put(420,357){\rule[-0.175pt]{0.361pt}{0.350pt}}
\put(422,356){\rule[-0.175pt]{0.542pt}{0.350pt}}
\put(424,355){\rule[-0.175pt]{0.542pt}{0.350pt}}
\put(426,354){\rule[-0.175pt]{0.542pt}{0.350pt}}
\put(428,353){\rule[-0.175pt]{0.542pt}{0.350pt}}
\put(431,352){\rule[-0.175pt]{1.084pt}{0.350pt}}
\put(435,351){\rule[-0.175pt]{1.084pt}{0.350pt}}
\put(440,350){\rule[-0.175pt]{3.493pt}{0.350pt}}
\put(454,351){\rule[-0.175pt]{1.084pt}{0.350pt}}
\put(459,352){\rule[-0.175pt]{0.542pt}{0.350pt}}
\put(461,353){\rule[-0.175pt]{0.542pt}{0.350pt}}
\put(463,354){\rule[-0.175pt]{0.542pt}{0.350pt}}
\put(465,355){\rule[-0.175pt]{0.542pt}{0.350pt}}
\put(468,356){\rule[-0.175pt]{0.361pt}{0.350pt}}
\put(469,357){\rule[-0.175pt]{0.361pt}{0.350pt}}
\put(471,358){\rule[-0.175pt]{0.361pt}{0.350pt}}
\put(472,359){\rule[-0.175pt]{0.361pt}{0.350pt}}
\put(474,360){\rule[-0.175pt]{0.361pt}{0.350pt}}
\put(475,361){\rule[-0.175pt]{0.361pt}{0.350pt}}
\put(477,362){\usebox{\plotpoint}}
\put(478,363){\usebox{\plotpoint}}
\put(479,364){\usebox{\plotpoint}}
\put(481,365){\usebox{\plotpoint}}
\put(482,366){\usebox{\plotpoint}}
\put(484,367){\usebox{\plotpoint}}
\put(485,368){\usebox{\plotpoint}}
\put(486,369){\usebox{\plotpoint}}
\put(488,370){\usebox{\plotpoint}}
\put(489,371){\usebox{\plotpoint}}
\put(490,372){\usebox{\plotpoint}}
\put(491,373){\usebox{\plotpoint}}
\put(492,374){\usebox{\plotpoint}}
\put(493,375){\usebox{\plotpoint}}
\put(494,376){\usebox{\plotpoint}}
\put(496,377){\usebox{\plotpoint}}
\put(497,378){\usebox{\plotpoint}}
\put(498,379){\usebox{\plotpoint}}
\put(499,380){\usebox{\plotpoint}}
\put(500,381){\usebox{\plotpoint}}
\put(501,382){\usebox{\plotpoint}}
\put(502,383){\usebox{\plotpoint}}
\put(503,384){\usebox{\plotpoint}}
\put(505,385){\usebox{\plotpoint}}
\put(506,386){\usebox{\plotpoint}}
\put(507,387){\usebox{\plotpoint}}
\put(508,388){\usebox{\plotpoint}}
\put(509,389){\usebox{\plotpoint}}
\put(510,390){\usebox{\plotpoint}}
\put(511,391){\usebox{\plotpoint}}
\put(512,392){\usebox{\plotpoint}}
\put(513,393){\usebox{\plotpoint}}
\put(514,394){\usebox{\plotpoint}}
\put(516,395){\usebox{\plotpoint}}
\put(517,396){\usebox{\plotpoint}}
\put(518,397){\usebox{\plotpoint}}
\put(519,398){\usebox{\plotpoint}}
\put(520,399){\usebox{\plotpoint}}
\put(521,400){\usebox{\plotpoint}}
\put(522,401){\usebox{\plotpoint}}
\put(524,402){\rule[-0.175pt]{0.361pt}{0.350pt}}
\put(525,403){\rule[-0.175pt]{0.361pt}{0.350pt}}
\put(527,404){\rule[-0.175pt]{0.361pt}{0.350pt}}
\put(528,405){\rule[-0.175pt]{0.361pt}{0.350pt}}
\put(530,406){\rule[-0.175pt]{0.361pt}{0.350pt}}
\put(531,407){\rule[-0.175pt]{0.361pt}{0.350pt}}
\put(533,408){\rule[-0.175pt]{0.434pt}{0.350pt}}
\put(534,409){\rule[-0.175pt]{0.434pt}{0.350pt}}
\put(536,410){\rule[-0.175pt]{0.434pt}{0.350pt}}
\put(538,411){\rule[-0.175pt]{0.434pt}{0.350pt}}
\put(540,412){\rule[-0.175pt]{0.434pt}{0.350pt}}
\put(541,413){\rule[-0.175pt]{2.409pt}{0.350pt}}
\put(552,414){\rule[-0.175pt]{0.542pt}{0.350pt}}
\put(554,413){\rule[-0.175pt]{0.542pt}{0.350pt}}
\put(556,412){\rule[-0.175pt]{0.542pt}{0.350pt}}
\put(558,411){\rule[-0.175pt]{0.542pt}{0.350pt}}
\put(561,410){\usebox{\plotpoint}}
\put(562,409){\usebox{\plotpoint}}
\put(563,408){\usebox{\plotpoint}}
\put(564,407){\usebox{\plotpoint}}
\put(565,406){\usebox{\plotpoint}}
\put(566,405){\usebox{\plotpoint}}
\put(567,404){\usebox{\plotpoint}}
\put(568,403){\usebox{\plotpoint}}
\put(569,402){\usebox{\plotpoint}}
\put(570,399){\rule[-0.175pt]{0.350pt}{0.361pt}}
\put(571,398){\rule[-0.175pt]{0.350pt}{0.361pt}}
\put(572,396){\rule[-0.175pt]{0.350pt}{0.361pt}}
\put(573,395){\rule[-0.175pt]{0.350pt}{0.361pt}}
\put(574,393){\rule[-0.175pt]{0.350pt}{0.361pt}}
\put(575,392){\rule[-0.175pt]{0.350pt}{0.361pt}}
\put(576,390){\rule[-0.175pt]{0.350pt}{0.361pt}}
\put(577,389){\rule[-0.175pt]{0.350pt}{0.361pt}}
\put(578,387){\rule[-0.175pt]{0.350pt}{0.361pt}}
\put(579,386){\rule[-0.175pt]{0.350pt}{0.361pt}}
\put(580,383){\rule[-0.175pt]{0.350pt}{0.642pt}}
\put(581,380){\rule[-0.175pt]{0.350pt}{0.642pt}}
\put(582,378){\rule[-0.175pt]{0.350pt}{0.642pt}}
\put(583,375){\rule[-0.175pt]{0.350pt}{0.642pt}}
\put(584,372){\rule[-0.175pt]{0.350pt}{0.642pt}}
\put(585,370){\rule[-0.175pt]{0.350pt}{0.642pt}}
\put(586,367){\rule[-0.175pt]{0.350pt}{0.642pt}}
\put(587,364){\rule[-0.175pt]{0.350pt}{0.642pt}}
\put(588,362){\rule[-0.175pt]{0.350pt}{0.642pt}}
\put(589,358){\rule[-0.175pt]{0.350pt}{0.883pt}}
\put(590,354){\rule[-0.175pt]{0.350pt}{0.883pt}}
\put(591,351){\rule[-0.175pt]{0.350pt}{0.883pt}}
\put(592,347){\rule[-0.175pt]{0.350pt}{0.883pt}}
\put(593,343){\rule[-0.175pt]{0.350pt}{0.883pt}}
\put(594,340){\rule[-0.175pt]{0.350pt}{0.883pt}}
\put(595,336){\rule[-0.175pt]{0.350pt}{0.883pt}}
\put(596,332){\rule[-0.175pt]{0.350pt}{0.883pt}}
\put(597,329){\rule[-0.175pt]{0.350pt}{0.883pt}}
\put(598,324){\rule[-0.175pt]{0.350pt}{1.151pt}}
\put(599,319){\rule[-0.175pt]{0.350pt}{1.151pt}}
\put(600,314){\rule[-0.175pt]{0.350pt}{1.151pt}}
\put(601,309){\rule[-0.175pt]{0.350pt}{1.151pt}}
\put(602,305){\rule[-0.175pt]{0.350pt}{1.151pt}}
\put(603,300){\rule[-0.175pt]{0.350pt}{1.151pt}}
\put(604,295){\rule[-0.175pt]{0.350pt}{1.151pt}}
\put(605,290){\rule[-0.175pt]{0.350pt}{1.151pt}}
\put(606,286){\rule[-0.175pt]{0.350pt}{1.151pt}}
\put(607,280){\rule[-0.175pt]{0.350pt}{1.373pt}}
\put(608,274){\rule[-0.175pt]{0.350pt}{1.373pt}}
\put(609,268){\rule[-0.175pt]{0.350pt}{1.373pt}}
\put(610,263){\rule[-0.175pt]{0.350pt}{1.373pt}}
\put(611,257){\rule[-0.175pt]{0.350pt}{1.373pt}}
\put(612,251){\rule[-0.175pt]{0.350pt}{1.373pt}}
\put(613,246){\rule[-0.175pt]{0.350pt}{1.373pt}}
\put(614,240){\rule[-0.175pt]{0.350pt}{1.373pt}}
\put(615,234){\rule[-0.175pt]{0.350pt}{1.373pt}}
\put(616,229){\rule[-0.175pt]{0.350pt}{1.373pt}}
\put(617,221){\rule[-0.175pt]{0.350pt}{1.900pt}}
\put(618,213){\rule[-0.175pt]{0.350pt}{1.900pt}}
\put(619,205){\rule[-0.175pt]{0.350pt}{1.900pt}}
\put(620,197){\rule[-0.175pt]{0.350pt}{1.900pt}}
\put(621,189){\rule[-0.175pt]{0.350pt}{1.900pt}}
\put(622,181){\rule[-0.175pt]{0.350pt}{1.900pt}}
\put(623,173){\rule[-0.175pt]{0.350pt}{1.900pt}}
\put(624,165){\rule[-0.175pt]{0.350pt}{1.900pt}}
\put(625,158){\rule[-0.175pt]{0.350pt}{1.900pt}}
\put(626,158){\usebox{\plotpoint}}
\sbox{\plotpoint}{\rule[-0.350pt]{0.700pt}{0.700pt}}%
\put(264,158){\usebox{\plotpoint}}
\put(264,158){\usebox{\plotpoint}}
\put(265,159){\usebox{\plotpoint}}
\put(266,160){\usebox{\plotpoint}}
\put(267,161){\usebox{\plotpoint}}
\put(268,162){\usebox{\plotpoint}}
\put(269,163){\usebox{\plotpoint}}
\put(270,164){\usebox{\plotpoint}}
\put(271,165){\usebox{\plotpoint}}
\put(272,166){\usebox{\plotpoint}}
\put(273,167){\usebox{\plotpoint}}
\put(274,168){\usebox{\plotpoint}}
\put(275,169){\usebox{\plotpoint}}
\put(276,170){\usebox{\plotpoint}}
\put(278,171){\usebox{\plotpoint}}
\put(279,172){\usebox{\plotpoint}}
\put(280,173){\usebox{\plotpoint}}
\put(281,174){\usebox{\plotpoint}}
\put(283,175){\usebox{\plotpoint}}
\put(284,176){\usebox{\plotpoint}}
\put(285,177){\usebox{\plotpoint}}
\put(286,178){\usebox{\plotpoint}}
\put(287,179){\usebox{\plotpoint}}
\put(288,180){\usebox{\plotpoint}}
\put(289,181){\usebox{\plotpoint}}
\put(290,182){\usebox{\plotpoint}}
\put(291,183){\usebox{\plotpoint}}
\put(292,184){\usebox{\plotpoint}}
\put(293,185){\usebox{\plotpoint}}
\put(294,186){\usebox{\plotpoint}}
\put(295,187){\usebox{\plotpoint}}
\put(296,188){\usebox{\plotpoint}}
\put(297,189){\usebox{\plotpoint}}
\put(298,190){\usebox{\plotpoint}}
\put(299,191){\usebox{\plotpoint}}
\put(300,192){\usebox{\plotpoint}}
\put(301,193){\usebox{\plotpoint}}
\put(302,194){\usebox{\plotpoint}}
\put(303,195){\usebox{\plotpoint}}
\put(304,196){\usebox{\plotpoint}}
\put(305,197){\usebox{\plotpoint}}
\put(306,198){\usebox{\plotpoint}}
\put(307,199){\usebox{\plotpoint}}
\put(308,200){\usebox{\plotpoint}}
\put(309,201){\usebox{\plotpoint}}
\put(310,202){\usebox{\plotpoint}}
\put(311,203){\usebox{\plotpoint}}
\put(312,204){\usebox{\plotpoint}}
\put(313,205){\usebox{\plotpoint}}
\put(315,206){\usebox{\plotpoint}}
\put(316,207){\usebox{\plotpoint}}
\put(317,208){\usebox{\plotpoint}}
\put(318,209){\usebox{\plotpoint}}
\put(320,210){\usebox{\plotpoint}}
\put(321,211){\usebox{\plotpoint}}
\put(322,212){\usebox{\plotpoint}}
\put(323,213){\usebox{\plotpoint}}
\put(324,214){\usebox{\plotpoint}}
\put(325,215){\usebox{\plotpoint}}
\put(326,216){\usebox{\plotpoint}}
\put(327,217){\usebox{\plotpoint}}
\put(328,218){\usebox{\plotpoint}}
\put(329,219){\usebox{\plotpoint}}
\put(330,220){\usebox{\plotpoint}}
\put(331,221){\usebox{\plotpoint}}
\put(332,222){\usebox{\plotpoint}}
\put(333,223){\usebox{\plotpoint}}
\put(334,224){\usebox{\plotpoint}}
\put(335,225){\usebox{\plotpoint}}
\put(336,226){\usebox{\plotpoint}}
\put(337,227){\usebox{\plotpoint}}
\put(338,228){\usebox{\plotpoint}}
\put(339,229){\usebox{\plotpoint}}
\put(340,230){\usebox{\plotpoint}}
\put(341,231){\usebox{\plotpoint}}
\put(342,232){\usebox{\plotpoint}}
\put(343,233){\usebox{\plotpoint}}
\put(344,234){\usebox{\plotpoint}}
\put(345,235){\usebox{\plotpoint}}
\put(346,236){\usebox{\plotpoint}}
\put(348,237){\usebox{\plotpoint}}
\put(349,238){\usebox{\plotpoint}}
\put(350,239){\usebox{\plotpoint}}
\put(351,240){\usebox{\plotpoint}}
\put(352,241){\usebox{\plotpoint}}
\put(353,242){\usebox{\plotpoint}}
\put(354,243){\usebox{\plotpoint}}
\put(355,244){\usebox{\plotpoint}}
\put(357,245){\usebox{\plotpoint}}
\put(358,246){\usebox{\plotpoint}}
\put(359,247){\usebox{\plotpoint}}
\put(360,248){\usebox{\plotpoint}}
\put(361,249){\usebox{\plotpoint}}
\put(362,250){\usebox{\plotpoint}}
\put(363,251){\usebox{\plotpoint}}
\put(364,252){\usebox{\plotpoint}}
\put(365,253){\usebox{\plotpoint}}
\put(366,254){\usebox{\plotpoint}}
\put(367,255){\usebox{\plotpoint}}
\put(368,256){\usebox{\plotpoint}}
\put(369,257){\usebox{\plotpoint}}
\put(370,258){\usebox{\plotpoint}}
\put(371,259){\usebox{\plotpoint}}
\put(372,260){\usebox{\plotpoint}}
\put(373,261){\usebox{\plotpoint}}
\put(374,262){\usebox{\plotpoint}}
\put(375,263){\usebox{\plotpoint}}
\put(376,264){\usebox{\plotpoint}}
\put(377,265){\usebox{\plotpoint}}
\put(378,266){\usebox{\plotpoint}}
\put(379,267){\usebox{\plotpoint}}
\put(380,268){\usebox{\plotpoint}}
\put(381,269){\usebox{\plotpoint}}
\put(382,270){\usebox{\plotpoint}}
\put(383,271){\usebox{\plotpoint}}
\put(385,272){\usebox{\plotpoint}}
\put(386,273){\usebox{\plotpoint}}
\put(387,274){\usebox{\plotpoint}}
\put(388,275){\usebox{\plotpoint}}
\put(389,276){\usebox{\plotpoint}}
\put(390,277){\usebox{\plotpoint}}
\put(391,278){\usebox{\plotpoint}}
\put(392,279){\usebox{\plotpoint}}
\put(394,280){\usebox{\plotpoint}}
\put(395,281){\usebox{\plotpoint}}
\put(396,282){\usebox{\plotpoint}}
\put(397,283){\usebox{\plotpoint}}
\put(398,284){\usebox{\plotpoint}}
\put(399,285){\usebox{\plotpoint}}
\put(400,286){\usebox{\plotpoint}}
\put(401,287){\usebox{\plotpoint}}
\put(402,288){\usebox{\plotpoint}}
\put(403,289){\usebox{\plotpoint}}
\put(404,290){\usebox{\plotpoint}}
\put(405,291){\usebox{\plotpoint}}
\put(406,292){\usebox{\plotpoint}}
\put(407,293){\usebox{\plotpoint}}
\put(408,294){\usebox{\plotpoint}}
\put(409,295){\usebox{\plotpoint}}
\put(410,296){\usebox{\plotpoint}}
\put(411,297){\usebox{\plotpoint}}
\put(413,298){\usebox{\plotpoint}}
\put(414,299){\usebox{\plotpoint}}
\put(415,300){\usebox{\plotpoint}}
\put(416,301){\usebox{\plotpoint}}
\put(417,302){\usebox{\plotpoint}}
\put(418,303){\usebox{\plotpoint}}
\put(419,304){\usebox{\plotpoint}}
\put(420,305){\usebox{\plotpoint}}
\put(421,306){\usebox{\plotpoint}}
\put(422,307){\usebox{\plotpoint}}
\put(423,308){\usebox{\plotpoint}}
\put(424,309){\usebox{\plotpoint}}
\put(425,310){\usebox{\plotpoint}}
\put(426,311){\usebox{\plotpoint}}
\put(427,312){\usebox{\plotpoint}}
\put(428,313){\usebox{\plotpoint}}
\put(429,314){\usebox{\plotpoint}}
\put(431,315){\usebox{\plotpoint}}
\put(432,316){\usebox{\plotpoint}}
\put(433,317){\usebox{\plotpoint}}
\put(434,318){\usebox{\plotpoint}}
\put(435,319){\usebox{\plotpoint}}
\put(436,320){\usebox{\plotpoint}}
\put(437,321){\usebox{\plotpoint}}
\put(438,322){\usebox{\plotpoint}}
\put(439,323){\usebox{\plotpoint}}
\put(440,324){\usebox{\plotpoint}}
\put(441,325){\usebox{\plotpoint}}
\put(442,326){\usebox{\plotpoint}}
\put(443,327){\usebox{\plotpoint}}
\put(444,328){\usebox{\plotpoint}}
\put(445,329){\usebox{\plotpoint}}
\put(446,330){\usebox{\plotpoint}}
\put(447,331){\usebox{\plotpoint}}
\put(448,332){\usebox{\plotpoint}}
\put(450,333){\usebox{\plotpoint}}
\put(451,334){\usebox{\plotpoint}}
\put(452,335){\usebox{\plotpoint}}
\put(453,336){\usebox{\plotpoint}}
\put(454,337){\usebox{\plotpoint}}
\put(455,338){\usebox{\plotpoint}}
\put(456,339){\usebox{\plotpoint}}
\put(457,340){\usebox{\plotpoint}}
\put(458,341){\usebox{\plotpoint}}
\put(459,342){\usebox{\plotpoint}}
\put(460,343){\usebox{\plotpoint}}
\put(461,344){\usebox{\plotpoint}}
\put(462,345){\usebox{\plotpoint}}
\put(463,346){\usebox{\plotpoint}}
\put(464,347){\usebox{\plotpoint}}
\put(465,348){\usebox{\plotpoint}}
\put(466,349){\usebox{\plotpoint}}
\put(468,350){\usebox{\plotpoint}}
\put(469,351){\usebox{\plotpoint}}
\put(470,352){\usebox{\plotpoint}}
\put(471,353){\usebox{\plotpoint}}
\put(472,354){\usebox{\plotpoint}}
\put(473,355){\usebox{\plotpoint}}
\put(474,356){\usebox{\plotpoint}}
\put(475,357){\usebox{\plotpoint}}
\put(476,358){\usebox{\plotpoint}}
\put(477,359){\usebox{\plotpoint}}
\put(478,360){\usebox{\plotpoint}}
\put(479,361){\usebox{\plotpoint}}
\put(480,362){\usebox{\plotpoint}}
\put(481,363){\usebox{\plotpoint}}
\put(482,364){\usebox{\plotpoint}}
\put(483,365){\usebox{\plotpoint}}
\put(484,366){\usebox{\plotpoint}}
\put(485,367){\usebox{\plotpoint}}
\put(487,368){\usebox{\plotpoint}}
\put(488,369){\usebox{\plotpoint}}
\put(489,370){\usebox{\plotpoint}}
\put(490,371){\usebox{\plotpoint}}
\put(491,372){\usebox{\plotpoint}}
\put(492,373){\usebox{\plotpoint}}
\put(493,374){\usebox{\plotpoint}}
\put(494,375){\usebox{\plotpoint}}
\put(495,376){\usebox{\plotpoint}}
\put(496,377){\usebox{\plotpoint}}
\put(497,378){\usebox{\plotpoint}}
\put(498,379){\usebox{\plotpoint}}
\put(499,380){\usebox{\plotpoint}}
\put(500,381){\usebox{\plotpoint}}
\put(501,382){\usebox{\plotpoint}}
\put(502,383){\usebox{\plotpoint}}
\put(503,384){\usebox{\plotpoint}}
\put(505,385){\usebox{\plotpoint}}
\put(506,386){\usebox{\plotpoint}}
\put(507,387){\usebox{\plotpoint}}
\put(508,388){\usebox{\plotpoint}}
\put(509,389){\usebox{\plotpoint}}
\put(510,390){\usebox{\plotpoint}}
\put(511,391){\usebox{\plotpoint}}
\put(512,392){\usebox{\plotpoint}}
\put(513,393){\usebox{\plotpoint}}
\put(514,394){\usebox{\plotpoint}}
\put(516,395){\usebox{\plotpoint}}
\put(517,396){\usebox{\plotpoint}}
\put(518,397){\usebox{\plotpoint}}
\put(519,398){\usebox{\plotpoint}}
\put(520,399){\usebox{\plotpoint}}
\put(521,400){\usebox{\plotpoint}}
\put(522,401){\usebox{\plotpoint}}
\put(523,402){\usebox{\plotpoint}}
\put(524,403){\usebox{\plotpoint}}
\put(525,404){\usebox{\plotpoint}}
\put(526,405){\usebox{\plotpoint}}
\put(527,406){\usebox{\plotpoint}}
\put(528,407){\usebox{\plotpoint}}
\put(529,408){\usebox{\plotpoint}}
\put(530,409){\usebox{\plotpoint}}
\put(531,410){\usebox{\plotpoint}}
\put(532,411){\usebox{\plotpoint}}
\put(533,412){\usebox{\plotpoint}}
\put(534,413){\usebox{\plotpoint}}
\put(535,414){\usebox{\plotpoint}}
\put(536,415){\usebox{\plotpoint}}
\put(537,416){\usebox{\plotpoint}}
\put(538,417){\usebox{\plotpoint}}
\put(539,418){\usebox{\plotpoint}}
\put(540,419){\usebox{\plotpoint}}
\put(542,420){\usebox{\plotpoint}}
\put(543,421){\usebox{\plotpoint}}
\put(544,422){\usebox{\plotpoint}}
\put(545,423){\usebox{\plotpoint}}
\put(546,424){\usebox{\plotpoint}}
\put(547,425){\usebox{\plotpoint}}
\put(548,426){\usebox{\plotpoint}}
\put(549,427){\usebox{\plotpoint}}
\put(550,428){\usebox{\plotpoint}}
\put(551,429){\usebox{\plotpoint}}
\put(553,430){\usebox{\plotpoint}}
\put(554,431){\usebox{\plotpoint}}
\put(555,432){\usebox{\plotpoint}}
\put(556,433){\usebox{\plotpoint}}
\put(557,434){\usebox{\plotpoint}}
\put(558,435){\usebox{\plotpoint}}
\put(559,436){\usebox{\plotpoint}}
\put(560,437){\usebox{\plotpoint}}
\put(561,438){\usebox{\plotpoint}}
\put(562,439){\usebox{\plotpoint}}
\put(563,440){\usebox{\plotpoint}}
\put(564,441){\usebox{\plotpoint}}
\put(565,442){\usebox{\plotpoint}}
\put(566,443){\usebox{\plotpoint}}
\put(567,444){\usebox{\plotpoint}}
\put(568,445){\usebox{\plotpoint}}
\put(569,446){\usebox{\plotpoint}}
\put(570,447){\usebox{\plotpoint}}
\put(571,448){\usebox{\plotpoint}}
\put(572,449){\usebox{\plotpoint}}
\put(573,450){\usebox{\plotpoint}}
\put(575,451){\usebox{\plotpoint}}
\put(576,452){\usebox{\plotpoint}}
\put(577,453){\usebox{\plotpoint}}
\put(578,454){\usebox{\plotpoint}}
\put(580,455){\usebox{\plotpoint}}
\put(581,456){\usebox{\plotpoint}}
\put(582,457){\usebox{\plotpoint}}
\put(583,458){\usebox{\plotpoint}}
\put(584,459){\usebox{\plotpoint}}
\put(585,460){\usebox{\plotpoint}}
\put(586,461){\usebox{\plotpoint}}
\put(587,462){\usebox{\plotpoint}}
\put(588,463){\usebox{\plotpoint}}
\put(589,464){\usebox{\plotpoint}}
\put(590,465){\usebox{\plotpoint}}
\put(591,466){\usebox{\plotpoint}}
\put(592,467){\usebox{\plotpoint}}
\put(593,468){\usebox{\plotpoint}}
\put(594,469){\usebox{\plotpoint}}
\put(595,470){\usebox{\plotpoint}}
\put(596,471){\usebox{\plotpoint}}
\put(597,472){\usebox{\plotpoint}}
\put(598,473){\usebox{\plotpoint}}
\put(599,474){\usebox{\plotpoint}}
\put(600,475){\usebox{\plotpoint}}
\put(601,476){\usebox{\plotpoint}}
\put(602,477){\usebox{\plotpoint}}
\put(603,478){\usebox{\plotpoint}}
\put(604,479){\usebox{\plotpoint}}
\put(605,480){\usebox{\plotpoint}}
\put(606,481){\usebox{\plotpoint}}
\put(607,482){\usebox{\plotpoint}}
\put(608,483){\usebox{\plotpoint}}
\put(609,484){\usebox{\plotpoint}}
\put(610,485){\usebox{\plotpoint}}
\put(612,486){\usebox{\plotpoint}}
\put(613,487){\usebox{\plotpoint}}
\put(614,488){\usebox{\plotpoint}}
\put(615,489){\usebox{\plotpoint}}
\put(617,490){\usebox{\plotpoint}}
\put(618,491){\usebox{\plotpoint}}
\put(619,492){\usebox{\plotpoint}}
\put(620,493){\usebox{\plotpoint}}
\put(621,494){\usebox{\plotpoint}}
\put(622,495){\usebox{\plotpoint}}
\put(623,496){\usebox{\plotpoint}}
\put(624,497){\usebox{\plotpoint}}
\put(625,498){\usebox{\plotpoint}}
\sbox{\plotpoint}{\rule[-0.500pt]{1.000pt}{1.000pt}}%
\put(505,158){\rule{.1pt}{.1pt}}
\put(505,164){\rule{.1pt}{.1pt}}
\put(505,170){\rule{.1pt}{.1pt}}
\put(505,175){\rule{.1pt}{.1pt}}
\put(505,181){\rule{.1pt}{.1pt}}
\put(505,187){\rule{.1pt}{.1pt}}
\put(505,193){\rule{.1pt}{.1pt}}
\put(505,199){\rule{.1pt}{.1pt}}
\put(505,205){\rule{.1pt}{.1pt}}
\put(505,210){\rule{.1pt}{.1pt}}
\put(505,216){\rule{.1pt}{.1pt}}
\put(505,222){\rule{.1pt}{.1pt}}
\put(505,228){\rule{.1pt}{.1pt}}
\put(505,234){\rule{.1pt}{.1pt}}
\put(505,240){\rule{.1pt}{.1pt}}
\put(505,245){\rule{.1pt}{.1pt}}
\put(505,251){\rule{.1pt}{.1pt}}
\put(505,257){\rule{.1pt}{.1pt}}
\put(505,263){\rule{.1pt}{.1pt}}
\put(505,269){\rule{.1pt}{.1pt}}
\put(505,275){\rule{.1pt}{.1pt}}
\put(505,280){\rule{.1pt}{.1pt}}
\put(505,286){\rule{.1pt}{.1pt}}
\put(505,292){\rule{.1pt}{.1pt}}
\put(505,298){\rule{.1pt}{.1pt}}
\put(505,304){\rule{.1pt}{.1pt}}
\put(505,310){\rule{.1pt}{.1pt}}
\put(505,315){\rule{.1pt}{.1pt}}
\put(505,321){\rule{.1pt}{.1pt}}
\put(505,327){\rule{.1pt}{.1pt}}
\put(505,333){\rule{.1pt}{.1pt}}
\put(505,339){\rule{.1pt}{.1pt}}
\put(505,345){\rule{.1pt}{.1pt}}
\put(505,350){\rule{.1pt}{.1pt}}
\put(505,356){\rule{.1pt}{.1pt}}
\put(505,362){\rule{.1pt}{.1pt}}
\put(505,368){\rule{.1pt}{.1pt}}
\put(505,374){\rule{.1pt}{.1pt}}
\put(505,380){\rule{.1pt}{.1pt}}
\put(505,385){\rule{.1pt}{.1pt}}
\end{picture}

\hfill

 \end{minipage}
 \begin{minipage}{4cm}
  % GNUPLOT: LaTeX picture
\setlength{\unitlength}{0.240900pt}
\ifx\plotpoint\undefined\newsavebox{\plotpoint}\fi
\sbox{\plotpoint}{\rule[-0.175pt]{0.350pt}{0.350pt}}%
\begin{picture}(469,612)(0,0)
\tenrm
\sbox{\plotpoint}{\rule[-0.175pt]{0.350pt}{0.350pt}}%
\put(264,158){\rule[-0.175pt]{87.206pt}{0.350pt}}
\put(264,158){\rule[-0.175pt]{0.350pt}{82.147pt}}
\put(264,158){\rule[-0.175pt]{87.206pt}{0.350pt}}
\put(626,158){\rule[-0.175pt]{0.350pt}{82.147pt}}
\put(264,499){\rule[-0.175pt]{87.206pt}{0.350pt}}
\put(300,465){\makebox(0,0)[l]{{\footnotesize $a>3$}}}
\put(518,127){\makebox(0,0){{\footnotesize $x_f$}}}
\put(566,131){\makebox(0,0){{\footnotesize $x_1$}}}
\put(432,131){\makebox(0,0){{\footnotesize $x_2$}}}
\put(264,158){\rule[-0.175pt]{0.350pt}{82.147pt}}
\put(264,158){\usebox{\plotpoint}}
\put(264,158){\rule[-0.175pt]{0.350pt}{2.355pt}}
\put(265,167){\rule[-0.175pt]{0.350pt}{2.355pt}}
\put(266,177){\rule[-0.175pt]{0.350pt}{2.355pt}}
\put(267,187){\rule[-0.175pt]{0.350pt}{2.355pt}}
\put(268,197){\rule[-0.175pt]{0.350pt}{2.355pt}}
\put(269,206){\rule[-0.175pt]{0.350pt}{2.355pt}}
\put(270,216){\rule[-0.175pt]{0.350pt}{2.355pt}}
\put(271,226){\rule[-0.175pt]{0.350pt}{2.355pt}}
\put(272,236){\rule[-0.175pt]{0.350pt}{2.355pt}}
\put(273,245){\rule[-0.175pt]{0.350pt}{1.638pt}}
\put(274,252){\rule[-0.175pt]{0.350pt}{1.638pt}}
\put(275,259){\rule[-0.175pt]{0.350pt}{1.638pt}}
\put(276,266){\rule[-0.175pt]{0.350pt}{1.638pt}}
\put(277,273){\rule[-0.175pt]{0.350pt}{1.638pt}}
\put(278,279){\rule[-0.175pt]{0.350pt}{1.638pt}}
\put(279,286){\rule[-0.175pt]{0.350pt}{1.638pt}}
\put(280,293){\rule[-0.175pt]{0.350pt}{1.638pt}}
\put(281,300){\rule[-0.175pt]{0.350pt}{1.638pt}}
\put(282,307){\rule[-0.175pt]{0.350pt}{1.638pt}}
\put(283,313){\rule[-0.175pt]{0.350pt}{1.365pt}}
\put(284,319){\rule[-0.175pt]{0.350pt}{1.365pt}}
\put(285,325){\rule[-0.175pt]{0.350pt}{1.365pt}}
\put(286,330){\rule[-0.175pt]{0.350pt}{1.365pt}}
\put(287,336){\rule[-0.175pt]{0.350pt}{1.365pt}}
\put(288,342){\rule[-0.175pt]{0.350pt}{1.365pt}}
\put(289,347){\rule[-0.175pt]{0.350pt}{1.365pt}}
\put(290,353){\rule[-0.175pt]{0.350pt}{1.365pt}}
\put(291,359){\rule[-0.175pt]{0.350pt}{1.365pt}}
\put(292,364){\rule[-0.175pt]{0.350pt}{0.990pt}}
\put(293,369){\rule[-0.175pt]{0.350pt}{0.990pt}}
\put(294,373){\rule[-0.175pt]{0.350pt}{0.990pt}}
\put(295,377){\rule[-0.175pt]{0.350pt}{0.990pt}}
\put(296,381){\rule[-0.175pt]{0.350pt}{0.990pt}}
\put(297,385){\rule[-0.175pt]{0.350pt}{0.990pt}}
\put(298,389){\rule[-0.175pt]{0.350pt}{0.990pt}}
\put(299,393){\rule[-0.175pt]{0.350pt}{0.990pt}}
\put(300,397){\rule[-0.175pt]{0.350pt}{0.990pt}}
\put(301,402){\rule[-0.175pt]{0.350pt}{0.642pt}}
\put(302,404){\rule[-0.175pt]{0.350pt}{0.642pt}}
\put(303,407){\rule[-0.175pt]{0.350pt}{0.642pt}}
\put(304,409){\rule[-0.175pt]{0.350pt}{0.642pt}}
\put(305,412){\rule[-0.175pt]{0.350pt}{0.642pt}}
\put(306,415){\rule[-0.175pt]{0.350pt}{0.642pt}}
\put(307,417){\rule[-0.175pt]{0.350pt}{0.642pt}}
\put(308,420){\rule[-0.175pt]{0.350pt}{0.642pt}}
\put(309,423){\rule[-0.175pt]{0.350pt}{0.642pt}}
\put(310,425){\usebox{\plotpoint}}
\put(311,427){\usebox{\plotpoint}}
\put(312,428){\usebox{\plotpoint}}
\put(313,429){\usebox{\plotpoint}}
\put(314,431){\usebox{\plotpoint}}
\put(315,432){\usebox{\plotpoint}}
\put(316,433){\usebox{\plotpoint}}
\put(317,435){\usebox{\plotpoint}}
\put(318,436){\usebox{\plotpoint}}
\put(319,437){\usebox{\plotpoint}}
\put(320,438){\usebox{\plotpoint}}
\put(320,439){\rule[-0.175pt]{0.434pt}{0.350pt}}
\put(321,440){\rule[-0.175pt]{0.434pt}{0.350pt}}
\put(323,441){\rule[-0.175pt]{0.434pt}{0.350pt}}
\put(325,442){\rule[-0.175pt]{0.434pt}{0.350pt}}
\put(327,443){\rule[-0.175pt]{0.434pt}{0.350pt}}
\put(328,444){\rule[-0.175pt]{0.723pt}{0.350pt}}
\put(332,443){\rule[-0.175pt]{0.723pt}{0.350pt}}
\put(335,442){\rule[-0.175pt]{0.723pt}{0.350pt}}
\put(338,441){\usebox{\plotpoint}}
\put(339,440){\usebox{\plotpoint}}
\put(340,439){\usebox{\plotpoint}}
\put(341,438){\usebox{\plotpoint}}
\put(343,437){\usebox{\plotpoint}}
\put(344,436){\usebox{\plotpoint}}
\put(345,435){\usebox{\plotpoint}}
\put(346,434){\usebox{\plotpoint}}
\put(348,431){\usebox{\plotpoint}}
\put(349,430){\usebox{\plotpoint}}
\put(350,429){\usebox{\plotpoint}}
\put(351,428){\usebox{\plotpoint}}
\put(352,426){\usebox{\plotpoint}}
\put(353,425){\usebox{\plotpoint}}
\put(354,424){\usebox{\plotpoint}}
\put(355,423){\usebox{\plotpoint}}
\put(356,422){\usebox{\plotpoint}}
\put(357,420){\rule[-0.175pt]{0.350pt}{0.401pt}}
\put(358,418){\rule[-0.175pt]{0.350pt}{0.401pt}}
\put(359,417){\rule[-0.175pt]{0.350pt}{0.401pt}}
\put(360,415){\rule[-0.175pt]{0.350pt}{0.401pt}}
\put(361,413){\rule[-0.175pt]{0.350pt}{0.401pt}}
\put(362,412){\rule[-0.175pt]{0.350pt}{0.401pt}}
\put(363,410){\rule[-0.175pt]{0.350pt}{0.401pt}}
\put(364,408){\rule[-0.175pt]{0.350pt}{0.401pt}}
\put(365,407){\rule[-0.175pt]{0.350pt}{0.401pt}}
\put(366,405){\rule[-0.175pt]{0.350pt}{0.428pt}}
\put(367,403){\rule[-0.175pt]{0.350pt}{0.428pt}}
\put(368,401){\rule[-0.175pt]{0.350pt}{0.428pt}}
\put(369,399){\rule[-0.175pt]{0.350pt}{0.428pt}}
\put(370,398){\rule[-0.175pt]{0.350pt}{0.428pt}}
\put(371,396){\rule[-0.175pt]{0.350pt}{0.428pt}}
\put(372,394){\rule[-0.175pt]{0.350pt}{0.428pt}}
\put(373,392){\rule[-0.175pt]{0.350pt}{0.428pt}}
\put(374,391){\rule[-0.175pt]{0.350pt}{0.428pt}}
\put(375,389){\rule[-0.175pt]{0.350pt}{0.385pt}}
\put(376,387){\rule[-0.175pt]{0.350pt}{0.385pt}}
\put(377,386){\rule[-0.175pt]{0.350pt}{0.385pt}}
\put(378,384){\rule[-0.175pt]{0.350pt}{0.385pt}}
\put(379,382){\rule[-0.175pt]{0.350pt}{0.385pt}}
\put(380,381){\rule[-0.175pt]{0.350pt}{0.385pt}}
\put(381,379){\rule[-0.175pt]{0.350pt}{0.385pt}}
\put(382,378){\rule[-0.175pt]{0.350pt}{0.385pt}}
\put(383,376){\rule[-0.175pt]{0.350pt}{0.385pt}}
\put(384,375){\rule[-0.175pt]{0.350pt}{0.385pt}}
\put(385,373){\rule[-0.175pt]{0.350pt}{0.401pt}}
\put(386,371){\rule[-0.175pt]{0.350pt}{0.401pt}}
\put(387,370){\rule[-0.175pt]{0.350pt}{0.401pt}}
\put(388,368){\rule[-0.175pt]{0.350pt}{0.401pt}}
\put(389,366){\rule[-0.175pt]{0.350pt}{0.401pt}}
\put(390,365){\rule[-0.175pt]{0.350pt}{0.401pt}}
\put(391,363){\rule[-0.175pt]{0.350pt}{0.401pt}}
\put(392,361){\rule[-0.175pt]{0.350pt}{0.401pt}}
\put(393,360){\rule[-0.175pt]{0.350pt}{0.401pt}}
\put(394,358){\rule[-0.175pt]{0.350pt}{0.375pt}}
\put(395,356){\rule[-0.175pt]{0.350pt}{0.375pt}}
\put(396,355){\rule[-0.175pt]{0.350pt}{0.375pt}}
\put(397,353){\rule[-0.175pt]{0.350pt}{0.375pt}}
\put(398,352){\rule[-0.175pt]{0.350pt}{0.375pt}}
\put(399,350){\rule[-0.175pt]{0.350pt}{0.375pt}}
\put(400,349){\rule[-0.175pt]{0.350pt}{0.375pt}}
\put(401,347){\rule[-0.175pt]{0.350pt}{0.375pt}}
\put(402,346){\rule[-0.175pt]{0.350pt}{0.375pt}}
\put(403,344){\usebox{\plotpoint}}
\put(404,343){\usebox{\plotpoint}}
\put(405,342){\usebox{\plotpoint}}
\put(406,340){\usebox{\plotpoint}}
\put(407,339){\usebox{\plotpoint}}
\put(408,338){\usebox{\plotpoint}}
\put(409,336){\usebox{\plotpoint}}
\put(410,335){\usebox{\plotpoint}}
\put(411,334){\usebox{\plotpoint}}
\put(412,333){\usebox{\plotpoint}}
\put(413,333){\usebox{\plotpoint}}
\put(413,333){\usebox{\plotpoint}}
\put(414,332){\usebox{\plotpoint}}
\put(415,331){\usebox{\plotpoint}}
\put(416,330){\usebox{\plotpoint}}
\put(417,329){\usebox{\plotpoint}}
\put(418,328){\usebox{\plotpoint}}
\put(419,327){\usebox{\plotpoint}}
\put(420,326){\usebox{\plotpoint}}
\put(421,325){\usebox{\plotpoint}}
\put(422,324){\usebox{\plotpoint}}
\put(423,323){\usebox{\plotpoint}}
\put(424,322){\usebox{\plotpoint}}
\put(425,321){\usebox{\plotpoint}}
\put(427,320){\usebox{\plotpoint}}
\put(428,319){\usebox{\plotpoint}}
\put(429,318){\usebox{\plotpoint}}
\put(430,317){\rule[-0.175pt]{0.723pt}{0.350pt}}
\put(434,316){\rule[-0.175pt]{0.723pt}{0.350pt}}
\put(437,315){\rule[-0.175pt]{0.723pt}{0.350pt}}
\put(440,314){\rule[-0.175pt]{3.132pt}{0.350pt}}
\put(453,315){\rule[-0.175pt]{0.723pt}{0.350pt}}
\put(456,316){\rule[-0.175pt]{0.723pt}{0.350pt}}
\put(459,317){\usebox{\plotpoint}}
\put(460,318){\usebox{\plotpoint}}
\put(461,319){\usebox{\plotpoint}}
\put(462,320){\usebox{\plotpoint}}
\put(464,321){\usebox{\plotpoint}}
\put(465,322){\usebox{\plotpoint}}
\put(466,323){\usebox{\plotpoint}}
\put(467,324){\usebox{\plotpoint}}
\put(469,325){\usebox{\plotpoint}}
\put(470,326){\usebox{\plotpoint}}
\put(471,327){\usebox{\plotpoint}}
\put(472,328){\usebox{\plotpoint}}
\put(473,329){\usebox{\plotpoint}}
\put(474,330){\usebox{\plotpoint}}
\put(475,331){\usebox{\plotpoint}}
\put(476,332){\usebox{\plotpoint}}
\put(477,333){\usebox{\plotpoint}}
\put(478,334){\usebox{\plotpoint}}
\put(479,335){\usebox{\plotpoint}}
\put(480,336){\usebox{\plotpoint}}
\put(481,338){\usebox{\plotpoint}}
\put(482,339){\usebox{\plotpoint}}
\put(483,340){\usebox{\plotpoint}}
\put(484,342){\usebox{\plotpoint}}
\put(485,343){\usebox{\plotpoint}}
\put(486,344){\usebox{\plotpoint}}
\put(487,345){\rule[-0.175pt]{0.350pt}{0.375pt}}
\put(488,347){\rule[-0.175pt]{0.350pt}{0.375pt}}
\put(489,349){\rule[-0.175pt]{0.350pt}{0.375pt}}
\put(490,350){\rule[-0.175pt]{0.350pt}{0.375pt}}
\put(491,352){\rule[-0.175pt]{0.350pt}{0.375pt}}
\put(492,353){\rule[-0.175pt]{0.350pt}{0.375pt}}
\put(493,355){\rule[-0.175pt]{0.350pt}{0.375pt}}
\put(494,356){\rule[-0.175pt]{0.350pt}{0.375pt}}
\put(495,358){\rule[-0.175pt]{0.350pt}{0.375pt}}
\put(496,359){\rule[-0.175pt]{0.350pt}{0.402pt}}
\put(497,361){\rule[-0.175pt]{0.350pt}{0.401pt}}
\put(498,363){\rule[-0.175pt]{0.350pt}{0.401pt}}
\put(499,364){\rule[-0.175pt]{0.350pt}{0.401pt}}
\put(500,366){\rule[-0.175pt]{0.350pt}{0.401pt}}
\put(501,368){\rule[-0.175pt]{0.350pt}{0.401pt}}
\put(502,369){\rule[-0.175pt]{0.350pt}{0.401pt}}
\put(503,371){\rule[-0.175pt]{0.350pt}{0.401pt}}
\put(504,373){\rule[-0.175pt]{0.350pt}{0.401pt}}
\put(505,374){\rule[-0.175pt]{0.350pt}{0.385pt}}
\put(506,376){\rule[-0.175pt]{0.350pt}{0.385pt}}
\put(507,378){\rule[-0.175pt]{0.350pt}{0.385pt}}
\put(508,379){\rule[-0.175pt]{0.350pt}{0.385pt}}
\put(509,381){\rule[-0.175pt]{0.350pt}{0.385pt}}
\put(510,383){\rule[-0.175pt]{0.350pt}{0.385pt}}
\put(511,384){\rule[-0.175pt]{0.350pt}{0.385pt}}
\put(512,386){\rule[-0.175pt]{0.350pt}{0.385pt}}
\put(513,387){\rule[-0.175pt]{0.350pt}{0.385pt}}
\put(514,389){\rule[-0.175pt]{0.350pt}{0.385pt}}
\put(515,391){\rule[-0.175pt]{0.350pt}{0.428pt}}
\put(516,392){\rule[-0.175pt]{0.350pt}{0.428pt}}
\put(517,394){\rule[-0.175pt]{0.350pt}{0.428pt}}
\put(518,396){\rule[-0.175pt]{0.350pt}{0.428pt}}
\put(519,398){\rule[-0.175pt]{0.350pt}{0.428pt}}
\put(520,399){\rule[-0.175pt]{0.350pt}{0.428pt}}
\put(521,401){\rule[-0.175pt]{0.350pt}{0.428pt}}
\put(522,403){\rule[-0.175pt]{0.350pt}{0.428pt}}
\put(523,405){\rule[-0.175pt]{0.350pt}{0.428pt}}
\put(524,406){\rule[-0.175pt]{0.350pt}{0.402pt}}
\put(525,408){\rule[-0.175pt]{0.350pt}{0.401pt}}
\put(526,410){\rule[-0.175pt]{0.350pt}{0.401pt}}
\put(527,411){\rule[-0.175pt]{0.350pt}{0.401pt}}
\put(528,413){\rule[-0.175pt]{0.350pt}{0.401pt}}
\put(529,415){\rule[-0.175pt]{0.350pt}{0.401pt}}
\put(530,416){\rule[-0.175pt]{0.350pt}{0.401pt}}
\put(531,418){\rule[-0.175pt]{0.350pt}{0.401pt}}
\put(532,420){\rule[-0.175pt]{0.350pt}{0.401pt}}
\put(533,421){\usebox{\plotpoint}}
\put(534,423){\usebox{\plotpoint}}
\put(535,424){\usebox{\plotpoint}}
\put(536,425){\usebox{\plotpoint}}
\put(537,426){\usebox{\plotpoint}}
\put(538,428){\usebox{\plotpoint}}
\put(539,429){\usebox{\plotpoint}}
\put(540,430){\usebox{\plotpoint}}
\put(541,431){\usebox{\plotpoint}}
\put(542,433){\usebox{\plotpoint}}
\put(543,434){\usebox{\plotpoint}}
\put(544,435){\usebox{\plotpoint}}
\put(545,436){\usebox{\plotpoint}}
\put(547,437){\usebox{\plotpoint}}
\put(548,438){\usebox{\plotpoint}}
\put(549,439){\usebox{\plotpoint}}
\put(550,440){\usebox{\plotpoint}}
\put(552,441){\rule[-0.175pt]{0.723pt}{0.350pt}}
\put(555,442){\rule[-0.175pt]{0.723pt}{0.350pt}}
\put(558,443){\rule[-0.175pt]{0.723pt}{0.350pt}}
\put(561,444){\rule[-0.175pt]{0.434pt}{0.350pt}}
\put(562,443){\rule[-0.175pt]{0.434pt}{0.350pt}}
\put(564,442){\rule[-0.175pt]{0.434pt}{0.350pt}}
\put(566,441){\rule[-0.175pt]{0.434pt}{0.350pt}}
\put(568,440){\rule[-0.175pt]{0.434pt}{0.350pt}}
\put(569,439){\usebox{\plotpoint}}
\put(570,437){\usebox{\plotpoint}}
\put(571,436){\usebox{\plotpoint}}
\put(572,435){\usebox{\plotpoint}}
\put(573,433){\usebox{\plotpoint}}
\put(574,432){\usebox{\plotpoint}}
\put(575,431){\usebox{\plotpoint}}
\put(576,429){\usebox{\plotpoint}}
\put(577,428){\usebox{\plotpoint}}
\put(578,427){\usebox{\plotpoint}}
\put(579,426){\usebox{\plotpoint}}
\put(580,423){\rule[-0.175pt]{0.350pt}{0.642pt}}
\put(581,420){\rule[-0.175pt]{0.350pt}{0.642pt}}
\put(582,418){\rule[-0.175pt]{0.350pt}{0.642pt}}
\put(583,415){\rule[-0.175pt]{0.350pt}{0.642pt}}
\put(584,412){\rule[-0.175pt]{0.350pt}{0.642pt}}
\put(585,410){\rule[-0.175pt]{0.350pt}{0.642pt}}
\put(586,407){\rule[-0.175pt]{0.350pt}{0.642pt}}
\put(587,404){\rule[-0.175pt]{0.350pt}{0.642pt}}
\put(588,402){\rule[-0.175pt]{0.350pt}{0.642pt}}
\put(589,397){\rule[-0.175pt]{0.350pt}{0.990pt}}
\put(590,393){\rule[-0.175pt]{0.350pt}{0.990pt}}
\put(591,389){\rule[-0.175pt]{0.350pt}{0.990pt}}
\put(592,385){\rule[-0.175pt]{0.350pt}{0.990pt}}
\put(593,381){\rule[-0.175pt]{0.350pt}{0.990pt}}
\put(594,377){\rule[-0.175pt]{0.350pt}{0.990pt}}
\put(595,373){\rule[-0.175pt]{0.350pt}{0.990pt}}
\put(596,369){\rule[-0.175pt]{0.350pt}{0.990pt}}
\put(597,365){\rule[-0.175pt]{0.350pt}{0.990pt}}
\put(598,359){\rule[-0.175pt]{0.350pt}{1.365pt}}
\put(599,353){\rule[-0.175pt]{0.350pt}{1.365pt}}
\put(600,348){\rule[-0.175pt]{0.350pt}{1.365pt}}
\put(601,342){\rule[-0.175pt]{0.350pt}{1.365pt}}
\put(602,336){\rule[-0.175pt]{0.350pt}{1.365pt}}
\put(603,331){\rule[-0.175pt]{0.350pt}{1.365pt}}
\put(604,325){\rule[-0.175pt]{0.350pt}{1.365pt}}
\put(605,319){\rule[-0.175pt]{0.350pt}{1.365pt}}
\put(606,314){\rule[-0.175pt]{0.350pt}{1.365pt}}
\put(607,307){\rule[-0.175pt]{0.350pt}{1.638pt}}
\put(608,300){\rule[-0.175pt]{0.350pt}{1.638pt}}
\put(609,293){\rule[-0.175pt]{0.350pt}{1.638pt}}
\put(610,286){\rule[-0.175pt]{0.350pt}{1.638pt}}
\put(611,280){\rule[-0.175pt]{0.350pt}{1.638pt}}
\put(612,273){\rule[-0.175pt]{0.350pt}{1.638pt}}
\put(613,266){\rule[-0.175pt]{0.350pt}{1.638pt}}
\put(614,259){\rule[-0.175pt]{0.350pt}{1.638pt}}
\put(615,252){\rule[-0.175pt]{0.350pt}{1.638pt}}
\put(616,246){\rule[-0.175pt]{0.350pt}{1.638pt}}
\put(617,236){\rule[-0.175pt]{0.350pt}{2.355pt}}
\put(618,226){\rule[-0.175pt]{0.350pt}{2.355pt}}
\put(619,216){\rule[-0.175pt]{0.350pt}{2.355pt}}
\put(620,206){\rule[-0.175pt]{0.350pt}{2.355pt}}
\put(621,197){\rule[-0.175pt]{0.350pt}{2.355pt}}
\put(622,187){\rule[-0.175pt]{0.350pt}{2.355pt}}
\put(623,177){\rule[-0.175pt]{0.350pt}{2.355pt}}
\put(624,167){\rule[-0.175pt]{0.350pt}{2.355pt}}
\put(625,158){\rule[-0.175pt]{0.350pt}{2.355pt}}
\put(626,158){\usebox{\plotpoint}}
\sbox{\plotpoint}{\rule[-0.350pt]{0.700pt}{0.700pt}}%
\put(264,158){\usebox{\plotpoint}}
\put(264,158){\usebox{\plotpoint}}
\put(265,159){\usebox{\plotpoint}}
\put(266,160){\usebox{\plotpoint}}
\put(267,161){\usebox{\plotpoint}}
\put(268,162){\usebox{\plotpoint}}
\put(269,163){\usebox{\plotpoint}}
\put(270,164){\usebox{\plotpoint}}
\put(271,165){\usebox{\plotpoint}}
\put(272,166){\usebox{\plotpoint}}
\put(273,167){\usebox{\plotpoint}}
\put(274,168){\usebox{\plotpoint}}
\put(275,169){\usebox{\plotpoint}}
\put(276,170){\usebox{\plotpoint}}
\put(278,171){\usebox{\plotpoint}}
\put(279,172){\usebox{\plotpoint}}
\put(280,173){\usebox{\plotpoint}}
\put(281,174){\usebox{\plotpoint}}
\put(283,175){\usebox{\plotpoint}}
\put(284,176){\usebox{\plotpoint}}
\put(285,177){\usebox{\plotpoint}}
\put(286,178){\usebox{\plotpoint}}
\put(287,179){\usebox{\plotpoint}}
\put(288,180){\usebox{\plotpoint}}
\put(289,181){\usebox{\plotpoint}}
\put(290,182){\usebox{\plotpoint}}
\put(291,183){\usebox{\plotpoint}}
\put(292,184){\usebox{\plotpoint}}
\put(293,185){\usebox{\plotpoint}}
\put(294,186){\usebox{\plotpoint}}
\put(295,187){\usebox{\plotpoint}}
\put(296,188){\usebox{\plotpoint}}
\put(297,189){\usebox{\plotpoint}}
\put(298,190){\usebox{\plotpoint}}
\put(299,191){\usebox{\plotpoint}}
\put(300,192){\usebox{\plotpoint}}
\put(301,193){\usebox{\plotpoint}}
\put(302,194){\usebox{\plotpoint}}
\put(303,195){\usebox{\plotpoint}}
\put(304,196){\usebox{\plotpoint}}
\put(305,197){\usebox{\plotpoint}}
\put(306,198){\usebox{\plotpoint}}
\put(307,199){\usebox{\plotpoint}}
\put(308,200){\usebox{\plotpoint}}
\put(309,201){\usebox{\plotpoint}}
\put(310,202){\usebox{\plotpoint}}
\put(311,203){\usebox{\plotpoint}}
\put(312,204){\usebox{\plotpoint}}
\put(313,205){\usebox{\plotpoint}}
\put(315,206){\usebox{\plotpoint}}
\put(316,207){\usebox{\plotpoint}}
\put(317,208){\usebox{\plotpoint}}
\put(318,209){\usebox{\plotpoint}}
\put(320,210){\usebox{\plotpoint}}
\put(321,211){\usebox{\plotpoint}}
\put(322,212){\usebox{\plotpoint}}
\put(323,213){\usebox{\plotpoint}}
\put(324,214){\usebox{\plotpoint}}
\put(325,215){\usebox{\plotpoint}}
\put(326,216){\usebox{\plotpoint}}
\put(327,217){\usebox{\plotpoint}}
\put(328,218){\usebox{\plotpoint}}
\put(329,219){\usebox{\plotpoint}}
\put(330,220){\usebox{\plotpoint}}
\put(331,221){\usebox{\plotpoint}}
\put(332,222){\usebox{\plotpoint}}
\put(333,223){\usebox{\plotpoint}}
\put(334,224){\usebox{\plotpoint}}
\put(335,225){\usebox{\plotpoint}}
\put(336,226){\usebox{\plotpoint}}
\put(337,227){\usebox{\plotpoint}}
\put(338,228){\usebox{\plotpoint}}
\put(339,229){\usebox{\plotpoint}}
\put(340,230){\usebox{\plotpoint}}
\put(341,231){\usebox{\plotpoint}}
\put(342,232){\usebox{\plotpoint}}
\put(343,233){\usebox{\plotpoint}}
\put(344,234){\usebox{\plotpoint}}
\put(345,235){\usebox{\plotpoint}}
\put(346,236){\usebox{\plotpoint}}
\put(348,237){\usebox{\plotpoint}}
\put(349,238){\usebox{\plotpoint}}
\put(350,239){\usebox{\plotpoint}}
\put(351,240){\usebox{\plotpoint}}
\put(352,241){\usebox{\plotpoint}}
\put(353,242){\usebox{\plotpoint}}
\put(354,243){\usebox{\plotpoint}}
\put(355,244){\usebox{\plotpoint}}
\put(357,245){\usebox{\plotpoint}}
\put(358,246){\usebox{\plotpoint}}
\put(359,247){\usebox{\plotpoint}}
\put(360,248){\usebox{\plotpoint}}
\put(361,249){\usebox{\plotpoint}}
\put(362,250){\usebox{\plotpoint}}
\put(363,251){\usebox{\plotpoint}}
\put(364,252){\usebox{\plotpoint}}
\put(365,253){\usebox{\plotpoint}}
\put(366,254){\usebox{\plotpoint}}
\put(367,255){\usebox{\plotpoint}}
\put(368,256){\usebox{\plotpoint}}
\put(369,257){\usebox{\plotpoint}}
\put(370,258){\usebox{\plotpoint}}
\put(371,259){\usebox{\plotpoint}}
\put(372,260){\usebox{\plotpoint}}
\put(373,261){\usebox{\plotpoint}}
\put(374,262){\usebox{\plotpoint}}
\put(375,263){\usebox{\plotpoint}}
\put(376,264){\usebox{\plotpoint}}
\put(377,265){\usebox{\plotpoint}}
\put(378,266){\usebox{\plotpoint}}
\put(379,267){\usebox{\plotpoint}}
\put(380,268){\usebox{\plotpoint}}
\put(381,269){\usebox{\plotpoint}}
\put(382,270){\usebox{\plotpoint}}
\put(383,271){\usebox{\plotpoint}}
\put(385,272){\usebox{\plotpoint}}
\put(386,273){\usebox{\plotpoint}}
\put(387,274){\usebox{\plotpoint}}
\put(388,275){\usebox{\plotpoint}}
\put(389,276){\usebox{\plotpoint}}
\put(390,277){\usebox{\plotpoint}}
\put(391,278){\usebox{\plotpoint}}
\put(392,279){\usebox{\plotpoint}}
\put(394,280){\usebox{\plotpoint}}
\put(395,281){\usebox{\plotpoint}}
\put(396,282){\usebox{\plotpoint}}
\put(397,283){\usebox{\plotpoint}}
\put(398,284){\usebox{\plotpoint}}
\put(399,285){\usebox{\plotpoint}}
\put(400,286){\usebox{\plotpoint}}
\put(401,287){\usebox{\plotpoint}}
\put(402,288){\usebox{\plotpoint}}
\put(403,289){\usebox{\plotpoint}}
\put(404,290){\usebox{\plotpoint}}
\put(405,291){\usebox{\plotpoint}}
\put(406,292){\usebox{\plotpoint}}
\put(407,293){\usebox{\plotpoint}}
\put(408,294){\usebox{\plotpoint}}
\put(409,295){\usebox{\plotpoint}}
\put(410,296){\usebox{\plotpoint}}
\put(411,297){\usebox{\plotpoint}}
\put(413,298){\usebox{\plotpoint}}
\put(414,299){\usebox{\plotpoint}}
\put(415,300){\usebox{\plotpoint}}
\put(416,301){\usebox{\plotpoint}}
\put(417,302){\usebox{\plotpoint}}
\put(418,303){\usebox{\plotpoint}}
\put(419,304){\usebox{\plotpoint}}
\put(420,305){\usebox{\plotpoint}}
\put(421,306){\usebox{\plotpoint}}
\put(422,307){\usebox{\plotpoint}}
\put(423,308){\usebox{\plotpoint}}
\put(424,309){\usebox{\plotpoint}}
\put(425,310){\usebox{\plotpoint}}
\put(426,311){\usebox{\plotpoint}}
\put(427,312){\usebox{\plotpoint}}
\put(428,313){\usebox{\plotpoint}}
\put(429,314){\usebox{\plotpoint}}
\put(431,315){\usebox{\plotpoint}}
\put(432,316){\usebox{\plotpoint}}
\put(433,317){\usebox{\plotpoint}}
\put(434,318){\usebox{\plotpoint}}
\put(435,319){\usebox{\plotpoint}}
\put(436,320){\usebox{\plotpoint}}
\put(437,321){\usebox{\plotpoint}}
\put(438,322){\usebox{\plotpoint}}
\put(439,323){\usebox{\plotpoint}}
\put(440,324){\usebox{\plotpoint}}
\put(441,325){\usebox{\plotpoint}}
\put(442,326){\usebox{\plotpoint}}
\put(443,327){\usebox{\plotpoint}}
\put(444,328){\usebox{\plotpoint}}
\put(445,329){\usebox{\plotpoint}}
\put(446,330){\usebox{\plotpoint}}
\put(447,331){\usebox{\plotpoint}}
\put(448,332){\usebox{\plotpoint}}
\put(450,333){\usebox{\plotpoint}}
\put(451,334){\usebox{\plotpoint}}
\put(452,335){\usebox{\plotpoint}}
\put(453,336){\usebox{\plotpoint}}
\put(454,337){\usebox{\plotpoint}}
\put(455,338){\usebox{\plotpoint}}
\put(456,339){\usebox{\plotpoint}}
\put(457,340){\usebox{\plotpoint}}
\put(458,341){\usebox{\plotpoint}}
\put(459,342){\usebox{\plotpoint}}
\put(460,343){\usebox{\plotpoint}}
\put(461,344){\usebox{\plotpoint}}
\put(462,345){\usebox{\plotpoint}}
\put(463,346){\usebox{\plotpoint}}
\put(464,347){\usebox{\plotpoint}}
\put(465,348){\usebox{\plotpoint}}
\put(466,349){\usebox{\plotpoint}}
\put(468,350){\usebox{\plotpoint}}
\put(469,351){\usebox{\plotpoint}}
\put(470,352){\usebox{\plotpoint}}
\put(471,353){\usebox{\plotpoint}}
\put(472,354){\usebox{\plotpoint}}
\put(473,355){\usebox{\plotpoint}}
\put(474,356){\usebox{\plotpoint}}
\put(475,357){\usebox{\plotpoint}}
\put(476,358){\usebox{\plotpoint}}
\put(477,359){\usebox{\plotpoint}}
\put(478,360){\usebox{\plotpoint}}
\put(479,361){\usebox{\plotpoint}}
\put(480,362){\usebox{\plotpoint}}
\put(481,363){\usebox{\plotpoint}}
\put(482,364){\usebox{\plotpoint}}
\put(483,365){\usebox{\plotpoint}}
\put(484,366){\usebox{\plotpoint}}
\put(485,367){\usebox{\plotpoint}}
\put(487,368){\usebox{\plotpoint}}
\put(488,369){\usebox{\plotpoint}}
\put(489,370){\usebox{\plotpoint}}
\put(490,371){\usebox{\plotpoint}}
\put(491,372){\usebox{\plotpoint}}
\put(492,373){\usebox{\plotpoint}}
\put(493,374){\usebox{\plotpoint}}
\put(494,375){\usebox{\plotpoint}}
\put(495,376){\usebox{\plotpoint}}
\put(496,377){\usebox{\plotpoint}}
\put(497,378){\usebox{\plotpoint}}
\put(498,379){\usebox{\plotpoint}}
\put(499,380){\usebox{\plotpoint}}
\put(500,381){\usebox{\plotpoint}}
\put(501,382){\usebox{\plotpoint}}
\put(502,383){\usebox{\plotpoint}}
\put(503,384){\usebox{\plotpoint}}
\put(505,385){\usebox{\plotpoint}}
\put(506,386){\usebox{\plotpoint}}
\put(507,387){\usebox{\plotpoint}}
\put(508,388){\usebox{\plotpoint}}
\put(509,389){\usebox{\plotpoint}}
\put(510,390){\usebox{\plotpoint}}
\put(511,391){\usebox{\plotpoint}}
\put(512,392){\usebox{\plotpoint}}
\put(513,393){\usebox{\plotpoint}}
\put(514,394){\usebox{\plotpoint}}
\put(516,395){\usebox{\plotpoint}}
\put(517,396){\usebox{\plotpoint}}
\put(518,397){\usebox{\plotpoint}}
\put(519,398){\usebox{\plotpoint}}
\put(520,399){\usebox{\plotpoint}}
\put(521,400){\usebox{\plotpoint}}
\put(522,401){\usebox{\plotpoint}}
\put(523,402){\usebox{\plotpoint}}
\put(524,403){\usebox{\plotpoint}}
\put(525,404){\usebox{\plotpoint}}
\put(526,405){\usebox{\plotpoint}}
\put(527,406){\usebox{\plotpoint}}
\put(528,407){\usebox{\plotpoint}}
\put(529,408){\usebox{\plotpoint}}
\put(530,409){\usebox{\plotpoint}}
\put(531,410){\usebox{\plotpoint}}
\put(532,411){\usebox{\plotpoint}}
\put(533,412){\usebox{\plotpoint}}
\put(534,413){\usebox{\plotpoint}}
\put(535,414){\usebox{\plotpoint}}
\put(536,415){\usebox{\plotpoint}}
\put(537,416){\usebox{\plotpoint}}
\put(538,417){\usebox{\plotpoint}}
\put(539,418){\usebox{\plotpoint}}
\put(540,419){\usebox{\plotpoint}}
\put(542,420){\usebox{\plotpoint}}
\put(543,421){\usebox{\plotpoint}}
\put(544,422){\usebox{\plotpoint}}
\put(545,423){\usebox{\plotpoint}}
\put(546,424){\usebox{\plotpoint}}
\put(547,425){\usebox{\plotpoint}}
\put(548,426){\usebox{\plotpoint}}
\put(549,427){\usebox{\plotpoint}}
\put(550,428){\usebox{\plotpoint}}
\put(551,429){\usebox{\plotpoint}}
\put(553,430){\usebox{\plotpoint}}
\put(554,431){\usebox{\plotpoint}}
\put(555,432){\usebox{\plotpoint}}
\put(556,433){\usebox{\plotpoint}}
\put(557,434){\usebox{\plotpoint}}
\put(558,435){\usebox{\plotpoint}}
\put(559,436){\usebox{\plotpoint}}
\put(560,437){\usebox{\plotpoint}}
\put(561,438){\usebox{\plotpoint}}
\put(562,439){\usebox{\plotpoint}}
\put(563,440){\usebox{\plotpoint}}
\put(564,441){\usebox{\plotpoint}}
\put(565,442){\usebox{\plotpoint}}
\put(566,443){\usebox{\plotpoint}}
\put(567,444){\usebox{\plotpoint}}
\put(568,445){\usebox{\plotpoint}}
\put(569,446){\usebox{\plotpoint}}
\put(570,447){\usebox{\plotpoint}}
\put(571,448){\usebox{\plotpoint}}
\put(572,449){\usebox{\plotpoint}}
\put(573,450){\usebox{\plotpoint}}
\put(575,451){\usebox{\plotpoint}}
\put(576,452){\usebox{\plotpoint}}
\put(577,453){\usebox{\plotpoint}}
\put(578,454){\usebox{\plotpoint}}
\put(580,455){\usebox{\plotpoint}}
\put(581,456){\usebox{\plotpoint}}
\put(582,457){\usebox{\plotpoint}}
\put(583,458){\usebox{\plotpoint}}
\put(584,459){\usebox{\plotpoint}}
\put(585,460){\usebox{\plotpoint}}
\put(586,461){\usebox{\plotpoint}}
\put(587,462){\usebox{\plotpoint}}
\put(588,463){\usebox{\plotpoint}}
\put(589,464){\usebox{\plotpoint}}
\put(590,465){\usebox{\plotpoint}}
\put(591,466){\usebox{\plotpoint}}
\put(592,467){\usebox{\plotpoint}}
\put(593,468){\usebox{\plotpoint}}
\put(594,469){\usebox{\plotpoint}}
\put(595,470){\usebox{\plotpoint}}
\put(596,471){\usebox{\plotpoint}}
\put(597,472){\usebox{\plotpoint}}
\put(598,473){\usebox{\plotpoint}}
\put(599,474){\usebox{\plotpoint}}
\put(600,475){\usebox{\plotpoint}}
\put(601,476){\usebox{\plotpoint}}
\put(602,477){\usebox{\plotpoint}}
\put(603,478){\usebox{\plotpoint}}
\put(604,479){\usebox{\plotpoint}}
\put(605,480){\usebox{\plotpoint}}
\put(606,481){\usebox{\plotpoint}}
\put(607,482){\usebox{\plotpoint}}
\put(608,483){\usebox{\plotpoint}}
\put(609,484){\usebox{\plotpoint}}
\put(610,485){\usebox{\plotpoint}}
\put(612,486){\usebox{\plotpoint}}
\put(613,487){\usebox{\plotpoint}}
\put(614,488){\usebox{\plotpoint}}
\put(615,489){\usebox{\plotpoint}}
\put(617,490){\usebox{\plotpoint}}
\put(618,491){\usebox{\plotpoint}}
\put(619,492){\usebox{\plotpoint}}
\put(620,493){\usebox{\plotpoint}}
\put(621,494){\usebox{\plotpoint}}
\put(622,495){\usebox{\plotpoint}}
\put(623,496){\usebox{\plotpoint}}
\put(624,497){\usebox{\plotpoint}}
\put(625,498){\usebox{\plotpoint}}
\sbox{\plotpoint}{\rule[-0.500pt]{1.000pt}{1.000pt}}%
\put(518,158){\rule{.1pt}{.1pt}}
\put(518,164){\rule{.1pt}{.1pt}}
\put(518,170){\rule{.1pt}{.1pt}}
\put(518,176){\rule{.1pt}{.1pt}}
\put(518,183){\rule{.1pt}{.1pt}}
\put(518,189){\rule{.1pt}{.1pt}}
\put(518,195){\rule{.1pt}{.1pt}}
\put(518,201){\rule{.1pt}{.1pt}}
\put(518,207){\rule{.1pt}{.1pt}}
\put(518,213){\rule{.1pt}{.1pt}}
\put(518,219){\rule{.1pt}{.1pt}}
\put(518,225){\rule{.1pt}{.1pt}}
\put(518,232){\rule{.1pt}{.1pt}}
\put(518,238){\rule{.1pt}{.1pt}}
\put(518,244){\rule{.1pt}{.1pt}}
\put(518,250){\rule{.1pt}{.1pt}}
\put(518,256){\rule{.1pt}{.1pt}}
\put(518,262){\rule{.1pt}{.1pt}}
\put(518,268){\rule{.1pt}{.1pt}}
\put(518,275){\rule{.1pt}{.1pt}}
\put(518,281){\rule{.1pt}{.1pt}}
\put(518,287){\rule{.1pt}{.1pt}}
\put(518,293){\rule{.1pt}{.1pt}}
\put(518,299){\rule{.1pt}{.1pt}}
\put(518,305){\rule{.1pt}{.1pt}}
\put(518,311){\rule{.1pt}{.1pt}}
\put(518,317){\rule{.1pt}{.1pt}}
\put(518,324){\rule{.1pt}{.1pt}}
\put(518,330){\rule{.1pt}{.1pt}}
\put(518,336){\rule{.1pt}{.1pt}}
\put(518,342){\rule{.1pt}{.1pt}}
\put(518,348){\rule{.1pt}{.1pt}}
\put(518,354){\rule{.1pt}{.1pt}}
\put(518,360){\rule{.1pt}{.1pt}}
\put(518,367){\rule{.1pt}{.1pt}}
\put(518,373){\rule{.1pt}{.1pt}}
\put(518,379){\rule{.1pt}{.1pt}}
\put(518,385){\rule{.1pt}{.1pt}}
\put(518,391){\rule{.1pt}{.1pt}}
\put(518,397){\rule{.1pt}{.1pt}}
\sbox{\plotpoint}{\rule[-0.250pt]{0.500pt}{0.500pt}}%
\put(566,158){\rule{.1pt}{.1pt}}
\put(566,165){\rule{.1pt}{.1pt}}
\put(566,173){\rule{.1pt}{.1pt}}
\put(566,180){\rule{.1pt}{.1pt}}
\put(566,187){\rule{.1pt}{.1pt}}
\put(566,194){\rule{.1pt}{.1pt}}
\put(566,202){\rule{.1pt}{.1pt}}
\put(566,209){\rule{.1pt}{.1pt}}
\put(566,216){\rule{.1pt}{.1pt}}
\put(566,224){\rule{.1pt}{.1pt}}
\put(566,231){\rule{.1pt}{.1pt}}
\put(566,238){\rule{.1pt}{.1pt}}
\put(566,245){\rule{.1pt}{.1pt}}
\put(566,253){\rule{.1pt}{.1pt}}
\put(566,260){\rule{.1pt}{.1pt}}
\put(566,267){\rule{.1pt}{.1pt}}
\put(566,275){\rule{.1pt}{.1pt}}
\put(566,282){\rule{.1pt}{.1pt}}
\put(566,289){\rule{.1pt}{.1pt}}
\put(566,296){\rule{.1pt}{.1pt}}
\put(566,304){\rule{.1pt}{.1pt}}
\put(566,311){\rule{.1pt}{.1pt}}
\put(566,318){\rule{.1pt}{.1pt}}
\put(566,326){\rule{.1pt}{.1pt}}
\put(566,333){\rule{.1pt}{.1pt}}
\put(566,340){\rule{.1pt}{.1pt}}
\put(566,347){\rule{.1pt}{.1pt}}
\put(566,355){\rule{.1pt}{.1pt}}
\put(566,362){\rule{.1pt}{.1pt}}
\put(566,369){\rule{.1pt}{.1pt}}
\put(566,377){\rule{.1pt}{.1pt}}
\put(566,384){\rule{.1pt}{.1pt}}
\put(566,391){\rule{.1pt}{.1pt}}
\put(566,398){\rule{.1pt}{.1pt}}
\put(566,406){\rule{.1pt}{.1pt}}
\put(566,413){\rule{.1pt}{.1pt}}
\put(566,420){\rule{.1pt}{.1pt}}
\put(566,428){\rule{.1pt}{.1pt}}
\put(566,435){\rule{.1pt}{.1pt}}
\put(566,442){\rule{.1pt}{.1pt}}
\put(432,158){\rule{.1pt}{.1pt}}
\put(432,162){\rule{.1pt}{.1pt}}
\put(432,166){\rule{.1pt}{.1pt}}
\put(432,170){\rule{.1pt}{.1pt}}
\put(432,174){\rule{.1pt}{.1pt}}
\put(432,178){\rule{.1pt}{.1pt}}
\put(432,182){\rule{.1pt}{.1pt}}
\put(432,186){\rule{.1pt}{.1pt}}
\put(432,191){\rule{.1pt}{.1pt}}
\put(432,195){\rule{.1pt}{.1pt}}
\put(432,199){\rule{.1pt}{.1pt}}
\put(432,203){\rule{.1pt}{.1pt}}
\put(432,207){\rule{.1pt}{.1pt}}
\put(432,211){\rule{.1pt}{.1pt}}
\put(432,215){\rule{.1pt}{.1pt}}
\put(432,219){\rule{.1pt}{.1pt}}
\put(432,223){\rule{.1pt}{.1pt}}
\put(432,227){\rule{.1pt}{.1pt}}
\put(432,231){\rule{.1pt}{.1pt}}
\put(432,235){\rule{.1pt}{.1pt}}
\put(432,239){\rule{.1pt}{.1pt}}
\put(432,243){\rule{.1pt}{.1pt}}
\put(432,247){\rule{.1pt}{.1pt}}
\put(432,252){\rule{.1pt}{.1pt}}
\put(432,256){\rule{.1pt}{.1pt}}
\put(432,260){\rule{.1pt}{.1pt}}
\put(432,264){\rule{.1pt}{.1pt}}
\put(432,268){\rule{.1pt}{.1pt}}
\put(432,272){\rule{.1pt}{.1pt}}
\put(432,276){\rule{.1pt}{.1pt}}
\put(432,280){\rule{.1pt}{.1pt}}
\put(432,284){\rule{.1pt}{.1pt}}
\put(432,288){\rule{.1pt}{.1pt}}
\put(432,292){\rule{.1pt}{.1pt}}
\put(432,296){\rule{.1pt}{.1pt}}
\put(432,300){\rule{.1pt}{.1pt}}
\put(432,304){\rule{.1pt}{.1pt}}
\put(432,308){\rule{.1pt}{.1pt}}
\put(432,313){\rule{.1pt}{.1pt}}
\put(432,317){\rule{.1pt}{.1pt}}
\end{picture}

\hfill

 \end{minipage}
 \vspace{-0.8cm}
}
{
\caption{\protect\capsize
Illustration af pitchforkbifurkationen for den logistiske
afbildning for parameterv{\ae}rdien $a=3$. Grafen for den
dobbeltiterede afbildning $f^2_a(x)$ b{\aa}de l{\o}ftes
over og s{\ae}nkes under kurven $y=x$, hvorved to
periodiske punkter genereres. Det tidligere stabile
fikspunkt $x_f$ er nu ustabilt.}
\label{fig:LogGraf}
}

Det skift, der er indtr{\aa}dt i dynamikken for den
logistiske afbildning $f_a(x)$, kaldes en
pitchforkbifurkation (eller gaffelgrensbifurkation). Vi ser
ogs{\aa} af lig\-ning~\ref{eq:Pitch}, at
pitchforkbifurkationen er {\ae}kvivalent med forekomsten af
en periodefordobling i sy\-stemets dynamik.

\vspace{4.0mm}
Hele den argumentation, vi har anvendt p{\aa} $f_a(x)$, kan
nu gentages for den dobbeltiterede afbildning $f_a^2(x)$,
idet der eksisterer et $a_1 \in ]a_0;4]$, s{\aa}ledes at
$f_a^2(x)$ underg{\aa}r en pitchforkbifurkation i $a=a_1$.
De to fikspunkter $x_1$ og $x_2$ bliver alts{\aa} ustabile
under generering af fire punkter, $x_1,\ldots,x_4$, p{\aa}
hvilke $f_{a_1}(x)$ vil v{\ae}re 4-periodisk.


\vspace{4.0mm}
Helt generelt viser det sig, at der i den logistiske
afbildning optr{\ae}der en kaskade af s{\aa}danne
pitchforbifurkationer for de respektive iterater af
$f_a^{2^p}(x)$ for v{\ae}rdierne $a_p$, hvor $p =
0,1,2,\ldots$. F{\o}lgen af bifurkationsv{\ae}rdier
$a_0,a_1,a_2,\ldots$ er endvidere konvergent med
gr{\ae}nsev{\ae}rdien $a_\infty$, svarende til at
``gr{\ae}nse\-cyklu\-ser\-nes'' periode g{\aa}r mod uendelig.

%%%%%%%%%%%%%%%%%%%%%%%%%%%%%%%%%%%%%%%%%%%%%%%%%%%%%%%%%%%%%%%%%%%%%%%%
%% figur
%%
%% beskrivelse : Illustration af konvergens mod
%%               Feigenbaums konstant
%% makroer     : PSTricks
%%%%%%%%%%%%%%%%%%%%%%%%%%%%%%%%%%%%%%%%%%%%%%%%%%%%%%%%%%%%%%%%%%%%%%%%
\begin{center}
 \begin{pspicture}(0,0)(9,2)
%  \psgrid[](0,0)(0,0)(9,2)
  \psline[linewidth=0.8pt,arrowinset=0]{->}(0.0,1.0)(8.0,1.0)
  \psline[linewidth=0.8pt]{-}(0.6,0.9)(0.6,1.1)
  \psline[linewidth=0.8pt]{-}(3.3,0.9)(3.3,1.1)
  \psline[linewidth=0.8pt]{-}(4.6,0.9)(4.6,1.1)
  \psline[linewidth=0.8pt]{-}(5.2,0.9)(5.2,1.1)
  \psline[linewidth=0.8pt]{-}(6.8,0.9)(6.8,1.1)
  \rput[tc]{*0}(0.6,0.7){\footnotesize $a_0$}
  \rput[tc]{*0}(3.3,0.7){\footnotesize $a_1$}
  \rput[tc]{*0}(4.6,0.7){\footnotesize $a_2$}
  \rput[tc]{*0}(5.2,0.7){\footnotesize $a_3$}
  \rput[tc]{*0}(6.8,0.7){\footnotesize $a_{\infty}$}
  \rput[tc]{*0}(6.0,0.7){\footnotesize $\ldots$}
  \rput[cc]{*0}(8.5,1.0){\footnotesize $a$}
 \end{pspicture}
\end{center}

Ydermere og nok s{\aa} vigtigt g{\ae}lder der i gr{\ae}nsen
$p \rightarrow \infty$ f{\o}lgende geometriske forhold

\begin{equation}
 \delta = \lim_{p \rightarrow \infty}
	  \frac{a_p - a_{p-1}}{a_{p+1}-a_p}
\end{equation}

Konstanten $\delta$ er universel og kaldes for 
{\em Feigenbaums konstant}. Ad numerisk vej er denne
bestemt til

\begin{equation}
 \delta = 4.6692016091\ldots
\end{equation}

\subsection{Torusbifurkationen}
\label{sec:TorusBif}
Langt mere kompliceret og stadig ikke fuldt forst{\aa}et er
torusbifurkationen. Som tidligere beskrevet svarer denne
til, at to komplekskonjugerede Floquetmultiplikatorer
$\lambda_{\pm}$ sk{\ae}rer enhedscirklen i det komplekse
plan. Skriver vi $\lambda_{\pm}$ som $e^{\pm i T_0}$ og de
tilh{\o}rende komplekse Floquetegenvektorer som $\Psi_\pm$,
ser vi, at den bev{\ae}gelse, der beskrives af en
perturbation $\epsilon(t)$ indeholdt i det ustabile
Floquetrum ${\rm span}(\Re\Psi_+,\Im\Psi_)$, vil v{\ae}re
lig

\begin{equation}
 \epsilon(t) = \lambda_+ {\bf p}(t) = e^{i T_0t} {\bf p}(t)
 \label{eq:TorusPert}
\end{equation}

hvor den til $\lambda_+$ h{\o}rende Floqueteksponent
$\sigma_+$ alts{\aa} opfylder $\sigma_+ = i T_0$. Endvidere
ser vi, at faktoren $e^{i T_0}$ er periodisk med perioden
$\frac{2\pi}{T_0}$. Pertubationen $\epsilon(t)$ kan nu
opf{\o}re sig p{\aa} to m{\aa}der

\begin{itemize}
  \item Hvis der findes $p,q \in \Z$ s{\aa}
  $T_0 = \frac{2\pi}{T}\frac{p}{q}$, da siges $T$ og $T_0$
  at v{\ae}re rationelt afh{\ae}ngige. I dette tilf{\ae}lde
  er $\epsilon(t)$ alts{\aa} periodisk, da der g{\ae}lder

  \begin{eqnarray}
    \epsilon(t+qT) & = & e^{i T_0t}e^{i T_0qT} {\bf
    p}(t)\nonumber\\ & = & e^{i T_0t}e^{i
    T_0\frac{2\pi}{T_0}p} {\bf p}(t)\nonumber\\ & = & e^{i
    T_0t}e^{i 2\pi p} {\bf p}(t)\nonumber\\ & = & e^{i
    T_0t}{\bf p}(t)\nonumber\\ & = & \epsilon(t)
  \end{eqnarray}

  Bev{\ae}gelsen vil alts{\aa} v{\ae}re en lukket kurve
  p{\aa} en torus, hvorfor vi taler om en {\em resonant}
  eller {\em fasel{\aa}st\/} dynamik.
  %%%%%%%%%%%%%%%%%%%%%%%%%%%%%%%%%%%%%%%%%%%%%%%%%%%
  \item For $T_0 \neq \frac{2\pi}{T}\frac{p}{q}$ for alle
  $p,q \in \Z$ vil banekurven p{\aa} torusen aldrig kunne
  v{\ae}re lukket, og da denne ydermere ikke kan sk{\ae}re
  sig selv, er bev{\ae}gelsen {\em t{\ae}t} p{\aa} torusen.
  I denne situation taler vi om en {\em kvasiperiodisk}
  bev{\ae}gelse.
\end{itemize}

Udfra disse betragtninger er det forholdsvist let at
forestille sig, at en serie af s{\aa}danne
torusbifurkationer vil kunne f{\o}re til en meget kompleks
dynamik i faserummet. Det var pr{\ae}cis s{\aa}danne
betragtninger, der i 1944 ledte den russiske fysiker L.D.
Landau til at foresl{\aa}, at en uendelig r{\ae}kke af
ikke-resonante torusbifurkationer kunne f{\o}re til en
kaotisk opf{\o}rsel af det dynamiske sy\-stem\footnote{I 1944
blev ordet ``kaos'' ikke anvendt som videnskabelig term,
hvorfor Landau benyttede sig af begrebet ``turbulens''.
Senere har det dog vist sig, at disse begreber bliver brugt
synonymt for beskrivelsen af komplekse f{\ae}nomener i
rumlige fysiske/kemiske sy\-stemer} i kraft af indf{\o}rslen
af et uendeligt antal fundamentalfrekvenser
$\omega_1,\ldots,\omega_\infty$ \cite{Landau}.

%%%%%%%%%%%%%%%%%%%%%%%%%%%%%%%%%%%%%%%%%%%%%%%%%%%%%%%%%%%%%%%%%%%%%%%%
%% figur
%%
%% beskrivelse : Illustration af Landaus hypotese
%% makroer     : PSTricks
%%%%%%%%%%%%%%%%%%%%%%%%%%%%%%%%%%%%%%%%%%%%%%%%%%%%%%%%%%%%%%%%%%%%%%%%
\begin{center}
\vspace{-0.5cm}
 \begin{pspicture}(0,0)(14,4)
%  \psgrid[](0,0)(0,0)(14,4)
  \psline[linewidth=0.8pt,arrowinset=0]{->}(0.8,1.0)(12.6,1.0)
  \psline[linewidth=0.8pt]{-}( 2.1,0.9)( 2.1,1.1)
  \psline[linewidth=0.8pt]{-}( 4.8,0.9)( 4.8,1.1)
  \psline[linewidth=0.8pt]{-}( 7.5,0.9)( 7.5,1.1)
  \psline[linewidth=0.8pt]{-}(11.2,0.9)(11.2,1.1)
  \psellipse[linewidth=0.6pt](7.5,2.5)(0.5,0.23)
  \psellipse[linewidth=0.6pt](7.5,2.5)(0.9,0.50)

  \psarc[linewidth=0.6pt](7.5,2.15){0.135}{90}{270}
  \psarc[linewidth=0.6pt,
         linestyle=dotted,
         dotsep=0.6pt](7.5,2.15){0.135}{270}{90}
  \pscircle[linewidth=0.6pt](4.8,2.5){0.5}
  \pscircle*[](2.1,2.5){0.1}
  \rput[tc]{*0}( 2.1,0.7){\footnotesize $a_0$}
  \rput[tc]{*0}( 4.8,0.7){\footnotesize $a_1$}
  \rput[tc]{*0}( 7.5,0.7){\footnotesize $a_2$}
  \rput[tc]{*0}(11.2,0.7){\footnotesize $a_{\infty}$}
  \rput[tc]{*0}(9.35,0.7){\footnotesize $\ldots$}
  \rput[cc]{*0}(11.2,2.5){\tiny $\{\omega_1,\ldots,\omega_\infty \}$}
  \rput[cc]{*0}(9.9,2.5){\tiny $\ldots$}
  \rput[br]{*0}( 4.4,2.8){\tiny $\omega_1$}
  \rput[br]{*0}(6.78,2.8){\tiny $\omega_1$}
  \rput[cc]{*0}(7.2,2.18){\tiny $\omega_2$}
  \rput[cc]{*0}(13.1,1.0){\footnotesize $a$}
  \psline[linewidth=0.6pt,arrowinset=0]{->}(2.9,2.5)(3.63,2.5)
  \psline[linewidth=0.6pt,arrowinset=0]{->}(5.6,2.5)(6.33,2.5)
  \psline[linewidth=0.6pt,arrowinset=0]{->}(8.7,2.5)(9.43,2.5)
 \end{pspicture}
\end{center}
\vspace{-0.6cm}

Umiddelbart virker Landaus hypotese plausibel, da det
dynamiske sy\-stems opf{\o}rsel vil v{\ae}re uhyre
kompliceret under tilstedev{\ae}relse af et stort antal
fundamentalfrekvenser. Eksperimentelt er denne hypotese dog
aldrig blevet bekr{\ae}ftet, da en uendelig kaskade af
torusbifurkationer tydeligt ville vise sig gennem en
Fouriertransformation, idet ingen diskrete linier ville
optr{\ae}de i de resulterende spektra. Dette er eksempelvis
afkr{\ae}ftet i det klassiske Rayleigh-B\'{e}rnard
eksperiment, hvor kaos/turbulens allerede indtr{\ae}der
efter den anden torusbifurkation.

\vspace{4.0mm}
Forklaringen herp{\aa} viser sig at v{\ae}re indeholdt i
Ruelle-Newhouse-Takens teo\-remet \cite{RTN}, der stringent
forklarer hvorledes kaos kan v{\ae}re forbundet med
torus\-bifurkationen. Dette teorem udsiger, at i faserummet
for et dynamisk sy\-stem med $\dim > 3$, hvis dynamik er
beskrevet p{\aa} en torus gene\-reret efter to successive
torusbifurkationer, vil der med positiv sandsynlighed
eksistere en stabil kaotisk tiltr{\ae}kker. Mere l{\o}st betyder
dette, at et sy\-stem, der underg{\aa}r to successive
torusbifurkationer, efter alt at d{\o}mme vil havne p{\aa}
en kaotisk tiltr{\ae}kker.

%%%%%%%%%%%%%%%%%%%%%%%%%%%%%%%%%%%%%%%%%%%%%%%%%%%%%%%%%%%%%%%%%%%%%%%%
%% figur
%%
%% beskrivelse : Illustration af Ruelle-Takens-Newhouse teormet
%% makroer     : PSTricks
%%%%%%%%%%%%%%%%%%%%%%%%%%%%%%%%%%%%%%%%%%%%%%%%%%%%%%%%%%%%%%%%%%%%%%%%
\begin{center}
 \begin{pspicture}(0,0)(14,4)
%  \psgrid[](0,0)(0,0)(14,4)
  \psline[linewidth=0.8pt,arrowinset=0]{->}(0.8,1.0)(12.6,1.0)
  \psline[linewidth=0.8pt]{-}( 2.1,0.9)( 2.1,1.1)
  \psline[linewidth=0.8pt]{-}( 4.8,0.9)( 4.8,1.1)
  \psline[linewidth=0.8pt]{-}( 7.5,0.9)( 7.5,1.1)
  \psline[linewidth=0.8pt]{-}(11.2,0.9)(11.2,1.1)
  \psellipse[linewidth=0.6pt](7.5,2.5)(0.5,0.23)
  \psellipse[linewidth=0.6pt](7.5,2.5)(0.9,0.50)

  \psarc[linewidth=0.6pt](7.5,2.15){0.135}{90}{270}
  \psarc[linewidth=0.6pt,
         linestyle=dotted,
         dotsep=0.6pt](7.5,2.15){0.135}{270}{90}
  \pscircle[linewidth=0.6pt](4.8,2.5){0.5}
  \pscircle*[](2.1,2.5){0.1}
  \rput[tc]{*0}( 2.1,0.7){\footnotesize $a_0$}
  \rput[tc]{*0}( 4.8,0.7){\footnotesize $a_1$}
  \rput[tc]{*0}( 7.5,0.7){\footnotesize $a_2$}
  \rput[tc]{*0}(11.2,0.7){\footnotesize $a_{\rm crit}$}
  \rput[cc]{*0}(13.1,1.0){\footnotesize $a$}
  \rput[cc]{*0}(11.55,2.5){
    \tiny$\left\{\mbox{\begin{minipage}{3.5cm}
     {$3T$-torus med frekven-\\
      ser $\omega_1$, $\omega_2$ og $\omega_3$ vil\\
      med positiv sandsynlighed\\
      henfalde til ``strange\\
      attractor'' ved en infinitesi-\\
      mal perturbation.}
      \end{minipage}
      }\right.$\normalsize}
  \rput[br]{*0}( 4.4,2.8){\tiny $\omega_1$}
  \rput[br]{*0}(6.78,2.8){\tiny $\omega_1$}
  \rput[cc]{*0}(7.2,2.18){\tiny $\omega_2$}
  \psline[linewidth=0.6pt,arrowinset=0]{->}(2.9,2.5)(3.63,2.5)
  \psline[linewidth=0.6pt,arrowinset=0]{->}(5.6,2.5)(6.33,2.5)
  \psline[linewidth=0.6pt,arrowinset=0]{->}(8.7,2.5)(9.43,2.5)
 \end{pspicture}
\end{center}

For en mere uddybende diskussion af torusbifurkationen og
dennes sammen\-h{\ae}ng med kaos se
f.eks.\ \cite[s.\ 94-98]{Marek2} eller
\cite[s.\ 147-149]{Schuster}.

\vspace{4.0mm}
Efter i en vis forstand temmelig teoretiske diskussion
{\o}nsker vi nu at diskutere et eksempel i form af en model
for en kemisk reaktion. Det er vores h{\aa}b, at dette
eksempel viser, hvor n{\o}dvendig teoretiske overvejelser
er for p{\aa} tilfredstillende vis at kunne forst{\aa} de
ikke-line{\ae}re egenskaber, der optr{\ae}der i komplekse
reaktionssy\-stemer.

\section{Et model eksempel -- {\em Brusselatoren}}
Vi {\o}nsker nu at pr{\ae}sentere en kemisk model samt en
modificering af denne, der udviser en r{\ae}kke af de
f{\ae}nomener, som blev pr{\ae}senteret i forrige af\-snit.
I 1968 formulerede den belgiske kemiker og senere
Nobelprismodtager Ilya Prigogine i samarbejde med R.\
Lefever en meget simplificeret model for et kemisk
reaktionssy\-stem kaldet {\em Brusselatoren}
\cite{Prig1,Prig2}. Denne model har sidenhen v{\ae}ret
anvendt i en lang r{\ae}kke sammenh{\ae}nge p{\aa} trods af
det faktum, at det s{\ae}t elementar\-reaktioner, der
udg{\o}r modellens skelet, er urealistisk i en fysisk eller
kemisk sammenh{\ae}ng.

\vspace{4.0mm}
Ikke desto mindre besidder Brusselatoren sin egen styrke,
idet den kan ana\-lyseres eksakt og hermed kan bidrage til
en st{\o}rre og dybere indsigt i en r{\ae}kke af de
egenskaber, der udvises af mere komplicerede og realistiske
modeller s{\aa}vel som i virkelige kemiske sy\-stemer.

\vspace{4.0mm}
Brusselatoren er $2$-dimensional og beskriver omdannelsen
af to reaktanter $A$ og $B$ til produkterne $D$ og $E$.
Omdannelsen finder sted via fire elementar\-reaktioner, der
involverer to intermediater $X$ og $Y$\footnote{I dette
tilf{\ae}lde skelner vi ikke mellem en kemisk forbindelse
$X_i$ eller dennes koncentration [$X_i$], hvorfor vi i
begge tilf{\ae}lde benytter notationen $X_i$.}

\begin{subequations}
 \begin{eqalignno}
  A      &\stackrel{k_1}{\lr} X     \\
  B + X  &\stackrel{k_2}{\lr} Y + D \\
  2X + Y &\stackrel{k_3}{\lr} 3X \label{brus-auto} \\
  X      &\stackrel{k_4}{\lr} E
 \end{eqalignno}
 \label{eq:brus}
\end{subequations}

Reaktion~\ref{brus-auto} er autokatalytisk (to $X$
molekyler omdannes i samme reaktion til tre $X$ molekyler).
I Brusselatoren er tilstedev{\ae}relsen af et
autokatalytisk trin n{\o}dvendigt, for at modellen kan
udvise oscillationer. Dette beh{\o}ver dog ikke at v{\ae}re
sandt i det generelle tilf{\ae}lde, idet forekomsten af
oscillationer ogs{\aa} kan v{\ae}re bestemt af andre
egenskaber ved den p{\aa}g{\ae}ld\-ende model. Antager vi
nu, at $A$ og $B$ er konstante, samt at
hastighedskonstanterne tilfredsstiller $k_1=k_2=k_3=k_4=1$
(svarende til at sy\-stemet er gjort dimensionsl{\o}st),
vil de kinetiske lig\-ninger for Brusselatoren tage formen

\begin{equation}
 \begin{array}{lcl}
  \dot{X} & = & A - (B+1)X + X^2Y \\
  \dot{Y} & = & BX - X^2Y
 \end{array}
 \label{eq:bruseq}
\end{equation}

I det f{\o}lgende vil vi benytte konstanten $B$ som
bifurkationsparameter med det m{\aa}l for {\o}je at
bestemme den v{\ae}rdi $B_c$ af $B$, for hvilken der
optr{\ae}der en Hopfbifurkation i Brusselatoren. Det er
forholdvist simpelt at indse, at punktet
$(X_f,Y_f)=(A,\frac{B}{A})$ udg{\o}r Brusselatorens eneste
station{\ae}re punkt. Udregner vi nu Jacobimatricen ${\bf
J}$ for lig\-ning~\ref{eq:bruseq} i det station{\ae}re
punkt $(A,\frac{B}{A})$, f{\aa}s

\begin{equation}
 {\bf J}_{(A,\frac{B}{A})} =
 \left[
 \begin{array}{cr}
  -(B+1)+2XY &  X^2 \\
      B-2XY  & -X^2
 \end{array}
 \right]
 =
 \left[
 \begin{array}{cr}
  B-1 &  A^2 \\
  -B  & -A^2
 \end{array}
 \right]
\end{equation}

Idet vi nu bygger p{\aa}, at egenv{\ae}rdierne
$\lambda_{1,2}$ for en $2 \times 2$ matrix, ${\bf A}$, kan
udtrykkes som $\lambda_{1,2} = \frac{1}{2}(\Tr{\bf A} \pm
\sqrt{\Tr^2{\bf A} -4\det{\bf A}})$, ser vi, at betingelsen
$\Tr {\bf A} = 0$ under foruds{\ae}tningen $\det {\bf A}>0$
tvinger $\lambda_{1,2}$ til at v{\ae}re rent imagin{\ae}re.
Da $\Tr {\bf J}_{(A,\frac{B}{A})} = 0 \Rightarrow B_c =
1+A^2$, ser vi, at egenv{\ae}rdierne for denne v{\ae}rdi af
$B$ tilfredstiller

\begin{equation}
 \lambda_{1,2}(B_c) = \pm i A
\end{equation}

%%%%%%%%%%%%%%%%%%%%%%%%%%%%%%%%%%%%%%%%%%%%%%%%%%%%%%%%%%%%%%%%%%%%%%%%
%% figur
%%
%% beskrivelse : tre faseprotr{\ae}tter for Brusselator
%% plt         : fig27.plt
%% dat         : fig27a.dat, fig27b.dat, fig27c.dat
%% tex         : fig27a.ps, fig27b.ps, fig27c.ps
%% type        : PSTricks, EPSF
%%%%%%%%%%%%%%%%%%%%%%%%%%%%%%%%%%%%%%%%%%%%%%%%%%%%%%%%%%%%%%%%%%%%%%%%
\renewcommand{\capfont}{\bf}
\begin{figure}[tbp]
\begin{center}
  \begin{pspicture}(0,0)(14,5)
%   \psgrid[](0,0)(0,0)(14,5)
   \rput[bl]{*0}(0.10,0.3){%
                           \epsfxsize= 4.57cm 
                           \epsfysize= 4.30cm 
                           \epsffile{fig57a.ps}}
   \rput[bl]{*0}(4.60,0.3){%
                           \epsfxsize= 4.57cm 
                           \epsfysize= 4.30cm 
                           \epsffile{fig57b.ps}}
   \rput[bl]{*0}(9.10,0.3){%
                           \epsfxsize= 4.57cm 
                           \epsfysize= 4.30cm 
                           \epsffile{fig57c.ps}}
  \rput[bl]{*0}( 4.0,4.0){\footnotesize a)}
  \rput[bl]{*0}( 8.5,4.0){\footnotesize b)}
  \rput[bl]{*0}(13.0,4.0){\footnotesize c)}
  \rput[bc]{*0}( 2.5,4.8){\footnotesize $B < B_c$}
  \rput[bc]{*0}( 7.0,4.8){\footnotesize $B > B_c$}
  \rput[bc]{*0}(11.5,4.8){\footnotesize $B \gg B_c$}
  \end{pspicture}
\end{center}
\vspace{-1.0cm}
\caption{\protect\capsize
	 Eksempler p{\aa} faseportr{\ae}tter for Brusselatoren
	 for tre kvalitativt forskellige valg af 
	 parameteren $B$. a) $B<B_c$ b) $B>B_c$ og c) $B \gg B_c$.
	}
\label{fig:BrusEks}
\end{figure}
\renewcommand{\capfont}{\rm}

Da der ydermere g{\ae}lder 

\begin{equation}
 \left.\frac{d\Re\lambda_{1,2}}{dB}\right|_{B_c} =
 \half (\Tr {'} \pm (\Tr{^2}-4\det)^{-\frac{1}{2}})(\Tr{'}\Tr) =
 \half \neq 0
\end{equation}

slutter vi alts{\aa}, at Brusselatoren underg{\aa}r en
Hopfbifurkation for $B=B_c=1+A^2$. For $B<B_c$ vil
sy\-stemets dynamik svare til en spiralerende
bev{\ae}gelse ind mod det stabile station{\ae}re punkt. For
$B>B_c$ er det station{\ae}re punkt nu ustabilt og en
stabil gr{\ae}nsecyklus $\gamma$ er dannet i faserummet.
For v{\ae}rdier af $B$ t{\ae}t p{\aa} $B_c$ er
gr{\ae}nsecyklusens svingninger sinusformede med amplitude
${\cal A}$ og frekvens $\omega$ givet som

\begin{subequations}
 \begin{eqalignno}
  {\cal A}(B) &= \sqrt{B-B_c}  \\
  \omega(B)   &= A
 \end{eqalignno}
\end{subequations}

For $B \gg B_c$ udviser Brusselatoren mere komplicerede
ikke-line{\ae}re svingninger, der dog stadig er simple i en
vis forstand, da disse er $1$-periodiske for samtlige valg
af $B>B_c$. Faseportr{\ae}tter for de tre situationer
$B<B_c$, $B>B_c$ og $B \gg B_c$ er vist i
figur~\ref{fig:BrusEks}.

\vspace{4.0mm}
I den oprindelige udgave af Brusselatoren
(lig\-ning~\ref{eq:brus}) optr{\ae}der der ingen komplekse
f{\ae}nomener i form af periodefordoblings\-bifurkationer,
to\-rus\-svingninger eller kaos. Det er dog ikke s{\ae}rlig
sv{\ae}rt at udvide Brusselatoren til en
h{\o}jere-dimensional model, i hvilken alle disse
f{\ae}nomener er tilstede. For at illustrere dette
pr{\ae}senterer vi nu et uddrag af en r{\ae}kke resultater,
s{\aa}ledes som disse er behandlet i \cite{Marek2}.

\vspace{4.0mm}
Lad os betragte et sy\-stem af to koblede Brusselatorer,
svarende til en model\-lering af den eksperimentelle
situation, hvor to CSTR-kamre er forbundet s{\aa}ledes, at
disse indbyrdes kan udveksle stof ved diffusion (se
fig~\ref{fig:KobBrus}). Lader vi nu $x_1$, $y_1$ og $x_2$,
$y_2$ v{\ae}re koncentrationerne af stofferne $X$ og $Y$ i
kammer~1 henholdsvis kammer~2 og benyttes betegnelsen $D_1$
og $D_2$ for intensiteten af den diffusionsbestemte
masseudveksling af $X$ og $Y$, finder vi, at det koblede
sy\-stems kinetik kan beskrives ved f{\o}lgende
4-dimensionale differentiallig\-ning

\boxfigure{t}{\textwidth}
{
 \vspace{0.5cm} 
 \small
  \unitlength=0.8mm
\linethickness{0.8pt}
\begin{picture}(130.00,83.00)(-12.5,0)
\put(10.00,30.00){\framebox(40.00,40.00)[cc]{reaktion 1}}
\put(90.00,30.00){\framebox(40.00,40.00)[cc]{reaktion 2}}
\put(70.00,50.00){\vector(-1,0){18.00}}
\put(70.00,50.00){\vector(1,0){18.00}}
\put(30.00,80.00){\vector(0,-1){8.00}}
\put(110.00,80.00){\vector(0,-1){8.00}}
\put(30.00,28.00){\vector(0,-1){8.00}}
\put(110.00,28.00){\vector(0,-1){8.00}}
\put(30.00,83.00){\makebox(0,0)[cc]{ind}}
\put(110.00,83.00){\makebox(0,0)[cc]{ind}}
\put(30.00,17.00){\makebox(0,0)[cc]{ud}}
\put(110.00,17.00){\makebox(0,0)[cc]{ud}}
\put(70.00,53.00){\makebox(0,0)[cc]{masse}}
\put(70.00,47.00){\makebox(0,0)[cc]{udveksling}}
\end{picture}

 \vspace{-0.8cm}
}
{
\caption{\protect\capsize
	 Skematisk illustration af den situation, hvor to
	 Brusselatorer t{\ae}nkes indbyrdes koblet gennem 
	 masseudveksling via diffusion.}
\label{fig:KobBrus}
}

\begin{equation}
 \begin{array}{lcll}
  \dot{x_1} & = & A - (B+1)x_1 + x_1^2y_1 & + D_1(x_2-x_1)\\
  \dot{y_1} & = & Bx_1 - x_1^2y_1         & + D_2(y_2-y_1)\\
  \dot{x_2} & = & A - (B+1)x_2 + x_2^2y_2 & + D_1(x_1-x_2)\\
  \dot{y_2} & = & Bx_2 - x_2^2y_2         & + D_2(y_1-y_2)\\
 \end{array}
 \label{eq:KobletBrusEq}
\end{equation}

Der kan foretages en lang r{\ae}kke matematiske
overvejelser vedr{\o}rende den symmetri, der er indlejret i
lig\-ning~\ref{eq:KobletBrusEq}. For eksempel er denne
invariant under transformationen ${\bf S}$

\begin{equation}
 {\bf S} =
 \left[
 \begin{array}{llll}
  0 & 0 & 1 & 0\\
  0 & 0 & 0 & 1\\
  0 & 1 & 0 & 0\\
  1 & 0 & 0 & 0
 \end{array}
 \right]
\end{equation}

S{\aa}danne symmetriovervejelser er generelt af stor
betydning under beregnings- og fortolkningsfasen af
bifurkations- og l{\o}sningsdiagrammer for s{\ae}dvanlige
differentiallig\-ninger, idet disse tillader en v{\ae}sentlig
simplificering af det p{\aa}\-g{\ae}ld\-ende problem. Vi
vil dog ikke komme n{\ae}rmere ind p{\aa} dette omr{\aa}de
her, men henviser istedet til \cite{Marek2} for en mere
uddybende diskussion.

\begin{table}[tbp]
 \renewcommand{\capfont}{\bf}
 \begin{minipage}{5cm}
 \begin{center}
  \begin{tabular}{|c|c|c|}                  \hline\hline
   $i$ & $D_1^i$       & $\delta_i$ \\ \hline
   0   & $1.172 000 0$ & $5.882$    \\ \hline
   1   & $1.189 399 3$ & $5.882$    \\ \hline
   2   & $1.192 388 1$ & $4.347$    \\ \hline
   3   & $1.193 075 6$ & $4.699$    \\ \hline
   4   & $1.193 221 9$ &            \\ \hline\hline
  \end{tabular}
 \end{center} 
 \end{minipage}
 \ \hfill \
 \begin{minipage}{8cm}
 \caption{\protect\capsize
   Tabellen viser v{\ae}rdier $D_1^i$ for
   diffusions\-konstanten $D_1$, for hvilke den koblede
   Brusselator underg{\aa}r en periodefordobling. Endvidere
   er angivet et numerisk estimat af forholdet $\delta_i =
   \frac{D_1^{i+1}-D_1^{i}}{D_1^{i+2}-D_1^{i+1}}$, der ses
   at konvergere mod Feigenbaums konstant
   $\delta=4.6692016091\ldots$. De angivede data stammer
   fra \protect\cite{Marek2}. }
 \label{tab:PerdobData}
\end{minipage}
\end{table} 
\renewcommand{\capfont}{\rm}

\vspace{4.0mm}
I stedet {\o}nsker vi at diskutere de omr{\aa}der i
parameterrummet $(B,D_1)$, for hvilke den koblede
Brusselator model giver anledning til periodefordoblings-
og torusbifurkationer. I det f{\o}lgende s{\ae}tter vi
$A=2$ og $\frac{D_1}{D_2}=0.1$ og unders{\o}ger
egenskaberne ved lig\-ning~\ref{eq:KobletBrusEq} under
variation af $B$ og $D_1$ ($B,D_1 > 0$). S{\ae}tter vi
$B=5.9$, finder man udfra numeriske beregninger, at en
superkritisk Hopfbifurkation forekommer i
lig\-ning~\ref{eq:KobletBrusEq} for $D_1=0.0904$. Den
herved opst{\aa}ede gr{\ae}nsecyklus genneml{\o}ber
derp{\aa} en kaskade af periodefordoblinger for en
r{\ae}kke successive for{\o}gelser af $D_1$ som illustreret
i tabel~\ref{tab:PerdobData}.

\vspace{4.0mm}
At de angivne v{\ae}rdier for $D_1$ faktisk
repr{\ae}senterer en serie af periodefordoblinger er
illustreret i figur~\ref{fig:Perdob}, hvor en r{\ae}kke
numeriske integrationer af lig\-ning~\ref{eq:KobletBrusEq}
er vist svarende til en r{\ae}kke valg af parameteren
$D_1$, s{\aa}ledes at $D_1 \in [D_1^{i};D_1^{i+1}]$. Som
det v{\ae}sentligste resultat ser vi dog, at det numerisk
bestemte forhold

\begin{equation}
 \delta_i = \frac{D_1^{i+1}-D_1^{i}}{D_1^{i+2}-D_1^{i+1}}
\end{equation}

for $i=3$ p{\aa} nydeligste vis er i overensstemmelse med
den v{\ae}rdi af $\delta$, der er bestemt for den
logistiske afbildning. Til sidst bem{\ae}rkes, at for
$D_1>1.194>D_1^4$ vil den koblede Brusselator udvise
kaotisk opf{\o}rsel.

%%%%%%%%%%%%%%%%%%%%%%%%%%%%%%%%%%%%%%%%%%%%%%%%%%%%%%%%%%%%%%%%%%%%%%%%
%% figur
%%
%% beskrivelse : Periodefordobling i koblet Brusselator:
%%               T, 2T, 4T, 8T
%% plt         : fig26.plt
%% dat         : fig26a.dat, fig26b.dat, fig26c.dat, fig26d.dat
%% tex         : fig26a.tex, fig26b.tex, fig26c.tex, fig26d.tex
%% makroer     : TeXDraw, PSTricks
%%%%%%%%%%%%%%%%%%%%%%%%%%%%%%%%%%%%%%%%%%%%%%%%%%%%%%%%%%%%%%%%%%%%%%%%
\boxfigure{t}{\textwidth}
{
\begin{center}
 \begin{pspicture}(0,-0.4)(14,10)
%  \psgrid[](0,0)(0,0)(14,10)
  \rput[tl]{*0}( 1.2,9.2){% GNUPLOT: LaTeX using TEXDRAW macros
\begin{texdraw}
\normalsize
\ifx\pathDEFINED\relax\else\let\pathDEFINED\relax
 \def\QtGfr{\ifx (\TGre \let\YhetT\cpath\else\let\YhetT\relax\fi\YhetT}
 \def\path (#1 #2){\move (#1 #2)\futurelet\TGre\QtGfr}
 \def\cpath (#1 #2){\lvec (#1 #2)\futurelet\TGre\QtGfr}
\fi
\drawdim pt
\setunitscale 0.24
\linewd 3
\textref h:L v:C
\path (132 51)(795 51)
\path (132 51)(132 489)
\linewd 4
\path (132 105)(147 105)
\path (795 105)(780 105)
\move (115 105)\textref h:R v:C \htext{{\footnotesize$1$}}
\path (132 160)(147 160)
\path (795 160)(780 160)
\move (115 160)\htext{{\footnotesize$2$}}
\path (132 215)(147 215)
\path (795 215)(780 215)
\move (115 215)\htext{{\footnotesize$3$}}
\path (132 270)(147 270)
\path (795 270)(780 270)
\move (115 270)\htext{{\footnotesize$4$}}
\path (132 324)(147 324)
\path (795 324)(780 324)
\move (115 324)\htext{{\footnotesize$5$}}
\path (132 379)(147 379)
\path (795 379)(780 379)
\move (115 379)\htext{{\footnotesize$6$}}
\path (132 434)(147 434)
\path (795 434)(780 434)
\move (115 434)\htext{{\footnotesize$7$}}
\path (198 51)(198 66)
\path (198 489)(198 474)
\move (198 17)\textref h:C v:C \htext{}
\path (264 51)(264 66)
\path (264 489)(264 474)
\move (264 17)\htext{}
\path (330 51)(330 66)
\path (330 489)(330 474)
\move (330 17)\htext{}
\path (397 51)(397 66)
\path (397 489)(397 474)
\move (397 17)\htext{}
\path (463 51)(463 66)
\path (463 489)(463 474)
\move (463 17)\htext{}
\path (530 51)(530 66)
\path (530 489)(530 474)
\move (530 17)\htext{}
\path (596 51)(596 66)
\path (596 489)(596 474)
\move (596 17)\htext{}
\path (663 51)(663 66)
\path (663 489)(663 474)
\move (663 17)\htext{}
\path (729 51)(729 66)
\path (729 489)(729 474)
\move (729 17)\htext{}
\path (132 51)(795 51)(795 489)(132 489)(132 51)
\linewd 3
\path (132 135)(132 135)(132 134)(132 133)(132 133)(133 132)
\cpath (133 132)(134 133)(135 135)(135 138)(135 138)
\cpath (135 145)(135 145)(136 157)(137 180)(138 226)
\cpath (138 301)(139 363)(140 378)(140 378)(141 369)
\cpath (141 369)(141 352)(141 352)(141 334)(142 316)
\cpath (143 300)(144 284)(144 270)(145 258)(146 246)
\cpath (146 235)(147 225)(147 216)(148 207)(149 200)
\cpath (150 192)(150 186)(151 179)(152 173)(152 167)
\cpath (153 162)(153 157)(154 153)(155 148)(156 144)
\cpath (156 141)(157 138)(158 136)(158 134)(159 133)
\cpath (159 133)(159 132)(159 132)(160 133)(161 136)
\cpath (162 141)(162 141)(162 146)(162 150)(163 165)
\cpath (163 196)(164 255)(165 333)(165 374)(165 378)
\cpath (166 376)(167 369)(167 363)(168 347)(168 345)
\cpath (168 326)(169 309)(169 293)(170 278)(171 265)
\cpath (171 252)(172 241)(173 231)(174 222)(174 213)
\cpath (175 204)(175 197)(176 189)(177 183)(177 177)
\cpath (178 171)(179 165)(180 160)(180 155)(180 151)
\cpath (181 147)(182 143)(183 140)(183 138)(184 135)
\cpath (185 134)(185 134)(186 132)(186 132)(186 133)
\cpath (186 134)(187 138)(187 138)(188 144)(188 144)
\cpath (189 155)(189 176)(190 216)(191 288)(192 357)
\cpath (192 378)(192 378)(192 372)(192 372)(193 355)
\cpath (193 355)(194 337)(195 319)(195 302)(196 287)
\cpath (197 273)(198 260)(198 248)(198 237)(199 227)
\cpath (200 218)(201 209)(201 201)(202 194)(203 186)
\cpath (204 180)(204 174)(204 168)(205 163)(206 158)
\cpath (207 153)(207 149)(208 145)(209 142)(210 139)
\cpath (210 136)(210 135)(211 133)(211 133)(212 132)
\cpath (212 132)(213 133)(213 135)(214 140)(214 140)
\cpath (214 144)(215 147)(216 162)(216 189)(216 243)
\cpath (217 321)(218 371)(219 378)(219 378)(219 372)
\cpath (219 366)(219 366)(220 350)(220 348)(221 329)
\cpath (221 312)(222 295)(222 281)(223 267)(224 255)
\cpath (225 243)(225 233)(226 223)(227 214)(227 206)
\cpath (228 198)(228 191)(229 183)(230 177)(231 171)
\cpath (231 166)(232 161)(233 156)(233 151)(234 147)
\cpath (234 144)(235 141)(236 138)(237 135)(237 134)
\cpath (237 134)(238 133)(238 132)(238 132)(239 134)
\cpath (240 137)(240 142)(240 142)(241 153)(242 171)
\cpath (243 207)(243 274)(244 348)(244 377)(245 378)
\cpath (245 374)(245 374)(246 362)(246 358)(246 358)
\cpath (246 340)(247 321)(248 305)(249 289)(249 275)
\cpath (250 261)(250 249)(251 239)(252 228)(252 219)
\cpath (253 210)(254 202)(255 195)(255 188)(255 181)
\cpath (256 175)(257 169)(258 164)(258 159)(259 154)
\cpath (260 150)(261 146)(261 142)(261 139)(262 137)
\cpath (263 135)(264 133)(264 133)(264 132)(264 132)
\cpath (265 133)(266 135)(267 138)(267 138)(267 144)
\cpath (267 146)(267 159)(268 183)(269 231)(270 308)
\cpath (270 366)(271 378)(271 378)(272 368)(272 368)
\cpath (272 354)(273 351)(273 333)(273 315)(274 298)
\cpath (275 283)(276 269)(276 257)(277 245)(278 234)
\cpath (279 225)(279 216)(279 207)(280 199)(281 192)
\cpath (282 185)(282 178)(283 172)(284 167)(285 162)
\cpath (285 156)(285 152)(286 148)(287 144)(288 141)
\cpath (288 138)(289 136)(290 134)(291 133)(291 133)
\cpath (291 132)(291 132)(291 134)(292 136)(293 141)
\cpath (293 141)(294 146)(294 150)(294 167)(295 200)
\cpath (296 261)(296 339)(297 375)(297 378)(297 375)
\cpath (298 365)(298 361)(299 346)(299 343)(300 324)
\cpath (300 307)(301 291)(302 277)(302 264)(303 252)
\cpath (303 240)(304 230)(305 221)(306 212)(306 204)
\cpath (307 196)(308 189)(308 182)(309 176)(309 170)
\cpath (310 165)(311 159)(312 155)(312 150)(313 147)
\cpath (313 143)(314 140)(315 137)(315 135)(316 133)
\cpath (316 133)(317 132)(317 132)(318 133)(318 135)
\cpath (319 138)(319 138)(319 144)(319 144)(320 156)
\cpath (321 178)(321 221)(322 294)(323 360)(324 378)
\cpath (324 378)(324 371)(324 371)(325 354)(325 354)
\cpath (325 336)(326 318)(327 300)(327 285)(328 271)
\cpath (329 258)(330 247)(330 236)(330 226)(331 217)
\cpath (332 208)(333 201)(333 193)(334 186)(335 180)
\cpath (336 174)(336 168)(336 162)(337 157)(338 153)
\cpath (339 149)(339 145)(340 141)(341 138)(342 136)
\cpath (342 135)(342 133)(342 133)(343 132)(343 132)
\cpath (344 133)(345 135)(345 140)(345 140)(346 145)
\cpath (346 148)(347 164)(348 192)(348 249)(348 327)
\cpath (349 372)(350 378)(350 377)(351 369)(351 364)
\cpath (351 348)(351 346)(352 327)(353 310)(354 294)
\cpath (354 279)(354 266)(355 254)(356 243)(357 232)
\cpath (357 222)(358 213)(359 205)(360 197)(360 190)
\cpath (360 183)(361 177)(362 171)(363 165)(363 160)
\cpath (364 156)(365 151)(366 147)(366 144)(366 140)
\cpath (367 138)(368 135)(369 134)(369 134)(369 132)
\cpath (369 132)(370 133)(371 134)(371 137)(371 137)
\cpath (372 143)(372 143)(372 153)(373 174)(374 212)
\cpath (375 281)(375 352)(376 378)(376 378)(377 373)
\cpath (377 373)(377 360)(377 357)(377 357)(378 339)
\cpath (378 321)(379 303)(380 288)(381 273)(381 261)
\cpath (382 249)(383 238)(383 228)(384 219)(384 210)
\cpath (385 201)(386 194)(387 187)(387 180)(388 174)
\cpath (388 168)(389 163)(390 159)(390 153)(391 150)
\cpath (392 146)(393 142)(393 139)(394 137)(394 135)
\cpath (395 133)(395 133)(396 132)(396 132)(396 133)
\cpath (397 135)(398 139)(398 139)(399 144)(399 147)
\cpath (399 160)(400 186)(400 237)(401 315)(402 369)
\cpath (402 378)(402 378)(403 373)(403 367)(403 367)
\cpath (404 351)(404 349)(405 330)(405 313)(406 297)
\cpath (406 282)(407 268)(408 255)(408 244)(409 234)
\cpath (410 224)(411 215)(411 207)(411 198)(412 191)
\cpath (413 184)(414 178)(414 172)(415 166)(416 161)
\cpath (417 156)(417 152)(417 147)(418 144)(419 141)
\cpath (420 138)(420 135)(421 134)(421 134)(422 133)
\cpath (423 132)(423 132)(423 134)(423 136)(424 141)
\cpath (424 141)(425 151)(426 169)(426 204)(427 267)
\cpath (428 343)(429 376)(429 378)(429 375)(429 375)
\cpath (429 363)(429 360)(429 360)(430 342)(431 323)
\cpath (432 306)(432 291)(433 276)(434 263)(435 251)
\cpath (435 240)(435 229)(436 220)(437 211)(438 203)
\cpath (438 195)(439 188)(440 182)(441 175)(441 170)
\cpath (441 164)(442 159)(443 154)(444 150)(444 146)
\cpath (445 143)(446 140)(446 137)(447 135)(447 133)
\cpath (447 133)(448 132)(448 132)(449 133)(450 135)
\cpath (450 138)(450 138)(451 145)(451 145)(452 157)
\cpath (452 180)(453 226)(453 302)(454 363)(455 378)
\cpath (455 378)(456 369)(456 369)(456 352)(456 352)
\cpath (457 333)(458 316)(458 299)(459 284)(459 270)
\cpath (460 258)(461 246)(462 235)(462 225)(463 216)
\cpath (463 207)(464 200)(465 192)(465 186)(466 179)
\cpath (467 173)(468 167)(468 162)(469 157)(469 153)
\cpath (470 148)(471 144)(471 141)(472 138)(473 136)
\cpath (474 134)(474 133)(474 133)(475 132)(475 132)
\cpath (475 133)(476 136)(477 141)(477 141)(477 146)
\cpath (477 150)(478 165)(479 196)(480 255)(480 333)
\cpath (481 374)(481 378)(481 376)(482 369)(482 363)
\cpath (483 347)(483 345)(483 326)(484 309)(485 293)
\cpath (486 278)(486 265)(486 252)(487 241)(488 231)
\cpath (489 222)(489 213)(490 204)(491 197)(492 189)
\cpath (492 183)(492 177)(493 171)(494 165)(495 160)
\cpath (495 155)(496 151)(497 147)(498 143)(498 140)
\cpath (498 138)(499 135)(500 134)(500 134)(501 132)
\cpath (501 132)(501 133)(502 134)(503 138)(503 138)
\cpath (504 144)(504 144)(504 155)(504 176)(505 216)
\cpath (506 288)(507 357)(507 378)(507 378)(508 372)
\cpath (508 372)(509 355)(509 355)(510 337)(510 318)
\cpath (510 302)(511 287)(512 273)(513 260)(513 248)
\cpath (514 237)(515 227)(516 218)(516 209)(516 201)
\cpath (517 193)(518 186)(519 180)(519 174)(520 168)
\cpath (521 163)(521 158)(522 153)(522 149)(523 145)
\cpath (524 142)(525 139)(525 136)(526 135)(527 133)
\cpath (527 133)(527 132)(527 132)(528 133)(528 135)
\cpath (529 140)(529 140)(530 144)(530 147)(531 162)
\cpath (531 189)(532 243)(533 321)(533 371)(534 378)
\cpath (534 378)(534 372)(534 366)(534 366)(535 350)
\cpath (535 348)(536 329)(537 312)(537 295)(538 281)
\cpath (539 267)(539 255)(540 243)(540 233)(541 223)
\cpath (542 214)(543 206)(543 198)(544 191)(544 183)
\cpath (545 177)(546 171)(546 166)(547 161)(548 156)
\cpath (549 151)(549 147)(550 144)(550 141)(551 138)
\cpath (552 135)(552 134)(552 134)(553 133)(554 132)
\cpath (554 132)(555 134)(555 137)(556 142)(556 142)
\cpath (556 153)(557 171)(558 207)(558 274)(559 348)
\cpath (560 378)(560 378)(561 374)(561 374)(561 362)
\cpath (561 358)(561 358)(561 340)(562 321)(563 305)
\cpath (564 289)(564 275)(565 261)(566 249)(567 239)
\cpath (567 228)(567 219)(568 210)(569 202)(570 195)
\cpath (570 188)(571 181)(572 175)(573 169)(573 164)
\cpath (573 159)(574 154)(575 150)(576 146)(576 142)
\cpath (577 139)(578 137)(579 135)(579 133)(579 133)
\cpath (579 132)(579 132)(580 133)(581 135)(582 138)
\cpath (582 138)(582 144)(582 146)(583 159)(584 183)
\cpath (585 231)(585 309)(585 366)(586 378)(586 378)
\cpath (587 368)(587 368)(588 354)(588 351)(588 332)
\cpath (589 315)(590 298)(591 283)(591 269)(591 257)
\cpath (592 245)(593 234)(594 225)(594 216)(595 207)
\cpath (596 199)(596 192)(597 185)(597 178)(598 172)
\cpath (599 167)(600 162)(600 156)(601 152)(602 148)
\cpath (602 144)(603 141)(603 138)(604 136)(605 134)
\cpath (606 133)(606 133)(606 132)(606 132)(607 134)
\cpath (608 136)(608 141)(608 141)(609 146)(609 150)
\cpath (609 167)(610 200)(611 261)(612 339)(612 375)
\cpath (612 378)(613 375)(613 365)(614 361)(614 345)
\cpath (614 343)(615 324)(615 307)(616 291)(617 277)
\cpath (618 264)(618 252)(619 240)(619 230)(620 221)
\cpath (621 212)(621 204)(622 196)(623 189)(624 182)
\cpath (624 176)(625 170)(625 165)(626 159)(627 155)
\cpath (627 150)(628 147)(629 143)(630 140)(630 137)
\cpath (631 135)(631 133)(631 133)(632 132)(632 132)
\cpath (633 133)(633 135)(634 138)(634 138)(635 144)
\cpath (635 144)(636 156)(636 178)(636 221)(637 295)
\cpath (638 360)(639 378)(639 378)(639 371)(639 371)
\cpath (640 354)(640 354)(641 336)(642 318)(642 300)
\cpath (642 285)(643 271)(644 258)(645 247)(645 236)
\cpath (646 226)(647 217)(648 208)(648 201)(648 193)
\cpath (649 186)(650 180)(651 174)(651 168)(652 162)
\cpath (653 157)(654 153)(654 149)(654 145)(655 141)
\cpath (656 138)(657 136)(657 134)(658 133)(658 133)
\cpath (659 132)(659 132)(660 133)(660 135)(660 140)
\cpath (660 140)(661 145)(661 148)(662 164)(663 192)
\cpath (663 249)(664 327)(665 372)(665 378)(666 377)
\cpath (666 369)(666 364)(666 348)(666 346)(667 327)
\cpath (668 310)(669 294)(669 279)(670 266)(671 254)
\cpath (672 243)(672 232)(672 222)(673 213)(674 205)
\cpath (675 197)(675 190)(676 183)(677 177)(677 171)
\cpath (678 165)(678 160)(679 156)(680 151)(681 147)
\cpath (681 144)(682 140)(683 138)(683 135)(684 134)
\cpath (684 134)(684 132)(684 132)(685 133)(686 134)
\cpath (687 137)(687 137)(687 143)(687 143)(688 153)
\cpath (689 174)(689 212)(690 281)(690 352)(691 378)
\cpath (691 378)(692 373)(692 373)(693 360)(693 357)
\cpath (693 357)(693 339)(694 320)(694 303)(695 288)
\cpath (696 273)(696 261)(697 249)(698 238)(699 228)
\cpath (699 219)(700 210)(700 201)(701 194)(702 187)
\cpath (702 180)(703 174)(704 168)(705 163)(705 158)
\cpath (706 153)(706 150)(707 146)(708 142)(708 139)
\cpath (709 137)(710 135)(711 133)(711 133)(711 132)
\cpath (711 132)(711 133)(712 135)(713 139)(713 139)
\cpath (714 144)(714 147)(714 160)(715 186)(716 237)
\cpath (717 315)(717 369)(717 378)(717 378)(718 373)
\cpath (718 366)(718 366)(719 351)(719 349)(720 330)
\cpath (720 313)(721 297)(722 282)(723 268)(723 255)
\cpath (723 244)(724 234)(725 224)(726 215)(726 207)
\cpath (727 198)(728 191)(729 184)(729 178)(729 172)
\cpath (730 166)(731 161)(732 156)(732 152)(733 147)
\cpath (734 144)(735 141)(735 138)(735 135)(736 134)
\cpath (736 134)(737 133)(738 132)(738 132)(738 134)
\cpath (739 136)(740 142)(740 142)(741 151)(741 169)
\cpath (741 204)(742 268)(743 344)(744 377)(744 378)
\cpath (744 375)(744 375)(745 363)(745 360)(745 360)
\cpath (746 342)(747 323)(747 306)(747 290)(748 276)
\cpath (749 263)(750 251)(750 240)(751 229)(752 220)
\cpath (752 211)(753 203)(753 195)(754 188)(755 182)
\cpath (756 175)(756 170)(757 164)(758 159)(758 154)
\cpath (759 150)(759 146)(760 143)(761 140)(762 137)
\cpath (762 135)(763 133)(763 133)(764 132)(764 132)
\cpath (764 133)(765 135)(765 138)(765 138)(766 145)
\cpath (766 145)(767 158)(768 181)(768 226)(769 302)
\cpath (769 363)(770 378)(770 378)(771 369)(771 369)
\cpath (771 352)(771 352)(772 333)(773 316)(774 299)
\cpath (774 284)(775 270)(775 258)(776 246)(777 235)
\cpath (777 225)(778 216)(779 207)(780 200)(780 192)
\cpath (781 186)(781 179)(782 173)(783 167)(783 162)
\cpath (784 157)(785 153)(786 148)(786 144)(786 141)
\cpath (787 138)(788 136)(789 134)(789 133)(789 133)
\cpath (790 132)(790 132)(791 133)(792 136)(792 141)
\cpath (792 141)(792 146)(792 150)(793 165)(794 196)
\cpath (795 255)(795 333)
\end{texdraw}
}
  \rput[tl]{*0}( 7.6,9.2){% GNUPLOT: LaTeX using TEXDRAW macros
\begin{texdraw}
\normalsize
\ifx\pathDEFINED\relax\else\let\pathDEFINED\relax
 \def\QtGfr{\ifx (\TGre \let\YhetT\cpath\else\let\YhetT\relax\fi\YhetT}
 \def\path (#1 #2){\move (#1 #2)\futurelet\TGre\QtGfr}
 \def\cpath (#1 #2){\lvec (#1 #2)\futurelet\TGre\QtGfr}
\fi
\drawdim pt
\setunitscale 0.24
\linewd 3
\textref h:L v:C
\path (132 51)(795 51)
\path (132 51)(132 489)
\linewd 4
\path (132 105)(147 105)
\path (795 105)(780 105)
\move (115 105)\textref h:R v:C \htext{}
\path (132 160)(147 160)
\path (795 160)(780 160)
\move (115 160)\htext{}
\path (132 215)(147 215)
\path (795 215)(780 215)
\move (115 215)\htext{}
\path (132 270)(147 270)
\path (795 270)(780 270)
\move (115 270)\htext{}
\path (132 324)(147 324)
\path (795 324)(780 324)
\move (115 324)\htext{}
\path (132 379)(147 379)
\path (795 379)(780 379)
\move (115 379)\htext{}
\path (132 434)(147 434)
\path (795 434)(780 434)
\move (115 434)\htext{}
\path (198 51)(198 66)
\path (198 489)(198 474)
\move (198 17)\textref h:C v:C \htext{}
\path (264 51)(264 66)
\path (264 489)(264 474)
\move (264 17)\htext{}
\path (330 51)(330 66)
\path (330 489)(330 474)
\move (330 17)\htext{}
\path (397 51)(397 66)
\path (397 489)(397 474)
\move (397 17)\htext{}
\path (463 51)(463 66)
\path (463 489)(463 474)
\move (463 17)\htext{}
\path (530 51)(530 66)
\path (530 489)(530 474)
\move (530 17)\htext{}
\path (596 51)(596 66)
\path (596 489)(596 474)
\move (596 17)\htext{}
\path (663 51)(663 66)
\path (663 489)(663 474)
\move (663 17)\htext{}
\path (729 51)(729 66)
\path (729 489)(729 474)
\move (729 17)\htext{}
\path (132 51)(795 51)(795 489)(132 489)(132 51)
\linewd 3
\path (132 166)(132 166)(132 173)(132 244)(133 399)(134 454)
\cpath (134 454)(135 444)(135 444)(135 420)(135 420)
\cpath (135 393)(136 370)(137 349)(138 330)(138 313)
\cpath (139 297)(140 284)(141 271)(141 259)(141 248)
\cpath (142 238)(143 228)(144 220)(144 212)(145 204)
\cpath (146 197)(146 190)(147 184)(147 178)(148 173)
\cpath (149 168)(150 164)(150 159)(151 156)(152 153)
\cpath (152 150)(153 149)(153 148)(153 148)(154 148)
\cpath (155 150)(156 154)(156 154)(156 161)(156 161)
\cpath (157 173)(158 192)(158 226)(159 272)(159 314)
\cpath (160 333)(161 333)(161 332)(162 323)(162 322)
\cpath (162 322)(162 308)(163 294)(163 279)(164 266)
\cpath (165 253)(165 242)(166 231)(167 222)(168 212)
\cpath (168 204)(169 195)(169 188)(170 181)(171 174)
\cpath (171 168)(172 162)(173 156)(174 151)(174 147)
\cpath (175 141)(175 137)(176 133)(177 129)(177 126)
\cpath (178 122)(179 119)(180 116)(180 114)(180 111)
\cpath (181 110)(181 110)(182 109)(183 108)(183 108)
\cpath (183 108)(184 109)(185 111)(186 115)(186 119)
\cpath (186 121)(186 121)(186 133)(187 157)(188 219)
\cpath (189 368)(189 451)(189 455)(190 448)(190 448)
\cpath (190 437)(191 425)(191 425)(192 399)(192 375)
\cpath (192 354)(193 334)(194 316)(195 301)(195 286)
\cpath (196 273)(197 261)(198 250)(198 240)(198 231)
\cpath (199 222)(200 213)(201 206)(201 198)(202 192)
\cpath (203 186)(204 180)(204 174)(204 169)(205 165)
\cpath (206 160)(207 157)(207 153)(208 151)(209 149)
\cpath (210 148)(210 148)(210 148)(210 148)(210 150)
\cpath (211 153)(211 153)(212 159)(212 159)(213 169)
\cpath (213 187)(214 217)(215 261)(216 306)(216 330)
\cpath (216 334)(216 333)(217 325)(217 324)(218 312)
\cpath (218 311)(219 297)(219 282)(220 269)(221 256)
\cpath (221 244)(222 234)(222 223)(223 214)(224 205)
\cpath (225 197)(225 189)(226 183)(227 176)(227 169)
\cpath (228 163)(228 158)(229 153)(230 147)(231 143)
\cpath (231 138)(232 134)(233 130)(233 126)(234 123)
\cpath (234 120)(235 117)(236 114)(237 112)(237 111)
\cpath (237 111)(238 109)(238 108)(239 108)(239 108)
\cpath (240 109)(240 111)(241 114)(242 117)(242 120)
\cpath (242 123)(243 129)(243 150)(244 200)(244 331)
\cpath (245 444)(246 455)(246 452)(246 444)(246 431)
\cpath (247 411)(247 405)(248 380)(249 358)(249 338)
\cpath (250 320)(250 304)(251 290)(252 276)(252 264)
\cpath (253 252)(254 243)(255 233)(255 224)(255 215)
\cpath (256 207)(257 200)(258 193)(258 186)(259 181)
\cpath (260 175)(261 170)(261 165)(261 162)(262 158)
\cpath (263 154)(264 152)(264 150)(265 148)(265 148)
\cpath (266 148)(266 148)(267 149)(267 152)(267 152)
\cpath (267 157)(267 157)(268 167)(269 183)(270 210)
\cpath (270 251)(271 298)(272 327)(273 334)(273 334)
\cpath (273 328)(273 327)(273 327)(273 315)(273 314)
\cpath (274 300)(275 285)(276 271)(276 258)(277 246)
\cpath (278 236)(279 225)(279 216)(279 207)(280 199)
\cpath (281 191)(282 184)(282 177)(283 171)(284 165)
\cpath (285 159)(285 153)(285 148)(286 144)(287 139)
\cpath (288 135)(288 131)(289 127)(290 123)(291 120)
\cpath (291 117)(291 114)(292 112)(293 111)(293 111)
\cpath (294 109)(294 108)(295 108)(295 108)(296 108)
\cpath (296 110)(297 113)(297 118)(297 118)(298 121)
\cpath (298 126)(299 144)(300 184)(300 293)(301 432)
\cpath (301 455)(302 454)(302 447)(302 436)(303 418)
\cpath (303 411)(303 385)(304 363)(305 342)(306 324)
\cpath (306 308)(307 293)(308 279)(308 267)(309 255)
\cpath (309 245)(310 235)(311 225)(312 217)(312 209)
\cpath (313 201)(313 195)(314 188)(315 182)(315 177)
\cpath (316 171)(317 167)(318 162)(318 159)(319 155)
\cpath (319 152)(320 150)(321 149)(321 148)(321 148)
\cpath (322 149)(323 151)(324 156)(324 156)(324 164)
\cpath (325 178)(325 202)(326 240)(327 288)(327 323)
\cpath (328 334)(328 334)(329 329)(329 329)(330 317)
\cpath (330 317)(330 303)(330 288)(331 274)(332 261)
\cpath (333 249)(333 238)(334 228)(335 218)(336 209)
\cpath (336 201)(336 193)(337 186)(338 179)(339 172)
\cpath (339 166)(340 160)(341 155)(342 150)(342 144)
\cpath (342 140)(343 135)(344 132)(345 128)(345 124)
\cpath (346 121)(347 118)(348 115)(348 113)(348 111)
\cpath (348 111)(349 110)(350 108)(351 108)(351 108)
\cpath (351 108)(352 110)(353 112)(354 117)(354 117)
\cpath (354 121)(354 124)(354 139)(355 171)(356 260)
\cpath (357 413)(357 455)(357 455)(358 444)(358 441)
\cpath (359 416)(359 416)(360 391)(360 367)(360 347)
\cpath (361 328)(362 311)(363 296)(363 282)(364 270)
\cpath (365 258)(366 247)(366 237)(366 228)(367 219)
\cpath (368 211)(369 203)(369 196)(370 189)(371 183)
\cpath (371 177)(372 172)(372 168)(373 163)(374 159)
\cpath (375 156)(375 153)(376 150)(377 149)(377 148)
\cpath (377 148)(378 148)(378 150)(379 154)(379 154)
\cpath (380 162)(381 174)(381 196)(382 231)(383 278)
\cpath (383 318)(384 333)(384 333)(384 331)(384 331)
\cpath (385 321)(385 320)(385 320)(386 306)(387 291)
\cpath (387 277)(388 264)(388 252)(389 240)(390 230)
\cpath (390 220)(391 211)(392 202)(393 195)(393 187)
\cpath (394 180)(394 174)(395 168)(396 162)(396 156)
\cpath (397 150)(398 146)(399 141)(399 137)(400 132)
\cpath (400 129)(401 125)(402 122)(402 118)(403 116)
\cpath (404 114)(405 111)(405 111)(405 110)(406 108)
\cpath (406 108)(406 108)(407 108)(408 109)(408 111)
\cpath (409 115)(410 120)(410 122)(410 122)(411 135)
\cpath (411 162)(411 232)(412 386)(413 453)(414 455)
\cpath (414 446)(414 446)(414 422)(414 422)(415 396)
\cpath (416 372)(417 351)(417 332)(417 315)(418 299)
\cpath (419 285)(420 272)(420 260)(421 249)(422 239)
\cpath (423 230)(423 221)(423 213)(424 205)(425 198)
\cpath (426 191)(426 185)(427 179)(428 174)(429 168)
\cpath (429 164)(429 160)(430 156)(431 153)(432 151)
\cpath (432 149)(433 148)(433 148)(434 148)(435 150)
\cpath (435 153)(435 153)(435 160)(435 160)(436 171)
\cpath (437 190)(438 222)(438 267)(439 311)(440 332)
\cpath (440 334)(441 333)(441 324)(441 323)(441 311)
\cpath (441 309)(442 295)(443 281)(444 267)(444 255)
\cpath (445 243)(446 232)(446 222)(447 213)(447 204)
\cpath (448 196)(449 189)(450 182)(450 175)(451 168)
\cpath (452 162)(452 157)(453 152)(453 147)(454 142)
\cpath (455 138)(456 133)(456 129)(457 126)(458 122)
\cpath (458 119)(459 117)(459 114)(460 112)(461 110)
\cpath (461 110)(462 109)(462 108)(463 108)(463 108)
\cpath (463 109)(464 111)(465 114)(465 118)(465 120)
\cpath (465 120)(466 132)(467 154)(468 210)(468 353)
\cpath (469 449)(469 454)(469 450)(470 439)(470 427)
\cpath (471 408)(471 402)(471 377)(472 355)(473 336)
\cpath (474 318)(474 302)(475 288)(475 275)(476 263)
\cpath (477 252)(477 241)(478 231)(479 222)(480 214)
\cpath (480 207)(481 199)(481 192)(482 186)(483 180)
\cpath (483 174)(484 170)(485 165)(486 161)(486 157)
\cpath (486 154)(487 151)(488 150)(489 148)(489 148)
\cpath (489 148)(489 148)(490 150)(491 153)(491 153)
\cpath (492 158)(492 158)(492 168)(492 186)(493 214)
\cpath (494 257)(495 303)(495 330)(496 334)(496 333)
\cpath (497 327)(497 326)(498 313)(498 312)(498 298)
\cpath (498 284)(499 270)(500 257)(501 246)(501 234)
\cpath (502 225)(503 215)(504 206)(504 198)(504 190)
\cpath (505 183)(506 177)(507 170)(507 164)(508 159)
\cpath (509 153)(510 148)(510 143)(510 138)(511 134)
\cpath (512 130)(513 126)(513 123)(514 120)(515 117)
\cpath (516 114)(516 112)(516 111)(516 111)(517 109)
\cpath (518 108)(519 108)(519 108)(519 108)(520 111)
\cpath (521 114)(521 119)(521 119)(522 122)(522 128)
\cpath (522 147)(523 192)(524 313)(525 439)(525 455)
\cpath (525 453)(525 446)(526 433)(526 414)(527 408)
\cpath (527 383)(528 360)(528 340)(529 322)(530 306)
\cpath (531 291)(531 278)(532 265)(533 254)(533 243)
\cpath (534 234)(534 225)(535 216)(536 208)(537 201)
\cpath (537 194)(538 187)(539 181)(539 176)(540 171)
\cpath (540 166)(541 162)(542 158)(543 155)(543 152)
\cpath (544 150)(544 148)(544 148)(545 148)(545 148)
\cpath (546 149)(546 151)(546 151)(547 156)(547 156)
\cpath (548 165)(549 180)(549 207)(550 246)(550 294)
\cpath (551 326)(552 334)(552 334)(552 328)(552 328)
\cpath (553 316)(553 315)(554 301)(555 287)(555 273)
\cpath (556 260)(556 248)(557 237)(558 226)(558 217)
\cpath (559 208)(560 200)(561 192)(561 185)(561 178)
\cpath (562 171)(563 165)(564 159)(564 154)(565 149)
\cpath (566 144)(567 139)(567 135)(567 131)(568 127)
\cpath (569 123)(570 120)(570 117)(571 115)(572 113)
\cpath (573 111)(573 111)(573 109)(573 108)(574 108)
\cpath (574 108)(575 108)(576 110)(576 113)(577 117)
\cpath (577 117)(577 122)(578 126)(579 141)(579 178)
\cpath (579 278)(580 424)(581 455)(581 455)(582 441)
\cpath (582 438)(582 413)(582 413)(583 388)(584 365)
\cpath (585 344)(585 326)(585 309)(586 294)(587 280)
\cpath (588 268)(588 256)(589 246)(590 236)(591 226)
\cpath (591 218)(591 210)(592 202)(593 195)(594 189)
\cpath (594 183)(595 177)(596 172)(596 167)(597 162)
\cpath (597 159)(598 156)(599 153)(600 150)(600 149)
\cpath (601 148)(601 148)(602 149)(602 150)(603 155)
\cpath (603 155)(603 163)(604 177)(605 199)(606 236)
\cpath (606 284)(607 321)(608 334)(608 334)(608 330)
\cpath (608 330)(609 318)(609 318)(609 304)(610 290)
\cpath (611 276)(612 263)(612 250)(613 239)(614 228)
\cpath (614 219)(615 210)(615 201)(616 194)(617 186)
\cpath (618 180)(618 173)(619 167)(619 161)(620 156)
\cpath (621 150)(621 145)(622 141)(623 136)(624 132)
\cpath (624 128)(625 124)(625 121)(626 118)(627 115)
\cpath (627 113)(628 111)(628 111)(629 110)(630 108)
\cpath (630 108)(630 108)(631 108)(631 110)(632 112)
\cpath (633 116)(633 116)(633 120)(633 123)(634 138)
\cpath (635 168)(636 248)(636 402)(636 455)(636 455)
\cpath (637 443)(637 443)(638 418)(638 418)(639 393)
\cpath (639 369)(640 348)(641 330)(642 312)(642 297)
\cpath (642 283)(643 270)(644 258)(645 248)(645 237)
\cpath (646 228)(647 219)(648 211)(648 204)(648 197)
\cpath (649 190)(650 184)(651 178)(651 173)(652 168)
\cpath (653 163)(654 159)(654 156)(654 153)(655 150)
\cpath (656 149)(657 148)(657 148)(657 148)(658 150)
\cpath (659 154)(659 154)(660 161)(660 161)(660 173)
\cpath (660 194)(661 227)(662 273)(663 315)(663 333)
\cpath (664 333)(664 332)(664 332)(665 322)(665 321)
\cpath (665 321)(666 308)(666 293)(666 279)(667 265)
\cpath (668 253)(669 242)(669 231)(670 221)(671 212)
\cpath (672 203)(672 195)(672 188)(673 180)(674 174)
\cpath (675 168)(675 162)(676 156)(677 151)(677 146)
\cpath (678 141)(678 137)(679 133)(680 129)(681 125)
\cpath (681 122)(682 119)(683 116)(683 114)(684 111)
\cpath (684 111)(684 110)(685 109)(686 108)(686 108)
\cpath (687 108)(687 109)(688 111)(689 115)(689 119)
\cpath (689 121)(689 121)(690 133)(690 159)(691 223)
\cpath (692 373)(693 452)(693 455)(693 447)(693 447)
\cpath (694 424)(694 424)(694 399)(695 374)(696 353)
\cpath (696 333)(697 316)(698 300)(699 286)(699 273)
\cpath (700 261)(700 250)(701 240)(702 231)(702 222)
\cpath (703 213)(704 205)(705 198)(705 192)(706 185)
\cpath (706 180)(707 174)(708 169)(708 165)(709 160)
\cpath (710 156)(711 153)(711 151)(711 149)(712 148)
\cpath (712 148)(713 148)(714 150)(714 153)(714 153)
\cpath (715 159)(715 159)(716 170)(717 188)(717 219)
\cpath (717 263)(718 308)(719 331)(720 334)(720 333)
\cpath (720 325)(720 324)(721 312)(721 311)(722 296)
\cpath (723 282)(723 268)(723 255)(724 244)(725 233)
\cpath (726 223)(726 213)(727 205)(728 197)(729 189)
\cpath (729 182)(729 175)(730 169)(731 163)(732 158)
\cpath (732 153)(733 147)(734 142)(735 138)(735 134)
\cpath (735 129)(736 126)(737 123)(738 120)(738 117)
\cpath (739 114)(740 112)(741 111)(741 111)(741 109)
\cpath (741 108)(742 108)(742 108)(743 109)(744 111)
\cpath (744 114)(744 117)(745 120)(745 123)(746 130)
\cpath (747 151)(747 203)(747 337)(748 446)(749 455)
\cpath (749 451)(750 441)(750 429)(750 411)(750 404)
\cpath (751 379)(752 357)(752 337)(753 320)(753 303)
\cpath (754 289)(755 276)(756 264)(756 252)(757 242)
\cpath (758 232)(758 223)(759 215)(759 207)(760 200)
\cpath (761 193)(762 186)(762 180)(763 175)(764 170)
\cpath (764 165)(765 161)(765 157)(766 154)(767 152)
\cpath (768 150)(768 148)(768 148)(769 148)(769 148)
\cpath (769 149)(770 152)(770 152)(771 157)(771 157)
\cpath (771 167)(772 183)(773 211)(774 253)(774 300)
\cpath (775 328)(775 334)(775 334)(776 327)(776 327)
\cpath (776 327)(777 314)(777 314)(777 299)(778 285)
\cpath (779 271)(780 258)(780 246)(781 235)(781 225)
\cpath (782 216)(783 207)(783 198)(784 191)(785 183)
\cpath (786 177)(786 171)(786 165)(787 159)(788 153)
\cpath (789 148)(789 144)(790 139)(791 135)(792 131)
\cpath (792 126)(792 123)(793 120)(794 117)(795 114)
\cpath (795 112)
\end{texdraw}
}
  \rput[tl]{*0}( 1.2,4.4){% GNUPLOT: LaTeX using TEXDRAW macros
\begin{texdraw}
\normalsize
\ifx\pathDEFINED\relax\else\let\pathDEFINED\relax
 \def\QtGfr{\ifx (\TGre \let\YhetT\cpath\else\let\YhetT\relax\fi\YhetT}
 \def\path (#1 #2){\move (#1 #2)\futurelet\TGre\QtGfr}
 \def\cpath (#1 #2){\lvec (#1 #2)\futurelet\TGre\QtGfr}
\fi
\drawdim pt
\setunitscale 0.24
\linewd 3
\textref h:L v:C
\path (132 51)(795 51)
\path (132 51)(132 489)
\linewd 4
\path (132 105)(147 105)
\path (795 105)(780 105)
\move (115 105)\textref h:R v:C \htext{{\footnotesize$1$}}
\path (132 160)(147 160)
\path (795 160)(780 160)
\move (115 160)\htext{{\footnotesize$2$}}
\path (132 215)(147 215)
\path (795 215)(780 215)
\move (115 215)\htext{{\footnotesize$3$}}
\path (132 270)(147 270)
\path (795 270)(780 270)
\move (115 270)\htext{{\footnotesize$4$}}
\path (132 324)(147 324)
\path (795 324)(780 324)
\move (115 324)\htext{{\footnotesize$5$}}
\path (132 379)(147 379)
\path (795 379)(780 379)
\move (115 379)\htext{{\footnotesize$6$}}
\path (132 434)(147 434)
\path (795 434)(780 434)
\move (115 434)\htext{{\footnotesize$7$}}
\path (198 51)(198 66)
\path (198 489)(198 474)
\move (198 17)\textref h:C v:C \htext{{\footnotesize$10$}}
\path (264 51)(264 66)
\path (264 489)(264 474)
\move (264 17)\htext{{\footnotesize$20$}}
\path (330 51)(330 66)
\path (330 489)(330 474)
\move (330 17)\htext{{\footnotesize$30$}}
\path (397 51)(397 66)
\path (397 489)(397 474)
\move (397 17)\htext{{\footnotesize$40$}}
\path (463 51)(463 66)
\path (463 489)(463 474)
\move (463 17)\htext{{\footnotesize$50$}}
\path (530 51)(530 66)
\path (530 489)(530 474)
\move (530 17)\htext{{\footnotesize$60$}}
\path (596 51)(596 66)
\path (596 489)(596 474)
\move (596 17)\htext{{\footnotesize$70$}}
\path (663 51)(663 66)
\path (663 489)(663 474)
\move (663 17)\htext{{\footnotesize$80$}}
\path (729 51)(729 66)
\path (729 489)(729 474)
\move (729 17)\htext{{\footnotesize$90$}}
\path (132 51)(795 51)(795 489)(132 489)(132 51)
\linewd 3
\path (132 419)(132 419)(132 405)(132 393)(133 370)(134 349)
\cpath (135 331)(135 315)(135 300)(136 287)(137 275)
\cpath (138 264)(138 253)(139 243)(140 234)(141 225)
\cpath (141 218)(141 210)(142 203)(143 197)(144 191)
\cpath (144 185)(145 180)(146 175)(146 171)(147 167)
\cpath (147 164)(148 162)(149 159)(150 159)(150 159)
\cpath (150 159)(151 160)(152 164)(152 164)(152 170)
\cpath (153 180)(153 196)(154 221)(155 252)(156 283)
\cpath (156 303)(157 308)(157 308)(158 303)(158 303)
\cpath (158 294)(158 294)(159 282)(159 270)(160 258)
\cpath (161 246)(162 236)(162 225)(163 216)(163 207)
\cpath (164 199)(165 192)(165 184)(166 177)(167 171)
\cpath (168 165)(168 159)(169 153)(169 148)(170 144)
\cpath (171 139)(171 135)(172 130)(173 126)(174 123)
\cpath (174 120)(175 117)(175 114)(176 111)(177 109)
\cpath (177 108)(177 108)(178 106)(179 105)(180 105)
\cpath (180 105)(180 106)(180 108)(181 111)(182 117)
\cpath (182 117)(183 120)(183 126)(183 144)(184 189)
\cpath (185 317)(186 447)(186 460)(186 457)(186 450)
\cpath (186 435)(187 424)(187 409)(188 384)(189 362)
\cpath (189 342)(190 324)(191 307)(192 293)(192 279)
\cpath (192 267)(193 255)(194 245)(195 235)(195 226)
\cpath (196 218)(197 210)(198 202)(198 195)(198 189)
\cpath (199 183)(200 177)(201 172)(201 168)(202 163)
\cpath (203 159)(204 156)(204 153)(204 151)(205 150)
\cpath (206 149)(206 149)(207 150)(207 152)(207 152)
\cpath (208 157)(208 157)(209 165)(210 180)(210 204)
\cpath (210 241)(211 287)(212 321)(213 331)(213 331)
\cpath (213 327)(213 327)(214 315)(214 315)(215 300)
\cpath (216 286)(216 273)(216 260)(217 248)(218 237)
\cpath (219 226)(219 216)(220 207)(221 199)(221 192)
\cpath (222 184)(222 177)(223 171)(224 165)(225 159)
\cpath (225 153)(226 147)(227 143)(227 138)(228 133)
\cpath (228 129)(229 125)(230 121)(231 117)(231 114)
\cpath (232 111)(233 108)(233 106)(234 104)(234 102)
\cpath (234 102)(235 101)(236 100)(237 100)(237 100)
\cpath (237 101)(238 102)(238 104)(239 108)(240 111)
\cpath (240 114)(240 114)(240 126)(241 150)(242 215)
\cpath (243 387)(243 466)(243 467)(243 462)(244 454)
\cpath (244 454)(244 427)(244 427)(245 401)(246 377)
\cpath (246 356)(247 336)(248 320)(249 305)(249 291)
\cpath (250 279)(250 267)(251 256)(252 246)(252 237)
\cpath (253 228)(254 220)(255 213)(255 205)(255 198)
\cpath (256 192)(257 187)(258 181)(258 177)(259 172)
\cpath (260 168)(261 165)(261 162)(261 160)(262 159)
\cpath (263 159)(263 159)(264 159)(264 162)(264 162)
\cpath (265 168)(265 168)(266 176)(267 190)(267 212)
\cpath (267 241)(268 274)(269 298)(270 308)(270 308)
\cpath (270 306)(270 306)(271 297)(271 297)(272 286)
\cpath (273 274)(273 262)(273 250)(274 239)(275 229)
\cpath (276 219)(276 210)(277 202)(278 194)(279 186)
\cpath (279 180)(279 173)(280 167)(281 161)(282 156)
\cpath (282 150)(283 145)(284 141)(285 136)(285 132)
\cpath (285 128)(286 124)(287 120)(288 117)(288 114)
\cpath (289 112)(290 110)(291 108)(291 108)(291 107)
\cpath (291 106)(292 105)(292 105)(293 106)(294 107)
\cpath (294 110)(295 114)(295 114)(295 119)(296 122)
\cpath (296 137)(297 169)(297 260)(298 420)(299 460)
\cpath (299 460)(300 447)(300 444)(300 418)(300 418)
\cpath (301 393)(302 369)(302 348)(303 330)(303 312)
\cpath (304 297)(305 284)(306 271)(306 259)(307 249)
\cpath (308 239)(308 229)(309 221)(309 213)(310 205)
\cpath (311 198)(312 191)(312 185)(313 180)(313 174)
\cpath (314 169)(315 165)(315 161)(316 157)(317 154)
\cpath (318 152)(318 150)(319 149)(319 149)(319 150)
\cpath (320 151)(321 155)(321 155)(321 162)(322 174)
\cpath (323 194)(324 227)(324 271)(325 312)(325 330)
\cpath (326 330)(326 329)(327 320)(327 319)(327 319)
\cpath (327 306)(328 291)(329 277)(330 264)(330 252)
\cpath (330 240)(331 230)(332 219)(333 210)(333 202)
\cpath (334 194)(335 186)(336 180)(336 173)(336 166)
\cpath (337 160)(338 155)(339 150)(339 144)(340 139)
\cpath (341 135)(342 130)(342 126)(342 122)(343 119)
\cpath (344 115)(345 112)(345 109)(346 107)(347 105)
\cpath (348 102)(348 102)(348 102)(348 101)(349 100)
\cpath (349 100)(350 100)(351 102)(351 103)(352 106)
\cpath (353 111)(353 111)(353 115)(354 121)(354 139)
\cpath (354 184)(355 322)(356 458)(357 467)(357 462)
\cpath (357 462)(357 453)(357 437)(357 437)(358 409)
\cpath (359 384)(360 363)(360 343)(360 325)(361 309)
\cpath (362 295)(363 282)(363 270)(364 260)(365 249)
\cpath (366 240)(366 231)(366 222)(367 215)(368 207)
\cpath (369 201)(369 195)(370 189)(371 183)(371 178)
\cpath (372 174)(372 170)(373 166)(374 163)(375 161)
\cpath (375 159)(376 159)(376 159)(377 159)(377 161)
\cpath (377 161)(378 165)(378 165)(378 173)(379 185)
\cpath (380 204)(381 231)(381 264)(382 292)(383 306)
\cpath (383 308)(383 307)(384 300)(384 300)(384 290)
\cpath (384 290)(385 278)(386 266)(387 254)(387 243)
\cpath (388 232)(388 222)(389 213)(390 204)(390 197)
\cpath (391 189)(392 182)(393 175)(393 169)(394 163)
\cpath (394 157)(395 152)(396 147)(396 142)(397 138)
\cpath (398 133)(399 129)(399 125)(400 122)(400 118)
\cpath (401 115)(402 113)(402 111)(403 108)(403 108)
\cpath (404 107)(405 106)(405 105)(405 105)(406 105)
\cpath (406 107)(407 108)(408 112)(408 116)(408 119)
\cpath (408 119)(409 130)(410 154)(411 216)(411 373)
\cpath (411 458)(412 460)(412 452)(412 452)(413 428)
\cpath (413 428)(414 401)(414 377)(415 355)(416 336)
\cpath (417 318)(417 303)(417 288)(418 276)(419 264)
\cpath (420 252)(420 242)(421 233)(422 224)(423 216)
\cpath (423 207)(423 201)(424 193)(425 187)(426 181)
\cpath (426 176)(427 171)(428 166)(429 162)(429 158)
\cpath (429 155)(430 153)(431 150)(432 150)(432 150)
\cpath (432 149)(432 149)(433 150)(434 153)(434 153)
\cpath (435 159)(435 159)(435 169)(435 186)(436 214)
\cpath (437 255)(438 300)(438 326)(438 331)(439 330)
\cpath (440 324)(440 323)(441 311)(441 310)(441 296)
\cpath (441 282)(442 268)(443 255)(444 244)(444 233)
\cpath (445 223)(446 213)(446 205)(447 197)(447 189)
\cpath (448 182)(449 175)(450 168)(450 162)(451 156)
\cpath (452 151)(452 146)(453 141)(453 136)(454 132)
\cpath (455 128)(456 123)(456 120)(457 116)(458 113)
\cpath (458 110)(459 108)(459 105)(460 103)(460 103)
\cpath (461 102)(462 101)(462 100)(463 100)(463 100)
\cpath (463 101)(464 102)(465 105)(465 109)(465 109)
\cpath (466 114)(466 117)(467 131)(468 163)(468 257)
\cpath (469 432)(469 467)(469 466)(470 462)(470 451)
\cpath (470 446)(471 432)(471 419)(471 393)(472 370)
\cpath (473 349)(474 331)(474 315)(475 300)(475 287)
\cpath (476 275)(477 264)(477 253)(478 243)(479 234)
\cpath (480 225)(480 218)(481 210)(481 204)(482 197)
\cpath (483 191)(483 185)(484 180)(485 175)(486 171)
\cpath (486 167)(486 164)(487 162)(488 159)(489 159)
\cpath (489 159)(489 159)(490 160)(491 164)(491 164)
\cpath (492 170)(492 180)(492 196)(493 220)(494 252)
\cpath (495 283)(495 303)(496 308)(496 308)(497 303)
\cpath (497 303)(498 294)(498 294)(498 282)(498 270)
\cpath (499 258)(500 246)(501 236)(501 225)(502 216)
\cpath (503 207)(504 199)(504 192)(504 184)(505 177)
\cpath (506 171)(507 165)(507 159)(508 153)(509 148)
\cpath (510 144)(510 139)(510 135)(511 130)(512 126)
\cpath (513 123)(513 120)(514 117)(515 114)(516 111)
\cpath (516 109)(516 108)(516 108)(517 106)(518 105)
\cpath (519 105)(519 105)(519 106)(520 108)(521 111)
\cpath (521 117)(521 117)(522 120)(522 126)(522 144)
\cpath (523 190)(524 317)(525 447)(525 460)(525 457)
\cpath (525 450)(526 435)(526 423)(527 409)(527 384)
\cpath (528 362)(528 342)(529 324)(530 307)(531 293)
\cpath (531 279)(532 267)(533 255)(533 245)(534 235)
\cpath (534 226)(535 218)(536 210)(537 202)(537 195)
\cpath (538 189)(539 183)(539 177)(540 172)(540 168)
\cpath (541 163)(542 159)(543 156)(543 153)(544 151)
\cpath (544 150)(545 149)(545 149)(546 150)(546 152)
\cpath (546 152)(547 157)(547 157)(548 165)(549 180)
\cpath (549 204)(550 241)(550 287)(551 321)(552 331)
\cpath (552 331)(552 327)(552 327)(553 315)(553 315)
\cpath (554 301)(555 286)(555 273)(556 260)(556 248)
\cpath (557 237)(558 226)(558 216)(559 207)(560 199)
\cpath (561 192)(561 184)(561 177)(562 171)(563 165)
\cpath (564 159)(564 153)(565 147)(566 143)(567 138)
\cpath (567 133)(567 129)(568 125)(569 121)(570 117)
\cpath (570 114)(571 111)(572 108)(573 106)(573 104)
\cpath (573 102)(573 102)(574 101)(575 100)(576 100)
\cpath (576 100)(576 101)(577 102)(578 104)(579 108)
\cpath (579 111)(579 114)(579 114)(579 126)(580 150)
\cpath (581 215)(582 387)(582 466)(582 467)(582 462)
\cpath (583 454)(583 454)(584 427)(584 427)(585 401)
\cpath (585 377)(585 356)(586 337)(587 320)(588 305)
\cpath (588 291)(589 279)(590 267)(591 256)(591 246)
\cpath (591 237)(592 228)(593 220)(594 213)(594 205)
\cpath (595 199)(596 192)(596 187)(597 181)(597 177)
\cpath (598 172)(599 168)(600 165)(600 162)(601 160)
\cpath (602 159)(602 159)(602 159)(603 159)(603 162)
\cpath (603 162)(604 168)(604 168)(605 176)(606 190)
\cpath (606 212)(607 241)(608 274)(608 298)(609 308)
\cpath (609 308)(609 306)(609 306)(610 297)(610 297)
\cpath (611 286)(612 274)(612 262)(613 250)(614 240)
\cpath (614 229)(615 219)(615 210)(616 202)(617 194)
\cpath (618 186)(618 180)(619 173)(619 167)(620 161)
\cpath (621 156)(621 150)(622 145)(623 141)(624 136)
\cpath (624 132)(625 128)(625 124)(626 120)(627 117)
\cpath (627 114)(628 112)(629 110)(630 108)(630 108)
\cpath (630 107)(631 106)(631 105)(631 105)(632 106)
\cpath (633 107)(633 110)(634 114)(634 114)(635 119)
\cpath (635 122)(636 136)(636 168)(636 259)(637 420)
\cpath (638 460)(638 460)(639 447)(639 444)(639 418)
\cpath (639 418)(640 393)(641 369)(642 348)(642 330)
\cpath (642 313)(643 297)(644 284)(645 271)(645 260)
\cpath (646 249)(647 239)(648 229)(648 221)(648 213)
\cpath (649 205)(650 198)(651 191)(651 185)(652 180)
\cpath (653 174)(654 169)(654 165)(654 161)(655 157)
\cpath (656 154)(657 152)(657 150)(658 149)(658 149)
\cpath (659 150)(660 151)(660 155)(660 155)(660 162)
\cpath (661 174)(662 194)(663 226)(663 271)(664 311)
\cpath (665 330)(665 330)(666 329)(666 320)(666 319)
\cpath (666 307)(666 306)(667 291)(668 277)(669 264)
\cpath (669 252)(670 240)(671 230)(672 219)(672 210)
\cpath (672 202)(673 194)(674 186)(675 180)(675 173)
\cpath (676 166)(677 161)(677 155)(678 150)(678 144)
\cpath (679 139)(680 135)(681 130)(681 126)(682 123)
\cpath (683 119)(683 115)(684 112)(684 109)(685 107)
\cpath (686 105)(687 102)(687 102)(687 102)(688 101)
\cpath (689 100)(689 100)(689 100)(690 101)(690 103)
\cpath (691 106)(692 111)(692 111)(692 115)(693 121)
\cpath (693 139)(694 184)(694 321)(695 458)(696 467)
\cpath (696 462)(696 462)(696 453)(696 437)(696 437)
\cpath (697 409)(698 384)(699 363)(699 343)(700 325)
\cpath (700 309)(701 295)(702 282)(702 270)(703 260)
\cpath (704 249)(705 240)(705 231)(706 223)(706 215)
\cpath (707 207)(708 201)(708 195)(709 189)(710 183)
\cpath (711 178)(711 174)(711 170)(712 166)(713 163)
\cpath (714 161)(714 159)(715 159)(715 159)(716 159)
\cpath (717 161)(717 161)(717 165)(717 165)(717 173)
\cpath (718 185)(719 204)(720 231)(720 264)(721 291)
\cpath (722 306)(722 308)(723 307)(723 300)(723 300)
\cpath (723 290)(723 290)(724 278)(725 266)(726 254)
\cpath (726 243)(727 232)(728 222)(729 213)(729 204)
\cpath (729 197)(730 189)(731 182)(732 175)(732 169)
\cpath (733 163)(734 157)(735 152)(735 147)(735 142)
\cpath (736 138)(737 133)(738 129)(738 125)(739 122)
\cpath (740 118)(741 115)(741 113)(741 111)(742 108)
\cpath (742 108)(743 107)(744 106)(744 105)(744 105)
\cpath (745 105)(746 107)(747 108)(747 112)(747 116)
\cpath (747 119)(747 119)(748 130)(749 154)(750 216)
\cpath (750 372)(751 458)(751 460)(752 453)(752 453)
\cpath (752 428)(752 428)(753 402)(753 377)(754 355)
\cpath (755 336)(756 318)(756 303)(757 288)(758 276)
\cpath (758 264)(759 252)(759 242)(760 233)(761 224)
\cpath (762 216)(762 207)(763 201)(764 194)(764 187)
\cpath (765 181)(765 176)(766 171)(767 166)(768 162)
\cpath (768 158)(769 155)(769 153)(770 150)(771 150)
\cpath (771 150)(771 149)(771 149)(772 150)(773 153)
\cpath (773 153)(774 159)(774 159)(774 169)(775 186)
\cpath (775 213)(776 255)(777 300)(777 326)(778 331)
\cpath (778 330)(779 324)(779 323)(780 311)(780 310)
\cpath (780 296)(781 282)(781 268)(782 255)(783 244)
\cpath (783 233)(784 223)(785 213)(786 205)(786 197)
\cpath (786 189)(787 182)(788 175)(789 168)(789 162)
\cpath (790 156)(791 151)(792 146)(792 141)(792 136)
\cpath (793 132)(794 128)(795 123)(795 120)
\end{texdraw}
}
  \rput[tl]{*0}( 7.6,4.4){% GNUPLOT: LaTeX using TEXDRAW macros
\begin{texdraw}
\normalsize
\ifx\pathDEFINED\relax\else\let\pathDEFINED\relax
 \def\QtGfr{\ifx (\TGre \let\YhetT\cpath\else\let\YhetT\relax\fi\YhetT}
 \def\path (#1 #2){\move (#1 #2)\futurelet\TGre\QtGfr}
 \def\cpath (#1 #2){\lvec (#1 #2)\futurelet\TGre\QtGfr}
\fi
\drawdim pt
\setunitscale 0.24
\linewd 3
\textref h:L v:C
\path (132 51)(795 51)
\path (132 51)(132 489)
\linewd 4
\path (132 105)(147 105)
\path (795 105)(780 105)
\move (115 105)\textref h:R v:C \htext{}
\path (132 160)(147 160)
\path (795 160)(780 160)
\move (115 160)\htext{}
\path (132 215)(147 215)
\path (795 215)(780 215)
\move (115 215)\htext{}
\path (132 270)(147 270)
\path (795 270)(780 270)
\move (115 270)\htext{}
\path (132 324)(147 324)
\path (795 324)(780 324)
\move (115 324)\htext{}
\path (132 379)(147 379)
\path (795 379)(780 379)
\move (115 379)\htext{}
\path (132 434)(147 434)
\path (795 434)(780 434)
\move (115 434)\htext{}
\path (198 51)(198 66)
\path (198 489)(198 474)
\move (198 17)\textref h:C v:C \htext{{\footnotesize$10$}}
\path (264 51)(264 66)
\path (264 489)(264 474)
\move (264 17)\htext{{\footnotesize$20$}}
\path (330 51)(330 66)
\path (330 489)(330 474)
\move (330 17)\htext{{\footnotesize$30$}}
\path (397 51)(397 66)
\path (397 489)(397 474)
\move (397 17)\htext{{\footnotesize$40$}}
\path (463 51)(463 66)
\path (463 489)(463 474)
\move (463 17)\htext{{\footnotesize$50$}}
\path (530 51)(530 66)
\path (530 489)(530 474)
\move (530 17)\htext{{\footnotesize$60$}}
\path (596 51)(596 66)
\path (596 489)(596 474)
\move (596 17)\htext{{\footnotesize$70$}}
\path (663 51)(663 66)
\path (663 489)(663 474)
\move (663 17)\htext{{\footnotesize$80$}}
\path (729 51)(729 66)
\path (729 489)(729 474)
\move (729 17)\htext{{\footnotesize$90$}}
\path (132 51)(795 51)(795 489)(132 489)(132 51)
\linewd 3
\path (132 321)(132 321)(132 312)(132 304)(133 290)(134 276)
\cpath (135 264)(135 252)(135 242)(136 232)(137 223)
\cpath (138 215)(138 207)(139 200)(140 192)(141 186)
\cpath (141 180)(141 174)(142 169)(143 165)(144 160)
\cpath (144 156)(145 153)(146 150)(146 147)(147 146)
\cpath (147 145)(147 145)(148 145)(149 147)(150 150)
\cpath (150 150)(150 157)(150 157)(151 169)(152 190)
\cpath (152 225)(153 276)(153 321)(154 341)(155 341)
\cpath (155 339)(155 339)(156 329)(156 327)(156 327)
\cpath (156 312)(157 297)(158 282)(158 268)(159 255)
\cpath (159 244)(160 233)(161 223)(162 213)(162 204)
\cpath (163 197)(163 189)(164 182)(165 175)(165 168)
\cpath (166 162)(167 156)(168 151)(168 146)(169 141)
\cpath (169 136)(170 132)(171 128)(171 123)(172 120)
\cpath (173 117)(174 113)(174 110)(175 108)(175 105)
\cpath (176 103)(176 103)(177 102)(177 101)(178 100)
\cpath (178 100)(179 100)(180 101)(180 102)(180 105)
\cpath (181 110)(181 110)(182 114)(182 118)(183 133)
\cpath (183 168)(184 271)(185 441)(185 466)(186 465)
\cpath (186 461)(186 444)(186 428)(186 417)(187 391)
\cpath (188 368)(189 348)(189 330)(190 313)(191 299)
\cpath (192 285)(192 273)(192 262)(193 252)(194 242)
\cpath (195 233)(195 225)(196 216)(197 210)(198 202)
\cpath (198 196)(198 190)(199 184)(200 179)(201 174)
\cpath (201 170)(202 166)(203 163)(204 161)(204 159)
\cpath (204 158)(204 158)(205 158)(206 159)(207 163)
\cpath (207 163)(207 169)(208 180)(209 196)(210 221)
\cpath (210 253)(210 285)(211 305)(212 309)(212 309)
\cpath (213 305)(213 305)(213 294)(213 294)(214 283)
\cpath (215 270)(216 258)(216 247)(216 236)(217 226)
\cpath (218 216)(219 207)(219 199)(220 192)(221 184)
\cpath (221 177)(222 171)(222 165)(223 159)(224 153)
\cpath (225 148)(225 143)(226 138)(227 134)(227 130)
\cpath (228 126)(228 122)(229 119)(230 115)(231 113)
\cpath (231 110)(232 108)(233 106)(233 106)(233 105)
\cpath (234 104)(234 103)(234 103)(235 104)(236 105)
\cpath (237 107)(237 111)(237 114)(238 117)(238 117)
\cpath (238 128)(239 152)(240 214)(240 375)(241 462)
\cpath (241 464)(242 455)(242 455)(243 429)(243 429)
\cpath (243 402)(244 378)(244 357)(245 337)(246 320)
\cpath (246 304)(247 290)(248 277)(249 265)(249 255)
\cpath (250 244)(250 234)(251 225)(252 217)(252 210)
\cpath (253 202)(254 195)(255 189)(255 183)(255 178)
\cpath (256 173)(257 168)(258 164)(258 160)(259 157)
\cpath (260 155)(261 153)(261 152)(261 152)(261 153)
\cpath (262 154)(263 157)(263 157)(264 164)(264 164)
\cpath (264 175)(265 193)(266 222)(267 262)(267 300)
\cpath (267 320)(268 324)(268 323)(269 315)(269 315)
\cpath (270 304)(270 303)(270 289)(271 276)(272 263)
\cpath (273 250)(273 239)(273 228)(274 219)(275 210)
\cpath (276 201)(276 193)(277 186)(278 179)(279 172)
\cpath (279 166)(279 160)(280 154)(281 149)(282 144)
\cpath (282 139)(283 134)(284 130)(285 126)(285 122)
\cpath (285 118)(286 114)(287 111)(288 108)(288 106)
\cpath (289 104)(290 102)(290 102)(291 101)(291 99)
\cpath (291 99)(291 99)(292 99)(293 100)(294 102)
\cpath (294 105)(295 110)(295 110)(295 113)(296 118)
\cpath (296 134)(297 172)(297 288)(298 449)(299 467)
\cpath (299 464)(299 455)(300 440)(300 440)(300 412)
\cpath (301 387)(302 365)(302 345)(303 327)(303 312)
\cpath (304 297)(305 284)(306 273)(306 261)(307 251)
\cpath (308 242)(308 233)(309 225)(309 217)(310 210)
\cpath (311 203)(312 196)(312 190)(313 185)(313 180)
\cpath (314 175)(315 171)(315 168)(316 165)(317 162)
\cpath (318 161)(318 160)(318 160)(319 161)(319 163)
\cpath (319 163)(320 168)(320 168)(321 174)(321 186)
\cpath (322 204)(323 230)(324 261)(324 288)(325 302)
\cpath (325 304)(325 303)(326 297)(326 297)(327 288)
\cpath (327 288)(327 276)(328 264)(329 252)(330 242)
\cpath (330 231)(330 222)(331 213)(332 204)(333 196)
\cpath (333 189)(334 181)(335 174)(336 168)(336 162)
\cpath (336 157)(337 152)(338 147)(339 142)(339 138)
\cpath (340 133)(341 129)(342 126)(342 122)(342 119)
\cpath (343 116)(344 114)(345 111)(345 109)(345 109)
\cpath (346 108)(347 108)(348 107)(348 107)(348 108)
\cpath (348 109)(349 112)(350 117)(350 117)(351 121)
\cpath (351 126)(351 144)(352 186)(353 302)(354 438)
\cpath (354 456)(354 455)(354 447)(354 435)(355 416)
\cpath (355 409)(356 384)(357 361)(357 341)(358 323)
\cpath (359 306)(360 292)(360 279)(360 266)(361 255)
\cpath (362 244)(363 234)(363 225)(364 216)(365 209)
\cpath (366 201)(366 194)(366 188)(367 182)(368 176)
\cpath (369 171)(369 166)(370 162)(371 158)(371 154)
\cpath (372 151)(372 149)(373 147)(374 147)(374 147)
\cpath (375 147)(375 148)(376 153)(376 153)(377 159)
\cpath (377 159)(377 171)(378 192)(378 228)(379 276)
\cpath (380 319)(381 337)(381 337)(381 335)(381 335)
\cpath (382 325)(382 324)(382 324)(383 309)(383 294)
\cpath (384 279)(384 266)(385 254)(386 242)(387 231)
\cpath (387 221)(388 212)(388 203)(389 195)(390 188)
\cpath (390 180)(391 174)(392 168)(393 162)(393 156)
\cpath (394 150)(394 145)(395 140)(396 135)(396 131)
\cpath (397 126)(398 123)(399 119)(399 115)(400 112)
\cpath (400 109)(401 107)(402 105)(402 102)(402 102)
\cpath (403 101)(404 100)(405 99)(405 99)(405 100)
\cpath (406 100)(406 102)(407 105)(408 109)(408 109)
\cpath (408 114)(408 117)(409 132)(410 165)(411 267)
\cpath (411 439)(411 466)(411 465)(412 461)(412 444)
\cpath (413 428)(413 417)(414 390)(414 368)(415 348)
\cpath (416 330)(417 314)(417 299)(417 286)(418 273)
\cpath (419 263)(420 252)(420 243)(421 234)(422 225)
\cpath (423 217)(423 210)(423 203)(424 197)(425 191)
\cpath (426 185)(426 180)(427 175)(428 171)(429 168)
\cpath (429 165)(429 162)(430 160)(431 159)(431 159)
\cpath (432 159)(432 162)(433 165)(433 165)(434 171)
\cpath (435 183)(435 199)(435 224)(436 255)(437 285)
\cpath (438 302)(438 306)(438 306)(439 301)(439 301)
\cpath (440 291)(440 291)(441 279)(441 267)(441 256)
\cpath (442 245)(443 234)(444 224)(444 215)(445 206)
\cpath (446 198)(446 190)(447 183)(447 177)(448 170)
\cpath (449 164)(450 158)(450 153)(451 147)(452 143)
\cpath (452 138)(453 134)(453 129)(454 126)(455 122)
\cpath (456 119)(456 116)(457 113)(458 111)(458 108)
\cpath (458 108)(459 107)(459 106)(460 105)(460 105)
\cpath (461 105)(462 106)(462 108)(463 111)(463 115)
\cpath (463 117)(463 117)(464 127)(465 148)(465 200)
\cpath (466 340)(467 453)(467 460)(468 456)(468 444)
\cpath (468 432)(468 432)(469 405)(469 381)(470 359)
\cpath (471 339)(471 321)(472 306)(473 291)(474 278)
\cpath (474 265)(475 254)(475 244)(476 234)(477 225)
\cpath (477 216)(478 209)(479 201)(480 195)(480 188)
\cpath (481 182)(481 177)(482 171)(483 167)(483 162)
\cpath (484 159)(485 156)(486 153)(486 150)(486 149)
\cpath (486 149)(487 149)(487 149)(488 150)(489 152)
\cpath (489 152)(489 157)(489 157)(490 166)(491 181)
\cpath (492 206)(492 245)(492 291)(493 323)(494 332)
\cpath (494 332)(495 326)(495 326)(495 315)(495 314)
\cpath (496 300)(497 285)(498 271)(498 258)(498 246)
\cpath (499 236)(500 225)(501 216)(501 207)(502 198)
\cpath (503 191)(504 183)(504 177)(504 170)(505 164)
\cpath (506 158)(507 153)(507 147)(508 142)(509 137)
\cpath (510 132)(510 129)(510 124)(511 120)(512 117)
\cpath (513 113)(513 110)(514 108)(515 105)(516 103)
\cpath (516 102)(516 102)(516 100)(517 99)(518 99)
\cpath (518 99)(519 99)(519 100)(520 102)(521 106)
\cpath (521 109)(521 111)(521 111)(522 121)(522 141)
\cpath (523 192)(524 345)(525 462)(525 466)(525 462)
\cpath (525 458)(525 458)(526 432)(526 432)(527 405)
\cpath (527 381)(528 359)(528 339)(529 323)(530 307)
\cpath (531 294)(531 281)(532 269)(533 258)(533 249)
\cpath (534 240)(534 231)(535 222)(536 215)(537 208)
\cpath (537 201)(538 195)(539 189)(539 184)(540 179)
\cpath (540 174)(541 171)(542 167)(543 165)(543 162)
\cpath (544 161)(544 161)(544 161)(545 162)(546 165)
\cpath (546 165)(546 170)(546 170)(547 178)(548 192)
\cpath (549 212)(549 240)(550 270)(550 292)(551 303)
\cpath (551 303)(552 301)(552 301)(552 294)(552 294)
\cpath (553 283)(554 272)(555 260)(555 249)(556 238)
\cpath (556 228)(557 219)(558 210)(558 201)(559 193)
\cpath (560 186)(561 179)(561 173)(561 166)(562 161)
\cpath (563 155)(564 150)(564 145)(565 141)(566 136)
\cpath (567 132)(567 128)(567 124)(568 121)(569 118)
\cpath (570 115)(570 113)(571 111)(571 111)(572 110)
\cpath (573 108)(573 108)(573 108)(573 108)(574 110)
\cpath (575 112)(576 116)(576 116)(576 120)(576 123)
\cpath (577 138)(578 167)(579 247)(579 401)(579 453)
\cpath (579 453)(580 442)(580 442)(581 417)(581 417)
\cpath (582 392)(582 368)(583 347)(584 328)(585 312)
\cpath (585 296)(585 282)(586 270)(587 258)(588 247)
\cpath (588 237)(589 228)(590 219)(591 210)(591 203)
\cpath (591 196)(592 189)(593 183)(594 177)(594 171)
\cpath (595 167)(596 162)(596 158)(597 154)(597 151)
\cpath (598 149)(599 147)(600 145)(600 145)(600 145)
\cpath (600 145)(601 146)(602 149)(602 149)(602 154)
\cpath (602 154)(603 164)(603 180)(604 209)(605 253)
\cpath (606 305)(606 336)(607 341)(607 341)(607 333)
\cpath (608 332)(608 320)(608 318)(609 303)(609 288)
\cpath (610 273)(611 261)(612 249)(612 237)(613 227)
\cpath (614 217)(614 208)(615 200)(615 192)(616 184)
\cpath (617 177)(618 171)(618 165)(619 159)(619 153)
\cpath (620 148)(621 143)(621 138)(622 134)(623 129)
\cpath (624 125)(624 121)(625 117)(625 114)(626 111)
\cpath (627 108)(627 106)(628 104)(629 102)(629 102)
\cpath (630 101)(630 100)(631 100)(631 100)(631 101)
\cpath (632 102)(633 104)(633 108)(634 111)(634 114)
\cpath (634 114)(635 125)(636 148)(636 210)(636 380)
\cpath (637 465)(638 467)(638 463)(638 455)(638 455)
\cpath (639 428)(639 428)(639 402)(640 378)(641 356)
\cpath (642 337)(642 321)(642 305)(643 291)(644 279)
\cpath (645 267)(645 256)(646 246)(647 237)(648 228)
\cpath (648 220)(648 213)(649 205)(650 199)(651 192)
\cpath (651 187)(652 181)(653 177)(654 172)(654 168)
\cpath (654 165)(655 162)(656 160)(657 159)(657 158)
\cpath (657 158)(658 159)(659 162)(659 162)(660 167)
\cpath (660 167)(660 175)(660 189)(661 210)(662 239)
\cpath (663 273)(663 297)(664 309)(664 309)(665 307)
\cpath (665 307)(666 299)(666 299)(666 287)(666 275)
\cpath (667 263)(668 251)(669 240)(669 230)(670 220)
\cpath (671 211)(672 202)(672 195)(672 187)(673 180)
\cpath (674 174)(675 167)(675 161)(676 156)(677 150)
\cpath (677 145)(678 141)(678 136)(679 132)(680 127)
\cpath (681 123)(681 120)(682 117)(683 114)(683 111)
\cpath (684 109)(684 107)(684 107)(685 105)(686 105)
\cpath (687 104)(687 104)(687 104)(688 105)(689 106)
\cpath (689 110)(690 114)(690 115)(690 118)(690 125)
\cpath (691 144)(692 192)(693 329)(693 453)(693 463)
\cpath (694 459)(694 451)(694 435)(695 423)(695 408)
\cpath (696 384)(696 361)(697 342)(698 324)(699 308)
\cpath (699 293)(700 280)(700 268)(701 257)(702 246)
\cpath (702 237)(703 228)(704 219)(705 211)(705 204)
\cpath (706 197)(706 190)(707 184)(708 179)(708 174)
\cpath (709 169)(710 165)(711 161)(711 158)(711 155)
\cpath (712 153)(713 152)(713 152)(714 151)(714 151)
\cpath (714 153)(715 156)(715 156)(716 161)(716 161)
\cpath (717 171)(717 186)(717 212)(718 249)(719 291)
\cpath (720 318)(720 325)(720 325)(721 319)(721 319)
\cpath (722 308)(722 308)(723 294)(723 280)(723 267)
\cpath (724 255)(725 243)(726 232)(726 222)(727 213)
\cpath (728 204)(729 196)(729 189)(729 181)(730 174)
\cpath (731 168)(732 162)(732 156)(733 150)(734 146)
\cpath (735 141)(735 136)(735 132)(736 127)(737 123)
\cpath (738 119)(738 116)(739 112)(740 109)(741 107)
\cpath (741 105)(741 102)(741 102)(742 101)(743 100)
\cpath (744 99)(744 99)(744 99)(745 99)(746 101)
\cpath (747 103)(747 108)(747 111)(747 114)(748 126)
\cpath (749 152)(750 225)(750 405)(751 467)(751 467)
\cpath (751 456)(752 450)(752 437)(752 423)(753 397)
\cpath (753 373)(754 353)(755 334)(756 318)(756 303)
\cpath (757 289)(758 277)(758 266)(759 255)(759 246)
\cpath (760 237)(761 228)(762 220)(762 213)(763 206)
\cpath (764 199)(764 193)(765 187)(765 182)(766 177)
\cpath (767 173)(768 169)(768 166)(769 164)(769 162)
\cpath (770 161)(770 161)(771 161)(771 162)(772 166)
\cpath (772 166)(773 172)(774 182)(774 197)(775 220)
\cpath (775 249)(776 278)(777 297)(777 303)(777 303)
\cpath (778 300)(778 300)(779 291)(779 291)(780 280)
\cpath (780 268)(781 256)(781 246)(782 235)(783 225)
\cpath (783 216)(784 207)(785 198)(786 191)(786 184)
\cpath (786 177)(787 171)(788 165)(789 159)(789 153)
\cpath (790 148)(791 144)(792 139)(792 135)(792 131)
\cpath (793 127)(794 123)(795 120)(795 117)
\end{texdraw}
}
  \rput[tc]{*0}( 4.1,0.1){\footnotesize tid/s}
  \rput[tc]{*0}(10.5,0.1){\footnotesize tid/s}
  \rput[cc]{*0}( 6.3,8.7){\footnotesize a)}
  \rput[cc]{*0}(12.7,8.7){\footnotesize b)}
  \rput[cc]{*0}( 6.3,3.9){\footnotesize c)}
  \rput[cc]{*0}(12.7,3.9){\footnotesize d)}
  \rput[cb]{*0}( 4.0,9.4){\footnotesize periode $T$}
  \rput[cb]{*0}(10.4,9.4){\footnotesize periode $2T$}
  \rput[cb]{*0}( 4.0,4.6){\footnotesize periode $4T$}
  \rput[cb]{*0}(10.4,4.6){\footnotesize periode $8T$}
  \rput[cc]{*0}( 0.5,7.3){\rotateleft{\footnotesize $x_1$}}
  \rput[cc]{*0}( 0.5,2.4){\rotateleft{\footnotesize $x_1$}}
 \end{pspicture}
\end{center}
}
{
\caption{\protect\capsize
Eksempler p{\aa} tidsr{\ae}kker opn{\aa}et ved numerisk
integration af lig\-ning~\protect\ref{eq:KobletBrusEq},
svarende til en r{\ae}kke successive periodefordoblinger i
den koblede Brusselator under variation af
diffusionskonstanten $D_1$. De valgte parameterv{\ae}rdier
for $D_1$ er: $1.160$, $1.180$, $1.191$ og $1.1928$ i
r{\ae}kkefl{\o}gen a) til d).}
\label{fig:Perdob}
}

\vspace{4.0mm}
Vi vender nu blikket mod et s{\ae}t parameterv{\ae}rdier,
for hvilke der ind\-tr{\ae}der en torusbifurkation i den
koblede Brusselator. Vi s{\ae}tter $B=5.5$ og unders{\o}ger
opf{\o}rslen i intervallet $0.052 < D_1 < 0.053$. For
$D_1>0.05295$ er l{\o}sningerne til
lig\-ning~\ref{eq:KobletBrusEq} simple svingninger. For
$D_1=0.05295$ optr{\ae}der en torusbifurkation, der giver
anledning til kvasiperiodisk bev{\ae}gelse. For lavere
v{\ae}rdier af $D_1$ bliver bev{\ae}gelsen resonant, idet
dynamikken fasel{\aa}ser p{\aa} torusen i form af en
$3T$-periodisk svingning. Formindskes $D_1$ yderligere,
bliver bev{\ae}gelsen igen kvasiperiodisk, en deformation
af torusen indtr{\ae}der, hvorp{\aa} en rent kaotisk
opf{\o}rsel iagttages. Den m{\aa}de, hvorp{\aa} sy\-stemets
dynamik manifisterer sig i intervallet $0.052 < D_1 <
0.053$, er kort opridset i tabel~\ref{tab:TorusData}.

\capsize
\begin{table}[htbp] 
 \renewcommand{\capfont}{\bf}
 \begin{center} 
  \begin{tabular}{|c|l|}                        \hline\hline
   $D_1$ {\em interval} & \multicolumn{1}{c|}{\em tiltr{\ae}kker type} \\ \hline
   $D_1 > 0.05295     $ & simple svingninger   \\ \hline
   $0.05295 \mdash 0.05246$ & kvasiperiodisk bane p{\aa} en torus \\ \hline
   $0.05246 \mdash 0.05244$ & $3T$-periodisk fasel{\aa}st kurve 
			      p{\aa} en torus\\ \hline
   $0.05244 \mdash 0.05238$ & kvasiperiodisk bane p{\aa} en torus \\ \hline
   $0.05238 \mdash 0.05206$ & strange attractor               \\ \hline
   $D_1 < 0.05206$          & station{\ae}rt punkt           \\\hline\hline
  \end{tabular}
 \end{center} 
 \caption{\protect\capsize
  De angivne data stammer fra \protect\cite{Marek2} og
  angiver de stabile tiltr{\ae}kkere, der forekommer for
  den koblede Brusselator i intervallet for forskellige
  v{\ae}rdier af diffusionskonstanten $D_1$.}
 \label{tab:TorusData}
\renewcommand{\capfont}{\rm}
\end{table} 
\normalsize

\vspace{4.0mm}
Vi har set, hvorledes en r{\ae}kke teoretiske forudsigelser
kan genfindes i en for\-holds\-vis simpel model for et
kemisk reaktionssy\-stem. Afslutningsvis vil vi beskrive,
hvilke af de diskuterede st{\o}rrelser, der er
tilg{\ae}ngelige udfra eksperimentelle m{\aa}linger.

\section{Eksperimentel bestemmelse af
Floquet\-multiplikatorer og -eks\-ponenter}

Et v{\ae}sentligt
problem vedr{\o}rende eksperimentel bestemmelse af de
karakteristiske Floquetst{\o}rrelser skyldes, at der endnu
ikke eksisterer et teoretisk grundlag, der kobler indholdet
af de eksperimentelle data til en teoretisk beskrivelse af
de kemiske oscillationer i et ikke-line{\ae}rt
parameteromr{\aa}de. Derudover har det vist sig, at
s{\aa}vel observation som stabilisering af
bifurkationsf{\ae}nomener, svarende til de tre
bifurkationstyper beskrevet i forrige af\-snit, stiller
store krav til n{\o}jagtigheden af det eksperimentelle
kontrol- og m{\aa}leudstyr.

\vspace{4.0mm}
Blandt andet af disse {\aa}rsager, har det indtil videre
ikke v{\ae}ret muligt eksperimentelt at tilvejebringe en
situation, hvor torusbifurkationer og
periodefordoblingsbifurkationer i kemiske sy\-stemer har
kunnet unders{\o}ges tilfredsstillende. F{\ae}nomenerne har
dog v{\ae}ret observeret eksperi\-mentelt, men af praktiske
{\aa}rsager er disse meget sv{\ae}re at stabilisere.
Eksempelvis er en periodefordobling i BZ-reaktionen
observeret (se \cite{BZPeriodDoubling} for en n{\ae}rmere
beskrivelse).

\vspace{4.0mm}
Befinder man sig derimod i det line{\ae}re
parameteromr{\aa}de i n{\ae}rheden af en superkritisk
Hopfbifurkation, er det dog muligt at give en vurdering af
st{\o}rrelses\-ordenen af den Floqueteksponent, der
beskriver bev{\ae}gelser omkring gr{\ae}nse\-cyklusens
plan. Vi har tidligere i kapitel~\ref{cha:Quench}
set, at den gr{\ae}nse\-cyklus, som beskriver de kemiske
oscillationer lige efter sy\-stemet har gennemg{\aa}et en
superkritisk Hopfbifurkation, med god tiln{\ae}rmelse
beskrives af udtrykket

\begin{equation}
 {\bf c}(t) = {\bf c}_f +
 a \left[ {\bf u}\cos \omega t + {\bf v}\sin \omega t \right]
\end{equation}

hvor $({\bf u},{\bf v})$ udg{\o}r henholdsvis real- og
imagin{\ae}rdel af de to komplekse egenvektorer til
Jacobimatricen. $\omega$ er imagin{\ae}rdelen i den
komplekse egenv{\ae}rdi $\alpha + i \omega$ h{\o}rende til
egenvektoren ${\bf u} - i{\bf v}$. Perturberes et
s{\aa}dant sy\-stem ved tils{\ae}tning af en blanding af
reaktionens reaktanter, s{\aa}ledes at perturbationsvektoren
er rettet mod det station{\ae}re punkt ${\bf c}_f$, vil den
resulterende udspiralering i n{\ae}rheden af ${\bf c}_f$
v{\ae}re beskrevet ved udtrykket

\begin{equation}
 {\bf c}(t) = {\bf c}_f + a e^{\alpha t} 
 \left[ {\bf u}\cos \omega t + {\bf v}\sin \omega t \right]
\end{equation}

%%%%%%%%%%%%%%%%%%%%%%%%%%%%%%%%%%%%%%%%%%%%%%%%%%%%%%%%%%%%%%%%%%%%%%%%
%% figur
%%
%% beskrivelse : Skematisk illustration af to bev{\ae}gelsestyper
%%               mod gr{\ae}nsecyklus
%% makroer     : PSTricks, PST-Plot
%%%%%%%%%%%%%%%%%%%%%%%%%%%%%%%%%%%%%%%%%%%%%%%%%%%%%%%%%%%%%%%%%%%%%%%%
\boxfigure{t}{\textwidth}
{
\newcommand{\temp}{\mbox{\scriptsize ${\bf c}(t)$}}
\begin{center}
   \begin{pspicture}(0,0)(14,4)
%   \psgrid[](0,0)(0,0)(14,4)
   \psline[linewidth=1.2pt,
           fillstyle=solid,
           fillcolor=lightgray]{-}(1.0,1.0)(5.0,1.0)(8.0,3.0)(4.0,3.0)(1.0,1.0)
   \psline[origin={-5,0},
           fillstyle=solid,
           fillcolor=lightgray,
           linewidth=1.2pt]{-}(1.0,1.0)(5.0,1.0)(8.0,3.0)(4.0,3.0)(1.0,1.0)
   \scriptsize
   \rput[cb]{*0}( 4.0,3.2){\scriptsize a)}
   \rput[cb]{*0}( 9.0,3.2){\scriptsize b)}
   \rput[br]{*0}( 7.2,2.6){\scriptsize $E^u$}
   \rput[br]{*0}(12.2,2.6){\scriptsize $E^u$}
   \rput[cc]{*0}( 2.5,1.2){\scriptsize $\gamma$}
   \rput[lc]{*0}( 8.8,1.72){\scriptsize $\gamma$}
   \rput[tc]{*0}( 7.68,2.58){\white\temp}
   \rput[tc]{*0}(10.70,1.92){\temp}
   \rput[tc]{*0}( 5.68,2.58){\temp}
   \parametricplot[linewidth=1.2pt,origin={-4.5,-2},plotpoints=100]{0}{360}%
    {t cos 1.5 t sin mul add %
     0.8 t sin mul}
   \parametricplot[origin={-4.5,-2},plotpoints=100]{0}{1440}%
    {t cos 1.5 t sin mul add 2.71 0.002 t mul neg exp mul %
     0.8 t sin mul 2.71 0.002 t mul neg exp mul}
   \parametricplot[arrows=->,
                   arrowinset=0,
                   arrowsize=1.5pt 2.25,
                   origin={-4.5,-2},
                   plotpoints=2]{180}{179}%
    {t cos 1.5 t sin mul add 2.71 0.002 t mul neg exp mul %
     0.8 t sin mul 2.71 0.002 t mul neg exp mul}
   \parametricplot[arrows=->,
                   arrowinset=0,
                   arrowsize=1.5pt 2.25,
                   origin={-4.5,-2},
                   plotpoints=2]{540}{539}%
    {t cos 1.5 t sin mul add 2.71 0.002 t mul neg exp mul %
     0.8 t sin mul 2.71 0.002 t mul neg exp mul}

   \parametricplot[linewidth=1.2pt,origin={-9.5,-2},plotpoints=100]{0}{360}%
    {0.5 t cos 1.5 t sin mul add mul %
     0.4 t sin mul}
   \parametricplot[origin={-9.5,-2},plotpoints=300]{0}{1080}%
    {0.5 t cos 1.5 t sin mul add mul 
     2.71 0.0025 t mul neg exp 1.3 mul 1 add mul %
     0.4 t sin mul
     2.71 0.0025 t mul neg exp 1.3 mul 1 add mul}
   \parametricplot[arrows=->,
                   arrowinset=0,
                   arrowsize=1.5pt 2.25,
                   origin={-9.5,-2},
                   plotpoints=2]{20}{21}%
    {0.5 t cos 1.5 t sin mul add mul 
     2.71 0.0025 t mul neg exp 1.3 mul 1 add mul %
     0.4 t sin mul
     2.71 0.0025 t mul neg exp 1.3 mul 1 add mul}
   \parametricplot[arrows=->,
                   arrowinset=0,
                   arrowsize=1.5pt 2.25,
                   origin={-9.5,-2},
                   plotpoints=2]{380}{381}%
    {0.5 t cos 1.5 t sin mul add mul
     2.71 0.0025 t mul neg exp 1.3 mul 1 add mul %
     0.4 t sin mul
     2.71 0.0025 t mul neg exp 1.3 mul 1 add mul}
  \end{pspicture}
\end{center}
}
{
\caption{\protect\capsize 
Skematisk illustration af de to typiske konvergensformer
mod en stabil gr{\ae}nsecyklus, $\gamma$, genereret ved en
superkritisk Hopfbifurkation. a) viser bev{\ae}gelsen fra
det station{\ae}re ud mod $\gamma$, hvorimod b) illustrerer
bev{\ae}gelsen udenfor $\gamma$ ind mod denne.}
\label{fig:LimitCycPert}
}

Vi forestiller os nu, at perturbationen er rettet v{\ae}k
fra gr{\ae}nsecyklus p{\aa} en s{\aa}dan m{\aa}de, at denne
stadig foreg{\aa}r i $({\bf u},{\bf v})$-planet. Systemets
responderende bev{\ae}gelse p{\aa} denne perturbation vil
stadig v{\ae}re en spiralering tilbage til
gr{\ae}nsecyklusen, men vil nu v{\ae}re beskrevet af et
udtryk, der involverer \'{e}n af de $n$
Floquet\-eksponenter\footnote{Floquetmultiplikatoren og
-eksponenten, der er associeret med denne bev{\ae}gelse,
vil her blive symoliseret ved $\lambda_2$ henholdsvis
$\sigma_2$} (se figur~\ref{fig:LimitCycPert} for en
illustration af disse to f{\ae}nomener). F{\o}lger vi
notationen fra lig\-ning~\ref{eq:PertExpan}, vil denne
bev{\ae}gelse alts{\aa} kunne udtrykkes som

\begin{equation} 
 {\bf c} (t) = {\bf c}_f + a {\bf p}_2(t) e^{\sigma_2 t} 
\end{equation} 

hvor den periodiske funktion ${\bf p}_2(t)$ m{\aa} kunne
skrives som 

\begin{equation}
 {\bf p}_2(t) = \left[ {\bf u}\cos \omega t + 
 {\bf v}\sin \omega t \right].
\end{equation}

Vi vil nu vise, hvilken relation der er g{\ae}ldende mellem
Floqueteksponenten $\sigma_2$ og realdelen $\alpha$ af den
komplekse egenv{\ae}rdi $\alpha + i \omega$. Dette
sp{\o}rgsm{\aa}l er f.eks.\ besvaret i
\cite{GinzburgLandau}, hvor den {\o}nskede relation
fremkommer som et af resultaterne i forbindelse med en
beskrivelse af m{\o}nsterdannelse og turbulens i kemiske
reaktions\-sy\-stemer ved hj{\ae}lp af den komplekse
Ginzburg-Landau lig\-ning. Sammenh{\ae}ngen kan dog
ogs{\aa} vises ved udnyttelse af normalformsteoremet (se
\cite{GuckenheimerHolmes} for en n{\ae}rmere beskrivelse af
dette). Denne tiln{\ae}rmelse til problem\-stil\-lingen vil
blive foretrukket her, da denne endnu ikke er blevet
beskrevet i litteraturen.

\vspace{4.0mm}
Vi forestiller os nu, at vi udelukkende er interesserede i
at beskrive det kemiske sy\-stems dynamik i $({\bf u},{\bf
v})$-planet, der som bekendt udsp{\ae}nder
gr{\ae}nsecyklus. Antager vi yderligere, at sy\-stemet
befinder sig tilpas t{\ae}t p{\aa} en superkritisk
Hopfbifurkation, da kan man vise \cite{GuckenheimerHolmes},
at den dynamik, der er indeholdt i den kinetiske lingning
$\dot{\bf c} = {\bf f}({\bf c},\mu)$, i $({\bf u},{\bf
v})$-planet med god approksimation vil v{\ae}re beskrevet
af det $2$-dimensionale differentiallig\-ningssy\-stem

\begin{equation}
 \begin{array}{lll}
  \dot{x} & = & x(\mu - (x^2 + y^2)) - \omega y\\
  \dot{y} & = & y(\mu - (x^2 + y^2)) + \omega x
 \end{array}
 \label{eq:SuperHopf}
\end{equation}

hvor vi har valgt ${\bf c}_f$ som origo i rummet span$({\bf
u},{\bf v})$. Systemet~\ref{eq:SuperHopf} kaldes for
normalformen for den superkritiske Hopfbifurkation og har
det trivielle station{\ae}re punkt $(0,0)$ med
tilh{\o}rende Jacobimatrix

\begin{equation}
 {\bf J}_{(0,0)} = 
 \left[
 \begin{array}{rr}
  \mu    & -\omega \\
  \omega &  \mu
 \end{array}
 \right]
\end{equation}

Heraf ses, at lig\-ning~\ref{eq:SuperHopf} netop underg{\aa}r
en superkritisk Hopfbifurkation for parameterv{\ae}rdien
$\mu=0$. Videre ser vi, at de to egenv{\ae}rdier for
${\bf J}_{(0,0)}$ tilfredsstiller $\lambda_{1,2} = \mu \pm
i \omega$, hvorfor realdelen $\alpha$ for det oprindelige
sy\-stem opfylder $\alpha = \mu$. Vi {\o}nsker nu at bestemme
Floqueteksponenten for den periodisk stabile l{\o}sning,
der findes til lig\-ning~\ref{eq:SuperHopf} for $\mu > 0$.
Transformeres lig\-ning~\ref{eq:SuperHopf} til pol{\ae}re
koordinater $(r,\theta)$, s{\aa}ledes at $x=r \cos \theta$
og $y=r \sin \theta$, vil sy\-stemet i den pol{\ae}re basis
v{\ae}re beskrevet af lig\-ningerne

\begin{equation}
 \begin{array}{lll}
  \dot{r}      & = & r(\mu - r^2)\\
  \dot{\theta} & = & \omega 
 \end{array}
 \label{eq:SuperHopfPolar}
\end{equation}

der med begyndelsesbetingelserne $r(0)=\sqrt{\mu}$ og
$\theta (0) = 0$ har l{\o}sningen $r(t) = \sqrt{\mu}$ og
$\theta (t) = \omega t$. Vi ser specielt, at $r(t)$ er
konstant, hvilket netop betyder, at denne l{\o}sning svarer
til en periodisk gr{\ae}nsecyklus i (x,y)-planet givet ved

\begin{subequations}
 \begin{eqalignno}
  x(t) &= \sqrt{\mu} \cos \omega t\\
  y(t) &= \sqrt{\mu} \sin \omega t
 \end{eqalignno}
 \label{eq:SuperHopfSol}
\end{subequations}

med amplituden $\sqrt{\mu}$ og perioden
$\frac{2\pi}{\omega}$. Udregnes nu Jacobimatricen for
lig\-ning~\ref{eq:SuperHopf} under inds{\ae}ttelse af
lig\-ning~\ref{eq:SuperHopfSol}, findes f{\o}lgende
differentiallig\-ning til bestemmelse af monodromimatricen
${\bf M}(T)$

\begin{equation}
 \dot{\bf M} = -
 \left[
 \begin{array}{cc}
  2\mu \cos^2\omega t & \sin 2\omega t + \omega \\
  \sin 2\omega t - \omega & 2\mu \sin^2\omega t
 \end{array}
 \right]
 {\bf M}, \mbox{\ \ hvor \ \ } {\bf M}(0) = {\bf I}
\end{equation}

Udnytter vi nu Abels identitet\footnote{Se 
f.eks.\ \cite[s.\ 198]{Saaty} for et bevis for Abels identitet }

\begin{equation}
 \det {\bf M}(T) = 
 \det {\bf M}(0) 
 \exp \left[\int_0^T \Tr {\bf J}(\tau) d\tau\right]
\end{equation}

samt at $\det {\bf M}(T) = \lambda_1\lambda_2 = \lambda_2 =
e^{\sigma_2 T}$ (da $\lambda_1 = 1$), finder vi f{\o}lgende
resultat for Floqueteksponenten $\sigma_2$

\begin{equation}
 \lambda_2 = \det {\bf M}(T) = 
 \exp -2\mu\int_0^T\negsp  d\tau = e^{-2\mu T} \Rightarrow
 \sigma_2 = -2 \mu
\end{equation}

Med andre ord vil realdelen $\alpha$ og Floqueteksponenten
$\sigma_2$ tilstr{\ae}kkelig t{\ae}t p{\aa} den
superkritiske Hopfbifurkation forholde sig til hinanden som

\begin{equation}
 \frac{\alpha}{\sigma_2} = -2
\end{equation}

Dette resultat er verificeret ved numeriske beregninger
\cite{SpecRapport} og har yderligere v{\ae}ret anvendt til
bestemmelse af $\sigma_2$ udfra eksperimentelle data
\cite{GinzburgLandau}.

