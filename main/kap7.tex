\chapter{Poincar\'{e}afbild\-ningen}
Vi har tidligere set eksempler p{\aa}, hvorledes oscillerende
egenskaber ved den generelle kinetiske ligning

\begin{equation}
 \dot{\bf c} = {\bf f}({\bf c},\mu)
 \label{eq:GenKin}
\end{equation}

kan diskuteres ved at betragte l{\o}sninger
repr{\ae}senteret ved tidsr{\ae}kker eller banekurver i
faserummet. Da ligning~\ref{eq:GenKin} sj{\ae}ldent kan
l{\o}ses analytisk, vil disse to repr{\ae}sentationsformer
oftest v{\ae}re bestemt ved numerisk integration.

\vspace{2.5mm}
{\O}nsker man at illustrere et givet systems opf{\o}rsel i
faserummet opst{\aa}r der dog et problem, n{\aa}r dettes
dimension er st{\o}rre end $3$, idet vi ikke l{\ae}ngere
fysisk vil kunne visualisere de tilh{\o}rende banekurver.
En metode, der elegant omg{\aa}r dette problem, er i 1899
udviklet af den franske matematiker og videskabsteoretiker
Henri Poincar\'{e} \cite{PoinOrig}. I stedet for direkte at
betragte selve banekurvernes forl{\o}b i faserummet
l{\o}ste Poincar\'{e} problemet ved at indskr{\ae}nke sig
til at betragte banekurvernes {\em sk{\ae}ringer} med et
passende plan i det p{\aa}g{\ae}ldende faserum. Herved
reduceres beskrivelsen af det dynamiske system fra
dimensionen $n$ til $n-1$.

\vspace{2.5mm}
Denne id\'{e} kan formuleres pr{\ae}cist ved hj{\ae}lp af
den s{\aa}kaldte Poincar\'{e}afbild\-ning, der har fundet
grobund som et yderst anvendeligt v{\ae}rkt{\o}j i studiet
af dynamiske systemer. I det f{\o}lgende vil vi teoretisk
redeg{\o}re for Poincar\'{e}\-afbild\-ningens vigtigste
egenskaber, hvorp{\aa} vi viser dennes anvendelse i en
analyse af torusbifurkationer i den koblede Brusselator
model. Til sidst illustreres hvorledes Poincar\'{e}
afbild\-ningen kan udnyttes til at unders{\o}ge
eksperimentelle data.

\section{Diskrete afbild\-ninger}
\label{sec:DiskMap}
Indledningsvis reserverer vi plads til en diskussion af det
diskrete dynamiske system

\begin{equation}
 {\bf x}_{n+1} = {\bf f}({\bf x}_n)
 \label{eq:DiskGen}
\end{equation}

da vi i den egentlige analyse af Poincar\'{e}afbild\-ningen
vil udnytte visse egen\-skaber ved denne type af
afbild\-ninger.

\vspace{4.0mm}
Lad ${\bf x}_f$ v{\ae}re et fikspunkt for
ligning~\ref{eq:DiskGen}, dvs.\ ${\bf f}({\bf x}_f) = {\bf
x}_f$. Idet vi nu perturberer ${\bf x}_f$ med st{\o}rrelsen
$\delta{\bf x}$, vil vi, ved at studere dynamikken for
${\bf f}({\bf x}_f+\delta{\bf x})$, formulere de
line{\ae}re stabilitetskriterier for fikspunktet ${\bf
x}_f$. Hertil betragter vi ${\bf f}^n({\bf x}_f+\delta{\bf
x})$, der ved Taylorudvikling til 1.\ orden samt anvendelse
af k{\ae}dereglen giver

\begin{eqnarray}
{\bf f}^n({\bf x}_f+\delta{\bf x}) & \simeq &
{\bf f}^n({\bf x}_f) +
\left.
 \frac{\partial {\bf f}^n}{\partial {\bf x}}
\right|_{{\bf x}_f} \negsp\delta{\bf x} \nonumber\\
                                   & \simeq &
{\bf x}_f +
\left(
 \left.
  \frac{\partial {\bf f}}{\partial {\bf x}}
 \right|_{{\bf x}_f}
\right)^{\!\!\! n}\delta{\bf x} \nonumber\\
                                   & \simeq &
{\bf x}_f + {\bf J}^n\delta{\bf x}
\end{eqnarray}

hvor ${\bf J}$ er Jacobimatricen for funktionen ${\bf
f}({\bf x})$ udviklet i fikspunktet ${\bf x}_f$. Nu
antages, at ${\bf J}$ kan diagonaliseres som ${\bf J} =
{\bf U}^{-1}{\bf \Lambda}{\bf U}$, hvor ${\bf \Lambda}$ er
en diagonal matrix med egenv{\ae}rdierne
$\lambda_1,\ldots,\lambda_n$ i diagonalen. Da kan ${\bf
f}^n({\bf x}_f+\delta{\bf x})$ udtrykkes som

\begin{eqnarray}
{\bf f}^n({\bf x}_f+\delta{\bf x}) & \simeq & 
{\bf x}_f + ({\bf U}^{-1}{\bf \Lambda}
{\bf U})^n\delta{\bf x} \nonumber\\
& \simeq & 
{\bf x}_f + \underbrace{
 ({\bf U}^{-1}{\bf \Lambda}{\bf U})
 ({\bf U}^{-1}{\bf \Lambda}{\bf U})
 \cdots
 ({\bf U}^{-1}{\bf \Lambda}{\bf U})}_{\mbox{$n$ gange}}
\delta{\bf x}\nonumber\\
& \simeq & 
{\bf x}_f + {\bf U}^{-1}{\bf \Lambda}^n{\bf U}
\delta{\bf x}\nonumber\\
{\bf f}^n({\bf x}_f+\delta{\bf x}) & \simeq & 
{\bf x}_f + {\bf U}^{-1} 
\left[ 
 \begin{array}{ccc} \lambda_1 & & \\
         & \ddots &    \\
         & & \lambda_n 
 \end{array}
\right]^n {\bf U} \delta{\bf x}
\end{eqnarray}

Da fikspunktet ${\bf x}$ er stabilt, n{\aa}r kravet $\lim_{n
\rightarrow \infty} ||{\bf f}^n({\bf x}_f+\delta{\bf x})||
= 0$ er opfyldt, ser vi, at de line{\ae}re
stabilitetskriterier for ${\bf x}_f$ bliver

\begin{itemize}
 \item ${\bf x}_f$ {\bf stabil}\\
 Egenv{\ae}rdierne $\lambda_1,\ldots,\lambda_n$ for ${\bf J}$ 
 opfylder: $|\lambda_j|<1$ for alle $j = 1,\ldots,n$.
 \item ${\bf x}_f$ {\bf ustabil}\\
 Mindst \'{e}n af egenv{\ae}rdierne $\lambda_1,\ldots,\lambda_n$ 
 for ${\bf J}$ opfylder: $|\lambda_j|>1$ for $j = 1,\ldots,n$.
\end{itemize}

Efter fasts{\ae}ttelsen af disse generelle
stabilitetskriterier er vi nu rustet til at g{\aa} i krig
med selve Poincar\'{e}afbild\-ningen

\section{Poincar\'{e}afbild\-ningen}
Vi betragter nu det kontinuerte dynamiske system

\begin{equation}
 \dot{\bf c} = {\bf f}({\bf c},\mu)
 \label{eq:GenKin2}
\end{equation}

svarende til vores s{\ae}dvanlige kintiske ligning. Lad os
antage, at der til denne ligning findes en
periodisk\footnote{Faktisk beh{\o}ver l{\o}sningen ikke
n{\o}dvendigvis at v{\ae}re periodisk, men blot
begr{\ae}nset i faserummet, hvilket vi da ogs{\aa} i det
f{\o}lgende skal se eksempler p{\aa} i form af kaotisk
opf{\o}rsel.} l{\o}sning $\varphi_t({\bf c}_p)$.
L{\o}sninger i n{\ae}rheden af denne angives ved
$\varphi_t({\bf c})$. Nu v{\ae}lger vi s{\aa} et hyperplan
$\Omega$ i faserummet defineret ved afbild\-ningen $S: \R^n
\mapsto \R$

\begin{equation}
 \Omega = \{{\bf c} = c_1,\ldots,c_n \in \R^n 
 | S(c_1,\ldots,c_n) = 0\}
\end{equation}

s{\aa}ledes at f{\o}lgende to krav er opfyldt

\begin{itemize}
 \item $\varphi_t({\bf c})$ sk{\ae}rer altid $\Omega$ transversalt
 \item $\varphi_t({\bf c})$ returnere altid til $\Omega$ i samme
 retning.
\end{itemize}

Lad os betragte et punkt ${\bf c} \in \Omega$ og definere
{\em returneringsafbild\-ningen} $T_\Omega({\bf c}):
\R\times\Omega \mapsto \R$ som den tid det tager
$\varphi_t({\bf c})$ at returnere til hyperplanet $\Omega$.
$T_\Omega$ afbild\-er alts{\aa} et givet punkt fra
Poincar\'{e}planet p{\aa} det tidsrum, det tager for
punktet at returnere til Poincar\'{e}planet under
bev{\ae}gelsen $\varphi_t({\bf c})$. Formelt har vi

\begin{equation}
 \varphi_{T_\Omega({\bf c})}({\bf c}) \in \Omega
 \mbox{\ n{\aa}r\ }
 {\bf c} \in \Omega
\end{equation}

Ved hj{\ae}lp af returneringsafbild\-ningen $T_\Omega$ kan vi
nu definere Poincar\'{e}\-afbild\-\-ningen $P: \Omega \mapsto
\Omega$ som

\begin{equation}
 P({\bf c}) = \varphi_{T_\Omega({\bf c})}({\bf c})
 \mbox{\ hvor\ }
 {\bf c} \in \Omega
\end{equation}

Uden bevis angiver vi nu f{\o}lgende egenskaber ved
Poincar\'{e}\-afbild\-\-ningen

\begin{itemize}
  \item Lad ${\bf c}_p \in \Omega$ og lad $\varphi_t({\bf
  c}_p)$ v{\ae}re $T$-periodisk, da g{\ae}lder ${\bf c}
  \rightarrow {\bf c}_p \Rightarrow T_\Omega({\bf c})
  \rightarrow T$. Ydermere ser vi, at der m{\aa} g{\ae}lde
  $P({\bf c}_p)={\bf c}_p$, hvorfor punktet ${\bf c}_p$
  p{\aa} banekurven for den periodiske l{\o}sning vil
  v{\ae}re fikspunkt for Poincar\'{e}afbild\-ningen $P$ (se
  figur~\ref{fig:PoinScheme}a).
  %%%%%%%%%%%%%%%%%%%%%%%%%%%%%%%%%%%%%%%%%%%%%%%%%%%%%%
  \item Lad ${\bf c}_p \in \Omega$ og lad $\varphi_t({\bf
  c}_p)$ v{\ae}re en resonant bev{\ae}gelse p{\aa} en torus
  med to fundamentalfrekvenser $\omega_1$ og $\omega_2$.
  Dermed findes $p$ og $q$ s{\aa} $\frac{p}{q} =
  \frac{\omega_1}{\omega_2}$, hvorfor punktet ${\bf c}_p$
  vil v{\ae}re et fikspunkt for den $q$ gange iterede
  Poincar\'{e}afbild\-ning: $P^q({\bf c}_p)={\bf c}_p$.
  P{\aa} Poincar\'{e}planet svarer dette til $q$ punkter
  p{\aa} en cirkelperiferi (se
  figur~\ref{fig:PoinScheme}b).
  %%%%%%%%%%%%%%%%%%%%%%%%%%%%%%%%%%%%%%%%%%%%%%%%%%%%%%
  \item Lad ${\bf c}_p \in \Omega$ og lad $\varphi_t({\bf
  c}_p)$ v{\ae}re en kvasiperiodisk bev{\ae}gelse p{\aa} en
  torus med to rationelt uafh{\ae}ngige
  fundamentalfrekvenser $\omega_1$ og $\omega_2$.
  Bev{\ae}g\-el\-sen p{\aa} torusen er t{\ae}t og
  dynamikken p{\aa} det tilh{\o}rende Poincar\'{e}plan vil
  derfor ogs{\aa} v{\ae}re t{\ae}t p{\aa} cirkelperiferien
  (se figur~\ref{fig:PoinScheme}c).
  %%%%%%%%%%%%%%%%%%%%%%%%%%%%%%%%%%%%%%%%%%%%%%%%%%%%%%
  \item Lad ${\bf c} \in \Omega$ og lad $\varphi_t({\bf
  c})$ v{\ae}re en l{\o}sning til et dynamisk system med en
  kaotisk dynamik. Da vil dynamikken p{\aa} det
  tilh{\o}rende Poincar\'{e}plan finde sted p{\aa} en
  fraktal tiltr{\ae}kker (se figur~\ref{fig:PoinScheme}d).
\end{itemize}

I det vi henviser til resultaterne fra
afsnit~\ref{sec:DiskMap}, ser vi, at en periodisk
l{\o}sning til ligning~\ref{eq:GenKin2} netop er stabil,
n{\aa}r samtlige egenv{\ae}rdier knyttet til
lineariseringen af Poincar\'{e}afbild\-ningen $P$ er numerisk
mindre end $1$. Betragter vi specielt Jacobimatricen
forbundet med denne linearisering, er denne pr.\ definition
givet som

\begin{equation}
 \left.\frac{\partial P}{\partial {\bf c}}\right|_{{\bf c}_p} =
 \left.\frac{\partial \varphi_t({\bf c})}
            {\partial {\bf c}}\right|_{{\bf c}_p,t=T} 
\end{equation}

hvoraf vi ser, at den lineariserede
Poincar\'{e}afbild\-ning netop kan betragtes som
restriktionen af monodromimatricen ${\bf M}(T)$ til
Poincar\'{e}planet $\Omega$. Stabilitets\-overvejelserne
for Poincar\'{e}afbild\-ningen bliver derfor identiske med
de resultater, der afledtes ved hj{\ae}lp af
monodromimatricen. Poincar\'{e}afbild\-ningens $n-1$
egenv{\ae}rdier er alts{\aa} identiske med de $n-1$
Floquetmultiplikatorer, der svarer til de perturbationer,
der ikke er i gr{\ae}nsecyklusens retning.

%%%%%%%%%%%%%%%%%%%%%%%%%%%%%%%%%%%%%%%%%%%%%%%%%%%%%%%%%%%%%%%%%%%%%%%%
%% figur
%%
%% beskrivelse : Fire typiske attraktorer p{\aa} Poincareplan
%% dat         : fig71.ps
%% makroer     : PSTricks, PST-Plot, EPSF
%%%%%%%%%%%%%%%%%%%%%%%%%%%%%%%%%%%%%%%%%%%%%%%%%%%%%%%%%%%%%%%%%%%%%%%%
\boxfigure{tbp}{\textwidth}
{
\begin{center}
 \begin{pspicture}(0,0)(14,10)
% \psgrid(0,0)(0,0)(14,10)
 \psline(1,1)(13,1)(13,9)(1,9)(1,1)
 \psline(7,1)(7,9)
 \psline(1,5)(13,5)
 \rput[cc]{*0}(1.5,8.5){\footnotesize a)}
 \rput[cc]{*0}(6.5,8.5){\footnotesize $\Omega$}
 \rput[cc]{*0}(7.5,8.5){\footnotesize b)}
 \rput[cc]{*0}(12.5,8.5){\footnotesize $\Omega$}
 \rput[cc]{*0}(1.5,4.5){\footnotesize c)}
 \rput[cc]{*0}(6.5,4.5){\footnotesize $\Omega$}
 \rput[cc]{*0}(7.5,4.5){\footnotesize d)}
 \rput[cc]{*0}(12.5,4.5){\footnotesize $\Omega$}
 \rput[cc]{*-22}( 4,3){\psellipse(0,0)(1.8,1)}
 \rput[cc]{* 15}(10,7){\psellipse[linestyle=dotted](0,0)(1.5,1)}

 \pscircle*[](4,7){0.075}
 \rput[tl]{*0}(4.1,6.9){\footnotesize ${\bf c}_p$}
 \psplot[plotpoints=30,linestyle=dotted,origin={-4,-7}]{-2.3}{2.3}{0.14 x 3 exp mul}
 \parametricplot[plotpoints=30,linestyle=dotted,origin={-4,-7}]{-1.8}{1.8}
 {0.14 t 3 exp mul 0.7 mul 0.7 t mul add 0.14 t 3 exp mul 1.2 mul -1.2 t mul add}
 \rput[cc]{*0}(10,3.3){\epsfxsize=6cm\epsfysize=4cm\epsffile{fig71.ps}}
\end{pspicture}
\end{center}
}
{
\caption{\protect\capsize
Fire typiske former for dynamik p{\aa} et Poincar\'{e}plan: a)
Et fikspunkt p{\aa} Poincar\'{e}planet svarer til, at systemet
bev{\ae}ger sig p{\aa} en gr{\ae}nsecyklus i faserummet. b)
En bev{\ae}gelse p{\aa} en fasel{\aa}st torus i faserummet
viser sig p{\aa} Poincar\'{e}planet som diskrete punkter p{\aa}
periferien af en cirkel/ellipse. c) Er bev{\ae}gelsen
derimod kvasiperiodisk bliver cirklen/ellipsen kontinuert,
da dynamikken nu er t{\ae}t p{\aa} den p{\aa}g{\ae}ldende
torus. d) En kaotisk dynamik i faserummet vil give
anledning til en m{\ae}ngde af punkter p{\aa}
Poincar\'{e}planet med en fraktal dimension.}
\label{fig:PoinScheme}
}

\vspace{4.0mm}
En analytisk udregning af Poincar\'{e}afbild\-ningen for et
givet dynamisk system er sj{\ae}ldent mulig, da dette
mindst kr{\ae}ver at det dynamiske system kan l{\o}ses
analytisk. I stedet kan problemstillingen angribes via
numeriske metoder. Da egenskaberne ved
Poincar\'{e}\-afbild\-ningen kan afledes numerisk ved at
udregne monodromimatricen \cite{SpecRapport,Marek,Marek2},
er man dog oftest kun interesseret i at bestemme
sk{\ae}ringen mellem en given banekurve og
Poincar\'{e}planet $\Omega$. En beskrivelse af en r{\ae}kke
numeriske algoritmer til at l{\o}se dette problem kan
f.eks.\ findes i \cite{Marek2,Henon,LorenzPoin}.

\vspace{4.0mm}
F{\o}r vi pr{\ae}senterer et eksempel p{\aa} Poincar\'{e}
afbild\-ningens anvendelse i kemien, vil vi, for at
klarg{\o}re de hidtil indf{\o}rte begreber, diskutere et
simpelt eksempel, der kan angribes analytisk og
forh{\aa}bentlig bringer de fleste ting p{\aa} plads. Lad
os igen betragte normalformen for den superkritiske
Hopfbifurkation

\begin{equation}
 \begin{array}{lll}
  \dot{x} & = & x(\mu - (x^2 + y^2)) - \omega y\\
  \dot{y} & = & y(\mu - (x^2 + y^2)) + \omega x
 \end{array}
 \label{eq:SuperHopf2}
\end{equation}

Vi har tidligere set, at ligning~\ref{eq:SuperHopf2} har en
stabil periodisk l{\o}sning for $\mu>0$, hvorfor vi
{\o}nsker at bestemme en Poincar\'{e}afbild\-ning for dette
system med s{\ae}rlig henblik p{\aa} at beskrive dynamikken
i n{\ae}rheden af den periodiske l{\o}sning. Indf{\o}res
pol{\ae}re koordinater, transformeres
ligning~\ref{eq:SuperHopf2} til systemet

\begin{equation}
 \begin{array}{lll}
  \dot{r}      & = & r(\mu-r^2)\\
  \dot{\theta} & = & \omega
 \end{array}
 \label{eq:SuperHopfPol}
\end{equation}

hvor $r$ og $\theta$ angiver henholdsvis radial- og
angul{\ae}rv{\ae}rdi i den pol{\ae}re basis. Dette
differentialligningssystem har som generel l{\o}sning

\begin{subequations}
 \begin{eqalignno}
 r(t)       &= 
 \sqrt{\frac{\mu}{1-[1-\frac{\mu}{r_0^2}]e^{-2\mu t}}},
 \mbox{\ }\mbox{\ } r_0^2 = x_0^2 + y_0^2 \\
 \theta (t) &= 
 \omega t + \theta_0,
 \mbox{\ }\mbox{\ } \theta_0 = \arctan \frac{y_0}{x_0}
 \end{eqalignno}
\label{eq:PolSolut}
\end{subequations}

%%%%%%%%%%%%%%%%%%%%%%%%%%%%%%%%%%%%%%%%%%%%%%%%%%%%%%%%%%%%%%%%%%%%%%%%
%% figur
%%
%% beskrivelse : Poincare afbild\-ning for superkritisk Hopf
%% plt         : fig72.plt
%% dat         : fig72.dat
%% tex         : fig72.ps
%% type        : PSTricks, EPSF
%%%%%%%%%%%%%%%%%%%%%%%%%%%%%%%%%%%%%%%%%%%%%%%%%%%%%%%%%%%%%%%%%%%%%%%%
\boxfigure{tbp}{\textwidth}
{
\begin{center}
  \begin{pspicture}(0,0)(13.4,7)
%  \psgrid[](0,0)(0,0)(13.4,7)
   \psline[linewidth=0.8pt,arrowinset=0]{->}(6.5,1.0)(6.5,6.2)
   \psline[linewidth=0.8pt,arrowinset=0]{->}(2.0,3.4)(12.0,3.4)
   \pscircle*[](  8.53,3.4){0.07}
   \pscircle*[](  9.07,3.4){0.07}
   \pscircle*[]( 11.07,3.4){0.07}
   \rput[bl]{*0}( 8.63,3.5){\scriptsize $x_3$}
   \rput[bl]{*0}( 9.17,3.5){\scriptsize $x_2$}
   \rput[bl]{*0}(11.17,3.5){\scriptsize $x_1$}
   \rput[bl]{*0}( 5.40,3.8){\scriptsize $\gamma$}
   \rput[cl]{*0}( 9.60,0.4){\scriptsize $[x(t),y(t)]$}
   \rput[cl]{*0}(12.20,3.4){\footnotesize $x$}
   \rput[cb]{*0}( 6.50,6.3){\footnotesize $y$}
   \rput[bl]{*0}(0.12,0.22){%
                           \epsfxsize= 12.00cm 
                           \epsfysize=  6.19cm 
                           \epsffile{fig72.ps}}
  \end{pspicture}
\end{center}
}
{
\caption{\protect\capsize
Figuren illustrerer, hvorledes en Poincar\'{e}afbild\-ning
kan konstrueres for den superkritiske Hopfbifurkation.
Poincar\'{e}planet $\Omega$ v{\ae}lges som den positive
$x$-akse, hvorved systemets diskrete dynamik p{\aa} dette
underrum kan studeres kvantitativt. Punkterne $x_1$, $x_2$,
$x_3$ og $x_4$ udg{\o}r de f{\o}rste fire punkter i
f{\o}lgen af iterater for Poincar\'{e}afbild\-ningen. Denne
har gr{\ae}nsev{\ae}rdien $\protect\sqrt{\mu}$, svarende
til gr{\ae}nsecyklusens sk{\ae}ring med $\Omega$.}
\label{fig:SuperHopfSchem}
}

Vi v{\ae}lger nu Poincar\'{e}planet $\Omega$ som den
positive $x$-akse ($\Omega = \R^+\verb+\+{\ 0}\ $) og {\o}nsker
at beskrive udviklingen af et vilk{\aa}rligt punkt $x_0 \in
\Omega$ under den dermed definerede Poincar\'{e}afbild\-ning
(se figur~\ref{fig:SuperHopfSchem} for en skematisk
illustration af $\Omega$). Perioden $T$ for en svingning er
$\frac{2\pi}{\omega}$ og begyndelsesbetingelserne findes
som $x_0=r_0$ samt $\theta_0=0$ (da $y_0=0$). Derfor er den
diskrete dynamik p{\aa} $\Omega$ givet ved f{\o}lgende
Poincar\'{e}afbild\-ning $P(x_n)$

\begin{equation}
 x_{n+1} = P(x_n) = 
 \sqrt{\frac{\mu}{1-[1-\frac{\mu}{x_n^2}]e^{-\frac{4\pi\mu}{\omega}}}}
\end{equation}

Vi har tidligere vist, at ligning~\ref{eq:SuperHopf2} har 
den periodiske l{\o}sning

\begin{subequations}
 \begin{eqalignno}
  x(t) &= \sqrt{\mu} \cos (\omega t + \theta_0)\\
  y(t) &= \sqrt{\mu} \sin (\omega t + \theta_0)
 \end{eqalignno}
 \label{eq:HopfNormSolut}
\end{subequations}

Vi ser, at denne l{\o}sning antager v{\ae}rdien
$\sqrt{\mu}$ under dennes sk{\ae}ring med
Poin\-car\'{e}\-planet $\Omega$, hvorfor $x_0=\sqrt{\mu}$
er et fikspunkt for Poincar\'{e}afbild\-ningen. (Den
samvittighedsfulde l{\ae}ser b{\o}r checke, at $x_0 =
\sqrt{\mu}$ faktisk er et fikspunkt for $P(x_n)$).
Kvalitativt kan dynamikken p{\aa} $\Omega$ alts{\aa}
illustreres som f{\o}lgende

%%%%%%%%%%%%%%%%%%%%%%%%%%%%%%%%%%%%%%%%%%%%%%%%%%%%%%%%%%%%%%%%%%%%%%%%
%% figur
%%
%% beskrivelse : Kvalitativ Illustration af bev{\ae}gelse
%%               p{\aa} 1-dimensionalt Poincare plan for superkritisk
%%               Hopf bifurkation
%% makroer     : PSTricks, PST-Plot
%%%%%%%%%%%%%%%%%%%%%%%%%%%%%%%%%%%%%%%%%%%%%%%%%%%%%%%%%%%%%%%%%%%%%%%%
\begin{center}
 \begin{pspicture}(0,0)(14,3)
%  \psgrid[](0,0)(0,0)(14,3)
  \psline[linewidth=0.8pt,arrowinset=0]{->}(1.8,1.0)(11.8,1.0)
  \pscircle*[]( 2.8,1.0){0.07}
  \pscircle*[]( 5.8,1.0){0.07}
  \pscircle*[]( 7.8,1.0){0.07}
  \pscircle*[]( 9.1,1.0){0.07}
  \pscircle*[]( 9.8,1.0){0.07}
  \pscircle*[](10.8,1.0){0.07}
  \rput[tc]{*0}( 2.8,0.8){\footnotesize $r_1$}
  \rput[tc]{*0}( 5.8,0.8){\footnotesize $r_2$}
  \rput[tc]{*0}( 7.8,0.8){\footnotesize $r_3$}
  \rput[tc]{*0}( 9.1,0.8){\footnotesize $r_4$}
  \rput[tc]{*0}( 9.8,0.8){\footnotesize $r_5$}
  \rput[tc]{*0}(10.8,0.8){\footnotesize $r_f$}
  \rput[cc]{*0}(10.3,1.1){\footnotesize $\ldots$}
  \rput[cc]{*0}(12.0,1.0){\footnotesize $\Omega$}
  \psline[linewidth=0.8pt,arrowinset=0]{->}(4.29,1.6)(4.4,1.6)
  \psline[linewidth=0.8pt,arrowinset=0]{->}(6.79,1.45)(6.9,1.45)
  \psline[linewidth=0.8pt,arrowinset=0]{->}(8.50,1.3)(8.55,1.3)
  \psline[linewidth=0.8pt,arrowinset=0]{->}(9.55,1.2)(9.60,1.2)
  \parametricplot[origin={-2.8,-1.0},linewidth=0.8pt]{0.0}{3.0}
   {t -4 0.6 mul 3 dup mul div t dup mul mul
       4 0.6 mul 3 div t mul add}
  \parametricplot[origin={-5.8,-1.0},linewidth=0.8pt]{0.0}{2.0}
   {t -4 0.45 mul 2 dup mul div t dup mul mul
       4 0.45 mul 2 div t mul add}
  \parametricplot[origin={-7.8,-1.0},linewidth=0.8pt]{0.0}{1.3}
   {t -4 0.3 mul 1.3 dup mul div t dup mul mul
       4 0.3 mul 1.3 div t mul add}
  \parametricplot[origin={-9.1,-1.0},linewidth=0.8pt]{0.0}{0.7}
   {t -4 0.2 mul 0.7 dup mul div t dup mul mul
       4 0.2 mul 0.7 div t mul add}
 \end{pspicture}
\end{center}

Udfra dette {\o}nsker vi nu at bestemme lineariseringen af
$P(x_n)$ i punktet $P(x_0 = \sqrt{\mu})$ og sammenligne
dennes egenv{\ae}rdi med den v{\ae}rdi, vi tidligere fandt
for den ikke-trivielle Floquetmultiplikator for
ligning~\ref{eq:SuperHopf2}. Bestemmer vi den afledede
$P'(x_n)$, finder vi

\begin{equation}
 P'(x_0) = 
 \sqrt{\frac{\mu}{1-[1-\frac{\mu}{x_0^2}]e^{-\frac{4\pi\mu}{\omega}}}}
 \mu^{-\frac{3}{2}} x_0^{-3}e^{-\frac{4\pi\mu}{\omega}}
\end{equation}

Ved inds{\ae}ttelse af $x_0=\sqrt{\mu}$ f{\aa}s

\begin{equation}
 \left.P'(x_0)\right|_{\sqrt{\mu}} = e^{-\frac{4\pi\mu}{\omega}}
\end{equation}

der jo pr{\ae}cis var den v{\ae}rdi for den tilsvarende
Floquetmultiplikator, vi tid\-ligere bestemte under
anvendelse af Abels identitet.


\section{Anvendelse af Poincar\'{e}afbild\-ningen i kemiske modeller}
\label{PoincareApplied}
Efter dette meget skematiske eksempel {\o}nsker vi nu at
diskutere, hvorledes Poincar\'{e}afbild\-ningen kan
anvendes til at studere torusbifurkationer i kemiske
reaktioner. Vi har tidligere i afsnit~\ref{sec:TorusBif}
diskuteret, hvordan to successive torus\-bifurkationer giver
anledning til kaotisk opf{\o}rsel i den koblede
Brusselatormodel, analogt med det tidligere omtalte
Ruelle-Takens-Newhouse teorem. Det viser sig, at en
geometrisk forst{\aa}else af dette f{\ae}nomen bedst kommer
til sin ret ved at studere Poincar\'{e}afbild\-ninger.

\vspace{4.0mm}
En r{\ae}kke ``fornuftige'' v{\ae}rdier for
diffusionskonstanten $D_1$ udv{\ae}lges nu i intervallet
$[0.05206;0.05295]$ og den hertil svarende stabile
tiltr{\ae}kkers egenskaber studeres ved hj{\ae}lp af en
Poincar\'{e}afbild\-ning. N{\aa}r vi bev{\ae}ger os ned
igennem det n{\ae}vnte interval, vil en r{\ae}kke
torusbifurkationer for{\aa}rsage, at f{\o}lgende sekvens af
stabile tiltr{\ae}kkere observeres

\begin{itemize}
  \item kvasiperiodisk tiltr{\ae}kker
  \item $3T$-periodisk tiltr{\ae}kker
  \item kvasiperiodisk tiltr{\ae}kker
  \item kaotisk tiltr{\ae}kker
\end{itemize}

I samtlige af disse parameteromr{\aa}der er foretaget en
numerisk integration, resulterende i en banekurve p{\aa} en
af de respektive tiltr{\ae}kkere (hvis de valgte
begyndelsesbetingelser ikke ligger p{\aa} tiltr{\ae}kkeren, da
integreres indtil dette er tilf{\ae}ldet og det sidst
integrerede punkt v{\ae}lges som begyndelsesbetingelse for en
ny integration). Ved integrationen udregnes st{\o}rrelsen


\begin{equation}
  H({\bf c}) = \langle {\bf h},{\bf c}-{\bf c}_{\Omega} \rangle,
  \label{eq:PoincareSign}
\end{equation}

hvor ${\bf h}$ er en normalvektor til Poincar\'{e}planet
${\bf c}_\Omega$. Et fortegnsskift for $H$ indikerer
alts{\aa}, at banekurven har ``krydset'' Poincar\'{e}planet
og et ${\bf c}^*\in\Omega$ kan approksimeres yderligere via
line{\ae}r interpolation. M{\ae}ngden af disse punkter $\{
{\bf c}^*_1, {\bf c}^*_2, \ldots\ \}$ udg{\o}r systemets
diskrete banekurve p{\aa} det tilh{\o}rende
Poincar\'{e}\-plan for den integrerede banekurve.
Poincar\'{e}afbild\-ningen h{\o}rende til de fire
tilf{\ae}lde fremg{\aa}r af figur~\ref{fig:PoinPlot}.

%%%%%%%%%%%%%%%%%%%%%%%%%%%%%%%%%%%%%%%%%%%%%%%%%%%%%%%%%%%%%%%%%%%%%%%%
%% figur
%%
%% beskrivelse : Diagrammer til Poincareafbild\-ninger for
%%               koblet Brusselator underg{\aa}ende successive
%%               torusbifurkationer
%% type        : PSTricks, Fotokopi
%%%%%%%%%%%%%%%%%%%%%%%%%%%%%%%%%%%%%%%%%%%%%%%%%%%%%%%%%%%%%%%%%%%%%%%%
\footnotesize
\renewcommand{\capfont}{\bf}
\begin{figure}[tbp]
 \begin{center}
  \psset{xunit=0.93cm,yunit=0.93cm}
  \begin{pspicture}(0,0)(14,9.33)
%   \psgrid(0,0)(0,0)(14,9.33)
    %%%%%%%%%%%%%%%%%%%%%%%%% axis %%%%%%%%%%%%%%%%%%%%%%%%%%
    \psline{-}(4.66,0.00)(0.00,0.00)(0.00,4.66)
    \psline[origin={-4.66,0}](4.66,0.00)(0.00,0.00)(0.00,4.66)
    \psline[origin={-9.32,0}](4.66,4.66)(4.66,0.00)(0.00,0.00)(0.00,4.66)
    \psline[origin={0,-4.66}](4.66,0.00)(0.00,0.00)(0.00,4.66)
                             (4.66,4.66)(4.66,0.00)
    \psline[origin={-4.66,-4.66}](4.66,0.00)(0.00,0.00)(0.00,4.66)
                                 (4.66,4.66)(4.66,0.00)
    \psline[origin={-9.32,-4.66}](4.66,0.00)(0.00,0.00)(0.00,4.66)
                                 (4.66,4.66)(4.66,0.00)
    %%%%%%%%%%%%%%%%%%%%%%%%% xtics 1 %%%%%%%%%%%%%%%%%%%%%%%%%%
    \psline(0.93,0.00)(0.93,0.1)\psline(1.87,0.00)(1.87,0.1)
    \psline(2.80,0.00)(2.80,0.1)\psline(3.73,0.00)(3.73,0.1)
    \psset{origin={-4.66,0}}
    \psline(0.93,0.00)(0.93,0.1)\psline(1.87,0.00)(1.87,0.1)
    \psline(2.80,0.00)(2.80,0.1)\psline(3.73,0.00)(3.73,0.1)
    \psset{origin={-9.32,0}}
    \psline(0.93,0.00)(0.93,0.1)\psline(1.87,0.00)(1.87,0.1)
    \psline(2.80,0.00)(2.80,0.1)\psline(3.73,0.00)(3.73,0.1)
    %%%%%%%%%%%%%%%%%%%%%%%%% xtics 2 %%%%%%%%%%%%%%%%%%%%%%%%%%
    \psset{origin={0,-4.66}}
    \psline(0.93,0.00)(0.93,0.1)\psline(1.87,0.00)(1.87,0.1)
    \psline(2.80,0.00)(2.80,0.1)\psline(3.73,0.00)(3.73,0.1)
    \psset{origin={-4.66,-4.66}}
    \psline(0.93,0.00)(0.93,0.1)\psline(1.87,0.00)(1.87,0.1)
    \psline(2.80,0.00)(2.80,0.1)\psline(3.73,0.00)(3.73,0.1)
    \psset{origin={-9.32,-4.66}}
    \psline(0.93,0.00)(0.93,0.1)\psline(1.87,0.00)(1.87,0.1)
    \psline(2.80,0.00)(2.80,0.1)\psline(3.73,0.00)(3.73,0.1)
    %%%%%%%%%%%%%%%%%%%%%%%%% xtics 3 %%%%%%%%%%%%%%%%%%%%%%%%%%
    \psset{origin={0,-4.56}}
    \psline(0.93,0.00)(0.93,0.1)\psline(1.87,0.00)(1.87,0.1)
    \psline(2.80,0.00)(2.80,0.1)\psline(3.73,0.00)(3.73,0.1)
    \psset{origin={-4.66,-4.56}}
    \psline(0.93,0.00)(0.93,0.1)\psline(1.87,0.00)(1.87,0.1)
    \psline(2.80,0.00)(2.80,0.1)\psline(3.73,0.00)(3.73,0.1)
    \psset{origin={-9.32,-4.56}}
    \psline(0.93,0.00)(0.93,0.1)\psline(1.87,0.00)(1.87,0.1)
    \psline(2.80,0.00)(2.80,0.1)\psline(3.73,0.00)(3.73,0.1)
    %%%%%%%%%%%%%%%%%%%%%%%%% ytics 1 %%%%%%%%%%%%%%%%%%%%%%%%%%
    \psset{origin={0,0}}
    \psline(0.00,0.93)(0.10,0.93)\psline(0.00,1.87)(0.10,1.87)
    \psline(0.00,2.80)(0.10,2.80)\psline(0.00,3.73)(0.10,3.73)
    \psset{origin={0,-4.66}}
    \psline(0.00,0.93)(0.10,0.93)\psline(0.00,1.87)(0.10,1.87)
    \psline(0.00,2.80)(0.10,2.80)\psline(0.00,3.73)(0.10,3.73)
    %%%%%%%%%%%%%%%%%%%%%%%%% ytics 2 %%%%%%%%%%%%%%%%%%%%%%%%%%
    \psset{origin={-4.66,0}}
    \psline(0.00,0.93)(0.10,0.93)\psline(0.00,1.87)(0.10,1.87)
    \psline(0.00,2.80)(0.10,2.80)\psline(0.00,3.73)(0.10,3.73)
    \psset{origin={-4.66,-4.66}}
    \psline(0.00,0.93)(0.10,0.93)\psline(0.00,1.87)(0.10,1.87)
    \psline(0.00,2.80)(0.10,2.80)\psline(0.00,3.73)(0.10,3.73)
    %%%%%%%%%%%%%%%%%%%%%%%%% ytics 3 %%%%%%%%%%%%%%%%%%%%%%%%%%
    \psset{origin={-4.56,0}}
    \psline(0.00,0.93)(0.10,0.93)\psline(0.00,1.87)(0.10,1.87)
    \psline(0.00,2.80)(0.10,2.80)\psline(0.00,3.73)(0.10,3.73)
    \psset{origin={-4.56,-4.66}}
    \psline(0.00,0.93)(0.10,0.93)\psline(0.00,1.87)(0.10,1.87)
    \psline(0.00,2.80)(0.10,2.80)\psline(0.00,3.73)(0.10,3.73)
    %%%%%%%%%%%%%%%%%%%%%%%%% ytics 4 %%%%%%%%%%%%%%%%%%%%%%%%%%
    \psset{origin={-9.32,0}}
    \psline(0.00,0.93)(0.10,0.93)\psline(0.00,1.87)(0.10,1.87)
    \psline(0.00,2.80)(0.10,2.80)\psline(0.00,3.73)(0.10,3.73)
    \psset{origin={-9.32,-4.66}}
    \psline(0.00,0.93)(0.10,0.93)\psline(0.00,1.87)(0.10,1.87)
    \psline(0.00,2.80)(0.10,2.80)\psline(0.00,3.73)(0.10,3.73)
    %%%%%%%%%%%%%%%%%%%%%%%%% ytics 5 %%%%%%%%%%%%%%%%%%%%%%%%%%
    \psset{origin={-9.23,0}}
    \psline(0.00,0.93)(0.10,0.93)\psline(0.00,1.87)(0.10,1.87)
    \psline(0.00,2.80)(0.10,2.80)\psline(0.00,3.73)(0.10,3.73)
    \psset{origin={-9.23,-4.66}}
    \psline(0.00,0.93)(0.10,0.93)\psline(0.00,1.87)(0.10,1.87)
    \psline(0.00,2.80)(0.10,2.80)\psline(0.00,3.73)(0.10,3.73)
    %%%%%%%%%%%%%%%%%%%%%%%%% x symbols %%%%%%%%%%%%%%%%%%%%%%%%
    \rput[tc](0.93,-.1){\footnotesize $1$}
    \rput[tc](1.87,-.1){\footnotesize $2$}
    \rput[tc](2.80,-.1){\footnotesize $3$}
    \rput[tc](3.73,-.1){\footnotesize $4$}
    \rput[tc](5.59,-.1){\footnotesize $1$}
    \rput[tc](6.53,-.1){\footnotesize $2$}
    \rput[tc](7.46,-.1){\footnotesize $3$}
    \rput[tc](8.39,-.1){\footnotesize $4$}
    \rput[tc](10.25,-.1){\footnotesize $1$}
    \rput[tc](11.19,-.1){\footnotesize $2$}
    \rput[tc](12.12,-.1){\footnotesize $3$}
    \rput[tc](13.05,-.1){\footnotesize $4$}
    %%%%%%%%%%%%%%%%%%%%%%%%% y symbols %%%%%%%%%%%%%%%%%%%%%%%%
    \rput[rc](-.1,0.93){\footnotesize $1$}
    \rput[rc](-.1,1.87){\footnotesize $2$}
    \rput[rc](-.1,2.80){\footnotesize $3$}
    \rput[rc](-.1,3.73){\footnotesize $4$}
    \rput[rc](-.1,5.59){\footnotesize $1$}
    \rput[rc](-.1,6.53){\footnotesize $2$}
    \rput[rc](-.1,7.46){\footnotesize $3$}
    \rput[rc](-.1,8.39){\footnotesize $4$}
    %%%%%%%%%%%%%%%%%%%%% other symbols %%%%%%%%%%%%%%%%%%%%%%%%
    \rput[cc]( 4.20,8.86){\footnotesize a)}
    \rput[cc]( 8.86,8.86){\footnotesize b)}
    \rput[cc](13.52,8.86){\footnotesize c)}
    \rput[cc]( 4.20,4.20){\footnotesize d)}
    \rput[cc]( 8.86,4.20){\footnotesize e)}
    \rput[cc](13.52,4.20){\footnotesize f)}
    \rput[tc]( 4.40,-.1){\footnotesize $x_1$}
    \rput[tc]( 9.06,-.1){\footnotesize $x_1$}
    \rput[tc](13.72,-.1){\footnotesize $x_1$}
    \rput[rc](-.1, 4.40){\footnotesize $x_2$}
    \rput[rc](-.1, 9.06){\footnotesize $x_2$}
  \end{pspicture}
 \end{center}
 \caption{\protect\capsize
 Overgang fra torus til kaos i den koblede Brusselatormodel
 illustreret ved hj{\ae}lp af Poincar\'{e}afbild\-ninger. a)
 Torus, $D_1$ = 0.05247. b) Kvasiperiodisk bev{\ae}gelse med
 rotationstal $\frac{1}{3}$, $D_1$ = 0.05245. c) Torus,
 $D_1$ = 0.05242. d) Kaotisk tiltr{\ae}kker (bem{\ae}rk
 desintegration af torusen), $D_1$ = 0.0523. e) Kaotisk
 tiltr{\ae}kker, $D_1$ = 0.0522. f) Kaotisk tiltr{\ae}kker, $D_1$ =
 0.0521. Figuren stammer fra \protect\cite{Marek2}.}
 \label{fig:PoinPlot}
\end{figure}
\normalsize
\renewcommand{\capfont}{\rm}

\vspace{4.0mm}
I figur~\ref{fig:PoinPlot}a ses, hvorledes bev{\ae}gelsen i
faserummet resulterer i en lukket kurve p{\aa} Poincar\'{e}
planet, hvilket jo netop er {\ae}kvivalent med, at
bev{\ae}gelsen p{\aa} tiltr{\ae}kkeren er t{\ae}t p{\aa} en
torus (svarende til forekomsten af to rationelt
uaf\-h{\ae}ng\-ige vinkelfrekvenser). For lidt lavere
v{\ae}rdier af $D_1$ (figur~\ref{fig:PoinPlot}b)
fasel{\aa}ser bev{\ae}gelsen p{\aa} torusen og banekurven
p{\aa} Poincar\'{e}planet udg{\o}res nu af et diskret
s{\ae}t best{\aa}ende af tre punkter (faktisk svarer dette
til et rotationstal\footnote{Begrebet ``rotationstal''
stammer fra teorien for s{\aa}kaldte 1-dimensionale
cirkelafbild\-ninger. Denne teori har i b{\aa}de kemi og
fysik vist sig meget fyldestg{\o}rende med hensyn til at
beskrive dynamiske systemer, hvis dynamik er karakteriseret
ved to eller flere vinkelfrekvenser. For en definition af
``rotationstallet'' $\rho$ samt en god introduktion til
``cirkelafbild\-ninger'' se f.eks.\
\protect\cite{Devaney}.} $\rho =\frac{1}{3}$).

\vspace{4.0mm}
I figur~\ref{fig:PoinPlot}c ses, hvorledes bev{\ae}gelsen
atter bliver kvasiperiodisk. For endnu lavere v{\ae}rdier
af $D_1$ dannes en slags ``vinger'' eller foldninger p{\aa}
torusen (figur~\ref{fig:PoinPlot}d), idet denne n{\ae}rmest
s{\o}nderrives. I figur~\ref{fig:PoinPlot}e-f er torusen
fuld\-st{\ae}n\-dig destrueret og bev{\ae}gelsen i
faserummet er nu kaotisk, hvorfor dynamikken p{\aa}
Poincar\'{e}planet nu finder sted p{\aa} en fraktal
tiltr{\ae}kker. Den kaotiske opf{\o}rsel introduceres
alts{\aa} i dette tilf{\ae}lde via en desintegration af den
torus, hvorp{\aa} dynamikken finder sted.

\vspace{4.0mm}
Vi har alts{\aa} set, hvordan studiet af et dynamisk system
og en fortolkning af en parameters indflydelse p{\aa}
dettes dynamik kan anskueligg{\o}res p{\aa} en visuelt let
forst{\aa}elig m{\aa}de ved at betragte systemets
opf{\o}rsel p{\aa} et Poincar\'{e}plan. I
afsnit~\ref{sec:TorusBif} unders{\o}gte vi den samme
opf{\o}rsel ved blot at betragte en r{\ae}kke tilh{\o}rende
tidsr{\ae}kker. Sammenlignes denne metode med den netop
gennem\-g{\aa}ede, ser vi tydeligt, at de konklusioner, der
kan drages ved blot at unders{\o}ge en tidsserie, er langt
svagere end den information, der tilvejebringes ved at
studere Poincar\'{e}\-afbild\-ninger. Eksempelvis er det ikke
umiddelbart muligt at drage nogle konklusioner om, hvorvidt
en bev{\ae}gelse er kvasiperiodisk eller fasel{\aa}st ved
blot at studere numerisk integrerede tidsr{\ae}kker.

\section{P{\aa}visning af deterministisk kaos i
eksperimentel data}
\label{sec:ExperimentalPoin}

Efter dette modeleksempel {\o}nsker vi nu at diskutere,
hvordan Poincar\'{e}\-afbild\-ninger kan anvendes til at
analysere data fra eksperimentelle m{\aa}linger p{\aa}
kemiske systemer. Situationen her er ganske anderledes, da
vi her, imods{\ae}t\-ning til modelberegninger, meget
sj{\ae}ldent er i stand til at m{\aa}le koncentrationen af
samtlige stoffer, der er involverede i den
p{\aa}g{\ae}ldende reaktion. Dette problem kan dog p{\aa}
en bekvem m{\aa}de l{\o}ses ved i stedet at betragte
s{\aa}kaldte {\em rekonstruktioner\/} af faserummet.
S{\aa}danne rekonstruktioner kr{\ae}ver udelukkende, at
koncentrationen af et enkelt stof kan m{\aa}les
eksperimentelt.

\vspace{4.0mm}
I \cite{Swinney} er en r{\ae}kke eksperimentelle
tidsr{\ae}kker fra BZ-reaktionen blevet unders{\o}gt ved
hj{\ae}lp af s{\aa}danne rekonstruktioner. BZ-reaktionen er
blevet unders{\o}gt i en CSTR-opstilling, hvor
koncentrationen af Br$^-$-ioner m{\aa}les ved hj{\ae}lp af
en bromidelektrode. Under passende fors{\o}gsbetingelser
observeredes en r{\ae}kke uregelm{\ae}ssige svingninger.
S{\aa}danne svingninger var p{\aa} dav{\ae}rende tidspunkt
(1982) velkendte i BZ-reaktionen, hvorimod der endnu
herskede tvivl om, hvorvidt disse beskrev s{\aa}kaldt
deterministisk kaos. Videre var det uklart, om den
underliggende dynamik bag en s{\aa}dan tidsr{\ae}kke
skyldtes en bev{\ae}gelse p{\aa} en s{\aa}kaldt {\em
strange attractor\/}.

\vspace{4.0mm}
I ovenn{\ae}vnte artikel beskrives, hvorledes
bev{\ae}gelsen i det oprindelige faserum kan rekonstrueres
ud fra disse uregelm{\ae}ssige svingninger. Kalder vi
koncentrationen af Br$^-$-ioner til tidspunktet $t$ i en
s{\aa}dan tidsr{\ae}kke for $B(t)$, da kan tidsr{\ae}kken bekvemt
angives ved hj{\ae}lp af f{\o}lgen $B(t_i)_{i \in \{ 1,2,\:
\ldots \} }$. Udfra denne f{\o}lge konstrueres nu et
s{\ae}t af vektorer i et $m$-dimensionalt faserum som

\begin{equation}
  \left[ B(t_i), B(t_i+T), \ldots, B(t_i + (m-1)T) \right]
  \label{eq:ReconVector}
\end{equation}

%%%%%%%%%%%%%%%%%%%%%%%%%%%%%%%%%%%%%%%%%%%%%%%%%%%%%%%%%%%%%%%%%%%%%%%%
%% figur
%%
%% beskrivelse : Figurer fra Physica 8D, 257, 1983
%% plt         : -------
%% dat         : -------
%% tex         : -------
%% type        : Fotokopi
%%%%%%%%%%%%%%%%%%%%%%%%%%%%%%%%%%%%%%%%%%%%%%%%%%%%%%%%%%%%%%%%%%%%%%%%
\boxfigure{tbp}{\textwidth}
{
 \vspace{12cm}
}
{
\caption{\protect\capsize
a) 3-dimensional rekonstruktion af en kaotisk
tiltr{\ae}kker foretaget ud fra en tidsr{\ae}kke m{\aa}lt
p{\aa} BZ-reaktionen. b) 2-dimensional projektion af
tiltr{\ae}kkeren fra a). c) 1-dimensional afbildning
konstrueret ved at plotte de ordnede par $(x_i,x_{i+1})$
svarende til de succesive v{\ae}rdier, der fremkommer ved
banekurvens sk{\ae}ring med den stiblede linie i b).
Samtlige figure stammer fra \protect\cite{Swinney}.
}
\label{fig:PoincarePlots}
}

hvor $t_i = i \Delta t$, $i = 1, \ldots, \infty$.
St{\o}rrelsen $T$ kaldes {\em en tidsforskydning\/}. Lad nu
$n$ v{\ae}re dimensionen af det faserum, der beskriver
dynamikken af det oprindelige kemiske system (vi antager,
at der indg{\aa}r n dynamiske stoffer i den kemiske
reaktion). Hvis den nye dimension $m$ opfylder $m \geq
2n+1$, da vil denne rekonstruerede banekurve have de samme
topologiske egenskaber som den banekurve, der beskriver
systemts opf{\o}rsel i det virkelige $n$-dimensionale
kemi\-ske faserum. Vi siger, at faseportr{\ae}ttet knyttet
til en s{\aa}dan rekonstruktion udg{\o}r en indlej\-ring
for den oprindelige tiltr{\ae}kker.

\vspace{4.0mm}
Som tidligere n{\ae}vnt i afsnit 1.2 betyder dette, at
karakteristiske eksponenter som egenv{\ae}rdier,
Floquetmultiplikatorer og Lyapunoveksponter vil v{\ae}re
bevaret under en s{\aa}dan rekonstruktion. Bev{\ae}gelsen
p{\aa} en given stabil tiltr{\ae}kker (gr{\ae}nsecyklus,
torus, kaotisk tiltr{\ae}kker, etc.) i det rekonstruerede
rum vil alts{\aa} v{\ae}re sammenlignelig med den
tilsvarende bev{\ae}gelse i det oprindelige faserum.

\vspace{4.0mm}
Med hensyn til BZ-reaktionen s{\aa} skaber en s{\aa}dan
rekonstruktion dog en r{\ae}kke umiddelbare problemer.
Denne reaktion menes idag at involvere ikke mindre end
ca.\ 30 dynamiske stoffer, hvilket betyder, at
rekonstruktionen skulle foretages i et 61-dimensionalt
faserum. Heldigvis kan $m$ dog godt v{\ae}lges betydeligt
mindre uden at systemets v{\ae}sentligste egenskaber mistes
ved en rekonstruktion. Eksempelvis vil simple svingninger i
et h{\o}jdimensionalt system som BZ-reaktionen stadig
v{\ae}re bevaret ved en 2-dimensional rekonstruktion, da de
tilh{\o}rende banekurver vil v{\ae}re lukkede.

\vspace{4.0mm}
P{\aa} baggrund af s{\aa}danne overvejelser har man i
\cite{Swinney} indskr{\ae}nket sig til at studere
3-dimensionale rekonstruktioner af de m{\aa}lte
tidsr{\ae}kke. En s{\aa}dan rekonstruktion af en kaotisk
tiltr{\ae}kker er vist i figur~\ref{fig:PoincarePlots}a.
Udfra denne v{\ae}lger man nu et 2-dimensionalt
Poincar\'{e}plan s{\aa}ledes, at banekurverne sk{\ae}rer
dette tranversalt. Selve Poincar\'{e}planet og dets
gennemsk{\ae}ringer af den rekonstruerede banekurve
fremg{\aa}r af figur~\ref{fig:PoincarePlots}b.

\vspace{4.0mm}
Lad os nu betragte et punkt $x_1$ p{\aa} Poincar\'{e}planet.
N{\aa}r banekurven har sk{\aa}ret Poincar\'{e}planet i $x_1$,
bev{\ae}ger denne sig videre ud p{\aa} tiltr{\ae}kkeren og
returnerer til planet i et punkt $x_2$. Til ethvert punkt
p{\aa} Poincar\'{e}planet kan vi alts{\aa} entydigt knytte et
returneringspunkt. Punktet $x_1$ definerer alts{\aa} en
f{\o}lge af s{\aa}danne successive returneringspunkter som

\begin{equation}
  \left[ (x_1,x_2), (x_2,x_3), (x_3,x_4), \ldots \right]
\end{equation}

Plotter vi nu dette s{\ae}t punkter $(x_i,x_{i+1})$, ses
disse umiddelbart at ligge p{\aa} en parametriserbar kurve,
som illustreret i figur~\ref{fig:PoincarePlots}c. Vi ser
med andre ord, at de uregelm{\ae}ssige svingninger, der her
er beskrevet for BZ-reaktionen, ikke er af tilf{\ae}ldig
natur, men derimod har en deterministisk karakter, da
disses egenskaber kan beskrives ved en kontinuert funktion.
Med andre ord kan ovenst{\aa}ende s{\ae}t af
returneringspunkter angives ved en punktf{\o}lge $(x_i)$,
s{\aa}ledes at de enkelte punkter rekursivt er givet som
$x_{i+1} = f(x_i)$, hvor $f$ alts{\aa} er en kontinuert
funktion.

\vspace{4.0mm}
Denne egenskab ved Poincar\'{e}afbild\-ningen er i
\cite{Swinney} yderligere blevet anvendt til at bestemme
Lyapunoveksponenten for $f$, der tilmed vil v{\ae}re den
st{\o}rste for det p{\aa}g{\ae}ldende datas{\ae}t. Denne
approksimeres ved at ``fitte'' punkterne i
figur~\ref{fig:PoincarePlots} til en kubisk spline.
Lyapunoveksponenten $\lambda$ for en 1-dimensional
afbild\-ning er defineret som

\begin{equation}
  \lambda = \lim_{N \rightarrow \infty} 
            \sum_{i=1}^N \ln |f'(x_i)|
  \label{eq:LyaExp}
\end{equation}

hvorfor differentialkvotienten $f'(x_i)$ kan udregnes ved
hj{\ae}lp af den kubiske spline og Lyapunoveksponenten
$\lambda$ kan hermed bestemmes. Udfra datas{\ae}ttet er
denne fundet til 

\begin{equation}
  \lambda = 0.3 \pm 0.1
\end{equation}

Da deterministisk kaos altid indeb{\ae}rer, at en eller
flere Lyapunoveksponenter er strengt st{\o}rre end nul, er
dette yderligere evidens for forekomsten af denne type kaos
i de beskrevne uregelm{\ae}ssige svingninger.

%%%%%%%%%%%%%%%%%%%%%%%%%%%%%%%%%%%%%%%%%%%%%%%%%%%%%%%%%%%%%%%%%%%%%%%%
%% figur
%%
%% beskrivelse : Skematisk 1D-Poincare funktion
%% plt         : fig33.plt
%% dat         : fig33.dat
%% tex         : fig33.tex
%% type        : TeXDraw
%%%%%%%%%%%%%%%%%%%%%%%%%%%%%%%%%%%%%%%%%%%%%%%%%%%%%%%%%%%%%%%%%%%%%%%%
\renewcommand{\capfont}{\bf}
\begin{figure}[t]
 \begin{minipage}{6cm}
  % GNUPLOT: LaTeX using TEXDRAW macros
\begin{texdraw}
\normalsize
\ifx\pathDEFINED\relax\else\let\pathDEFINED\relax
 \def\QtGfr{\ifx (\TGre \let\YhetT\cpath\else\let\YhetT\relax\fi\YhetT}
 \def\path (#1 #2){\move (#1 #2)\futurelet\TGre\QtGfr}
 \def\cpath (#1 #2){\lvec (#1 #2)\futurelet\TGre\QtGfr}
\fi
\drawdim pt
\setunitscale 0.24
\linewd 3
\textref h:L v:C
\linewd 4
\path (133 61)(816 61)(816 707)(133 707)(133 61)
\move (183 39)\textref h:C v:C \htext{{\footnotesize${\rm I}_1$}}
\move (300 39)\htext{{\footnotesize${\rm I}_2$}}
\move (598 39)\htext{{\footnotesize${\rm I}_3$}}
\path (133 61)(133 61)(136 163)(140 251)(143 323)(147 385)
\cpath (150 434)(154 475)(157 507)(161 535)(164 558)
\cpath (167 577)(170 596)(174 613)(177 629)(181 643)
\cpath (184 656)(188 666)(191 676)(195 684)(198 692)
\cpath (202 697)(205 701)(208 705)(212 707)
\path (224 707)(225 707)(229 705)(232 703)(236 700)(239 696)
\cpath (243 691)(246 685)(250 679)(253 672)(256 664)
\cpath (259 656)(263 646)(266 636)(270 625)(273 613)
\cpath (277 600)(280 586)(284 573)(287 558)(291 544)
\cpath (294 530)(297 517)(300 503)(304 490)(307 477)
\cpath (311 464)(314 451)(318 439)(321 427)(325 415)
\cpath (328 404)(332 395)(335 385)(338 375)(341 366)
\cpath (345 357)(348 348)(352 340)(355 332)(359 323)
\cpath (362 315)(366 308)(369 301)(373 294)(376 287)
\cpath (379 280)(383 274)(386 268)(389 261)(393 254)
\cpath (396 248)(400 242)(403 236)(407 231)(410 225)
\cpath (414 221)(417 215)(421 211)(424 206)(427 203)
\cpath (430 198)(434 195)(437 191)(441 188)(444 184)
\cpath (448 181)(451 178)(455 175)(458 171)(462 169)
\cpath (465 166)(468 163)(471 161)(474 158)(478 155)
\cpath (481 152)(485 151)(488 148)(492 146)(495 143)
\cpath (499 142)(502 140)(506 138)(509 135)(512 134)
\cpath (516 132)(519 130)(522 128)(526 126)(529 125)
\cpath (533 124)(536 122)(540 120)(543 119)(547 117)
\cpath (550 116)(554 115)(557 113)(560 112)(563 110)
\cpath (567 109)(570 107)(574 107)(577 106)(581 104)
\cpath (584 103)(588 102)(591 101)(595 99)(598 98)
\cpath (601 98)(604 97)(608 96)(611 95)(615 94)
\cpath (618 93)(622 92)(625 91)(629 90)(632 89)
\cpath (636 89)(639 88)(642 87)(645 86)(649 86)
\cpath (652 85)(656 84)(659 84)(663 83)(666 82)
\cpath (670 82)(673 81)(677 80)(680 80)(683 80)
\cpath (687 79)(690 78)(693 78)(697 77)(700 76)
\cpath (704 75)(707 75)(711 74)(714 73)(718 73)
\cpath (721 72)(725 71)(728 71)(731 71)(734 70)
\cpath (738 70)(741 69)(745 69)(748 68)(752 68)
\cpath (755 67)(759 67)(762 66)(766 66)(769 65)
\cpath (772 65)(775 64)(779 64)(782 63)(786 63)
\cpath (789 63)(793 62)(796 62)(800 62)(803 62)
\cpath (807 62)(810 62)(813 61)(816 61)
\linewd 3
\path (220 61)(220 61)(220 95)(220 129)(220 162)(220 197)
\cpath (220 231)(220 265)(220 299)(220 332)(220 367)
\cpath (220 401)(220 435)(220 468)(220 503)(220 537)
\cpath (220 571)(220 605)(220 638)(220 673)(220 707)
\path (375 61)(375 61)(375 72)(375 85)(375 97)(375 108)
\cpath (375 121)(375 133)(375 144)(375 157)(375 169)
\cpath (375 181)(375 193)(375 205)(375 217)(375 229)
\cpath (375 241)(375 253)(375 265)(375 277)(375 289)
\end{texdraw}

 \end{minipage}
 \ \hfill \
 \protect\vspace{-0.5cm}
 \begin{minipage}{7cm}
  \caption{\protect\capsize
   Den viste graf illu\-strerer en ske\-ma\-tisk
   frem\-stil\-ling af funktionen $f$, der be\-skri\-ver
   Poincar\'{e}afbild\-ningen i det re\-kon\-stru\-erede
   faserum. Ved en gentagen foldning og str{\ae}kning af de
   tre intervaller $I_1$, $I_2$ og $I_3$ genereres en
   kaotisk dynamik for de successive iterater af $f$. En
   dynamik med disse egenskaber siges at finde sted p{\aa}
   en ``strange attractor''.\protect\vspace*{\fill}}
 \end{minipage}
 \label{fig:StrechFold}
 \normalsize
\vspace{5mm}
\end{figure}
\renewcommand{\capfont}{\rm}

\vspace{4.0mm}
Sluttelig vil vi vise, hvorledes man kan drage yderligere
konklusioner ved\-r{\o}rende den tiltr{\ae}kker, hvorp{\aa}
dynamikken for BZ-reaktionen foreg{\aa}r. Lad os betragte
funktionen $f$, s{\aa}ledes som denne skematisk er gengivet
i figur~\ref{fig:StrechFold}. Heraf fremg{\aa}r, at $f$
beskriver en surjektiv afbild\-ning af et interval $I$
p{\aa} $I$. Opdel nu $I$ i tre disjunkte delintervaller
$I_1$, $I_2$ og $I_3$ som vist i
figur~\ref{fig:StrechFold}. Lad os unders{\o}ge billedet
af disse tre delintervaller under $f$. Udfra en rent
grafisk argumentation f{\aa}s

\begin{subequations}
 \begin{eqalignno}
  f(I_1) &= I_1 \cup I_2 \cup I_3\\
  f(I_2) &= I_3\\
  f(I_3) &= I_1 \cup I_2
 \end{eqalignno}
 \label{eq:SubIntervalImage}
\end{subequations}

Dette betyder, at intervallet $I$ samtidig
str{\ae}kkes og foldes under afbild\-ningen $f$. Skematisk
kan dette angives som

%%%%%%%%%%%%%%%%%%%%%%%%%%%%%%%%%%%%%%%%%%%%%%%%%%%%%%%%%%%%%%%%%%%%%%%%
%% figur
%%
%% beskrivelse : Illustration af foldning og str{\ae}kning 
%%               for 1D-afbild\-ning i Swinneys artikel
%% type        : PSTricks
%%%%%%%%%%%%%%%%%%%%%%%%%%%%%%%%%%%%%%%%%%%%%%%%%%%%%%%%%%%%%%%%%%%%%%%%
\vspace{4.0mm}
\begin{center}
  \begin{pspicture}(0,0)(5,3)
%   \psgrid[subgriddiv=1,griddots=10,gridlabels=7pt](0,0)(0,0)(5,3)
    \psline[linewidth=0.8pt]{|-|}(0,0)(5,0)
    \psline[linearc=0.2,linewidth=0.8pt]{|-|}(0,1.7)(5,1.7)(5,1.3)(2.5,1.3)
    \psline[linewidth=0.8pt]{|-|}(0,3)(5,3)
    \psline[linewidth=0.8pt,arrowinset=0]{->}(1,2.7)(1,2.0)
    \psline[linewidth=0.8pt,arrowinset=0]{->}(1,1.2)(1,0.5)
    \psline[linewidth=0.8pt,arrowinset=0](0.75,3.1)(0.75,2.9)
    \psline[linewidth=0.8pt,arrowinset=0](2.5,3.1)(2.5,2.9)
    \psline[linewidth=0.8pt,arrowinset=0](2.5,1.8)(2.5,1.6)
    \psline[linewidth=0.8pt,arrowinset=0](2.5,-.1)(2.5,.1)
    \rput[bc]{*0}(0.45,3.1){\footnotesize $I_1$}
    \rput[bc]{*0}(1.63,3.1){\footnotesize $I_2$}
    \rput[bc]{*0}(3.75,3.1){\footnotesize $I_3$}
    \rput[lc]{*0}(1.3,2.35){\footnotesize str{\ae}kning}
    \rput[lc]{*0}(1.3,0.75){\footnotesize foldning}
  \end{pspicture}
\end{center}

\vspace{4.0mm}
Betragter vi en iteration af intervallet $I$ under $f$, vil
dette foldes og str{\ae}kkes p{\aa} en yderst kompliceret
m{\aa}de, der visuelt minder om den gentagne foldning og
udruling af wienerbr{\o}dsdej. Kvalitativt betyder dette,
at punkter t{\ae}t p{\aa} hinanden vil fjerne sig fra
hinanden med en eksponentiel hastighed. En tiltr{\ae}kker med
s{\aa}danne egenskaber kaldes en {\em strange attractor\/}.
Vi ser, at den dynamik, der karakteriserer de n{\ae}vnte
tidsr{\ae}kker for BZ-reaktionen, svarer til en
bev{\ae}gelse p{\aa} en strange attractor.



